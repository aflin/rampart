\subsection{Geographical coordinate functions}

The geographical coordinate functions allow for efficient processing
of latitude / longitude operations.  They allow for the conversion of
a latitude/longitude pair into a single ``geocode'', which is a single
\verb`long` value that contains both values.  This can be used to
easily compare it to other geocodes (for distance calculations) or for
finding other geocodes that are within a certain distance.

% - - - - - - - - - - - - - - - - - - - - - - - - - - - - - - - - - - - - - -
\subsubsection{azimuth2compass}

\begin{verbatim}
  azimuth2compass(double azimuth [, int resolution [, int verbosity]])
\end{verbatim}

The \verb`azimuth2compass` function converts a numerical azimuth value
(degrees of rotation from 0 degrees north) and converts it into a
compass heading, such as \verb`N` or \verb`Southeast`.  The exact text
returned is controlled by two optional parameters, \verb`resolution`
and \verb`verbosity`.

\verb`Resolution` determines how fine-grained the values returned
are.  There are 4 possible values:

\begin{itemize}
\item \verb`1` - Only the four cardinal directions are used (N, E, S,
W)
\item \verb`2` \em{(default)} - Inter-cardinal directions (N, NE, E,
etc.)
\item \verb`3` - In-between inter-cardinal directions (N, NNE, NE, ENE,
E, etc.)
\item \verb`4` - ``by'' values (N, NbE, NNE, NEbN, NE, NEbE, ENE, EbN,
E, etc.)
\end{itemize}

\verb`Verbosity` affects how verbose the resulting text is.  There are
two possible values:

\begin{itemize}
\item \verb`1` \em{(default)} - Use initials for direction values (N,
NbE, NNE, etc.)
\item \verb`2` - Use full text for direction values (North, North by
east, North-northeast, etc.)
\end{itemize}

For an azimuth value of \verb`105`, here are some example results of
\verb`azimuth2compass`:

\begin{verbatim}
azimuth2compass(105): E
azimuth2compass(105, 3): ESE
azimuth2compass(105, 4): EbS
azimuth2compass(105, 1, 2): East
azimuth2compass(105, 3, 2): East-southeast
azimuth2compass(105, 4, 2): East by south
\end{verbatim}

% - - - - - - - - - - - - - - - - - - - - - - - - - - - - - - - - - - - - - -
\subsubsection{azimuthgeocode}

\begin{verbatim}
  azimuthgeocode(geocode1, geocode2 [, method])
\end{verbatim}

The \verb`azimuthgeocode` function calculates the directional heading
going from one geocode to another.  It returns a number between 0-360
where 0 is north, 90 east, etc., up to 360 being north again.

The third, optional \verb`method` parameter can be used to specify
which mathematical method is used to calculate the direction.  There
are two possible values:

\begin{itemize}
\item \verb`greatcircle` {\em(default)} - The ``Great Circle'' method
is a highly accurate tool for calculating distances and directions on
a sphere.  It is used by default.

\item \verb`pythagorean` - Calculations based on the pythagorean
method can also be used.  They're faster, but less accurate as the
core formulas don't take the curvature of the earth into
consideration.  Some internal adjustments are made, but the values are
less accurate than the \verb`greatcircle` method, especially over long
distances and with paths that approach the poles.
\end{itemize}

\EXAMPLE 

For examples of using \verb`azimuthgeocode`, see the \verb`geocode`
script in the \verb`texis/samples` directory.

% - - - - - - - - - - - - - - - - - - - - - - - - - - - - - - - - - - - - - -
\subsubsection{azimuthlatlon}

\begin{verbatim}
  azimuthlatlon(lat1, lon1, lat2, lon2, [, method])
\end{verbatim}

The \verb`azimuthlatlon` function calculates the directional heading
going from one latitude-longitude point to another.  It operates
identically to \verb`azimuthgeocode`, except azimuthlatlon takes its
parameters in a pair of latitude-longitude points instead of geocode
values.

The third, optional \verb`method` parameter can be used to specify
which mathematical method is used to calculate the direction.  There
are two possible values:

\begin{itemize}
\item \verb`greatcircle` {\em(default)} - The ``Great Circle'' method
is a highly accurate tool for calculating distances and directions on
a sphere.  It is used by default.

\item \verb`pythagorean` - Calculations based on the pythagorean
method can also be used.  They're faster, but less accurate as the
core formulas don't take the curvature of the earth into
consideration.  Some internal adjustments are made, but the values are
less accurate than the \verb`greatcircle` method, especially over long
distances and with paths that approach the poles.

\end{itemize}

% - - - - - - - - - - - - - - - - - - - - - - - - - - - - - - - - - - - - - -
\subsubsection{dms2dec, dec2dms}

\begin{verbatim}
  dms2dec(dms)
  dec2dms(dec)
\end{verbatim}

The \verb`dms2dec` and \verb`dec2dms` functions are for changing back
and forth between the deprecated Texis/Vortex ``degrees minutes
seconds'' (DMS) format (west-positive) and ``decimal degree'' format
for latitude and longitude coordinates.  All SQL geographical functions
expect decimal degree parameters (the Vortex \verb`<code2geo>` and
\verb`<geo2code>` Vortex functions expect Texis/Vortex DMS).

Texis/Vortex DMS values are of the format $DDDMMSS$.  For example,
35\degree 15' would be represented as 351500.

In decimal degrees, a degree is a whole digit, and minutes \& seconds
are represented as fractions of a degree.  Therefore, 35\degree 15'
would be 35.25 in decimal degrees.

Note that the Texis/Vortex DMS format has {\em west}-positive
longitudes (unlike ISO 6709 DMS format), and decimal
degrees have {\em east}-positive longitudes.  It is up to the caller
to flip the sign of longitudes where needed.

% - - - - - - - - - - - - - - - - - - - - - - - - - - - - - - - - - - - - - -
\subsubsection{distgeocode}

\begin{verbatim}
  distgeocode(geocode1, geocode2 [, method] )
\end{verbatim}

The \verb`distgeocode` function calculates the distance, in miles,
between two given geocodes.  It uses the ``Great Circle'' method for
calculation by default, which is very accurate.  A faster, but less
accurate, calculation can be done with the Pythagorean theorem.  It is
not designed for distances on a sphere, however, and becomes somewhat  
inaccurate at larger distances and on paths that approach the poles.
To use the Pythagorean theorem, pass a third string parameter,
``\verb`pythagorean`'', to force that method.  ``\verb`greatcircle`''
can also be specified as a method. 

For example:
\begin{itemize}
\item New York (JFK) to Cleveland (CLE), the Pythagorean method is off by 
.8 miles (.1\%)
\item New York (JFK) to Los Angeles (LAX), the Pythagorean method is off by 
22.2 miles (.8\%)
\item New York (JFK) to South Africa (PLZ), the Pythagorean method is off by
430 miles (5.2\%)
\end{itemize}

\EXAMPLE 

For examples of using \verb`distgeocode`, see the \verb`geocode`
script in the \verb`texis/samples` directory.

\SEE

\verb`distlatlon`

For a very fast method that leverages geocodes for selecting cities
within a certain radius, see the \verb`<code2geo>` and \verb`<geo2code>`
functions in the Vortex manual.

% - - - - - - - - - - - - - - - - - - - - - - - - - - - - - - - - - - - - - -
\subsubsection{distlatlon}

\begin{verbatim}
  distlatlon(lat1, lon1, lat2, lon2 [, method] )
\end{verbatim}

The \verb`distlatlon` function calculates the distance, in miles,
between two points, represented in latitude/longitude pairs in decimal
degree format.

Like \verb`distgeocode`, it uses the ``Great Circle'' method by
default, but can be overridden to use the faster, less accurate
Pythagorean method if ``\verb`pythagorean`'' is passed as the optional
\verb`method` parameter.

For example:
\begin{itemize}
\item New York (JFK) to Cleveland (CLE), the Pythagorean method is off by 
.8 miles (.1\%)
\item New York (JFK) to Los Angeles (LAX), the Pythagorean method is off by 
22.2 miles (.8\%)
\item New York (JFK) to South Africa (PLZ), the Pythagorean method is off by
430 miles (5.2\%)
\end{itemize}

\SEE

\verb`distgeocode`

% - - - - - - - - - - - - - - - - - - - - - - - - - - - - - - - - - - - - - -

\subsubsection{latlon2geocode, latlon2geocodearea}

\begin{verbatim}
  latlon2geocode(lat[, lon])
  latlon2geocodearea(lat[, lon], radius)
\end{verbatim}

The \verb`latlon2geocode` function encodes a given latitude/longitude
coordinate into one \verb`long` return value.  This encoded value -- a
``geocode'' value -- can be indexed and used with a special variant of
Texis' \verb`BETWEEN` operator for bounded-area searches of a
geographical region.

The \verb`latlon2geocodearea` function generates a bounding area
centered on the coordinate.  It encodes a given latitude/longitude
coordinate into a {\em two-} value \verb`varlong`.  The returned
geocode value pair represents the southwest and northeast corners of a
square box centered on the latitude/longitude coordinate, with sides
of length two times \verb`radius` (in decimal degrees).  This bounding
area can be used with the Texis \verb`BETWEEN` operator for fast
geographical searches.  \verb`latlon2geocodearea` was added in
version 6.00.1299627000 20110308; it replaces the deprecated Vortex
\verb`<geo2code>` function.

The \verb`lat` and \verb`lon` parameters are \verb`double`s in the
decimal degrees format.  (To pass $DDDMMSS$ ``degrees minutes
seconds'' (DMS) format values, convert them first with \verb`dms2dec`
or \verb`parselatitude()`/\verb`parselongitude()`.).  Negative numbers
represent south latitudes and west longitudes, i.e. these functions
are east-positive, and decimal format (unlike Vortex \verb`<geo2code>`
which is west-positive, and DMS-format).

Valid values for latitude are -90 to 90 inclusive.  Valid values for
longitude are -360 to 360 inclusive.  A longitude value less than -180
will have 360 added to it, and a longitude value greater than 180 will
have 360 subtracted from it.  This allows longitude values to continue
to increase or decrease when crossing the International Dateline, and
thus avoid a non-linear ``step function''.  Passing invalid \verb`lat`
or \verb`lon` values to \verb`latlon2geocode` will return -1.  These
changes were added in version 5.01.1193956000 20071101.

In version 5.01.1194667000 20071109 and later, the \verb`lon`
parameter is optional: both latitude and longitude (in that order) may
be given in a single space- or comma-separated text (\verb`varchar`)
value for \verb`lat`.  Also, a \verb`N`/\verb`S` suffix (for latitude)
or \verb`E`/\verb`W` suffix (for longitude) may be given; \verb`S` or
\verb`W` will negate the value.

  In version 6.00.1300154000 20110314 and later, the latitude and/or
longitude may have just about any of the formats supported by
\verb`parselatitude()`/\verb`parselongitude()`
(p.~\pageref{parselatitudeSqlFunc}), provided they are disambiguated
(e.g. separate parameters; or if one parameter, separated by a comma
and/or fully specified with degrees/minutes/seconds).

\EXAMPLE
\begin{samepage}
% NOTE: this is tested in Vortex test454:
\begin{verbatim}
  -- Populate a table with latitude/longitude information:
  create table geotest(city varchar(64), lat double, lon double,
                       geocode long);
  insert into geotest values('Cleveland, OH, USA', 41.4,  -81.5,  -1);
  insert into geotest values('Seattle, WA, USA',   47.6, -122.3,  -1);
  insert into geotest values('Dayton, OH, USA',    39.75, -84.19, -1);
  insert into geotest values('Columbus, OH, USA',  39.96, -83.0,  -1);
  -- Prepare for geographic searches:
  update geotest set geocode = latlon2geocode(lat, lon);
  create index xgeotest_geocode on geotest(geocode);
  -- Search for cities within a 3-degree-radius "circle" (box)
  -- of Cleveland, nearest first:
  select city, lat, lon, distlatlon(41.4, -81.5, lat, lon) MilesAway
  from geotest
  where geocode between (select latlon2geocodearea(41.4, -81.5, 3.0))
  order by 4 asc;
\end{verbatim}
\end{samepage}

For more examples of using \verb`latlon2geocode`, see the
\verb`geocode` script in the \verb`texis/samples` directory.

\CAVEATS

The geocode values returned by \verb`latlon2geocode` and
\verb`latlon2geocodearea` are platform-dependent in format and
accuracy, and should not be copied across platforms. On platforms with
32-bit \verb`long`s a geocode value is accurate to about 32 seconds
(around half a mile, depending on latitude).  -1 is returned for
invalid input values (in version 5.01.1193955804 20071101 and later).

NOTES

The geocodes produced by these functions are compatible with the codes
used by the deprecated Vortex functions \verb`<code2geo>` and
\verb`<geo2code>`.  However, the \verb`<code2geo>` and
\verb`<geo2code>` functions take Texis/Vortex DMS format ($DDDMMSS$
``degrees minutes seconds'', as described in the \verb`dec2dms` and
\verb`dms2dec` SQL functions).

\SEE

\verb`geocode2lat`, \verb`geocode2lon`

% - - - - - - - - - - - - - - - - - - - - - - - - - - - - - - - - - - - - - -
\subsubsection{geocode2lat, geocode2lon}

\begin{verbatim}
  geocode2lat(geocode)
  geocode2lon(geocode)
\end{verbatim}

The \verb`geocode2lat` and \verb`geocode2lon` functions decode a
geocode into a latitude or longitude coordinate, respectively.  The
returned coordinate is in the decimal degrees format.  In
version 5.01.1193955804 20071101 and later, an invalid geocode value
(e.g. -1) will return NaN (Not a Number).

If you want $DDDMMSS$ ``degrees minutes seconds'' (DMS) format, you
can use \verb`dec2dms` to convert it.

\EXAMPLE

\begin{verbatim}
  select city, geocode2lat(geocode), geocode2lon(geocode) from geotest;
\end{verbatim}

\CAVEATS

As with \verb`latlon2geocode`, the \verb`geocode` value is platform-dependent
in accuracy and format, so it should not be copied across platforms,
and the returned coordinates from \verb`geocode2lat` and
\verb`geocode2lon` may differ up to about half a minute from the
original coordinates (due to the finite resolution of a \verb`long`).
In version 5.01.1193955804 20071101 and later, an invalid geocode value
(e.g. -1) will return \verb`NaN` (Not a Number).

\SEE

\verb`latlon2geocode`

% - - - - - - - - - - - - - - - - - - - - - - - - - - - - - - - - - - - - - -
\subsubsection{parselatitude, parselongitude}
\label{parselatitudeSqlFunc}

\begin{verbatim}
  parselatitude(latitudeText)
  parselongitude(longitudeText)
\end{verbatim}

The \verb`parselatitude` and \verb`parselongitude` functions parse a
text (\verb`varchar`) latitude or longitude coordinate, respectively,
and return its value in decimal degrees as a \verb`double`.  The
coordinate should be in one of the following forms (optional parts in
square brackets):

[$H$] $nnn$ [$U$] [\verb`:`] [$H$] [$nnn$ [$U$] [\verb`:`] [$nnn$ [$U$]]] [$H$] \\
$DDMM$[$.MMM$...] \\
$DDMMSS$[$.SSS$...]

where the terms are:

\begin{itemize}
  \item $nnn$ \\

    A number (integer or decimal) with optional plus/minus sign.  Only
    the first number may be negative, in which case it is a south
    latitude or west longitude.  Note that this is true even for
    $DDDMMSS$ (DMS) longitudes -- i.e. the ISO 6709 east-positive
    standard is followed, not the deprecated Texis/Vortex
    west-positive standard.

  \item $U$ \\
    A unit (case-insensitive):
    \begin{itemize}
      \item \verb`d`
      \item \verb`deg`
      \item \verb`deg.`
      \item \verb`degrees`
      \item \verb`'` (single quote) for minutes
      \item \verb`m`
      \item \verb`min`
      \item \verb`min.`
      \item \verb`minutes`
      \item \verb`"` (double quote) for seconds
      \item \verb`s` (iff \verb`d`/\verb`m` also used for degrees/minutes)
      \item \verb`sec`
      \item \verb`sec.`
      \item \verb`seconds`
      \item Unicode degree-sign (U+00B0), in ISO-8559-1 or UTF-8
    \end{itemize}
    If no unit is given, the first number is assumed to be degrees,
    the second minutes, the third seconds.  Note that ``\verb`s`'' may
    only be used for seconds if ``\verb`d`'' and/or ``\verb`m`'' was
    also used for an earlier degrees/minutes value; this is to help
    disambiguate ``seconds'' vs. ``southern hemisphere''.

  \item $H$ \\
    A hemisphere (case-insensitive):
    \begin{itemize}
      \item \verb`N`
      \item \verb`north`
      \item \verb`S`
      \item \verb`south`
      \item \verb`E`
      \item \verb`east`
      \item \verb`W`
      \item \verb`west`
    \end{itemize}
    A longitude hemisphere may not be given for a latitude, and
    vice-versa.

  \item $DD$ \\

    A two- or three-digit degree value, with optional sign.  Note that
    longitudes are east-positive ala ISO 6709, not west-positive like
    the deprecated Texis standard.

  \item $MM$ \\

    A two-digit minutes value, with leading zero if needed to make two digits.

  \item $.MMM$... \\

    A zero or more digit fractional minute value.

  \item $SS$ \\

    A two-digit seconds value, with leading zero if needed to make two digits.

  \item $.SSS$... \\

    A zero or more digit fractional seconds value.

\end{itemize}

  Whitespace is generally not required between terms in the first
format.  A hemisphere token may only occur once.
Degrees/minutes/seconds numbers need not be in that order, if units
are given after each number.  If a 5-integer-digit $DDDMM$[$.MMM$...]
format is given and the degree value is out of range (e.g. more than
90 degrees latitude), it is interpreted as a $DMMSS$[$.SSS$...] value
instead.  To force $DDDMMSS$[$.SSS$...] for small numbers, pad with
leading zeros to 6 or 7 digits.

\EXAMPLE

\begin{verbatim}
insert into geotest(lat, lon)
  values(parselatitude('54d 40m 10"'),
         parselongitude('W90 10.2'));
\end{verbatim}

\CAVEATS

An invalid or unparseable latitude or longitude value will return
\verb`NaN` (Not a Number).  Extra unparsed/unparsable text may be
allowed (and ignored) after the coordinate in most instances.
Out-of-range values (e.g. latitudes greater than 90 degrees) are
accepted; it is up to the caller to bounds-check the result.  The
\verb`parselatitude` and \verb`parselongitude` SQL functions were
added in version 6.00.1300132000 20110314.

% ----------------------------------------------------------------------------
