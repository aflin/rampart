\chapter{Metamorph: The Program Inside the Program}

\section{Background}

Deep inside of Texis stands Metamorph, the original
incarnation of Thunderstone's text retrieval methodology.
While Metamorph has grown into Texis, its intelligent
text query language has remained intact. It carries with
it a whole philosophy of concept based text retrieval
which sets a layer of assumptions about how text, particularly
English, should be processed to retain its meaning.

Metamorph is made reference to in many places throughout
Thunderstone's program documentation.  While originally
it was a stand-alone program, it lives on as functions
which are used at all levels of the search, chiefly
thought about only in relation to the formulation of queries
themselves, and attendant processing of the vocabulary
base of the language itself.

Most of what you need to know about Metamorph is included
at the appropriate places in our documentation.  This
section is devoted to some of the aspects of Metamorph
which were documented early on in the Research \& Development
of our product line, but may still be of interest for
context rich applications or just to the curious user.

\section{Original Design Mandate}

Based on their past experience, many people have become accustomed to
certain fixed ways of viewing the information dilemma; i.e., how to
deal with too much available stored information.  However, most of
these accepted solutions contain inherent flaws which can be readily
seen and solved once one realizes that there are choices now which we
did not have not too long ago.

Thunderstone took a fresh approach to text analysis and data
search problems from the very core; that is, in the methods of pattern
matching itself.  It has always been our premise that if we could
develop pattern matching algorithms fast and precise enough, we could
take a different approach to data research which would provide more
accurate and reliable results, while being less labor intensive in the
processing of the raw data.

Different algorithms are most efficient for different tasks; therefore
we have learned how to make use of the best tool for the best job by
integrating a number of smaller programs through a senior program.
The Metamorph Search Engine calls upon the right pattern matcher for
the right job, encompassing several different pattern matchers:  PPM
(Parallel Pattern Matcher), SPM (Single Pattern Matcher), Wildcard '*'
matcher, REX (Regular EXpression pattern matcher), XPM (ApproXimate
Pattern Matcher), and NPM (Numeric Pattern Matcher).  Each pattern
matcher handles a certain class of search problems and is optimized
for a particular type of task, making the overall search environment
as fast and efficient as possible.

In light of the rapid advances being made in hardware configurations,
we have maintained that the best approach to the creation of all data
manipulation tools must be totally software solutions.  In this way we
can as required modify our software to take advantage of new hardware
breakthroughs, rather than be tied to outdated hardware systems.  We
believe this represents a more cost effective solution to our clients
so that only their hardware budget concerns need be weighty, knowing
our software will be flexible enough to be portable to the hardware
configuration of their choice.

By way of example, we are in fact operational on more than 25
different Unix platforms, evidencing more portability than any other
software company around.  Using our Application Program
Interfaces (API's), we can extend portability of the Search Engine to
almost any platform. And now that Metamorph is inside Texis,
a complete and multi-faceted Relational DataBase Management System,
its use can be extended truly to any application that might
be desired, now in this exciting global village of
totally connected Internet and World Wide Web applications.

Some hardware pattern matching solutions have surfaced along the way,
some quite impressive.  However, there are certain inherent problems
with these techniques no matter how fast they appear to be.  What can
be located is limited, and there is limited portability.  If the host
configuration is greatly changed, an entirely new piece of hardware
will in most cases be required.  Metamorph on the other hand, being
entirely software, always has retained the potential of being moved to
a new hardware environment.  Our technology has shown flexibility over
time in DOS, Unix, MVS, OS/2, MACH, Macintosh, Windows, and NT
environments, spanning micro, mini and mainframe applications.


\section{Tokenization and Inverted Files}

Two leading techniques in text retrieval that have heretofore been
used have been file inversion and tokenization.  The dominant problem
with both of these techniques is that they require modification of the
original data to be searched in order for it to be accessible to the
data retrieval tool.  The second problem, which has deeper
ramifications, is that in order to perform file inversion or
tokenization such programs make certain predisposed determinations
about the lexical items that will be later identified.

How good such programs are will depend in great part upon their
ability to identify and then locate specified lexical items.  In most
cases the set of lexical items identified by the inversion or
tokenization routines is simply insufficient to guarantee the
retrieval of all things that one might want to search for.  But even
where the set of identifiable lexical items is reasonably good, one
always has a certain basic limitation to contend with:  one will never
be able to locate a superset of the lexical item listed in the look up
table.

It is for these reasons that when we make use of indexing techniques,
we supplement them with a final linear text read where required.  Many
content oriented definitive searches must contain a linear read of
context to make accurate determinations involving relevance.

In systems where a lookup table exists containing either file pointers
or tokens, context is missing.  You cannot search for something next
to something else, as no adequate record of related locations of items
is contained in the lookup table.  You may yet find what you are
looking for, but you may have to convert the file to its original form
before you can do so.

It will be hard to find a program which stores any, let alone all,
possible combinations of words making up phrases, idiomatic
expressions, and acronyms in the lookup table.  While you can look for
``Ming'', or you can look for ``Ling'', you cannot directly look for
``Ming-Ling''.  Another tricky category is that of regular expressions
involving combinations of lexical items.  If you are searching the
Bible by verse, you want to find a pattern of ``digit, digit, colon,
digit''.  This cannot be done when occurrences of digits are stored
separately from the occurrences of colons.

Our own database tool Texis has a modifiable lexical analyzer, which
goes further than other indexing programs to attend to this problem.
However, a linear free text scan still gives the maximum flexibility
for looking for any type of pattern in relation to another pattern,
and therefore has been included in Texis as part of the search
technique used to qualify hits.

Metamorph is the search engine inside of Texis which contains maximum
capability for identification of a diverse variety of lexical items.
No other program has such an extended capability to recognize these
items.  We can look for special expressions by themselves or in
proximity to sets of concepts.  Logical set operators 'and', 'or', and
'not' are applied to whole sets of lexical items, rather than just
single, specified lexical items.  Because Metamorph is so fast,
benchmarked even in its very early years of development as searching
up to 4.5 megabytes per second in some Unix environments, we can read
text live where required and get extremely impressive results.

Where stream text is involved, such as for a message analysis system
where large amounts of information are coming in at a steady rate, or
a news alert profiling system, you could not practically speaking
tokenize all the data as it was coming in before reading it to find
out if there was anything worth attending to in that data.  Using a
tokenized system to search incoming messages at a first pass would be
very unwieldy, as well as inefficient and lacking in discretion.
Indexes are more appropriately useful when searching large amounts of
archived data.  Texis makes use of Metamorph search engines internally
where required to read the stream text in all of its rich context but
without losing speed of efficiency.

Where Texis sends out a Metamorph query, it is fast and thorough in
its search and its retrieval.  A parser picks out phrases without
having to mark them as such, along with any known associations.
Search questions are automatically expanded to sets of concepts
without you having to tell it what those sets are, or having to
knowledge engineer connections in the files.  Regular expressions can
be located which other systems might miss.  Numeric quantities entered
as text and misspellings and typos can be found.  Intersections of all
the permutations of the defined sets can be located, within a defined
unit of text defined by the user.  No other search program is capable
of this.

Even were you to find a comparable Database Management System with
which to manage your text (which we challenge you to do!), at the
bottom line, you could not find all the specific things that the
Metamorph search engine would let you find.  In Texis we now have a
completely robust body; inside, it yet retains the heart and soul of
Metamorph.


\section{Metamorph Query Language Highlights}

Metamorph allows you to search for intersections of sets of lexical
items, while also performing prefix and suffix morpheme processing.
Once your target is found the question arises:  what rules govern
proximity of the items you wish to find?  In traditional searching
tools, this has been done only on a line by line basis, or by using
some quantitative proximity range.  Metamorph can search by an
intelligent textual unit, a sentence.  Whether searching by paragraph,
page, chart entry, or memo, in all respects it is intended that the
user may define real qualitative units of communication inside of
which the concepts he is interested in connecting are located.

The user can specify right within his or her query the delimiters of
choice:  i.e., he can look within a sentence, paragraph, a proximity
of 500 characters, or a specially defined textual unit such as a memo.
To the degree that lexical items can be defined and located as
beginning and end delimiters, your intersections will be located
within those parameters.

REX, Metamorph's Regular EXpression pattern matcher, can be used
outside Texis as a special text processing tool.  REX can locate
uniquely repeated patterns in files, such as headers, footers,
captions, diagram references, and so on.  If the existing patterns
aren't adequate to your needs, you can put them into your files rather
easily.  For example, using REX's incrementing counter and its search
and replace facility, one could locate paragraph starts and number
them.  Such pattern identification can be made use of by other
applications.

Metamorph allows for editing word sets, by hand or using the Backref
program.  This means that you may select which associations you would
like in connection to any search; you can create your own concept sets
permanently for future use.  You can fine tune the search to use
associations of only a certain part of speech.  You can enter all
known spelling variations of any particular search word in the same
way.  You can generally customize the program to include your own
nomenclature and vocabulary, making it increasingly intelligent the
longer it is in use.

You can call up the ApproXimate Pattern Matcher (XPM) and tell it to
look for a certain percentage of proximity to an entered string,
finding misspelled names and typos.  You can also look for numeric
quantities entered as text with the Numeric Pattern Matcher (NPM),
finding "four score and seven years ago" in the Gettysburg address
when searching for events 80 to 100 years ago.

The Metamorph Query Language was designed so that the text searcher
can get rudimentary satisfaction of result right away without needing
to know much of anything.  At the same time, a more complex query can
be written with just a little self-training time on the advanced
search syntax possibilities.  We like to say that there's {\em
nothing} that can't be found with a Metamorph query.  This flexibility
enhancing Texis, means the system designer setting up the search
environment and wanting to customize it to certain applications can
accomplish all his goals.

Texis, with Metamorph inside it, is intended to be a modular set of
tools to attack the formidable problem of how to get at and deal with
large quantities of information, when you don't really know what you
want to know or where to find it; and in the most dynamic, efficient,
and pragmatic way possible.  It is intended for discrete analysis
where the human supplies the final cognition.


\section{Your Basic Metamorph Search}

Metamorph has often been classified as a form of Artificial
Intelligence since its functions fall into the categories of knowledge
acquisition, natural language processing, and intelligent text
retrieval.

The software attempts in its own way to understand your question,
represent its understanding to the data in the files, and come up with
relevant responses as retrieved portions of full text information
which best correspond to your questions.

Metamorph's vocabulary is around 250,000+ word connections,
constructed in a dense web of associations and equivalences.  Search
parameters can be adjusted to dynamically dictate surface and deep
inference.  The program's responses can be controlled so that they are
direct or abstract in relation to your questions.  Proximity of
concept can be fine tuned so as to qualify degree of relevance,
providing matches which are sometimes concrete, sometimes abstract.

Think of your text as a field of information which was put together by
a human being for a stated purpose; the Equivalence File acts as an
intelligent language filter through which relevant associations
occurring as common denominators can be located and retrieved out of
the information in those files.

Metamorph retrieves data as a match to queries for response from any
text.  Metamorph can search files which are not flat ASCII, but it is
the ASCII characters which will be recognized.

To help you get started, some demo text files are supplied with the
Texis package.  These are files in the ``\verb`c:\morph3\text`''
directory if you installed on Drive C in DOS or Windows; or in the
``\verb`/usr/local/morph3/text`'' directory if you are on Unix:

\begin{tabbing}
{\bf Filenamexxx} \= {\bf Description}  \kill
{\bf Filename} \> {\bf Description}  \\
alien       \> science fiction excerpts \\
constn      \> the US Constitution \\
declare     \> the US Declaration of Independence \\
garden      \> descriptive prose \\
kids        \> children's adventure stories \\
liberty     \> Patrick Henry's ``liberty or death'' speech \\
events      \> downloaded news information from mid 1990 \\
qadhafi     \> magazine interview with Mohamar Qadhafi \\
socrates    \> summary from Plato's Republic about Socrates.
\end{tabbing}

Let's say you have set things up one way or another to be searching
these demo text files, and have Texis set up to type in a query.  The
easiest thing to do is think of a few concepts or keywords you'd like
to see matched near each other in a sentence.  For example, ``power''
and ``struggle''.  This can be entered on the query line as
``\verb`power struggle`'', where your default proximity is set to
search by sentence.  Or if it has not, you can enter your query
as ``\verb`power struggle w/sent`''.  If you have enabled
Metamorph hit markup, you might retrieve a passage of text which
could be set up to look something like this:

\begin{screen}
\begin{verbatim}
File: c:\morph3\text\events
PageNo: 11
Query: power struggle

 million whites and 28 million blacks.  []Charges range from using
 excessive FORCE on antigovernment protests and torture of detainees
 to openly backing the ANC's rival Zulu movement Inkatha in bloody
 CLASHES in Natal province.[]
     "It's a Frankenstein which has been created and inherited by a
 racist set-up," Slovo told a news conference, noting the ANC
 "reserves the right" to resume the armed struggle "should the
 government fail to carry out its undertakings."
\end{verbatim}
\end{screen}

In this example, this section of text was selected because the
sentence marked by [] blocks at beginning and end matched
the search request; i.e.:

\begin{quote}
Charges range from using excessive {\bf force} on anti-government protests
and torture of detainees to openly backing the ANC's rival Zulu movement
Inkatha in bloody {\bf clashes} in Natal province.
\end{quote}

This sentence was selected as a hit because it contained a concept
match to both concepts entered on the query line:  i.e.,
``\verb`force`'' matched ``\verb`power`'', and ``\verb`clashes`''
matched ``\verb`struggle`''.

\section{More Complex Query Syntax}

As you learn more about how the Metamorph Query Language works, your
queries can become more complex, if so desired.  Two key factors which
are part of any query, along with a statement of the search items you
are looking for, are intersection quantity and delimited text unit.

Search items can be weighted, by marking them for inclusion with a
plus sign `\verb`+`', or for exclusion with a minus sign `\verb`-`'.
All other search items are considered equally weighted.

It is understood that the maximum number of unmarked search items will
always be looked for, unless a designation follows those sets of
`\verb`@#`'; i.e., the at sign `\verb`@`' followed by the desired
number of intersections; i.e., \verb`0-9+`.  Designating ``\verb`@0`''
would mean ``zero intersections required''.

     For example, one might enter this query:
\begin{verbatim}
     +Near East @1 military political economic involvement
\end{verbatim}

This query would locate any sentences which definitely contained a
reference to one of the countries in the Near East set, and also
contained at least one intersection of 2 of the 4 specified search
sets:  ``\verb`military`'', ``\verb`political`'', ``\verb`economic`'',
and ``\verb`involvement`''.  Thus it would retrieve the following
sentence, where the words in curly braces indicate the set to which
the preceding word member belongs:

\begin{quote}
      Troops in {\bf Turkey} {\em \{Near East\}} became {\bf engaged}
      {\em \{involvement\}} in a heated {\bf battle} {\em \{military\}}
      when a training exercise was misinterpreted as a hostile initiative.
\end{quote}

You can further qualify the type of search results you are after by
changing the delimiters which dictate the proximity of entered
concepts.  This can be done by adding to the query line
``\verb`w/delim`'' where ``\verb`delim`'' is either ``\verb`line`'',
``\verb`sent`'', ``\verb`para`'', or ``\verb`page`''.  (NOTE:
``\verb`page`'' only works where page formatting characters exist in
the text.)  Or you can just choose an arbitrary number of characters
to search within; like:  ``\verb`w/250`'' to indicate a window of 500
characters (250 forward, 250 back) around the first search item found.

     Therefore you could change the nature of the above query:
\begin{verbatim}
     +Near East @1 military political economic involvement w/para
\end{verbatim}

By adding a delimiter specification ``\verb`w/para`'', you are
instructing Metamorph to search by paragraph rather than by sentence.
The required proximity of concept is now much broader.  More
paragraphs will be found to fit all the search requirements than
sentences.  But the sentences are likely to be closer matches to the
query, as the correlation of concept had to be a closer match.

In addition to entering English questions and keywords, Metamorph has
several special pattern matchers which allow the user to search for
practically any type of expression.  Any such expression is a valid
search item, which can be assigned a logic operator, and searched for
in proximity to other keywords, concepts, or expressions, as outlined
in the previous section.

All the above things are covered elsewhere in detail. Their strength
can be drawn upon where very specific types of results are desired.

Metamorph lets you create new and varied viewpoints as to what the
originator of the files might have intended, without hours of
preprocessing or knowledge engineering.  These new views and
impressions can be informative, educational, and useful, enhancing
content analysis and data correlation.
