\chapter{Queries Involving Calculated Values}{\label{chp:CalcQuer}}

While Texis focuses on manipulation of textual information, data can
also be operated on numerically.  Queries can be constructed which
combine calculated values with text search.

To illustrate the material in this chapter, we'll use an employee
table which the Personnel Department keeps to manage salaries and
benefits.  A sampling of the data stored in this table follows:

%\begin{screen}
\begin{verbatim}
  EID  ENAME               DEPT   SALARY   BENEFITS
  101  Aster, John A.      MKT    32000    FULL
  102  Barrington, Kyle    MGT    45000    FULL
  103  Chapman, Margaret   LIB    22000    PART
  104  Jackson, Herbert    RND    30000    FULL
  105  Price, Stella       FIN    42000    FULL
  106  Sanchez, Carla      MKT    35000    FULL
  107  Smith, Roberta      RND    25000    PART
\end{verbatim}
%\end{screen}

\section{Arithmetic Calculations}

This section covers the computational features of Texis.  They are
adequate to allow the user to perform computations on the data and/or
retrieve rows based on conditions involving computations.  For
example, you can adjust salaries for a 5 percent across-the-board
increase, or you can compute weekly salaries (i.e., salary divided by
52).

Arithmetic calculations are performed on fields, or columns, in the
database.  An {\em arithmetic expression} is used to describe the
desired computation.  The expression consists of column names and
numeric constants connected by parentheses and arithmetic operators.
Table~\ref{tab:ArithOp} shows the arithmetic operators used in Texis.

\begin{table}
\caption{Arithmetic Operators Supported in Texis}{\label{tab:ArithOp}}
\begin{center}
\begin{tabular}{|l|l|l|} \hline
Arithmetic Operation  & Texis Operator & Example              \\ \hline\hline
Addition              &  \verb`+`      & \verb`SALARY + 2000` \\
Subtraction           &  \verb`-`      & \verb`SALARY - 1000` \\
Multiplication        &  \verb`*`      & \verb`SALARY * 1.05` \\
Division              &  \verb`/`      & \verb`SALARY / 26`   \\ \hline
\end{tabular}
\end{center}
\end{table}

Typically, the arithmetic expression is used in the \verb`SELECT` clause to
perform calculations on data stored in the table.

{\bf Example:}
Next year every employee will receive a 5 percent salary increase.
List the names of each employee, his or her current salary, and next
year's salary.

Enter this statement:
\begin{verbatim}
     SELECT  ENAME, SALARY, SALARY * 1.05
     FROM    EMPLOYEE ;
\end{verbatim}
Where ``\verb`SALARY * 1.05`'' is the arithmetic expression.

The results are:

%\begin{screen}
\begin{verbatim}
  ENAME               SALARY     SALARY * 1.05
  Aster, John A.      32000      33600
  Barrington, Kyle    45000      47250
  Chapman, Margaret   22000      23100
  Jackson, Herbert    30000      31500
  Price, Stella       42000      44100
  Sanchez, Carla      35000      36750
  Smith, Roberta      25000      26250
\end{verbatim}
%\end{screen}

The expression ``\verb`SALARY * 1.05`'' results in each value in the
salary column being multiplied by 1.05.  The results are then
displayed in a new column that is labeled \verb`SALARY * 1.05`.

If more than one arithmetic operator is used in an arithmetic
expression, parentheses can be used to control the order in which the
arithmetic calculations are performed.  The operations enclosed in
parentheses are computed before operations that are not enclosed in
parentheses.  For example, the expression:
\begin{verbatim}
     12 * (SALARY + BONUS)
\end{verbatim}
means bonus is added to salary, and then this result is multiplied by
12.

If parentheses are omitted or if several operations are included
within the parentheses, the order in which calculations are performed
is as follows:

\begin{enumerate}
\item First, all multiplication, division and modulo\footnote{The
  modulo operator ({\tt \%}) was added in Texis version 8; it is
  supported for integral types.} operations are performed.
\item Then, all addition and subtraction operations are performed.
\end{enumerate}

For example, in the expression:
\begin{verbatim}
     SALARY + SALARY * .05
\end{verbatim}
the value in the SALARY column is multiplied by .05, and then the
salary value is added to this intermediate result.

When two or more computations in an expression are at the same level
(e.g., multiplication and division), the operations are executed from
left to right.  For example, in the expression:
\begin{verbatim}
     SALARY / 12 * 1.05
\end{verbatim}
the salary value is first divided by 12, and then this result is
multiplied by 1.05.

Arithmetic calculation can also be used in a \verb`WHERE` clause to select
rows based on a calculated condition.  In addition, arithmetic
expressions can be used in the {\tt HAVING} and {\tt ORDER BY} clauses, which will
be discussed in later sections of this chapter.

{\bf Example:}
List the names of all employees earning a monthly salary above \$3000.

This query:
\begin{verbatim}
     SELECT  ENAME
     FROM    EMPLOYEE
     WHERE   (SALARY/12) > 3000 ;
\end{verbatim}
results in:

%\begin{screen}
\begin{verbatim}
  ENAME
  Barrington, Kyle
  Price, Stella
\end{verbatim}
%\end{screen}

The rows in the \verb`EMPLOYEE` table are retrieved if the condition ``salary
divided by 12'' is greater than \$3000.  This was true only for
Barrington and for Price, whose annual salaries (respectively \$45,000
and \$42,000) are greater than \$3000 when divided by 12 months.

\section{Manipulating Information By Date}

In Texis dates are stored as integers representing an absolute number
of seconds from January 1, 1970, Greenwich Mean Time.  This is done
for efficiency, and to avoid confusions stemming from differences in
relative times assigned to files from different time zones.
The allowable range of years is 1970 through 2037. Years between 1902 and
1970 may be stored and compared for equality (\verb`=`) but will not compare
correctly using less than (\verb`<`) and greater than (\verb`>`).

Counters may also be treated as dates for comparison purposes. They
may be compared to date fields or date strings. When compared with
dates only the date portion of the counter is considered and the
sequence number is ignored.

% MAW 09-18-94 - rewrote in terms of parsetim(), add above para for counter
%To obtain dates in English in the manner one is used to seeing them, a
%{\em date function} is used to calculate a relative value derived from
%the absolute value stored.  This can be used on any table columns
%which store time or date information with the
%{\em date}
%data type, as defined in Chapter~\ref{chp:TabDef}.
%%{\em time} or

The comparison operators as given in Table~\ref{tab:CompOp} are used
to compare date values, so that dates may be used as qualifying
statements in the \verb`WHERE` clause.

{\bf Example:}
The Strategic Planning and Intelligence Department is responsible for
polling online news information on a daily basis, looking for
information relevant to Acme's ongoing business.  Articles of interest
are stored in an archived \verb`NEWS` table which retains the full text of
the article along with its subject, byline, source, and date.  The
date column is named NDATE, for ``News Date'', as ``date'' is a
special reserved SQL name and can't be used for column names.

A Date field may be compared to a number representing the number of
seconds since 1/1/70 0:0:0 GMT (e.g.:  778876248).  It may also be
compared to a string representing a human readable date in the format
\verb`'YYYY-MM-DD [HH:MM[:SS] [AM|PM]]'` (e.g.:  \verb`'1994-03-05 06:30
pm'` or \verb`'1994-07-04'`).  The date string may also be preceded by
``\verb`begin of`'' or ``\verb`end of`'' meaning the first or last second
of a day, respectively.

Enter this query:
\begin{verbatim}
     SELECT   NDATE, SUBJECT
     FROM     NEWS
     WHERE    NDATE BETWEEN 'begin of 1993-07-30'
                        AND 'end of 1993-07-30' ;
\end{verbatim}

Although the date column is stored with an absolute value, it is
converted to the correct relative value when displayed.  However, a
date assigned to a file is to the second, and to match that time, you
must match the same number of seconds.  Stating the date as
\verb`1993-07-30` refers to a particular second of that day.  An
article which came in at 2 p.m. would not match in seconds.  Thus you
state the range of seconds that span the 24 hour period called
``\verb`'1993-07-30'`'' by specifying a range between the first to last
moment of the day.

In this example, all the articles which were saved from July 30, 1993
are displayed with their subject lines.  The date as formatted by
Texis when displaying the date column is the format used inside the
single quotes.  It is put in quotes because it is a text string rather
than an absolute value.

Dates are usually used to limit the amount of text retrieved based on
some other search requirement, and would be so used along with other
qualifying statements in the \verb`WHERE` clause.  The next query is
identical to the last, but it adds another requirement.

\begin{verbatim}
     SELECT   NDATE, SUBJECT
     FROM     NEWS
     WHERE    NDATE BETWEEN 'begin of 1993-07-30'
                        AND 'end of 1993-07-30'
     AND      BODY LIKE 'bill gates' ;
\end{verbatim}

Now we can retrieve articles from July 30, 1993, but only a list of
those articles whose text body mentions Bill Gates.  A listing of Date
and Subject of the article will be displayed, as dictated in \verb`SELECT`.
Now we know which articles are available and can pick any we would
want to read in full.

This example uses a text query to find sentences in the body of the
information with reference to ``Bill Gates''.  Use of this type of
query in the \verb`LIKE` clause is explained in Chapter~\ref{Chp:MMLike}.
The following articles are retrieved:

%\begin{screen}
\begin{verbatim}
  NDATE                SUBJECT
  1993-30-07 04:46:04  High-Technology R&D Has Lost Its Cost-Effect...
  1993-30-07 13:10:08  Heavy R&D Spending No Longer the Magic Route...
\end{verbatim}
%\end{screen}

Date fields can use any of the comparison operators as shown in
Table~\ref{tab:CompOp} to manipulate information.  We could
broaden the date range of this search by increasing the {\tt BETWEEN}
range, or we could do it as follows:

\begin{verbatim}
     SELECT   NDATE, SUBJECT
     FROM     NEWS
     WHERE    BODY LIKE 'bill gates'
     AND      NDATE > 'begin of 1993-07-30'
     AND      NDATE < 'end of 1993-08-01' ;
\end{verbatim}

Remember that the actual value of the date is in a number of seconds.
Therefore, greater than (\verb`>`) translates to ``a greater number of
seconds than the stated value'', and therefore means ``newer than'', while
lesser than (\verb`<`) translates to ``a fewer number of seconds than the
stated value'', and therefore means ``older than''.

This would increase the output list to include dates in the specified
range; that is, between July 30th and August 1st 1993.

%\begin{screen}
\begin{verbatim}
  NDATE       SUBJECT
  1993-07-30 04:46:04  High-Technology R&D Has Lost Its Cost-Effect...
  1993-07-30 13:10:08  Heavy R&D Spending No Longer the Magic Route...
  1993-07-31 07:56:44  Microsoft-Novell battle out in the open
  1993-07-31 16:40:28  Microsoft to Undergo Justice Department Scrutiny
  1993-08-01 09:50:24  Justice Dept. Reportedly to Study Complaints ...
\end{verbatim}
%\end{screen}

Date strings have some additional operators, ``today'' and ``now''.  When used
following DATE they are converted to today's date and time in seconds
for both ``today'' and ``now''.  A time period of seconds, minutes, hours,
days, weeks, or months, can also be specified. A leading plus (\verb`+`)
or minus (\verb`-`) may also be specified to indicate past or
future.
Using our example from the \verb`NEWS` table, the form of the command would
be:

\begin{verbatim}
     SELECT   NDATE, SUBJECT
     FROM     NEWS
     WHERE    NDATE > '-7 days' ;
\end{verbatim}

This query requests all articles less than seven days old and would
produce a list of their subjects and date.

\begin{verbatim}
     SELECT   NDATE, SUBJECT
     FROM     NEWS
     WHERE    NDATE < '-1 minute'
       AND    NDATE > '-1 hour' ;
\end{verbatim}

This query would produce a list of articles which came in over the
last hour.
The date must be older than 1 minute ago, but
newer than 1 hour ago.

\section{Summarizing Values:
{\tt GROUP BY} Clause and
Aggregate Functions}

So far, the examples presented have shown how to retrieve and
manipulate values from individual rows in a table.  In this section,
we will illustrate how summary information can be obtained from groups
of rows in a table.

Often we find it useful to group data by some characteristic of the
group, such as department or division, or benefit level, so that
summary statistics about the group (totals, averages, etc.) can be
calculated.  For example, to calculate average departmental salaries,
the user could group the salaries of all employees by department.  In
Texis, the {\tt GROUP BY} clause is used to divide the rows of a table into
groups that have matching values in one or more columns.  The form of
this clause is:
\begin{verbatim}
     GROUP BY   column-name1 [,column-name2] ...
\end{verbatim}
and it fits into the \verb`SELECT` expression in the following manner.
\begin{verbatim}
     SELECT     column-name1 [,column-name2] ...
     FROM       table-name
     [WHERE     search-condition]
     [GROUP BY  column-name1 [,column-name2] ... ]
     [ORDER BY  column-name1 [DESC] [,column-name2] [DESC] ] ... ;
\end{verbatim}

The column(s) listed in the {\tt GROUP BY} clause are used to form groups.
The grouping is based on rows with the same value in the specified
column or columns being placed in the same group.  It is important to
note that grouping is conceptual; the table is not physically
rearranged.

As an extension Texis also allows the {\tt GROUP BY} clause to consist of
expressions instead of just column names.  This should be used with
caution, and the same expression should be used in the \verb`SELECT` as
in the {\tt GROUP BY} clause.  This is especially true if the expression
will fold multiple values together, such as dividing a number by
1000 to group quantities together if they are in the same 1000.  If
you select SALARY, and {\tt GROUP BY SALARY/1000} you will see one sample
salary from the matching group.

The {\tt GROUP BY} clause is normally used along with five built-in, or
``aggregate'' functions.  These functions perform special operations
on an entire table or on a set, or group, of rows rather than on each
row and then return one row of values for each group.

Table~\ref{tab:AggFunc} lists the aggregate functions available with
Texis.

\begin{table}
\caption{Texis Aggregate Function Names}{\label{tab:AggFunc}}
\begin{center}
\begin{tabular}{|l|l|l|} \hline
Function Name     & Meaning                                 & Example            \\ \hline\hline
SUM(column name)  & Total of the values in a numeric column & \verb`SUM(SALARY)` \\
AVG(column name)  & Average of the values in a column       & \verb`AVG(SALARY)` \\
MAX(column name)  & Largest value in a column               & \verb`MAX(SALARY)` \\
MIN(column name)  & Smallest value in a column              & \verb`MIN(SALARY)` \\
COUNT(*)          & Count of the number of rows selected    & \verb`COUNT(*)`    \\ \hline
\end{tabular}
\end{center}
\end{table}

Aggregate functions are used in place of column names in the \verb`SELECT`
statement.  The form of the function is:
\begin{verbatim}
     Function name ([DISTINCT] argument)
\end{verbatim}

In all situations the argument represents the column name to which the
function applies.  For example, if the sum of all salaries is needed,
then the function SUM is used and the argument is the column SALARY.
When COUNT is used an asterisk (*) can be placed within the parentheses
instead of a column name to count all the rows without regard to
field.

If the DISTINCT keyword is used then only the unique values are processed.
This is most useful with COUNT to find the number of unique values.  If
you use DISTINCT then you must supply a column name.  DISTINCT will work
with the other aggregate functions, although there is typically very little
need for them.  The DISTINCT feature was added in version 4.00.1002000000

{\bf Example:}
What is the average salary paid in each department?

Enter this statement:
\begin{verbatim}
     SELECT     DEPT, AVG(SALARY)
     FROM       EMPLOYEE
     GROUP BY   DEPT ;
\end{verbatim}
{\bf Syntax Notes:}
\begin{itemize}
\item \verb`AVG` is the aggregate function name.
\item \verb`(SALARY)` is the column on which the average is computed.
\item\verb`DEPT` is the column by which the rows will be grouped.
\end{itemize}

The above statement will produce the following results:

%\begin{screen}
\begin{verbatim}
  DEPT      AVG(SALARY)

  MKT       33500
  MGT       45000
  LIB       22000
  RND       27500
  FIN       42000
\end{verbatim}
%\end{screen}

In this query, all rows in the \verb`EMPLOYEE` table that have the same
department codes are grouped together.  The aggregate function AVG is
calculated for the salary column in each group.  The department code
and the average departmental salary are displayed for each department.

A \verb`SELECT` clause that contains an aggregate function cannot contain any
column name that does not apply to a group; for example:

The statement:
\begin{verbatim}
     SELECT     ENAME, AVG(SALARY)
     FROM       EMPLOYEE
     GROUP BY   DEPT ;
\end{verbatim}
results in the message
\begin{verbatim}
     Error at Line 1: Not a GROUP BY Expression
\end{verbatim}

It is not permissible to include column names in a \verb`SELECT` clause that
are not referenced in the {\tt GROUP BY} clause.  The only column names that
can be displayed, along with aggregate functions, must be listed in
the {\tt GROUP BY} clause.  Since \verb`ENAME` is not included in the {\tt GROUP BY}
clause, an error message results.

{\bf Example:}
The chair of the Marketing Department plans to participate in a
national salary survey for employees in Marketing Departments.
Determine the average salary paid to the Marketing Department
employees.

This statement:
\begin{verbatim}
     SELECT     COUNT(*), AVG(SALARY)
     FROM       EMPLOYEE
     WHERE      DEPT = 'MKT'
\end{verbatim}
Results in:

%\begin{screen}
\begin{verbatim}
  COUNT(*)   AVG(SALARY)

  2          33500
\end{verbatim}
%\end{screen}

In this example, the aggregate function AVG is used in a \verb`SELECT`
statement that has a \verb`WHERE` clause.  Texis selects the rows that
represent Marketing Department employees and then applies the
aggregate function to these rows.

You can divide the rows of a table into groups based on values in more
than one column.  For example, you might want to compute total salary
by department and then, within a department, want subtotals by
benefits classification.

{\bf Example:}
What is the total salary paid by benefits classification in each
department?

Enter this statement:
\begin{verbatim}
     SELECT     DEPT, BENEFITS, SUM(SALARY)
     FROM       EMPLOYEE
     GROUP BY   DEPT, BENEFITS ;
\end{verbatim}
In this example, we are grouping by department, and within department,
by benefits classification.

We'll get the following results:

%\begin{screen}
\begin{verbatim}
  DEPT      BENEFITS    SUM(SALARY)

  FIN       FULL        42000
  LIB       PART        22000
  MGT       FULL        45000
  MKT       FULL        67000
  RND       FULL        30000
  RND       PART        25000
\end{verbatim}
%\end{screen}

In this query, the rows are grouped by department and, within each
department, employees with the same benefits are grouped so that
totals can be computed.  Notice that the columns DEPT and BENEFITS can
appear in the \verb`SELECT` statement since both columns appear in the GROUP
BY clause.

If the {\tt GROUP BY} clause is omitted when an aggregate function is used,
then the entire table is considered as one group, and the group
function displays a single value for the entire table.

{\bf Example:}
What is the total salary paid to all employees?

The statement:
\begin{verbatim}
     SELECT     SUM(SALARY)
     FROM       EMPLOYEE ;
\end{verbatim}
results in:

%\begin{screen}
\begin{verbatim}
  SUM(SALARY)

  231000
\end{verbatim}
%\end{screen}

\section{Groups With Conditions: {\tt HAVING} Clause}

Sometimes you may want to specify a condition that applies to groups
rather than to individual rows.  For example, you might want a list of
departments where the average departmental salary is above \$30,000.
To express such a query, the {\tt HAVING} clause is used.  This clause
specifies which groups should be selected and is used in combination
with the {\tt GROUP BY} clause.  The form of this clause is as follows:
\begin{verbatim}
     [GROUP BY  column-name1 [,column-name2] ...
     [HAVING    search-condition ]
\end{verbatim}

Conditions in the {\tt HAVING} clause are applied after groups are formed.
The search condition of the {\tt HAVING} clause examines the grouped rows
and produces a row for each group where the search condition in the
{\tt HAVING} clause is true.  The clause is similar to the \verb`WHERE` clause,
except the {\tt HAVING} clause applies to groups.

{\bf Example:}
Which departments have an average salary above \$30,000? Order the
results by average salary, with highest average salary appearing
first.

The statement:
\begin{verbatim}
     SELECT     DEPT, AVG(SALARY) AS AVG_SALARY
     FROM       EMPLOYEE
     GROUP BY   DEPT
     HAVING     AVG_SALARY > 30000
     ORDER BY   AVG_SALARY DESC ;
\end{verbatim}

{\bf Syntax Notes:}
\begin{itemize}
\item When {\tt HAVING} is used, it always follows a {\tt GROUP BY} clause.
\item When referring to aggregate values in the {\tt HAVING} and {\tt ORDER BY} clauses
of a {\tt GROUP BY} you must assign an alternative name to the field, and use
that in the {\tt HAVING} and {\tt ORDER BY} clauses.
%\item The `\verb`2`' in the {\tt ORDER BY} clause refers to the second item
%listed in the \verb`SELECT` clause.  Therefore the basis of the sort is the
%computed value ``\verb`AVG(SALARY)`''.
\end{itemize}

The results are:

%\begin{screen}
\begin{verbatim}
  DEPT      AVG_SALARY

  MGT       45000
  FIN       42000
  MKT       33500
\end{verbatim}
%\end{screen}

In this query, the average salary for all departments is computed, but
only the names of those departments having an average salary above
\$30,000 are displayed.  Notice that Research and Development's
average of \$27,500 is not displayed, nor is the Library's average of
\$22,000.

The {\tt GROUP BY} clause does not sort the results, thus the need for the
{\tt ORDER BY} clause.  Finally, note that the {\tt ORDER BY} clause must be
placed after the {\tt GROUP BY} and {\tt HAVING} clauses.

This chapter has covered the computational capabilities of Texis.  In
the next chapter, you will learn how to develop more complex queries by
using the join operation and the nesting of queries.

% >>>>> DO NOT EDIT: this file generated from function.src <<<<<
% -*- mode: LaTeX -*-
% $Id$
%
%  Use:        For:               PDF render:   Online render:
%  ----        ----               ----------    --------------
%  \verb`...`  Keywords           Fixed-font    <code> fixed-font red-on-pink
%  {\tt ... }  User input         Fixed-font    <tt> fixed-font
%  {\bf ...}   Settings/sections  Bold          <b> bold

\section{Server functions}
The Texis server has a number of functions built into it which can operate
on fields.  This can occur anywhere an expression can occur in a SQL
statement.  It is possible that the server at your site has been extended
with additional functions.  Each of the arguments can be either a single
field name, or another expression.

% ----------------------------------------------------------------------------
\subsection{File functions}

% - - - - - - - - - - - - - - - - - - - - - - - - - - - - - - - - - - - - - -
\subsubsection{fromfile, fromfiletext}

The \verb`fromfile` and \verb`fromfiletext` functions read a file.
The syntax is
\begin{verbatim}
   fromfile(filename[, offset[, length]])
   fromfiletext(filename[, offset[, length]])
\end{verbatim}

These functions take one required, and two optional arguments.  The
first argument is the filename.  The second argument is an offset into
the file, and the third argument is the length of data to read.  If the
second argument is omitted then the file will be read from the
beginning.  If the third argument is omitted then the file will be read
to the end.  The result is the contents of the file.  This can be used
to load data into a table.  For example if you have an indirect field
and you wish to see the contents of the file you can issue SQL similar
to the following.

The difference between the two functions is the type of data that is
returned.  \verb`fromfile` will return varbyte data, and
\verb`fromfiletext` will return varchar data.  If you are using the
functions to insert data into a field you should make sure that you
use the appropriate function for the type of field you are inserting
into.

\begin{verbatim}
     SELECT  FILENAME, fromfiletext(FILENAME)
     FROM    DOCUMENTS
     WHERE   DOCID = 'JT09113' ;
\end{verbatim}

The results are:

\begin{screen}
\begin{verbatim}
  FILENAME            fromfiletext(FILENAME)
  /docs/JT09113.txt   This is the text contained in the document
  that has an id of JT09113.
\end{verbatim}
\end{screen}

% - - - - - - - - - - - - - - - - - - - - - - - - - - - - - - - - - - - - - -
\subsubsection{totext}

Converts data or file to text.  The syntax is
\begin{verbatim}
   totext(filename[, args])
   totext(data[, args])
\end{verbatim}

This function will convert the contents of a file, if the argument given is
an indirect, or else the result of the expression, and convert it to text.
It does this by calling the program \verb'anytotx', which must be in the
path.  The \verb`anytotx` program (obtained from Thunderstone) will handle
\verb`PDF` as well as many other file formats.

As of version 2.06.935767000 the \verb`totext` command will take an optional
second argument which contains arguments to the \verb`anytotx` program.
See the documentation for \verb`anytotx` for details on its arguments.

\begin{verbatim}
     SELECT  FILENAME, totext(FILENAME)
     FROM    DOCUMENTS
     WHERE   DOCID = 'JT09113' ;
\end{verbatim}

The results are:

\begin{screen}
\begin{verbatim}
  FILENAME            totext(FILENAME)
  /docs/JT09113.pdf   This is the text contained in the document
  that has an id of JT09113.
\end{verbatim}
\end{screen}

% - - - - - - - - - - - - - - - - - - - - - - - - - - - - - - - - - - - - - -
\subsubsection{toind}

Create a Texis managed indirect file.  The syntax is
\begin{verbatim}
   toind(data)
\end{verbatim}

This function takes the argument, stores it into a file, and returns the
filename as an \verb`indirect` type.  This is most often used in combination
with \verb`fromfile` to create a Texis managed file.  For example:

\begin{verbatim}
     INSERT  INTO DOCUMENTS
     VALUES('JT09114', toind(fromfile('srcfile')))
\end{verbatim}

The database will now contain a pointer to a copy of \verb|srcfile|, which
will remain searchable even if the original is changed or removed.  An
important point to note is that any changes to \verb|srcfile| will not be
reflected in the database, unless the table row's \verb`indirect` column
is modified (even to the save value, this just tells Texis to re-index it).

% - - - - - - - - - - - - - - - - - - - - - - - - - - - - - - - - - - - - - -
\subsubsection{canonpath}

  Returns canonical version of a file path, i.e. fully-qualified and
without symbolic links:

\begin{verbatim}
  canonpath(path[, flags])
\end{verbatim}

The optional \verb`flags` is a set of bit flags: bit 0 set if error
messages should be issued, bit 1 set if the return value should be
empty instead of \verb`path` on error.  Added in version
5.01.1139446515 20060208.

% - - - - - - - - - - - - - - - - - - - - - - - - - - - - - - - - - - - - - -
\subsubsection{pathcmp}

  File path comparison function; like C function \verb`strcmp()` but
for paths:

\begin{verbatim}
  pathcmp(pathA, pathB)
\end{verbatim}

Returns an integer indicating the sort order of \verb`pathA` relative
to \verb`pathB`: 0 if \verb`pathA` is the same as \verb`pathB`, less
than 0 if \verb`pathA` is less than \verb`pathB`, greater than 0 if
\verb`pathA` is greater than \verb`pathB`.  Paths are compared
case-insensitively if and only if the OS is case-insensitive for
paths, and OS-specific alternate directory separators are considered
the same (e.g. ``\verb`\`'' and ``\verb`/`'' in Windows).  Multiple
consecutive directory separators are considered the same as one.  A
trailing directory separator (if not also a leading separator) is
ignored.  Directory separators sort lexically before any other
character.

  Note that the paths are only compared lexically: no attempt is made
to resolve symbolic links, ``{\tt ..}'' path components, etc.  Note
also that no inference should be made about the magnitude of negative
or positive return values: greater magnitude does not necessarily
indicate greater lexical ``separation'', nor should it be assumed that
comparing the same two paths will always yield the same-magnitude
value in future versions.  Only the sign of the return value is
significant.  Added in version 5.01.1139446515 20060208.

% - - - - - - - - - - - - - - - - - - - - - - - - - - - - - - - - - - - - - -
\subsubsection{basename}

  Returns the base filename of a given file path.

\begin{verbatim}
  basename(path)
\end{verbatim}

The basename is the contents of \verb`path` after the last path separator.
No filesystem checks are performed, as this is a text/parsing function;
thus ``\verb`.`'' and ``\verb`..`'' are not significant.
Added in version 7.00.1352510000 20121109.

% - - - - - - - - - - - - - - - - - - - - - - - - - - - - - - - - - - - - - -
\subsubsection{dirname}

  Returns the directory part of a given file path.

\begin{verbatim}
  dirname(path)
\end{verbatim}

The directory is the contents of \verb`path` before the last path separator
(unless it is significant -- e.g. for the root directory -- in which case
it is retained).  Added in version 7.00.1352510000 20121109.
No filesystem checks are performed, as this is a text/parsing function;
thus ``\verb`.`'' and ``\verb`..`'' are not significant.

% - - - - - - - - - - - - - - - - - - - - - - - - - - - - - - - - - - - - - -
\subsubsection{fileext}

  Returns the file extension of a given file path.

\begin{verbatim}
  fileext(path)
\end{verbatim}

  The file extension starts with and includes a dot.  The file extension
is only considered present in the basename of the path, i.e. after the
last path separator.  Added in version 7.00.1352510000 20121109.

% - - - - - - - - - - - - - - - - - - - - - - - - - - - - - - - - - - - - - -
\subsubsection{joinpath}

Joins one or more file/directory path arguments into a merged path,
inserting/removing a path separator between arguments as needed.
Takes one to 5 path component arguments.  E.g.:

\begin{verbatim}
  joinpath('one', 'two/', '/three/four', 'five')
\end{verbatim}

yields

\begin{verbatim}
  one/two/three/four/five
\end{verbatim}

Added in version 7.00.1352770000 20121112.  Redundant path separators
internal to an argument are not removed, nor are ``{\tt .}'' and ``{\tt
  ..}'' path components removed.  Prior to version 7.07.1550082000
20190213 redundant path separators between arguments were not
removed.

% - - - - - - - - - - - - - - - - - - - - - - - - - - - - - - - - - - - - - -
\subsubsection{joinpathabsolute}

Like \verb`joinpath`, except that a second or later argument that is
an absolute path will overwrite the previously-merged path.  E.g.:

\begin{verbatim}
  joinpathabsolute('one', 'two', '/three/four', 'five')
\end{verbatim}

yields

\begin{verbatim}
  /three/four/five
\end{verbatim}

Under Windows, partially absolute path arguments -- e.g. ``{\tt
  /dir}'' or ``{\tt C:dir}'' where the drive or dir is still relative
-- are considered absolute for the sake of overwriting the merge.

Added in version 7.00.1352770000 20121112.  Redundant path separators
internal to an argument are not removed, nor are ``{\tt .}'' and
``{\tt ..}'' path components removed.  Prior to version
7.07.1550082000 20190213 partially absolute arguments were not
considered absolute.

% ----------------------------------------------------------------------------

\subsection{String Functions}

% - - - - - - - - - - - - - - - - - - - - - - - - - - - - - - - - - - - - - -
\subsubsection{abstract}

Generate an abstract of a given portion of text.  The syntax is
\begin{verbatim}
   abstract(text[, maxsize[, style[, query]]])
\end{verbatim}

  The abstract will be less than \verb`maxsize` characters long, and
will attempt to end at a word boundary.  If \verb`maxsize` is not
specified (or is less than or equal to 0) then a default size of 230
characters is used.

  The \verb`style` argument is a string or integer, and allows a
choice between several different ways of creating the abstract.
Note that some of these styles require the \verb`query` argument as
well, which is a Metamorph query to look for:

\begin{itemize}
  \item \verb`dumb` (0) \\
    Start the abstract at the top of the document.

  \item \verb`smart` (1) \\
    This style will look for the first meaningful chunk of text,
    skipping over any headers at the top of the text.  This is the
    default if neither \verb`style` nor \verb`query` is given.

  \item \verb`querysingle` (2) \\
    Center the abstract contiguously on the best occurence of
    \verb`query` in the document.

  \item \verb`querymultiple` (3) \\
    Like \verb`querysingle`, but also break up the abstract into
    multiple sections (separated with ``\verb`...`'') if needed to
    help ensure all terms are visible.  Also take care with URLs to
    try to show the start and end.

  \item \verb`querybest` \\
    An alias for the best available query-based style; currently the
    same as \verb`querymultiple`.  Using \verb`querybest` in a script
    ensures that if improved styles become available in future
    releases, the script will automatically ``upgrade'' to the best
    style.
\end{itemize}

  If no \verb`query` is given for the \verb`query`$...$ modes, they
fall back to \verb`dumb` mode.  If a \verb`query` is given with a {\em
non-}\verb`query`$...$ mode (\verb`dumb`/\verb`smart`), the mode is
promoted to \verb`querybest`.  The current locale and index
expressions also have an effect on the abstract in the
\verb`query`$...$ modes, so that it more closely reflects an
index-obtained hit.

\begin{verbatim}
     SELECT     abstract(STORY, 0, 1, 'power struggle')
     FROM       ARTICLES
     WHERE      ARTID = 'JT09115' ;
\end{verbatim}

% - - - - - - - - - - - - - - - - - - - - - - - - - - - - - - - - - - - - - -
\subsubsection{text2mm}

Generate \verb`LIKEP` query.  The syntax is
\begin{verbatim}
   text2mm(text[, maxwords])
\end{verbatim}

This function will take a text expression, and produce a list of words
that can be given to \verb`LIKER` or \verb`LIKEP` to find similar
documents.  \verb`text2mm` takes an optional second argument which
specifies how many words should be returned.  If this is not specified
then 10 words are returned.  Most commonly \verb`text2mm` will be given the
name of a field.  If it is an \verb`indirect` field you will need to call
\verb|fromfile| as shown below:

\begin{verbatim}
     SELECT     text2mm(fromfile(FILENAME))
     FROM       DOCUMENTS
     WHERE      DOCID = 'JT09115' ;
\end{verbatim}

You may also call it as \verb`texttomm()` instead of \verb`text2mm()` .

% - - - - - - - - - - - - - - - - - - - - - - - - - - - - - - - - - - - - - -
\subsubsection{keywords}

Generate list of keywords.  The syntax is
\begin{verbatim}
   keywords(text[, maxwords])
\end{verbatim}

{\tt keywords} is similar to {\tt text2mm} but produces a list of
phrases, with a linefeed separating them.  The difference between
{\tt text2mm} and {\tt keywords} is that {\tt keywords} will maintain
the phrases.
{\tt keywords} also takes an optional second
argument which indicates how many words or phrases should be returned.

% - - - - - - - - - - - - - - - - - - - - - - - - - - - - - - - - - - - - - -
\subsubsection{length}

Returns the length in characters of a \verb`char` or \verb`varchar`
expression, or number of strings/items in other types.  The syntax is
\begin{verbatim}
  length(value[, mode])
\end{verbatim}

For example:

\begin{verbatim}
     SELECT  NAME, length(NAME)
     FROM    SYSTABLES
\end{verbatim}

The results are:

\begin{screen}
\begin{verbatim}
  NAME                length(NAME)
 SYSTABLES               9
 SYSCOLUMNS             10
 SYSINDEX                8
 SYSUSERS                8
 SYSPERMS                8
 SYSTRIG                 7
 SYSMETAINDEX           12
\end{verbatim}
\end{screen}

  The optional \verb`mode` argument is a
\verb`stringcomparemode`-style compare mode to use; see the Vortex manual
on {\tt <apicp stringcomparemode>} for details on syntax and the
default.  If \verb`mode` is not given, the current {\tt apicp
stringcomparemode} is used.  Currently the only pertinent \verb`mode`
flag is ``{\tt iso-8859-1}'', which determines whether to interpret
\verb`value` as ISO-8859-1 or UTF-8.  This can alter how many characters long
the string appears to be, as UTF-8 characters are variable-byte-sized,
whereas ISO-8859-1 characters are always mono-byte.  The \verb`mode`
argument was added in version 6.

  In version 5.01.1226622000 20081113 and later, if given a \verb`strlst`
type \verb`value`, \verb`length()` returns the number of string values
in the list.  For other types, it returns the number of values, e.g.
for \verb`varint` it returns the number of integer values.

% - - - - - - - - - - - - - - - - - - - - - - - - - - - - - - - - - - - - - -
\subsubsection{lower}

Returns the text expression with all letters in lower-case. The syntax is
\begin{verbatim}
  lower(text[, mode])
\end{verbatim}

For example:

\begin{verbatim}
     SELECT  NAME, lower(NAME)
     FROM    SYSTABLES
\end{verbatim}

The results are:

\begin{screen}
\begin{verbatim}
  NAME                lower(NAME)
 SYSTABLES            systables
 SYSCOLUMNS           syscolumns
 SYSINDEX             sysindex
 SYSUSERS             sysusers
 SYSPERMS             sysperms
 SYSTRIG              systrig
 SYSMETAINDEX         sysmetaindex
\end{verbatim}
\end{screen}

Added in version 2.6.932060000.

  The optional \verb`mode` argument is a string-folding mode in the
same format as {\tt <apicp stringcomparemode>}; see the Vortex manual
for details on the syntax and default.  If \verb`mode` is unspecified,
the current {\tt apicp stringcomparemode} setting -- with ``{\tt +lowercase}''
aded -- is used.  The \verb`mode` argument was added in version 6.

% - - - - - - - - - - - - - - - - - - - - - - - - - - - - - - - - - - - - - -
\subsubsection{upper}

Returns the text expression with all letters in upper-case. The sytax is
\begin{verbatim}
  upper(text[, mode])
\end{verbatim}

For example:

\begin{verbatim}
     SELECT  NAME, upper(NAME)
     FROM    SYSTABLES
\end{verbatim}

The results are:

\begin{screen}
\begin{verbatim}
  NAME                upper(NAME)
 SYSTABLES            SYSTABLES
 SYSCOLUMNS           SYSCOLUMNS
 SYSINDEX             SYSINDEX
 SYSUSERS             SYSUSERS
 SYSPERMS             SYSPERMS
 SYSTRIG              SYSTRIG
 SYSMETAINDEX         SYSMETAINDEX
\end{verbatim}
\end{screen}

Added in version 2.6.932060000.

  The optional \verb`mode` argument is a string-folding mode in the
same format as {\tt <apicp stringcomparemode>}; see the Vortex manual
for details on the syntax and default.  If \verb`mode` is unspecified,
the current {\tt apicp stringcomparemode} setting -- with ``{\tt
+uppercase}'' added -- is used.  The \verb`mode` argument was added in
version 6.

% - - - - - - - - - - - - - - - - - - - - - - - - - - - - - - - - - - - - - -
\subsubsection{initcap}

Capitalizes text.  The syntax is
\begin{verbatim}
  initcap(text[, mode])
\end{verbatim}

Returns the text expression with the first letter of each word in
title case (i.e. upper case), and all other letters in lower-case.
For example:

\begin{verbatim}
     SELECT  NAME, initcap(NAME)
     FROM    SYSTABLES
\end{verbatim}

The results are:

\begin{screen}
\begin{verbatim}
  NAME                initcap(NAME)
 SYSTABLES            Systables
 SYSCOLUMNS           Syscolumns
 SYSINDEX             Sysindex
 SYSUSERS             Sysusers
 SYSPERMS             Sysperms
 SYSTRIG              Systrig
 SYSMETAINDEX         Sysmetaindex
\end{verbatim}
\end{screen}

Added in version 2.6.932060000.

  The optional \verb`mode` argument is a string-folding mode in the
same format as {\tt <apicp stringcomparemode>}; see the Vortex manual
for details on the syntax and default.  If \verb`mode` is unspecified,
the current {\tt apicp stringcomparemode} setting -- with ``{\tt +titlecase}''
added -- is used.  The \verb`mode` argument was added in version 6.

% - - - - - - - - - - - - - - - - - - - - - - - - - - - - - - - - - - - - - -
\subsubsection{sandr}

Search and replace text.
\begin{verbatim}
   sandr(search, replace, text)
\end{verbatim}

Returns the text expression with the search REX expression replaced
with the replace expression.  See the REX documentation and the
Vortex sandr function documentation for complete syntax of the search
and replace expressions.

\begin{verbatim}
     SELECT  NAME, sandr('>>=SYS=', 'SYSTEM TABLE ', NAME) DESC
     FROM    SYSTABLES
\end{verbatim}

The results are:

\begin{screen}
\begin{verbatim}
  NAME                DESC
 SYSTABLES            SYSTEM TABLE TABLES
 SYSCOLUMNS           SYSTEM TABLE COLUMNS
 SYSINDEX             SYSTEM TABLE INDEX
 SYSUSERS             SYSTEM TABLE USERS
 SYSPERMS             SYSTEM TABLE PERMS
 SYSTRIG              SYSTEM TABLE TRIG
 SYSMETAINDEX         SYSTEM TABLE METAINDEX
\end{verbatim}
\end{screen}

Added in version 3.0

% - - - - - - - - - - - - - - - - - - - - - - - - - - - - - - - - - - - - - -
\subsubsection{separator}

Returns the separator character from its \verb`strlst` argument,
as a \verb`varchar` string:

\begin{verbatim}
   separator(strlstValue)
\end{verbatim}

  This can be used in situations where the \verb`strlstValue` argument
may have a nul character as the separator, in which case simply
converting \verb`strlstValue` to \verb`varchar` and looking at the
last character would be incorrect.  Added in version 5.01.1226030000
20081106.

% - - - - - - - - - - - - - - - - - - - - - - - - - - - - - - - - - - - - - -
\subsubsection{stringcompare}

  Compares its string (\verb`varchar`) arguments \verb`a` and
\verb`b`, returning -1 if \verb`a` is less than \verb`b`, 0 if they
are equal, or 1 if \verb`a` is greater than \verb`b`:

\begin{verbatim}
  stringcompare(a, b[, mode])
\end{verbatim}

  The strings are compared using the optional \verb`mode` argument,
which is a string-folding mode in the same format as
{\tt <apicp stringcomparemode>}; see the Vortex manual for details on
the syntax and default.  If \verb`mode` is unspecified, the current
{\tt apicp stringcomparemode} setting is used.  Function added
in version 6.00.1304108000 20110429.

% - - - - - - - - - - - - - - - - - - - - - - - - - - - - - - - - - - - - - -
\subsubsection{stringformat}

  Returns its arguments formatted into a string (\verb`varchar`), like
the equivalent Vortex function \verb`<strfmt>` (based on the C
function \verb`sprintf()`):

\begin{verbatim}
  stringformat(format[, arg[, arg[, arg[, arg]]]])
\end{verbatim}

  The \verb`format` argument is a \verb`varchar` string that describes
how to print the following argument(s), if any.  See the Vortex manual
for \verb`<strfmt>` for details.  Added in version 6.00.1300386000
20110317.

% ----------------------------------------------------------------------------

\subsection{Math functions}

The following basic math functions are available in Texis:
\verb`acos`, \verb`asin`, \verb`atan`, \verb`atan2`, \verb`ceil`, \verb`cos`,
\verb`cosh`, \verb`exp`, \verb`fabs`, \verb`floor`, \verb`fmod`,
\verb`log`, \verb`log10`, \verb`pow`, \verb`sin`, \verb`sinh`, \verb`sqrt`,
\verb`tan`, \verb`tanh`.

All of the above functions call the ANSI C math library function
of the same name, and return a result of type \verb`double`.
\verb`pow`, \verb`atan2` and \verb`fmod` take two double arguments,
the remainder take one double argument.  Added in version:
2.6.931790000

In addition, the following math-related functions are available:

\begin{itemize}
  \item \verb`isNaN(x)` \\
    Returns 1 if \verb`x` is a float or double NaN (Not a Number) value,
    0 if not.  This function should be used to test for NaN, rather than
    using the equality operator (e.g. \verb`x = 'NaN'`), because the
    IEEE standard defines \verb`NaN == NaN` to be false, not true as
    might be expected.  Added in version 5.01.1193955406 20071101.
\end{itemize}

% ----------------------------------------------------------------------------

\subsection{Date functions}

The following date functions are available in Texis:
\verb`dayname`, \verb`month`, \verb`monthname`, \verb`dayofmonth`,
\verb`dayofweek`, \verb`dayofyear`, \verb`quarter`, \verb`week`,
\verb`year`, \verb`hour`, \verb`minute`, \verb`second`.

All the functions take a date as an argument.  \verb`dayname` and \verb`monthname` will return a string with the full day or month name based on the current
locale, and the others return a number.

The \verb`dayofweek` function returns 1 for Sunday.  The quarter is
based on months, so April 1st is the first day of quarter 2.  Week 1
begins with the first Sunday of the year.

Added in version: 3.0.948300000

The following functions were added in version 3.01.990400000:
\verb`monthseq`, \verb`weekseq` and \verb`dayseq` which will return the
number of months, weeks and days since an arbitrary past date.  These
can be used when comparing dates to see how many months, weeks or days
separate them.

% ----------------------------------------------------------------------------

\subsection{Bit manipulation functions}

  These functions are used to manipulate integers as bit fields.  This
can be useful for efficient set operations (e.g. set membership,
intersection, etc.).  For example, categories could be mapped to
sequential bit numbers, and a row's category membership stored
compactly as bits of an \verb`int` or \verb`varint`, instead of using a
string list.  Category membership can then be quickly determined with
\verb`bitand` on the integer.

  In the following functions, bit field arguments \verb`a` and
\verb`b` are \verb`int` or \verb`varint` (32 bits per integer, all platforms).
Argument \verb`n` is any integer type.  Bits are numbered starting
with 0 as the least-significant bit of the first integer.  31 is the
most-significant bit of the first integer, 32 is the least-significant
bit of the second integer (if a multi-value \verb`varint`), etc.
These functions were added in version 5.01.1099455599 of Nov 2 2004.

\begin{itemize}
  \item \verb`bitand(a, b)` \\
    Returns the bit-wise AND of \verb`a` and \verb`b`.  If one
    argument is shorter than the other, it will be expanded with
    0-value integers.

  \item \verb`bitor(a, b)` \\
    Returns the bit-wise OR of \verb`a` and \verb`b`.  If one argument
    is shorter than the other, it will be expanded with 0-value integers.

  \item \verb`bitxor(a, b)` \\
    Returns the bit-wise XOR (exclusive OR) of \verb`a` and \verb`b`.
    If one argument is shorter than the other, it will be expanded with
    0-value integers.

  \item \verb`bitnot(a)` \\
    Returns the bit-wise NOT of \verb`a`.

  \item \verb`bitsize(a)` \\
    Returns the total number of bits in \verb`a`, i.e. the highest
    bit number plus 1.

  \item \verb`bitcount(a)` \\
    Returns the number of bits in \verb`a` that are set to 1.

  \item \verb`bitmin(a)` \\
    Returns the lowest bit number in \verb`a` that is set to 1.
    If none are set to 1, returns -1.

  \item \verb`bitmax(a)` \\
    Returns the highest bit number in \verb`a` that is set to 1.
    If none are set to 1, returns -1.

  \item \verb`bitlist(a)` \\
    Returns the list of bit numbers of \verb`a`, in ascending order,
    that are set to 1, as a \verb`varint`.  Returns a single -1 if
    no bits are set to 1.

  \item \verb`bitshiftleft(a, n)` \\
    Returns \verb`a` shifted \verb`n` bits to the left, with 0s padded
    for bits on the right.  If \verb`n` is negative, shifts right instead.

  \item \verb`bitshiftright(a, n)` \\
    Returns \verb`a` shifted \verb`n` bits to the right, with 0s padded
    for bits on the left (i.e. an unsigned shift).  If \verb`n` is
    negative, shifts left instead.

  \item \verb`bitrotateleft(a, n)` \\
    Returns \verb`a` rotated \verb`n` bits to the left, with left
    (most-significant) bits wrapping around to the right.  If \verb`n`
    is negative, rotates right instead.

  \item \verb`bitrotateright(a, n)` \\
    Returns \verb`a` rotated \verb`n` bits to the right, with right
    (least-significant) bits wrapping around to the left.  If \verb`n`
    is negative, rotates left instead.

  \item \verb`bitset(a, n)` \\
    Returns \verb`a` with bit number \verb`n` set to 1.  \verb`a` will
    be padded with 0-value integers if needed to reach \verb`n` (e.g.
    \verb`bitset(5, 40)` will return a \verb`varint(2)`).

  \item \verb`bitclear(a, n)` \\
    Returns \verb`a` with bit number \verb`n` set to 0.  \verb`a` will
    be padded with 0-value integers if needed to reach \verb`n` (e.g.
    \verb`bitclear(5, 40)` will return a \verb`varint(2)`).

  \item \verb`bitisset(a, n)` \\
    Returns 1 if bit number \verb`n` is set to 1 in \verb`a`, 0 if not.
\end{itemize}

% ----------------------------------------------------------------------------

\subsection{Internet/IP address functions}

The following functions manipulate IP network and/or host addresses;
most take \verb`inet` style argument(s).
This is an IPv4
address string, optionally followed by a netmask.

For IPv4, the format is dotted-decimal, i.e.
$N$[{\tt .}$N$[{\tt .}$[N${\tt .}$N$]]] where $N$ is a decimal, octal
or hexadecimal integer from 0 to 255.  If $x < 4$ values of $N$ are
given, the last $N$ is taken as the last $5-x$ bytes instead of 1
byte, with missing bytes padded to the right.  E.g. {\tt 192.258} is
valid and equivalent to {\tt 192.1.2.0}: the last $N$ is 2 bytes in
size, and covers 5 - 2 = 3 needed bytes, including 1 zero pad to the
right.  Conversely, {\tt 192.168.4.1027} is not valid: the last $N$
is too large.

An IPv4 address may optionally be followed by a netmask, either of
the form {\tt /}$B$ or {\tt :}$IPv4$, where $B$ is a decimal, octal or
hexadecimal netmask integer from 0 to 32, and $IPv4$ is a
dotted-decimal IPv4 address of the same format described above.  If an
{\tt :}$IPv4$ netmask is given, only the largest contiguous set of
most-significant 1 bits are used (because netmasks are contiguous).
If no netmask is given, it will be calculated from standard IPv4 class
A/B/C/D/E rules, but will be large enough to include all given bytes
of the IP.  E.g. {\tt 1.2.3.4} is Class A which has a netmask of 8,
but the netmask will be extended to 32 to include all 4 given bytes.


In version 7.07.1554395000 20190404 and later, error messages are
reported.

  The \verb`inet` functions were added in version 5.01.1113268256 of
Apr 11 2005 and include the following.  See also the Vortex
\verb`<urlutil>` equivalents:

\begin{itemize}
  \item \verb`inetabbrev(inet)` \\

    Returns a possibly shorter-than-canonical representation of
    \verb`$inet`, where trailing zero byte(s) of an IPv4 address may
    be omitted.  All bytes of the network, and leading non-zero bytes
    of the host, will be included.  E.g.  {\tt <urlutil inetabbrev
      "192.100.0.0/24">} returns {\tt 192.100.0/24}.  The {\tt /}$B$
    netmask is included, except if (in version 7.07.1554840000
    20190409 and later) the network is host-only (i.e. netmask is the
    full size of the IP address).  Empty string is returned on error.

  \item \verb`inetcanon(inet)` \\

    Returns canonical representation of \verb`$inet`.  For IPv4, this
    is dotted-decimal with all 4 bytes.
    The {\tt /}$B$ netmask is included, except if (in version
    7.07.1554840000 20190409 and later) the network is host-only
    (i.e. netmask is the full size of the IP address).  Empty string
    is returned on error.

  \item \verb`inetnetwork(inet)` \\
    Returns string IP address with the network bits of \verb`inet`,
    and the host bits set to 0.  Empty string is returned on error.

  \item \verb`inethost(inet)` \\
    Returns string IP address with the host bits of \verb`inet`,
    and the network bits set to 0.  Empty string is returned on error.

  \item \verb`inetbroadcast(inet)` \\
    Returns string IP broadcast address for \verb`inet`, i.e. with
    the network bits, and host bits set to 1.  Empty string is
    returned on error.

  \item \verb`inetnetmask(inet)` \\
    Returns string IP netmask for \verb`inet`, i.e. with the
    network bits set to 1, and host bits set to 0.  Empty string is
    returned on error.

  \item \verb`inetnetmasklen(inet)` \\
    Returns integer netmask length of \verb`inet`.  -1 is returned
    on error.
% $
  \item \verb`inetcontains(inetA, inetB)` \\
    Returns 1 if \verb`inetA` contains \verb`inetB`, i.e. every
    address in \verb`inetB` occurs within the \verb`inetA` network.
    0 is returned if not, or -1 on error.

  \item \verb`inetclass(inet)` \\
    Returns class of \verb`inet`, e.g. {\tt A}, {\tt B}, {\tt C},
    {\tt D}, {\tt E} or {\tt classless} if a different netmask is
    used (or the address is IPv6).  Empty string is returned on error.

  \item \verb`inet2int(inet)` \\

    Returns integer representation of IP network/host bits of
    \verb`$inet` (i.e. without netmask); useful for compact storage of
    address as integer(s) instead of string.
    Returns -1 is returned on error (note that -1 may also be
    returned for an all-ones IP address, e.g. {\tt 255.255.255.255}).

  \item \verb`int2inet(i)` \\
    Returns \verb`inet` string for
    1- or 4-value {\tt varint} \verb`$i`
    taken as an IP address.  Since no netmask can be stored in the
    integer form of an IP address, the returned IP string will not
    have a netmask.  Empty string is returned on error.

\end{itemize}

% - - - - - - - - - - - - - - - - - - - - - - - - - - - - - - - - - - - - - -
\subsubsection{urlcanonicalize}

Canonicalize a URL.  Usage:
\begin{verbatim}
   urlcanonicalize(url[, flags])
\end{verbatim}

Returns a copy of \verb`url`, canonicalized according to
case-insensitive comma-separated \verb`flags`, which are zero or more of:

\begin{itemize}

  \item \verb`lowerProtocol` \\

    Lower-cases the protocol.

  \item \verb`lowerHost` \\

    Lower-cases the hostname.

  \item \verb`removeTrailingDot` \\

    Removes trailing dot(s) in hostname.

  \item \verb`reverseHost` \\

    Reverse the host/domains in the hostname.  E.g.
    {\tt http://host.example.com/} becomes
    {\tt http://com.example.host/}.  This can be used to put the
    most-significant part of the hostname leftmost.

  \item \verb`removeStandardPort` \\

    Remove the port number if it is the standard port for the protocol.

  \item \verb`decodeSafeBytes` \\

    URL-decode safe bytes, where semantics are unlikely to change.
    E.g. ``\verb`%41`'' becomes ``\verb`A`'', but ``\verb`%2F`''
    remains encoded, because it would decode to ``\verb`/`''.

  \item \verb`upperEncoded` \\

    Upper-case the hex characters of encoded bytes.

  \item \verb`lowerPath` \\

    Lower-case the (non-encoded) characters in the path.  May be used
    for URLs known to point to case-insensitive filesystems,
    e.g. Windows.

  \item \verb`addTrailingSlash` \\

    Adds a trailing slash to the path, if no path is present.

\end{itemize}

Default flags are all but \verb`reverseHost`, \verb`lowerPath`.  A
flag may be prefixed with the operator \verb`+` to append the flag to
existing flags; \verb`-` to remove the flag from existing flags; or
\verb`=` (default) to clear existing flags first and then set the
flag.  Operators remain in effect for subsequent flags until the next
operator (if any) is used.  Function added in Texis version 7.05.

% ----------------------------------------------------------------------------

\subsection{Geographical coordinate functions}

The geographical coordinate functions allow for efficient processing
of latitude / longitude operations.  They allow for the conversion of
a latitude/longitude pair into a single ``geocode'', which is a single
\verb`long` value that contains both values.  This can be used to
easily compare it to other geocodes (for distance calculations) or for
finding other geocodes that are within a certain distance.

% - - - - - - - - - - - - - - - - - - - - - - - - - - - - - - - - - - - - - -
\subsubsection{azimuth2compass}

\begin{verbatim}
  azimuth2compass(double azimuth [, int resolution [, int verbosity]])
\end{verbatim}

The \verb`azimuth2compass` function converts a numerical azimuth value
(degrees of rotation from 0 degrees north) and converts it into a
compass heading, such as \verb`N` or \verb`Southeast`.  The exact text
returned is controlled by two optional parameters, \verb`resolution`
and \verb`verbosity`.

\verb`Resolution` determines how fine-grained the values returned
are.  There are 4 possible values:

\begin{itemize}
\item \verb`1` - Only the four cardinal directions are used (N, E, S,
W)
\item \verb`2` \em{(default)} - Inter-cardinal directions (N, NE, E,
etc.)
\item \verb`3` - In-between inter-cardinal directions (N, NNE, NE, ENE,
E, etc.)
\item \verb`4` - ``by'' values (N, NbE, NNE, NEbN, NE, NEbE, ENE, EbN,
E, etc.)
\end{itemize}

\verb`Verbosity` affects how verbose the resulting text is.  There are
two possible values:

\begin{itemize}
\item \verb`1` \em{(default)} - Use initials for direction values (N,
NbE, NNE, etc.)
\item \verb`2` - Use full text for direction values (North, North by
east, North-northeast, etc.)
\end{itemize}

For an azimuth value of \verb`105`, here are some example results of
\verb`azimuth2compass`:

\begin{verbatim}
azimuth2compass(105): E
azimuth2compass(105, 3): ESE
azimuth2compass(105, 4): EbS
azimuth2compass(105, 1, 2): East
azimuth2compass(105, 3, 2): East-southeast
azimuth2compass(105, 4, 2): East by south
\end{verbatim}

% - - - - - - - - - - - - - - - - - - - - - - - - - - - - - - - - - - - - - -
\subsubsection{azimuthgeocode}

\begin{verbatim}
  azimuthgeocode(geocode1, geocode2 [, method])
\end{verbatim}

The \verb`azimuthgeocode` function calculates the directional heading
going from one geocode to another.  It returns a number between 0-360
where 0 is north, 90 east, etc., up to 360 being north again.

The third, optional \verb`method` parameter can be used to specify
which mathematical method is used to calculate the direction.  There
are two possible values:

\begin{itemize}
\item \verb`greatcircle` {\em(default)} - The ``Great Circle'' method
is a highly accurate tool for calculating distances and directions on
a sphere.  It is used by default.

\item \verb`pythagorean` - Calculations based on the pythagorean
method can also be used.  They're faster, but less accurate as the
core formulas don't take the curvature of the earth into
consideration.  Some internal adjustments are made, but the values are
less accurate than the \verb`greatcircle` method, especially over long
distances and with paths that approach the poles.
\end{itemize}

\EXAMPLE 

For examples of using \verb`azimuthgeocode`, see the \verb`geocode`
script in the \verb`texis/samples` directory.

% - - - - - - - - - - - - - - - - - - - - - - - - - - - - - - - - - - - - - -
\subsubsection{azimuthlatlon}

\begin{verbatim}
  azimuthlatlon(lat1, lon1, lat2, lon2, [, method])
\end{verbatim}

The \verb`azimuthlatlon` function calculates the directional heading
going from one latitude-longitude point to another.  It operates
identically to \verb`azimuthgeocode`, except azimuthlatlon takes its
parameters in a pair of latitude-longitude points instead of geocode
values.

The third, optional \verb`method` parameter can be used to specify
which mathematical method is used to calculate the direction.  There
are two possible values:

\begin{itemize}
\item \verb`greatcircle` {\em(default)} - The ``Great Circle'' method
is a highly accurate tool for calculating distances and directions on
a sphere.  It is used by default.

\item \verb`pythagorean` - Calculations based on the pythagorean
method can also be used.  They're faster, but less accurate as the
core formulas don't take the curvature of the earth into
consideration.  Some internal adjustments are made, but the values are
less accurate than the \verb`greatcircle` method, especially over long
distances and with paths that approach the poles.

\end{itemize}

% - - - - - - - - - - - - - - - - - - - - - - - - - - - - - - - - - - - - - -
\subsubsection{dms2dec, dec2dms}

\begin{verbatim}
  dms2dec(dms)
  dec2dms(dec)
\end{verbatim}

The \verb`dms2dec` and \verb`dec2dms` functions are for changing back
and forth between the deprecated Texis/Vortex ``degrees minutes
seconds'' (DMS) format (west-positive) and ``decimal degree'' format
for latitude and longitude coordinates.  All SQL geographical functions
expect decimal degree parameters (the Vortex \verb`<code2geo>` and
\verb`<geo2code>` Vortex functions expect Texis/Vortex DMS).

Texis/Vortex DMS values are of the format $DDDMMSS$.  For example,
35\degree 15' would be represented as 351500.

In decimal degrees, a degree is a whole digit, and minutes \& seconds
are represented as fractions of a degree.  Therefore, 35\degree 15'
would be 35.25 in decimal degrees.

Note that the Texis/Vortex DMS format has {\em west}-positive
longitudes (unlike ISO 6709 DMS format), and decimal
degrees have {\em east}-positive longitudes.  It is up to the caller
to flip the sign of longitudes where needed.

% - - - - - - - - - - - - - - - - - - - - - - - - - - - - - - - - - - - - - -
\subsubsection{distgeocode}

\begin{verbatim}
  distgeocode(geocode1, geocode2 [, method] )
\end{verbatim}

The \verb`distgeocode` function calculates the distance, in miles,
between two given geocodes.  It uses the ``Great Circle'' method for
calculation by default, which is very accurate.  A faster, but less
accurate, calculation can be done with the Pythagorean theorem.  It is
not designed for distances on a sphere, however, and becomes somewhat  
inaccurate at larger distances and on paths that approach the poles.
To use the Pythagorean theorem, pass a third string parameter,
``\verb`pythagorean`'', to force that method.  ``\verb`greatcircle`''
can also be specified as a method. 

For example:
\begin{itemize}
\item New York (JFK) to Cleveland (CLE), the Pythagorean method is off by 
.8 miles (.1\%)
\item New York (JFK) to Los Angeles (LAX), the Pythagorean method is off by 
22.2 miles (.8\%)
\item New York (JFK) to South Africa (PLZ), the Pythagorean method is off by
430 miles (5.2\%)
\end{itemize}

\EXAMPLE 

For examples of using \verb`distgeocode`, see the \verb`geocode`
script in the \verb`texis/samples` directory.

\SEE

\verb`distlatlon`

For a very fast method that leverages geocodes for selecting cities
within a certain radius, see the \verb`<code2geo>` and \verb`<geo2code>`
functions in the Vortex manual.

% - - - - - - - - - - - - - - - - - - - - - - - - - - - - - - - - - - - - - -
\subsubsection{distlatlon}

\begin{verbatim}
  distlatlon(lat1, lon1, lat2, lon2 [, method] )
\end{verbatim}

The \verb`distlatlon` function calculates the distance, in miles,
between two points, represented in latitude/longitude pairs in decimal
degree format.

Like \verb`distgeocode`, it uses the ``Great Circle'' method by
default, but can be overridden to use the faster, less accurate
Pythagorean method if ``\verb`pythagorean`'' is passed as the optional
\verb`method` parameter.

For example:
\begin{itemize}
\item New York (JFK) to Cleveland (CLE), the Pythagorean method is off by 
.8 miles (.1\%)
\item New York (JFK) to Los Angeles (LAX), the Pythagorean method is off by 
22.2 miles (.8\%)
\item New York (JFK) to South Africa (PLZ), the Pythagorean method is off by
430 miles (5.2\%)
\end{itemize}

\SEE

\verb`distgeocode`

% - - - - - - - - - - - - - - - - - - - - - - - - - - - - - - - - - - - - - -

\subsubsection{latlon2geocode, latlon2geocodearea}

\begin{verbatim}
  latlon2geocode(lat[, lon])
  latlon2geocodearea(lat[, lon], radius)
\end{verbatim}

The \verb`latlon2geocode` function encodes a given latitude/longitude
coordinate into one \verb`long` return value.  This encoded value -- a
``geocode'' value -- can be indexed and used with a special variant of
Texis' \verb`BETWEEN` operator for bounded-area searches of a
geographical region.

The \verb`latlon2geocodearea` function generates a bounding area
centered on the coordinate.  It encodes a given latitude/longitude
coordinate into a {\em two-} value \verb`varlong`.  The returned
geocode value pair represents the southwest and northeast corners of a
square box centered on the latitude/longitude coordinate, with sides
of length two times \verb`radius` (in decimal degrees).  This bounding
area can be used with the Texis \verb`BETWEEN` operator for fast
geographical searches.  \verb`latlon2geocodearea` was added in
version 6.00.1299627000 20110308; it replaces the deprecated Vortex
\verb`<geo2code>` function.

The \verb`lat` and \verb`lon` parameters are \verb`double`s in the
decimal degrees format.  (To pass $DDDMMSS$ ``degrees minutes
seconds'' (DMS) format values, convert them first with \verb`dms2dec`
or \verb`parselatitude()`/\verb`parselongitude()`.).  Negative numbers
represent south latitudes and west longitudes, i.e. these functions
are east-positive, and decimal format (unlike Vortex \verb`<geo2code>`
which is west-positive, and DMS-format).

Valid values for latitude are -90 to 90 inclusive.  Valid values for
longitude are -360 to 360 inclusive.  A longitude value less than -180
will have 360 added to it, and a longitude value greater than 180 will
have 360 subtracted from it.  This allows longitude values to continue
to increase or decrease when crossing the International Dateline, and
thus avoid a non-linear ``step function''.  Passing invalid \verb`lat`
or \verb`lon` values to \verb`latlon2geocode` will return -1.  These
changes were added in version 5.01.1193956000 20071101.

In version 5.01.1194667000 20071109 and later, the \verb`lon`
parameter is optional: both latitude and longitude (in that order) may
be given in a single space- or comma-separated text (\verb`varchar`)
value for \verb`lat`.  Also, a \verb`N`/\verb`S` suffix (for latitude)
or \verb`E`/\verb`W` suffix (for longitude) may be given; \verb`S` or
\verb`W` will negate the value.

  In version 6.00.1300154000 20110314 and later, the latitude and/or
longitude may have just about any of the formats supported by
\verb`parselatitude()`/\verb`parselongitude()`
(p.~\pageref{parselatitudeSqlFunc}), provided they are disambiguated
(e.g. separate parameters; or if one parameter, separated by a comma
and/or fully specified with degrees/minutes/seconds).

\EXAMPLE
\begin{samepage}
% NOTE: this is tested in Vortex test454:
\begin{verbatim}
  -- Populate a table with latitude/longitude information:
  create table geotest(city varchar(64), lat double, lon double,
                       geocode long);
  insert into geotest values('Cleveland, OH, USA', 41.4,  -81.5,  -1);
  insert into geotest values('Seattle, WA, USA',   47.6, -122.3,  -1);
  insert into geotest values('Dayton, OH, USA',    39.75, -84.19, -1);
  insert into geotest values('Columbus, OH, USA',  39.96, -83.0,  -1);
  -- Prepare for geographic searches:
  update geotest set geocode = latlon2geocode(lat, lon);
  create index xgeotest_geocode on geotest(geocode);
  -- Search for cities within a 3-degree-radius "circle" (box)
  -- of Cleveland, nearest first:
  select city, lat, lon, distlatlon(41.4, -81.5, lat, lon) MilesAway
  from geotest
  where geocode between (select latlon2geocodearea(41.4, -81.5, 3.0))
  order by 4 asc;
\end{verbatim}
\end{samepage}

For more examples of using \verb`latlon2geocode`, see the
\verb`geocode` script in the \verb`texis/samples` directory.

\CAVEATS

The geocode values returned by \verb`latlon2geocode` and
\verb`latlon2geocodearea` are platform-dependent in format and
accuracy, and should not be copied across platforms. On platforms with
32-bit \verb`long`s a geocode value is accurate to about 32 seconds
(around half a mile, depending on latitude).  -1 is returned for
invalid input values (in version 5.01.1193955804 20071101 and later).

NOTES

The geocodes produced by these functions are compatible with the codes
used by the deprecated Vortex functions \verb`<code2geo>` and
\verb`<geo2code>`.  However, the \verb`<code2geo>` and
\verb`<geo2code>` functions take Texis/Vortex DMS format ($DDDMMSS$
``degrees minutes seconds'', as described in the \verb`dec2dms` and
\verb`dms2dec` SQL functions).

\SEE

\verb`geocode2lat`, \verb`geocode2lon`

% - - - - - - - - - - - - - - - - - - - - - - - - - - - - - - - - - - - - - -
\subsubsection{geocode2lat, geocode2lon}

\begin{verbatim}
  geocode2lat(geocode)
  geocode2lon(geocode)
\end{verbatim}

The \verb`geocode2lat` and \verb`geocode2lon` functions decode a
geocode into a latitude or longitude coordinate, respectively.  The
returned coordinate is in the decimal degrees format.  In
version 5.01.1193955804 20071101 and later, an invalid geocode value
(e.g. -1) will return NaN (Not a Number).

If you want $DDDMMSS$ ``degrees minutes seconds'' (DMS) format, you
can use \verb`dec2dms` to convert it.

\EXAMPLE

\begin{verbatim}
  select city, geocode2lat(geocode), geocode2lon(geocode) from geotest;
\end{verbatim}

\CAVEATS

As with \verb`latlon2geocode`, the \verb`geocode` value is platform-dependent
in accuracy and format, so it should not be copied across platforms,
and the returned coordinates from \verb`geocode2lat` and
\verb`geocode2lon` may differ up to about half a minute from the
original coordinates (due to the finite resolution of a \verb`long`).
In version 5.01.1193955804 20071101 and later, an invalid geocode value
(e.g. -1) will return \verb`NaN` (Not a Number).

\SEE

\verb`latlon2geocode`

% - - - - - - - - - - - - - - - - - - - - - - - - - - - - - - - - - - - - - -
\subsubsection{parselatitude, parselongitude}
\label{parselatitudeSqlFunc}

\begin{verbatim}
  parselatitude(latitudeText)
  parselongitude(longitudeText)
\end{verbatim}

The \verb`parselatitude` and \verb`parselongitude` functions parse a
text (\verb`varchar`) latitude or longitude coordinate, respectively,
and return its value in decimal degrees as a \verb`double`.  The
coordinate should be in one of the following forms (optional parts in
square brackets):

[$H$] $nnn$ [$U$] [\verb`:`] [$H$] [$nnn$ [$U$] [\verb`:`] [$nnn$ [$U$]]] [$H$] \\
$DDMM$[$.MMM$...] \\
$DDMMSS$[$.SSS$...]

where the terms are:

\begin{itemize}
  \item $nnn$ \\

    A number (integer or decimal) with optional plus/minus sign.  Only
    the first number may be negative, in which case it is a south
    latitude or west longitude.  Note that this is true even for
    $DDDMMSS$ (DMS) longitudes -- i.e. the ISO 6709 east-positive
    standard is followed, not the deprecated Texis/Vortex
    west-positive standard.

  \item $U$ \\
    A unit (case-insensitive):
    \begin{itemize}
      \item \verb`d`
      \item \verb`deg`
      \item \verb`deg.`
      \item \verb`degrees`
      \item \verb`'` (single quote) for minutes
      \item \verb`m`
      \item \verb`min`
      \item \verb`min.`
      \item \verb`minutes`
      \item \verb`"` (double quote) for seconds
      \item \verb`s` (iff \verb`d`/\verb`m` also used for degrees/minutes)
      \item \verb`sec`
      \item \verb`sec.`
      \item \verb`seconds`
      \item Unicode degree-sign (U+00B0), in ISO-8559-1 or UTF-8
    \end{itemize}
    If no unit is given, the first number is assumed to be degrees,
    the second minutes, the third seconds.  Note that ``\verb`s`'' may
    only be used for seconds if ``\verb`d`'' and/or ``\verb`m`'' was
    also used for an earlier degrees/minutes value; this is to help
    disambiguate ``seconds'' vs. ``southern hemisphere''.

  \item $H$ \\
    A hemisphere (case-insensitive):
    \begin{itemize}
      \item \verb`N`
      \item \verb`north`
      \item \verb`S`
      \item \verb`south`
      \item \verb`E`
      \item \verb`east`
      \item \verb`W`
      \item \verb`west`
    \end{itemize}
    A longitude hemisphere may not be given for a latitude, and
    vice-versa.

  \item $DD$ \\

    A two- or three-digit degree value, with optional sign.  Note that
    longitudes are east-positive ala ISO 6709, not west-positive like
    the deprecated Texis standard.

  \item $MM$ \\

    A two-digit minutes value, with leading zero if needed to make two digits.

  \item $.MMM$... \\

    A zero or more digit fractional minute value.

  \item $SS$ \\

    A two-digit seconds value, with leading zero if needed to make two digits.

  \item $.SSS$... \\

    A zero or more digit fractional seconds value.

\end{itemize}

  Whitespace is generally not required between terms in the first
format.  A hemisphere token may only occur once.
Degrees/minutes/seconds numbers need not be in that order, if units
are given after each number.  If a 5-integer-digit $DDDMM$[$.MMM$...]
format is given and the degree value is out of range (e.g. more than
90 degrees latitude), it is interpreted as a $DMMSS$[$.SSS$...] value
instead.  To force $DDDMMSS$[$.SSS$...] for small numbers, pad with
leading zeros to 6 or 7 digits.

\EXAMPLE

\begin{verbatim}
insert into geotest(lat, lon)
  values(parselatitude('54d 40m 10"'),
         parselongitude('W90 10.2'));
\end{verbatim}

\CAVEATS

An invalid or unparseable latitude or longitude value will return
\verb`NaN` (Not a Number).  Extra unparsed/unparsable text may be
allowed (and ignored) after the coordinate in most instances.
Out-of-range values (e.g. latitudes greater than 90 degrees) are
accepted; it is up to the caller to bounds-check the result.  The
\verb`parselatitude` and \verb`parselongitude` SQL functions were
added in version 6.00.1300132000 20110314.

% ----------------------------------------------------------------------------

\subsection{JSON functions}

The JSON functions allow for the manipulation of \verb`varchar` fields
and literals as JSON objects.

\subsubsection{JSON Path Syntax}
The JSON Path syntax is standard Javascript object access, using \verb`$` to
represent the entire document.  If the document is an object the path must
start \verb`$.`, and if an array \verb`$[`.

\subsubsection{JSON Field Syntax}
\label{jsoncomputedfield}
In addition to using the JSON functions it is possible to access elements
in a \verb`varchar` field that holds JSON as if it was a field itself.
This allows for creation of indexes, searching and sorting efficiently.
Arrays can also be fetched as \verb`strlst` to make use of those features,
e.g. \verb`SELECT Json.$.name FROM tablename WHERE 'SQL' IN Json.$.skills[*];`

% - - - - - - - - - - - - - - - - - - - - - - - - - - - - - - - - - - - - - -
\subsubsection{isjson}

\begin{verbatim}
  isjson(JsonDocument)
\end{verbatim}

The \verb`isjson` function returns 1 if the document is valid JSON,
0 otherwise.

\begin{verbatim}
isjson('{ "type" : 1 }'): 1
isjson('{}'): 1
isjson('json this is not'): 0
\end{verbatim}

% - - - - - - - - - - - - - - - - - - - - - - - - - - - - - - - - - - - - - -
\subsubsection{json\_format}

\begin{verbatim}
  json_format(JsonDocument, FormatOptions)
\end{verbatim}

The \verb`json_format` formats the \verb`JsonDocument` according to
\verb`FormatOptions`.  Multiple options can be provided either space
or comma separated.

Valid \verb`FormatOptions` are:
\begin{itemize}
\item COMPACT - remove all unnecessary whitespace
\item INDENT(N) - print the JSON with each object or array member on a new line,
indented by N spaces to show structure
\item SORT-KEYS - sort the keys in the object.  By default the order is preserved
\item EMBED - omit the enclosing \verb`{}` or \verb`[]` is using the snippet in another object
\item ENSURE\_ASCII - encode all Unicode characters outside the ASCII range
\item ENCODE\_ANY - if not a valid JSON document then encode into a JSON literal, e.g. to encode a string.
\item ESCAPE\_SLASH - escape forward slash \verb`/` as \verb`\/`
\end{itemize}

% - - - - - - - - - - - - - - - - - - - - - - - - - - - - - - - - - - - - - -
\subsubsection{json\_type}

\begin{verbatim}
  json_type(JsonDocument)
\end{verbatim}

The \verb`json_type` function returns the type of the JSON object or element.
Valid responses are:
\begin{itemize}
\item OBJECT
\item ARRAY
\item STRING
\item INTEGER
\item DOUBLE
\item NULL
\item BOOLEAN
\end{itemize}

Assuming a field \verb`Json` containing:
{
  "items" : [
    {
      "Num" : 1,
      "Text" : "The Name",
      "First" : true
    },
    {
      "Num" : 2.0,
      "Text" : "The second one",
      "First" : false
    }
    ,
    null
  ]
}
\begin{verbatim}
json_type(Json): OBJECT
json_type(Json.$.items[0]): OBJECT
json_type(Json.$.items): ARRAY
json_type(Json.$.items[0].Num): INTEGER
json_type(Json.$.items[1].Num): DOUBLE
json_type(Json.$.items[0].Text): STRING
json_type(Json.$.items[0].First): BOOLEAN
json_type(Json.$.items[2]): NULL
\end{verbatim}

% - - - - - - - - - - - - - - - - - - - - - - - - - - - - - - - - - - - - - -
\subsubsection{json\_value}

\begin{verbatim}
  json_value(JsonDocument, Path)
\end{verbatim}

The \verb`json_value` extracts the value identified by \verb`Path` from
\verb`JsonDocument`.  \verb`Path` is a varchar in the JSON Path Syntax.
This will return a scalar value.  If \verb`Path` refers to an array,
object, or invalid path no value is returned.

Assuming the same Json field from the previous examples:
\begin{verbatim}
json_value(Json, '$'):
json_value(Json, '$.items[0]'):
json_value(Json, '$.items'):
json_value(Json, '$.items[0].Num'): 1
json_value(Json, '$.items[1].Num'): 2.0
json_value(Json, '$.items[0].Text'): The Name
json_value(Json, '$.items[0].First'): true
json_value(Json, '$.items[2]'):
\end{verbatim}


% - - - - - - - - - - - - - - - - - - - - - - - - - - - - - - - - - - - - - -
\subsubsection{json\_query}

\begin{verbatim}
  json_query(JsonDocument, Path)
\end{verbatim}

The \verb`json_query` extracts the object or array identified by \verb`Path`
from \verb`JsonDocument`.  \verb`Path` is a varchar in the JSON Path Syntax.
This will return either an object or an array value.  If \verb`Path` refers
to a scalar no value is returned.

Assuming the same Json field from the previous examples:

\verb`json_query(Json, '$')`\\
\verb`---------------------`\\
\verb`{"items":[{"Num":1,"Text":"The Name","First":true},`\split
\verb`{"Num":2.0,"Text":"The second one","First":false},null]}`

\verb`json_query(Json, '$.items[0]')`\\
\verb`------------------------------`\\
\verb`{"Num":1,"Text":"The Name","First":true}`

\verb`json_query(Json, '$.items')`\\
\verb`---------------------------`\\
\verb`[{"Num":1,"Text":"The Name","First":true},`\split
\verb`{"Num":2.0,"Text":"The second one","First":false},null]`

The following will return an empty string as they refer to scalars
or non-existent keys.
\begin{verbatim}
json_query(Json, '$.items[0].Num')
json_query(Json, '$.items[1].Num')
json_query(Json, '$.items[0].Text')
json_query(Json, '$.items[0].First')
json_query(Json, '$.items[2]')
\end{verbatim}


% - - - - - - - - - - - - - - - - - - - - - - - - - - - - - - - - - - - - - -
\subsubsection{json\_modify}

\begin{verbatim}
  json_modify(JsonDocument, Path, NewValue)
\end{verbatim}

The \verb`json_modify` function returns a modified version of JsonDocument
with the key at Path replaced by NewValue.

If \verb`Path` starts with {\tt append\textvisiblespace } then the NewValue is appended to the
array referenced by {\tt Path}.  It is an error it {\tt Path} refers to anything
other than an array.

\begin{verbatim}
json_modify('{}', '$.foo', 'Some "quote"')
------------------------------------------
{"foo":"Some \"quote\""}

json_modify('{ "foo" : { "bar": [40, 42] } }', 'append $.foo.bar', 99)
----------------------------------------------------------------------
{"foo":{"bar":[40,42,99]}}

json_modify('{ "foo" : { "bar": [40, 42] } }', '$.foo.bar', 99)
---------------------------------------------------------------
{"foo":{"bar":99}}
\end{verbatim}
% - - - - - - - - - - - - - - - - - - - - - - - - - - - - - - - - - - - - - -
\subsubsection{json\_merge\_patch}

\begin{verbatim}
  json_merge_patch(JsonDocument, Patch)
\end{verbatim}

The \verb`json_merge_patch` function provides a way to patch a target JSON
document with another JSON document.  The patch function conforms to
(href=https://tools.ietf.org/html/rfc7386) RFC 7386

Keys in \verb`JsonDocument` are replaced if found in \verb`Patch`.  If the
value in \verb`Patch` is \verb`null` then the key will be removed in the
target document.

\begin{verbatim}
json_merge_patch('{"a":"b"}',          '{"a":"c"}'
--------------------------------------------------
{"a":"c"}

json_merge_patch('{"a": [{"b":"c"}]}', '{"a": [1]}'
---------------------------------------------------
{"a":[1]}

json_merge_patch('[1,2]',              '{"a":"b", "c":null}'
------------------------------------------------------------
{"a":"b"}

% - - - - - - - - - - - - - - - - - - - - - - - - - - - - - - - - - - - - - -
\subsubsection{json\_merge\_preserve}

\begin{verbatim}
  json_merge_preserve(JsonDocument, Patch)
\end{verbatim}

The \verb`json_merge_preserve` function provides a way to patch a target JSON
document with another JSON document while preserving the content that exists
in the target document.

Keys in \verb`JsonDocument` are merged if found in \verb`Patch`.  If the same
key exists in both the target and patch file the result will be an array with
the values from both target and patch.

If the
value in \verb`Patch` is \verb`null` then the key will be removed in the
target document.

\begin{verbatim}
json_merge_preserve('{"a":"b"}',          '{"a":"c"}'
-----------------------------------------------------
{"a":["b","c"]}

json_merge_preserve('{"a": [{"b":"c"}]}', '{"a": [1]}'
------------------------------------------------------
{"a":[{"b":"c"},1]}

json_merge_preserve('[1,2]',              '{"a":"b", "c":null}'
---------------------------------------------------------------
[1,2,{"a":"b","c":null}]

\end{verbatim}


% - - - - - - - - - - - - - - - - - - - - - - - - - - - - - - - - - - - - - -
% ----------------------------------------------------------------------------

\subsection{Other Functions}

% - - - - - - - - - - - - - - - - - - - - - - - - - - - - - - - - - - - - - -
\subsubsection{exec}

Execute an external command.  The syntax is
\begin{verbatim}
   exec(commandline[, INPUT[, INPUT[, INPUT[, INPUT]]]]);
\end{verbatim}

Allows execution of an external command.  The first argument is the
command to execute.  Any subsequent arguments are written to the standard
input of the process.  The standard output of the command is read as the
return from the function.

This function allows unlimited extensibility of Texis, although if a
particular function is being used often then it should be linked into the
Texis server to avoid the overhead of invoking another process.

For example this could be used to OCR text.  If you have a program which
will take an image filename on the command line, and return the text on
standard out you could issue SQL as follows:

\begin{verbatim}
     UPDATE     DOCUMENTS
     SET        TEXT = exec('ocr '+IMGFILE)
     WHERE      TEXT = '';
\end{verbatim}

Another example would be if you wanted to print envelopes from names and
addresses in a table you might use the following SQL:

\begin{verbatim}
     SELECT	exec('envelope ', FIRST_NAME+' '+LAST_NAME+'
     ', STREET + '
     ', CITY + ', ' + STATE + ' ' + ZIP)
     FROM ADDRESSES;
\end{verbatim}

Notice in this example the addition of spaces and line-breaks between the
fields.  Texis does not add any delimiters between fields or arguments
itself.

% - - - - - - - - - - - - - - - - - - - - - - - - - - - - - - - - - - - - - -
\subsubsection{mminfo}

This function lets you obtain Metamorph info.  You have the choice of
either just getting the portions of the document which were the hits, or
you can also get messages which describe each hit and subhits.

The SQL to use is as follows:

\begin{verbatim}
    SELECT mminfo(query,data[,nhits,[0,msgs]]) from TABLE
           [where CONDITION];
\end{verbatim}

\begin{description}
\item[query]

Query should be a string containing a metamorph query.

\item[data]

The text to search. May be literal data or a field from the table.

\item[nhits]

The maximum number of hits to return.  If it is 0, which
is the default, you will get all the hits.

\item[msgs]

An integer; controls what information is returned. A bit-wise OR of
any combination of the following values:
\begin{itemize}
\item 1 to get matches and offset/length information
\item 2 to suppress text from \verb`data` which matches; printed by default
\item 4 to get a count of hits (up to \verb`nhits`)
\item 8 to get the hit count in a numeric parseable format
\item 16 to get the offset/length in the original query of each search set
\end{itemize}

Set offset/length information (value 16) is of the form:
\begin{verbatim}
Set N offset/len in query: setoff setlen
\end{verbatim}
Where \verb`N` is the set number (starting with 1), \verb`setoff` is
the byte offset from the start of the query where set \verb`N` is,
and \verb`setlen` is the length of the set.
This information is available in version 5.01.1220640000 20080905
and later.

Hit offset/length information is of the form:
\begin{verbatim}
300 <Data from Texis> offset length suboff sublen [suboff sublen]..
301 End of Metamorph hit
\end{verbatim}
Where:
\begin{itemize}
\item offset is the offset within the data of the overall hit context
      (sentence, paragraph, etc...)
\item length is the length of the overall hit context
\item suboff is the offset within the hit of a matching term
\item sublen is the length of the matching term
\item suboff and sublen will be repeated for as many terms as are
      required to satisfy the query.
\end{itemize}

\end{description}


\begin{verbatim}
Example:
   select mminfo('power struggle @0 w/.',Body,0,0,1) inf from html
          where Title\Meta\Body like 'power struggle';
Would give something of the form:

300 <Data from Texis> 62 5 0 5
power
301 End of Metamorph hit
300 <Data from Texis> 2042 5 0 5
power
301 End of Metamorph hit
300 <Data from Texis> 2331 5 0 5
POWER
301 End of Metamorph hit
300 <Data from Texis> 2892 8 0 8
STRUGGLE
301 End of Metamorph hit
\end{verbatim}

% - - - - - - - - - - - - - - - - - - - - - - - - - - - - - - - - - - - - - -
\subsubsection{convert}
\label{convertSqlFunction}

The convert function allows you to change the type of an expression.
The syntax is
\begin{verbatim}
   CONVERT(expression, 'type-name'[, 'mode'])
\end{verbatim}
The type name should in general be in lower case.

This
can be useful in a number of situations.  Some cases where you might want
to use convert are
\begin{itemize}
\item  The display format for a different format is more useful.  For example
you might want to convert a field of type COUNTER to a DATE field, so you
can see when the record was inserted, for example:

\begin{verbatim}
    SELECT convert(id, 'date')
    FROM   LOG;
\end{verbatim}

\begin{smscreen}
\begin{verbatim}
    CONVERT(id, 'date')
    1995-01-27 22:43:48
\end{verbatim}
\end{smscreen}
\item  If you have an application which is expecting data in a particular
type you can use convert to make sure you will receive the correct type.
\end{itemize}


Caveat: Note that in Texis version 7 and later, \verb`convert()`ing
data from/to \verb`varbyte`/\verb`varchar` no longer converts the data
to/from hexadecimal by default (as was done in earlier versions) in
programs other than \verb`tsql`; it is now preserved as-is (though
truncated at nul for \verb`varchar`).  See the \verb`bintohex()` and
\verb`hextobin()` functions (p.~\pageref{bintohexSqlFunction}) for
hexadecimal conversion, and the \verb`hexifybytes` SQL property
(p.~\pageref{hexifybytesProperty}) for controlling automatic hex
conversion.

Also in Texis version 7 and later, an optional third argument may be
given to \verb`convert()`, which is a \verb`varchartostrlstsep` mode
value (p.~\pageref{`varchartostrlstsep'}).  This third argument may
only be supplied when converting to type \verb`strlst` or
\verb`varstrlst`.  It allows the separator character or mode to be
conveniently specified locally to the conversion, instead of having to
alter the global \verb`varchartostrlstsep` mode.


% - - - - - - - - - - - - - - - - - - - - - - - - - - - - - - - - - - - - - -
\subsubsection{seq}

Returns a sequence number.  The number can be initialized to any value,
and the increment can be defined for each call.  The syntax is:

\begin{verbatim}
	seq(increment [, init])
\end{verbatim}

If {\tt init} is given then the sequence number is initialized to that value,
which will be the value returned.  It is then incremented by {\tt increment}.
If {\tt init} is not specified then the current value will be retained.  The
initial value will be zero if {\tt init} has not been specified.

Examples of typical use:

\begin{verbatim}
     SELECT  NAME, seq(1)
     FROM    SYSTABLES
\end{verbatim}

The results are:

\begin{screen}
\begin{verbatim}
  NAME                seq(1)
 SYSTABLES               0
 SYSCOLUMNS              1
 SYSINDEX                2
 SYSUSERS                3
 SYSPERMS                4
 SYSTRIG                 5
 SYSMETAINDEX            6
\end{verbatim}
\end{screen}

\begin{verbatim}
     SELECT  seq(0, 100)
     FROM    SYSDUMMY;

     SELECT  NAME, seq(1)
     FROM    SYSTABLES
\end{verbatim}

The results are:

\begin{screen}
\begin{verbatim}
  seq(0, 100)
     100

  NAME                seq(1)
 SYSTABLES             100
 SYSCOLUMNS            101
 SYSINDEX              102
 SYSUSERS              103
 SYSPERMS              104
 SYSTRIG               105
 SYSMETAINDEX          106
\end{verbatim}
\end{screen}

% - - - - - - - - - - - - - - - - - - - - - - - - - - - - - - - - - - - - - -
\subsubsection{random}

Returns a random {\tt int}.  The syntax is:

\begin{verbatim}
	random(max [, seed])
\end{verbatim}

If {\tt seed} is given then the random number generator is seeded to
that value.  The random number generator will only be seeded once in
each session, and will be randomly seeded on the first call if no seed
is supplied.  The {\tt seed} parameter is ignored in the second and
later calls to {\tt random} in a process.

The returned number is always non-negative, and never larger than the
limit of the C lib's random number generator (typically either 32767
or 2147483647).  If {\tt max} is non-zero, then the returned number
will also be less than {\tt max}.

This function is typically used to either generate a random number for
later use, or to generate a random ordering of result records by adding
{\tt random} to the {\tt ORDER BY} clause.

Examples of typical use:

\begin{verbatim}
     SELECT  NAME, random(100)
     FROM    SYSTABLES
\end{verbatim}

The results might be:

\begin{screen}
\begin{verbatim}
  NAME                random(100)
 SYSTABLES               90
 SYSCOLUMNS              16
 SYSINDEX                94
 SYSUSERS                96
 SYSPERMS                 1
 SYSTRIG                 84
 SYSMETAINDEX            96
\end{verbatim}
\end{screen}

\begin{verbatim}
     SELECT  ENAME
     FROM    EMPLOYEE
     ORDER BY random(0);
\end{verbatim}

The results would be a list of employees in a random order.

% - - - - - - - - - - - - - - - - - - - - - - - - - - - - - - - - - - - - - -
\subsubsection{bintohex}
\label{bintohexSqlFunction}

  Converts a binary (\verb`varbyte`) value into a hexadecimal string.

\begin{verbatim}
    bintohex(varbyteData[, 'stream|pretty'])
\end{verbatim}

  A string (\verb`varchar`) hexadecimal representation of the
\verb`varbyteData` parameter is returned.  This can be useful to
visually examine binary data that may contain non-printable or nul
bytes.  The optional second argument is a comma-separated string of
any of the following flags:

\begin{itemize}
  \item \verb`stream`: Use the default output mode: a continuous
    stream of hexadecimal bytes, i.e. the same format that
    \verb`convert(varbyteData, 'varchar')` would have returned in
    Texis version 6 and earlier.

  \item \verb`pretty`: Return a ``pretty'' version of the data: print
    16 byte per line, space-separate the hexadecimal bytes, and print
    an ASCII dump on the right side.
\end{itemize}

The \verb`bintohex()` function was added in Texis version 7.  Caveat:
Note that in version 7 and later, \verb`convert()`ing data from/to
\verb`varbyte`/\verb`varchar` no longer converts the data to/from
hexadecimal by default (as was done in earlier versions) in programs
other than \verb`tsql`; it is now preserved as-is (though truncated at
nul for \verb`varchar`).  See the \verb`hexifybytes` SQL property
(p.~\pageref{hexifybytesProperty}) to change this.

% - - - - - - - - - - - - - - - - - - - - - - - - - - - - - - - - - - - - - -
\subsubsection{hextobin}

  Converts a hexadecimal stream to its binary representation.

\begin{verbatim}
    hextobin(hexString[, 'stream|pretty'])
\end{verbatim}

  The hexadecimal \verb`varchar` string \verb`hexString` is converted
to its binary representation, and the \verb`varbyte` result returned.
The optional second argument is a comma-separated string of any of the
following flags:

\begin{itemize}
  \item \verb`stream`: Only accept the \verb`stream` format of
    \verb`bintohex()`, i.e. a stream of hexadecimal bytes, the same
    format that \verb`convert(varbyteData, 'varchar')` would have
    returned in Texis version 6 and earlier.  Whitespace is
    acceptable, but only between (not within) hexadecimal bytes.
    Case-insensitive.  Non-conforming data will result in an error
    message and the function failing.

  \item \verb`pretty`: Accept either \verb`stream` or \verb`pretty`
    format data; if the latter, only the hexadecimal bytes are parsed
    (e.g. ASCII column is ignored).  Parsing is more liberal, but
    may be confused if the data deviates significantly from either
    format.
\end{itemize}

The \verb`hextobin()` function was added in Texis version 7.  Caveat:
Note that in version 7 and later, \verb`convert()`ing data from/to
\verb`varbyte`/\verb`varchar` no longer converts the data to/from
hexadecimal by default (as was done in earlier versions) in programs
other than \verb`tsql`; it is now preserved as-is (though truncated at
nul for \verb`varchar`).  See the \verb`hexifybytes` SQL property
(p.~\pageref{hexifybytesProperty}) to change this.


% - - - - - - - - - - - - - - - - - - - - - - - - - - - - - - - - - - - - - -
\subsubsection{identifylanguage}

  Tries to identify the predominant language of a given string.  By
  returning a probability in addition to the identified language, this
  function can also serve as a test of whether the given string is
  really natural-language text, or perhaps binary/encoded data
  instead.  Syntax:

\begin{verbatim}
    identifylanguage(text[, language[, samplesize]])
\end{verbatim}

  The return value is a two-element \verb`strlst`: a probability and a
  language code.  The probability is a value from {\tt 0.000} to
  {\tt 1.000} that the {\tt text} argument is composed in the
  language named by the returned language code.  The language code is
  a two-letter ISO-639-1 code.

  If an ISO-639-1 code is given for the optional {\tt language}
  argument, the probability for that particular language is returned,
  instead of for the highest-probability language of the
  known/built-in languages (currently {\tt de}, {\tt es}, {\tt fr},
  {\tt ja}, {\tt pl}, {\tt tr}, {\tt da}, {\tt en}, {\tt eu},
  {\tt it}, {\tt ko}, {\tt ru}).

  The optional third argument {\tt samplesize} is the initial integer
  size in bytes of the {\tt text} to sample when determining language;
  it defaults to 16384.  The {\tt samplesize} parameter was added in
  version 7.01.1382113000 20131018.

  Note that since a \verb`strlst` value is returned, the probability
  is returned as a \verb`strlst` element, not a \verb`double` value,
  and thus should be cast to \verb`double` during comparisons.  In
  Vortex with \verb`arrayconvert` on (the default), the return value
  will be automatically split into a two-element Vortex \verb`varchar`
  array.

  The \verb`identifylanguage()` function is experimental, and its
  behavior, syntax, name and/or existence are subject to change
  without notice.  Added in version 7.01.1381362000 20131009.

% - - - - - - - - - - - - - - - - - - - - - - - - - - - - - - - - - - - - - -
\subsubsection{lookup}
\label{lookup_SqlFunction}

By combining the \verb`lookup()` function with a \verb`GROUP BY`, a
column may be grouped into bins or ranges -- e.g. for price-range
grouping -- instead of distinct individual values.  Syntax:

\begin{verbatim}
    lookup(keys, ranges[, names])
\end{verbatim}

The {\tt keys} argument is one (or more, e.g. \verb`strlst`) values to
look up; each is searched for in the {\tt ranges} argument, which is
one (or more, e.g. \verb`strlst`) ranges.  All range(s) that the given
key(s) match will be returned.  If the {\tt names} argument is given,
the corresponding {\tt names} value(s) are returned instead; this
allows ranges to be renamed into human-readable values.  If {\tt
  names} is given, the number of its values must equal the number of
ranges.

Each range is a pair of values (lower and upper bounds) separated by
``{\tt ..}'' (two periods).  The range is optionally surrounded by
square (bound included) or curly (bound excluded) brackets.  E.g.:

\begin{verbatim}
[10..20}
\end{verbatim}

denotes the range 10 to 20: including 10 (``{\tt [}'') but not
  including (``{\tt \}}'') 20.  Both an upper and lower bracket must
  be given if either is present (though they need not be the same
  type).  The default if no brackets are given is to include the lower
  bound but exclude the upper bound; this makes consecutive ranges
  non-overlapping, if they have the same upper and lower bound and no
  brackets (e.g. ``0..10,10..20'').  Either bound may be omitted, in
  which case that bound is unlimited.  Each range's lower bound must
  not be greater than its upper bound, nor equal if either bound is
  exclusive.

If a {\tt ranges} value is not {\tt varchar}/{\tt char}, or does not
contain ``{\tt ..}'', its entire value is taken as a single inclusive
lower bound, and the exclusive upper bound will be the next {\tt
  ranges} value's lower bound (or unlimited if no next value).
E.g. the \verb`varint` lower-bound list:

\begin{verbatim}
0,10,20,30
\end{verbatim}

is equivalent to the \verb`strlst` range list:

\begin{verbatim}
[0..10},[10..20},[20..30},[30..]
\end{verbatim}

By using the \verb`lookup()` function in a \verb`GROUP BY`, a column may
be grouped into ranges.  For example, given a table {\tt Products}
with the following SKUs and \verb`float` prices:

\begin{verbatim}
    SKU    Price
    ------------
    1234   12.95
    1235    5.99
    1236   69.88
    1237   39.99
    1238   29.99
    1239   25.00
    1240   50.00
    1241   -2.00
    1242  499.95
    1243   19.95
    1244    9.99
    1245  125.00
\end{verbatim}

they may be grouped into price ranges (with most-products first) with
this SQL:

\begin{samepage}
{\small
\begin{verbatim}
SELECT   lookup(Price, convert('0..25,25..50,50..,', 'strlst', 'lastchar'),
     convert('Under $25,$25-49.99,$50 and up,', 'strlst', 'lastchar'))
       PriceRange, count(SKU) NumberOfProducts
FROM Products
GROUP BY lookup(Price, convert('0..25,25..50,50..,', 'strlst', 'lastchar'),
     convert('Under $25,$25-49.99,$50 and up,', 'strlst', 'lastchar'))
ORDER BY 2 DESC;
\end{verbatim}
}
\end{samepage}

or this Vortex:

\begin{samepage}
\begin{verbatim}
<$binValues =   "0..25"      "25..50"     "50..">
<$binDisplays = "Under $$25" "$$25-49.99" "$$50 and up">
<sql row "select lookup(Price, $binValues, $binDisplays) PriceRange,
              count(SKU) NumberOfProducts
          from Products
          group by lookup(Price, $binValues, $binDisplays)
          order by 2 desc">
  <fmt "%10s: %d\n" $PriceRange $NumberOfProducts>
</sql>
\end{verbatim}
\end{samepage}

which would give these results:

\begin{verbatim}
  PriceRange NumberOfProducts
------------+------------+
Under $25,              4
$50 and up,             4
$25-49.99,              3
                        1
\end{verbatim}

The trailing commas in {\tt PriceRange} values are due to them being
\verb`strlst` values, for possible multiple ranges.  Note the empty
{\tt PriceRange} for the fourth row: the -2 {\tt Price} matched
no ranges, and hence an empty {\tt PriceRange} was returned for
it.

\CAVEATS

The \verb`lookup()` function as described above was added in Texis
version 7.06.1528745000 20180611.

A different version of the \verb`lookup()` function was first added in
version 7.01.1386980000 20131213: it only took the second range syntax
variant (single lower bound); range values had to be in ascending
order (by {\tt keys} type); only the first matching range was
returned; and if a key did not match any range the first range was
returned.

\SEE

\verb`lookupCanonicalizeRanges`, \verb`lookupParseRange`

% - - - - - - - - - - - - - - - - - - - - - - - - - - - - - - - - - - - - - -
\subsubsection{lookupCanonicalizeRanges}
\label{lookupCanonicalizeRanges_SqlFunction}

The \verb`lookupCanonicalizeRanges()` function returns the canonical
version(s) of its {\tt ranges} argument, which is zero or more ranges
of the syntaxes acceptable to \verb`lookup()`
(p.~\pageref{lookup_SqlFunction}):

\begin{verbatim}
    lookupCanonicalizeRanges(ranges, keyType)
\end{verbatim}

The canonical version always includes both a lower and upper
inclusive/exclusive bracket/brace, both lower and upper bounds (unless
unlimited), the ``{\tt ..}'' range operator, and is independent of
other ranges that may be in the sequence.

The {\tt keyType} parameter is a \verb`varchar` string denoting the
SQL type of the key field that would be looked up in the given
range(s).  This ensures that comparisons are done correctly.  E.g. for
a \verb`strlst` range list of ``{\tt 0,500,1000}'', {\tt keyType}
should be ``{\tt integer}'', so that ``{\tt 500}'' is not compared
alphabetically with ``{\tt 1000}'' and considered invalid (greater than).

This function can be used to verify the syntax of a range, or to
transform it into a standard form for \verb`lookupParseRange()`
(p.~\pageref{lookupParseRange_SqlFunction}).

\CAVEATS

For an implicit-upper-bound range, the upper bound is determined by
the {\em next} range's lower bound.  Thus the full list of ranges (if
multiple) should be given to \verb`lookupCanonicalizeRanges()` -- even
if only one range needs to be canonicalized -- so that each range gets
its proper bounds.

The \verb`lookupCanonicalizeRanges()` function was added in version
7.06.1528837000 20180612.  The {\tt keyType} parameter was added
in version 7.06.1535500000 20180828.

\SEE

\verb`lookup`, \verb`lookupParseRange`

% - - - - - - - - - - - - - - - - - - - - - - - - - - - - - - - - - - - - - -
\subsubsection{lookupParseRange}
\label{lookupParseRange_SqlFunction}

The \verb`lookupParseRange()` function parses a single
\verb`lookup()`-style range into its constituent parts, returning them
as strings in one \verb`strlst` value.  This can be used by Vortex
scripts to edit a range.  Syntax:

\begin{verbatim}
    lookupParseRange(range, parts)
\end{verbatim}

The {\tt parts} argument is zero or more of the following part tokens
as strings:

\begin{itemize}
  \item \verb`lowerInclusivity`: Returns the inclusive/exclusive operator
    for the lower bound, e.g. ``{\tt \{}'' or ``{\tt [}''
  \item \verb`lowerBound`: Returns the lower bound
  \item \verb`rangeOperator`: Returns the range operator, e.g. ``{\tt ..}''
  \item \verb`upperBound`: Returns the upper bound
  \item \verb`upperInclusivity`: Returns the inclusive/exclusive operator
    for the upper bound, e.g. ``{\tt \}}'' or ``{\tt ]}''
\end{itemize}

If a requested part is not present, an empty string is returned for
that part.  The concatenation of the above listed parts, in the above
order, should equal the given range.  Non-string range arguments are
not supported.

The \verb`lookupParseRange()` function was added in
version 7.06.1528837000 20180612.

\EXAMPLE

\begin{verbatim}
    lookupParseRange('10..20', 'lowerInclusivity')
\end{verbatim}

would return a single empty-string \verb`strlst`, as there is no
lower-bound inclusive/exclusive operator in the range ``{\tt 10..20}''.

\begin{verbatim}
    lookupParseRange('10..20', 'lowerBound')
\end{verbatim}

would return a \verb`strlst` with the single value ``{\tt 10}''.

\CAVEATS

For an implicit-upper-bound range, the upper bound is determined by
the {\em next} range's lower bound.  Since \verb`lookupParseRange()`
only takes one range, passing such a range to it may result in an
incorrect (unlimited) upper bound.  Thus the full list of ranges (if
multiple) should always be given to \verb`lookupCanonicalizeRanges()`
first, and only then the desired canonicalized range passed to
\verb`lookupParseRange()`.

\SEE

\verb`lookup`, \verb`lookupCanonicalizeRanges`

% - - - - - - - - - - - - - - - - - - - - - - - - - - - - - - - - - - - - - -
\subsubsection{hasFeature}

Returns 1 if given feature is supported, 0 if not (or unknown).
The syntax is:

\begin{verbatim}
	hasFeature(feature)
\end{verbatim}

where {\tt feature} is one of the following \verb`varchar` tokens:

\begin{itemize}
  \item \verb`RE2`  For RE2 regular expression support in REX
\end{itemize}

This function is typically used in Vortex scripts to test if a feature
is supported with the current version of Texis, and if not, to work
around that fact if possible.  For example:

\begin{verbatim}
     <if hasFeature( "RE2" ) = 1>
       ... proceed with RE2 expressions ...
     <else>
       ... use REX instead ...
     </if>
\end{verbatim}

Note that in a Vortex script that does not support {\tt hasFeature()}
itself, such an \verb`<if>` statement will still compile and run,
but will be false (with an error message).

Added in version 7.06.1481662000 20161213.  Some feature tokens were
added in later versions.

% - - - - - - - - - - - - - - - - - - - - - - - - - - - - - - - - - - - - - -
\subsubsection{ifNull}

Substitute another value for NULL values.  Syntax:

\begin{verbatim}
   ifNull(testVal, replaceVal)
\end{verbatim}

  If \verb`testVal` is a SQL NULL value, then \verb`replaceVal` (cast
to the type of \verb`testVal`) is returned; otherwise \verb`testVal`
is returned.  This function can be used to ensure that NULL value(s)
in a column are replaced with a non-NULL value, if a non-NULL value
is required:

\begin{verbatim}
    SELECT ifNull(myColumn, 'Unknown') FROM myTable;
\end{verbatim}

Added in version 7.02.1405382000 20140714.  Note that SQL NULL is not
yet fully supported in Texis (including in tables).  See also
\verb`isNull`.

% - - - - - - - - - - - - - - - - - - - - - - - - - - - - - - - - - - - - - -
\subsubsection{isNull}

Tests a value, and returns a \verb`long` value of 1 if NULL, 0 if not.
Syntax:

\begin{verbatim}
   isNull(testVal)
\end{verbatim}

\begin{verbatim}
    SELECT isNull(myColumn) FROM myTable;
\end{verbatim}

Added in version 7.02.1405382000 20140714.  Note that SQL NULL is
not yet fully supported in Texis (including in tables).  Also note
that Texis \verb`isNull` behavior differs from some other SQL
implementations; see also \verb`ifNull`.

% - - - - - - - - - - - - - - - - - - - - - - - - - - - - - - - - - - - - - -
\subsubsection{xmlTreeQuickXPath}

Extracts information from an XML document.

\begin{verbatim}
	xmlTreeQuickXPath(string xmlRaw, string xpathQuery
        [, string[] xmlns)
\end{verbatim}

Parameters:
\begin{itemize}
\item \verb`xmlRaw` - the plain text of the xml document you want to
  extract information from
\item \verb`xpathQuery` - the XPath expression that identifies the
  nodes you want to extract the data from
\item \verb`xmlns` {\em(optional)} - an array of \verb`prefix=URI`
  namespaces to use in the XPath query
\end{itemize}

Returns:
\begin{itemize}
\item String values of the node from the XML document \verb`xmlRaw`
  that match \verb`xpathQuery`
\end{itemize}

\verb`xmlTreeQuickXPath` allows you to easily extract information from
an XML document in a one-shot function.  It is intended to be used in
SQL statements to extract specific information from a field that
contains XML data.

It is essentially a one statement version of the following:
\begin{verbatim}
    <$doc = (xmlTreeNewDocFromString($xmlRaw))>
    <$xpath = (xmlTreeNewXPath($doc))>
    <$nodes = (xmlTreeXPathExecute($xpathQuery))>
    <loop $nodes>
        <$ret = (xmlTreeGetAllContent($nodes))>
        <$content = $content $ret>
    </loop>
\end{verbatim}

\EXAMPLE
if the \verb`xmlData` field of a table has content like this:
\begin{verbatim}
<extraInfo>
    <price>8.99</price>
    <author>John Doe</author>
    <isbn>978-0-06-051280-4</isbn>
</extraInfo>
\end{verbatim}

Then the following SQL statement will match that row:
\begin{verbatim}
SELECT * from myTable where xmlTreeQuickXPath(data,
'/extraInfo/author') = 'John Doe'
\end{verbatim}


\chapter{Advanced Queries}{\label{Chp:AdvQuer}}

This chapter is divided into three sections.  The first one focuses on
using the join operation to retrieve data from multiple tables.  The
second section covers nesting of queries, also known as subqueries.
The final section introduces several advanced query techniques,
including self-joins, correlated subqueries, subqueries using the
EXISTS operator.
%EXISTS operator, and combining query results using the UNION operator.

\section{Retrieving Data From Multiple Tables}

All the queries looked at so far have been answered by accessing data
from one table.  Sometimes, however, answers to a query may require
data from two or more tables.

For example, for the Corporate Librarian to display a list of
contributing authors with their long form department name requires
data from the \verb`REPORT` table (author) and data from the \verb`DEPARTMENT` table
(department name).  Obtaining the data you need requires the ability
to combine two or more tables.  This process is commonly referred to
as ``{\em joining the tables}''.

Two or more tables can be combined to form a single table by using the
{\em join operation}.  The join operation is based on the premise that
there is a logical association between two tables based on a common
attribute that links the tables.  Therefore, there must be a common
column in each table for a join operation to be executed.  For
example, both the \verb`REPORT` table and the \verb`DEPARTMENT` table have the
department identification code in common.  Thus, they can be joined.

Joining two tables in Texis is accomplished by using a \verb`SELECT`
statement.  The general form of the \verb`SELECT` statement when a join
operation is involved is:
\begin{verbatim}
     SELECT   column-name1 [,column-name2] ...
     FROM     table-name1, table-name2
     WHERE    table-name1.column-name = table-name2.column-name ;
\end{verbatim}
The combination of table name with column name as stated in the \verb`WHERE`
clause describes the Join condition.

\subsubsection{Command Discussion}

\begin{enumerate}
\item A join operation pulls data from two or more tables listed in
the \verb`FROM` clause.  These tables represent the source of the data to be
joined.

\item The \verb`WHERE` clause specifies the relationship between the tables
to be joined.  This relationship represents the {\em join condition}.
Typically, the join condition expresses a relationship between rows
from each table that match on a common attribute.

\item When the tables to be joined have the same column name, the
column name is prefixed with a table name in order for Texis to know
from which table the column comes.  Texis uses the notation:
\begin{verbatim}
     table-name.column-name
\end{verbatim}
The table name in front of the column name is referred to as a {\em
qualifier}.

\item The common attributes in the join condition need not have the
same column name, but they should represent the same kind of
information.  For example, where the attribute representing names of
people submitting resumes was named \verb`RNAME` in table 1, and the
attribute for names of employees was named \verb`ENAME` in table 2, you could
still join the tables on the common character field by specifying:

\begin{verbatim}
     WHERE table-name1.RNAME = table-name2.ENAME
\end{verbatim}

While the above is true, it is still a good rule of thumb in database
design to give the same name to all columns referring to data of the
same type and meaning.  Columns which are designed to be a key, and
intended as the basis for joining tables would normally be given the
same name.

\item If a row from one of the tables never satisfies the join
condition, that row will not appear in the joined table.

\item The tables are joined together, and then Texis extracts the
data, or columns, listed in the \verb`SELECT` clause.

\item Although tables can be combined if you omit the \verb`WHERE` clause,
this would result in a table of all possible combinations of rows from
the tables in the \verb`FROM` clause.  This output is usually not intended,
nor meaningful, and can waste much computer processing time.
Therefore, be careful in forming queries that involve multiple tables.
\end{enumerate}

{\bf Example:}
The corporate librarian wants to distribute a list of authors who have
contributed reports to the corporate library, along with the name of
that author's department.  To fulfill this request, data from both the
\verb`REPORT` table (author) and the \verb`DEPARTMENT` table (department name) are
needed.

You would enter this statement:
\begin{verbatim}
     SELECT   AUTHOR, DNAME
     FROM     REPORT, DEPARTMENT
     WHERE    REPORT.DEPT = DEPARTMENT.DEPT ;
\end{verbatim}

{\bf Syntax Notes:}
\begin{itemize}
\item REPORT and DEPARTMENT indicate the tables to be joined.

\item The \verb`WHERE` clause statement defines the condition for the join.

\item The notation ``\verb`REPORT.`'' in ``\verb`REPORT.DEPT`'', and
``\verb`DEPARTMENT.`'' in ``\verb`DEPARTMENT.DEPT`'' are the
qualifiers which indicate from which table to find the column.
\end{itemize}

This statement will result in the following joined table:

%\begin{screen}
\begin{verbatim}
  AUTHOR                   DNAME
  Jackson, Herbert         Research and Development
  Sanchez, Carla           Product Marketing and Sales
  Price, Stella            Finance and Accounting
  Smith, Roberta           Research and Development
  Aster, John A.           Product Marketing and Sales
  Jackson, Herbert         Research and Development
  Barrington, Kyle         Management and Administration
\end{verbatim}
%\end{screen}

In this query, we are joining data from the REPORT and the DEPARTMENT
tables.  The common attribute in these two tables is the department
code.  The conditional expression:
\begin{verbatim}
     REPORT.DEPT = DEPARTMENT.DEPT
\end{verbatim}
is used to describe how the rows in the two tables are to be matched.
Each row of the joined table is the result of combining a row from the
\verb`REPORT` table and a row from the \verb`DEPARTMENT` table for each comparison
with matching codes.

To further illustrate how the join works, look at the rows in the
\verb`REPORT` table below where DEPT is ``MKT'':

%\begin{screen}
\begin{verbatim}
  TITLE                    AUTHOR           DEPT FILENAME
  Disappearing Ink         Jackson, Herbert RND  /data/rnd/ink.txt
> INK PROMOTIONAL CAMPAIGN SANCHEZ, CARLA   MKT  /data/MKT/PROMO.RPT
  Budget for 4Q 92         Price, Stella    FIN  /data/ad/4q.rpt
  Round Widgets            Smith, Roberta   RND  /data/rnd/widge.txt
> PAPERCLIPS               ASTER, JOHN A.   MKT  /data/MKT/CLIP.RPT
  Color Panorama           Jackson, Herbert RND  /data/rnd/color.txt
  Meeting Schedule         Barrington, Kyle MGT  /data/mgt/when.rpt
\end{verbatim}
%\end{screen}

Now look at the rows in the \verb`DEPARTMENT` table below where DEPT is
``MKT''.  These are matching rows since the department code (``MKT'')
is the same.

%\begin{screen}
\begin{verbatim}
  DEPT DNAME                               DHEAD      DIV  BUDGET
  MGT  Management and Administration       Barrington CORP 22000
  FIN  Finance and Accounting              Price      CORP 26000
  LEG  Corporate Legal Support             Thomas     CORP 28000
  SUP  Supplies and Procurement            Sweet      CORP 10500
  REC  Recruitment and Personnel           Harris     CORP 15000
  RND  Research and Development            Jones      PROD 27500
  MFG  Manufacturing                       Washington PROD 32000
  CSS  Customer Support and Service        Ferrer     PROD 11000
> MKT  PRODUCT MARKETING AND SALES         BROWN      PROD 25000
  ISM  Information Systems Management      Dedrich    INFO 22500
  LIB  Corporate Library                   Krinski    INFO 18500
  SPI  Strategic Planning and Intelligence Peters     INFO 28500
\end{verbatim}
%\end{screen}

The matching rows can be conceptualized as combining a row from the
\verb`REPORT` table with a matching row from the \verb`DEPARTMENT` table.  Below is
a sample of rows from both tables, matched on the department code
``MKT'':

%\begin{screen}
\begin{verbatim}
DEPT DNAME     DHEAD DIV  BUDGET TITLE      AUTHOR  FILENAME
MKT  Marketing Brown PROD 25000  Ink        Sanchez /data/mkt/promo.rpt
MKT  Marketing Brown PROD 25000  Paperclips Aster   /data/mkt/clip.rpt
\end{verbatim}
%\end{screen}

This operation is carried out for all matching rows; i.e., each row in
the \verb`REPORT` table is combined, or matched, with a row having the same
department code in the \verb`DEPARTMENT` table:

%\begin{screen}
\begin{verbatim}
DEPT DNAME      DHEAD DIV  BUDGET TITLE      AUTHOR  FILENAME
RND  Research   Jones PROD 27500  Ink        Jackson /data/rnd/ink.txt
MKT  Marketing  Brown PROD 25000  Ink Promo  Sanchez /data/mkt/promo.rpt
FIN  Finance    Price CORP 26000  Budget     Price   /data/ad/4q.rpt
RND  Research   Jones PROD 27500  Widgets    Smith   /data/rnd/widge.txt
MKT  Marketing  Brown PROD 25000  Paperclips Aster   /data/mkt/clip.rpt
RND  Research   Jones PROD 27500  Panorama   Jackson /data/rnd/color.txt
MGT  Management Barri CORP 22000  Schedule   Barring /data/mgt/when.rpt
\end{verbatim}
%\end{screen}

The columns requested in the \verb`SELECT` statement determine the final
output for the joined table:

%\begin{screen}
\begin{verbatim}
  AUTHOR                   DNAME
  Jackson, Herbert         Research and Development
  Sanchez, Carla           Product Marketing and Sales
  Price, Stella            Finance and Accounting
  Smith, Roberta           Research and Development
  Aster, John A.           Product Marketing and Sales
  Jackson, Herbert         Research and Development
  Barrington, Kyle         Management and Administration
\end{verbatim}
%\end{screen}

Observe that the joined table does not include any data on several
departments from the \verb`DEPARTMENT` table, where that department did not
produce any contributing authors as listed in the \verb`REPORT` table.  The
joined table includes only rows where a match has occurred between
rows in both tables.  If a row in either table does not match any row
in the other table, the row is not included in the joined table.

In addition, notice that the DEPT column is not included in the final
joined table.  Only two columns are included in the joined table since
just two columns are listed in the \verb`SELECT` clause, and DEPT is not one
of them.

The next example illustrates that conditions other than the join
condition can be used in the \verb`WHERE` clause.  It also shows that even
though the results come from a single table, the solution may require
that data from two or more tables be joined in the \verb`WHERE` clause.

{\bf Example:}
Assume that you cannot remember the department code for Research and
Development, but you want to know the titles of all reports submitted
from that department.

Enter this statement:
\begin{verbatim}
     SELECT   TITLE
     FROM     DEPARTMENT, REPORT
     WHERE    DNAME = 'RESEARCH AND DEVELOPMENT'
       AND    REPORT.DEPT = DEPARTMENT.DEPT ;
\end{verbatim}

{\bf Syntax Notes:}
\begin{itemize}
\item The tables to be joined are listed after \verb`FROM`.
\item The condition for the join operation is specified after {\tt AND} (as
part of \verb`WHERE`).
\end{itemize}

The results follow:

%\begin{screen}
\begin{verbatim}
  TITLE
  Innovations in Disappearing Ink
  Improvements in Round Widgets
  Ink Color Panorama
\end{verbatim}
%\end{screen}

Since you don't know Research and Development's department code, you
use the department name found in the \verb`DEPARTMENT` table in order to find
the row that stores Research and Development's code, which is `RND'.
Conceptually, visualize the join operation to occur as follows:

\begin{enumerate}
\item The conditional expression {\tt DNAME = 'RESEARCH AND DEVELOPMENT'}
references one row from the \verb`DEPARTMENT` table; i.e., the `RND' row.

\item Now that the RND code is known, this row in the \verb`DEPARTMENT` table
is joined with the rows in the \verb`REPORT` table that have DEPT = RND.
The joined table represents the titles of the reports submitted by
authors from the Research and Development department.
\end{enumerate}

As the next example illustrates, more than two tables can be joined
together.

{\bf Example:}
Provide a list of salaries paid to those people in the Product
Division who contributed reports to the Corporate Library.  The report
should include the author's name, department name, and annual salary.

You would enter this statement:
\begin{verbatim}
     SELECT   AUTHOR, DNAME, SALARY
     FROM     REPORT, DEPARTMENT, EMPLOYEE
     WHERE    DEPARTMENT.DIV = 'PROD'
       AND    REPORT.DEPT = DEPARTMENT.DEPT
       AND    REPORT.DEPT = EMPLOYEE.DEPT ;
\end{verbatim}

{\bf Syntax Notes:}
\begin{itemize}
\item The order of the joins in the \verb`WHERE` clause is not important.

\item The three tables to be joined are listed after \verb`FROM`.

\item The first {\tt AND} statement (in \verb`WHERE` clause) is the condition for
joining the REPORT and \verb`DEPARTMENT` tables.

\item The second {\tt AND} statement (in \verb`WHERE` clause) is the condition for
joining the REPORT and \verb`EMPLOYEE` tables.

\item While department code happens to be a column which all three
tables have in common, it would be possible to join two tables with a
common column, and the other two tables with a different common
column, such as \verb`ENAME` in the \verb`EMPLOYEE` table and AUTHOR in the REPORT
table.  (The latter would not be as efficient, nor as reliable, so
department name was chosen instead.)
\end{itemize}

The results would be:

%\begin{screen}
\begin{verbatim}
  AUTHOR               DNAME                           SALARY
  Jackson, Herbert     Research and Development        30000
  Sanchez, Carla       Product Marketing and Sales     35000
  Smith, Roberta       Research and Development        25000
  Aster, John A.       Product Marketing and Sales     32000
\end{verbatim}
%\end{screen}

In this example, data from three tables (REPORT, DEPARTMENT, \verb`EMPLOYEE`)
are joined together.

Conceptually, the \verb`DEPARTMENT` table references the rows that contain
PROD; this gives us the departments in the Product Division.  The
departments in the Product Division (RND, MFG, CSS, MKT) are matched
against the departments in the DEPT column of the \verb`REPORT` table.  The
tables are joined for the Research and Development (RND) and Product
Marketing and Sales (MKT) departments.  This yields an intermediate
table containing all the columns from both the DEPARTMENT and REPORT
tables for RND and MKT rows.

This intermediate table is joined with the \verb`EMPLOYEE` table, based on
the second join condition {\tt REPORT.DEPT = EMPLOYEE.DEPT} to form
a combination of columns from all 3 tables, for the matching rows.

Finally, the \verb`SELECT` clause indicates which columns in the intermediate
joined table that you want displayed.  Thus the author, department
name, and annual salary are shown as in the above example.

As a final point, the order in which you place the conditions in the
\verb`WHERE` clause does not affect the way Texis accesses the data.  Texis
contains an ``{\em optimizer}'' in its underlying software, which
chooses the best access path to the data based on factors such as
index availability, size of tables involved, number of unique values
in an indexed column, and other statistical information.  Thus, the
results would not be affected by writing the same query in the
following order:
\begin{verbatim}
     SELECT   AUTHOR, DNAME, SALARY
     FROM     REPORT, DEPARTMENT, EMPLOYEE
     WHERE    REPORT.DEPT = EMPLOYEE.DEPT
       AND    REPORT.DEPT = DEPARTMENT.DEPT
       AND    DEPARTMENT.DIV = 'PROD' ;
\end{verbatim}

\section{Nesting Queries}

At times you may wish to retrieve rows in one table based on
conditions in a related table.  For example, suppose Personnel needed
to call in any employees in the Information Division receiving only
partial benefits, to discuss options for upgrading to the full benefit
program.  To answer this query, you have to retrieve the names of all
departments in the Information Division, found in the DEPARTMENT
table, and then the employees with partial benefits in the Information
Division departments, found in the \verb`EMPLOYEE` table.

In other situations, you may want to formulate a query from one table
that required you to make two passes through the table in order to
obtain the desired results.  For example, you may want to retrieve a
list of staff members earning a salary higher than Jackson, but you
don't know Jackson's salary.  To answer this query, you first find
Jackson's salary; then you compare the salary of each staff member to
his.

One approach is to develop a {\em subquery}, which involves embedding
a query (\verb`SELECT`-\verb`\verb`FROM``-\verb`WHERE` block) within the \verb`WHERE` clause of another
query.  This is sometimes referred to as a ``{\em nested query}''.

The format of a nested query is:
\begin{verbatim}
     SELECT   column-name1 [,column-name2]
     FROM     table-name
     WHERE    column-name IN
       (SELECT   column-name
        FROM     table-name
        WHERE    search-condition) ;
\end{verbatim}

{\bf Syntax Notes:}
\begin{itemize}
\item The first \verb`SELECT`-\verb`\verb`FROM``-\verb`WHERE` block is the outer query.
\item The second \verb`SELECT`-\verb`\verb`FROM``-\verb`WHERE` block in parentheses is the
subquery.
\item The {\tt IN} operator is normally used if the inner query returns many
rows and one column.
\end{itemize}

\subsubsection{Command Discussion}

Here are some points concerning the use of nested queries:

\begin{enumerate}
\item The above statement contains two \verb`SELECT`-\verb`FROM`-\verb`WHERE` blocks.  The
portion in parentheses is called the subquery.  The subquery is
evaluated first; then the outer query is evaluated based on the result
of the subquery.  In effect, the nested query can be looked at as
being equivalent to:
\begin{verbatim}
     SELECT   column-name1 [,column-name2] ...
     FROM     table-name
     WHERE    column-name IN (set of values from the subquery) ;
\end{verbatim}
where the set of values is determined from the inner \verb`SELECT`-\verb`FROM`-\verb`WHERE`
block.

\item The {\tt IN} operator is used to link the outer query to the subquery
when the subquery returns a set of values (one or more).  Other
comparison operators, such as \verb`<`, \verb`>`, \verb`=`, etc., can
be used to link an outer query to a subquery when the subquery returns
a single value.

\item The subquery must have only a single column or expression in the
\verb`SELECT` clause, so that the resulting set of values can be passed back
to the next outer query for evaluation.

\item You are not limited to one subquery.  Though it isn't advised,
there could be as many as 16 levels of subqueries, with no fixed
limitation except limits of memory and disk-space on the machine in
use.  Any of the operators (\verb`IN`, \verb`=`, \verb`<`, \verb`>`,
etc.) can be used to link the subquery to the next higher level.

%\item You can use a subquery only if the values in the outer, or main,
%query come from a single table.  If the desired output comes from more than
%one table, a join operation is required.
\end{enumerate}

{\bf Example:}
List the names of all personnel in the Information Division by entering
this statement:

\begin{verbatim}
     SELECT   ENAME
     FROM     EMPLOYEE
     WHERE    DEPT IN
       (SELECT   DEPT
        FROM     DEPARTMENT
        WHERE    DIV = 'INFO') ;
\end{verbatim}
Parentheses are placed around the subquery, as shown below the outer
\verb`WHERE` clause.

The results are:

%\begin{screen}
\begin{verbatim}
  ENAME
  Chapman, Margaret
  Dedrich, Franz
  Krinski, Wanda
  Peters, Robert
\end{verbatim}
%\end{screen}

To understand how this expression retrieves its results, work from the
bottom up in evaluating the \verb`SELECT` statement.  In other words, the
subquery is evaluated first.  This results in a set of values that can
be used as the basis for the outer query.  The innermost \verb`SELECT` block
retrieves the following set of department codes, as departments
in the Information (`INFO') Division: ISM, LIB, SPI.

In the outermost \verb`SELECT` block, the {\tt IN} operator tests whether any
department code in the \verb`EMPLOYEE` table is contained in the set of
department codes values retrieved from the inner \verb`SELECT` block; i.e.,
ISM, LIB, or SPI.

In effect, the outer \verb`SELECT` block is equivalent to:
\begin{verbatim}
     SELECT   ENAME
     FROM     EMPLOYEE
     WHERE    DEPT IN ('ISM', 'LIB', 'SPI') ;
\end{verbatim}
where the values in parentheses are values from the subquery.

Thus, the employee names Chapman, Dedrich, Krinski and Peters are
retrieved.

Subqueries can be nested several levels deep within a query, as the
next example illustrates.

{\bf Example:}
Acme Industrial's ink sales are up, and management wishes to reward
everyone in the division(s) most responsible.  List the names of all
employees in any division whose personnel have contributed reports on
ink to the corporate library, along with their department and benefit
level.

Use this statement:
\begin{verbatim}
     SELECT   ENAME, DEPT, BENEFITS
     FROM     EMPLOYEE
     WHERE    DEPT IN
       (SELECT   DEPT
        FROM     DEPARTMENT
        WHERE    DIV IN
          (SELECT   DIV
           FROM     DEPARTMENT
           WHERE    DEPT IN
             (SELECT   DEPT
              FROM     REPORT
              WHERE    TITLE  LIKE 'ink') ) ) ;
\end{verbatim}
{\tt IN} is used for each subquery since in each case it is possible to
retrieve several values.  You could use `\verb`=`' instead where you
knew only one value would be retrieved; e.g. where you wanted only the
division with the greatest number of reports rather than all divisions
contributing reports.

Results of the above nested query are:

%\begin{screen}
\begin{verbatim}
  ENAME                DEPT   BENEFITS
  Aster, John A.       MKT    FULL
  Jackson, Herbert     RND    FULL
  Sanchez, Carla       MKT    FULL
  Smith, Roberta       MKT    PART
  Jones, David         RND    FULL
  Washington, G.       MFG    FULL
  Ferrer, Miguel       CSS    FULL
  Brown, Penelope      MKT    FULL
\end{verbatim}
%\end{screen}

Again, remember that a nested query is evaluated from the bottom up;
i.e., from the innermost query to the outermost query.  First, a
text search is done ({\tt TITLE LIKE 'INK'}) of report titles from the
\verb`REPORT` table.  Two such titles are located: ``Disappearing Ink'' by
Herbert Jackson from Research and Development (RND), and ``Ink
Promotional Campaign'' by Carla Sanchez from Product Marketing and
Sales (MKT).  Thus the results of the innermost query produces a list
of two department codes: RND and MKT.

Once the departments are known, a search is done of the DEPARTMENT
table, to locate the division or divisions to which these departments
belong.  Both departments belong to the Product Division (PROD); thus
the results of the next subquery produces one item: PROD.

A second pass is made through the same table, DEPARTMENT, to find all
departments which belong to the Product Division.  This search
produces a list of four Product Division departments: MKT, RND, MFG,
and CSS, adding Manufacturing as well as Customer Support and Service
to the list.

This list is passed to the outermost query so that the \verb`EMPLOYEE` table
may be searched for all employees in those departments.  The final
listing is retrieved, as above.

Here is another example specifically designed to illustrate the use of
a subquery making two passes through the same table to find the
desired results.

{\bf Example:}
List the names of employees who have salaries greater than that of
Herbert Jackson.  Assume you do not know Jackson's salary.

Enter this statement:
\begin{verbatim}
     SELECT   ENAME, SALARY
     FROM     EMPLOYEE
     WHERE    SALARY >
       (SELECT   SALARY
        FROM     EMPLOYEE
        WHERE    ENAME = 'Jackson, Herbert') ;
\end{verbatim}
The compare operator \verb`>` can be used (as could \verb`=` and other
compare operators) where a single value only will be returned from the
subquery.

Using the sample information in our \verb`EMPLOYEE` table, the results are as
follows:

%\begin{screen}
\begin{verbatim}
  ENAME              SALARY
  Aster, John A.     32000
  Barrington, Kyle   45000
  Price Stella       42000
  Sanchez, Carla     35000
\end{verbatim}
%\end{screen}

The subquery searches the \verb`EMPLOYEE` table and returns the value
\verb`30000`, the salary listed for Herbert Jackson.  Then the outer
\verb`SELECT` block searches the \verb`EMPLOYEE` table again to retrieve all
employees with \verb`SALARY > 30000`.  Thus the above employees with
higher salaries are retrieved.

\section{Forming Complex Queries}

The situations covered in this section are more technical than most
end users have need to conceptualize.  However, a system administrator
may require such complex query structures to efficiently obtain the
desired results.

\subsection{Joining a Table to Itself}

In some situations, you may find it necessary to join a table to
itself, as though you were joining two separate tables.  This is
referred to as a {\em self join}.  In the self join, the combined
result consists of two rows from the same table.

For example, suppose that within the \verb`EMPLOYEE` table, personnel are
assigned a RANK of ``STAFF'', ``DHEAD'', and so on.  To obtain a list
of employees that includes employee name and the name of his or her
department head requires the use of a self join.

To join a table to itself, the table name appears twice in the \verb`FROM`
clause.  To distinguish between the appearance of the same table name,
a temporary name, called an {\em alias} or a {\em correlation name},
is assigned to each mention of the table name in the \verb`FROM` clause.  The
form of the \verb`FROM` clause with an alias is:
\begin{verbatim}
     FROM   table-name [alias1] [,table-name [alias2] ] ...
\end{verbatim}

To help clarify the meaning of the query, the alias can be used as a
qualifier, in the same way that the table name serves as a qualifier,
in \verb`SELECT` and \verb`WHERE` clauses.

{\bf Example:}
As part of an analysis of Acme's salary structure, you want to
identify the names of any regular staff who are earning more than a
department head.

Enter this query:
\begin{verbatim}
     SELECT   STAFF.ENAME, STAFF.SALARY
     FROM     EMPLOYEE DHEAD, EMPLOYEE STAFF
     WHERE    DHEAD.RANK = 'DHEAD' AND STAFF.RANK = 'STAFF'
       AND    STAFF.SALARY > DHEAD.SALARY ;
\end{verbatim}
Using a sampling of information from the \verb`EMPLOYEE` table, we would
get these results:

%\begin{screen}
\begin{verbatim}
  ENAME               SALARY

  Sanchez, Carla      35000
\end{verbatim}
%\end{screen}

In this query, the \verb`EMPLOYEE` table, using the alias feature, is treated
as two separate tables named \verb`DHEAD` and \verb`STAFF`, as shown here (in
shortened form):

%\begin{screen}
\begin{verbatim}
  DHEAD Table                         STAFF Table
  EID ENAME   DEPT RANK  BEN  SALARY  EID ENAME   DEPT RANK  BEN  SALARY
  101 Aster   MKT  STAFF FULL 32000   101 Aster   MKT  STAFF FULL 32000
  109 Brown   MKT  DHEAD FULL 37500   109 Brown   MKT  DHEAD FULL 37500
  103 Chapman LIB  STAFF PART 22000   103 Chapman LIB  STAFF PART 22000
  110 Krinski LIB  DHEAD FULL 32500   110 Krinski LIB  DHEAD FULL 32500
  106 Sanchez MKT  STAFF FULL 35000   106 Sanchez MKT  STAFF FULL 35000
\end{verbatim}
%\end{screen}

Now the join operation can be made use of, as if there were two
separate tables, evaluated as follows.

First, using the following compound condition:
\begin{verbatim}
     DHEAD.RANK = 'DHEAD' AND STAFF.RANK = 'STAFF'
\end{verbatim}
each department head record (Brown, Krinski) in the \verb`DHEAD` table is
joined with each staff record (Aster, Chapman, Sanchez) from the \verb`STAFF`
table to form the following intermediate result:

%\begin{screen}
\begin{verbatim}
  DHEAD Table                         STAFF Table
  EID ENAME   DEPT RANK  BEN  SALARY  EID ENAME   DEPT RANK  BEN  SALARY
  109 Brown   MKT  DHEAD FULL 37500   101 Aster   MKT  STAFF FULL 32000
  109 Brown   MKT  DHEAD FULL 37500   103 Chapman LIB  STAFF PART 22000
  109 Brown   MKT  DHEAD FULL 37500   106 Sanchez MKT  STAFF FULL 35000
  110 Krinski LIB  DHEAD FULL 32500   101 Aster   MKT  STAFF FULL 32000
  110 Krinski LIB  DHEAD FULL 32500   103 Chapman LIB  STAFF PART 22000
  110 Krinski LIB  DHEAD FULL 32500   106 Sanchez MKT  STAFF FULL 35000
\end{verbatim}
%\end{screen}

Notice that every department head row is combined with each staff
record.

Next, using the condition:
\begin{verbatim}
       STAFF.SALARY > DHEAD.SALARY
\end{verbatim}
for each row of the joined table, the salary value from the \verb`STAFF`
portion is compared with the corresponding salary value from the \verb`DHEAD`
portion.  If \verb`STAFF.SALARY` is greater than \verb`DHEAD.SALARY`, then
\verb`STAFF.ENAME` and \verb`STAFF.SALARY` are retrieved in the final table.

The only row in the joined table satisfying this condition of staff
salary being greater than department head salary is the last one,
where Carla Sanchez from Marketing, at a salary of \$35,000, is
earning more than Wanda Krinski, as department head for the Corporate
Library, at a salary of \$32,500.

\subsection{Correlated Subqueries}

All the previous examples of subqueries evaluated the innermost query
completely before moving to the next level of the query. Some queries,
however, cannot be completely evaluated before the outer, or main,
query is evaluated.  Instead, the search condition of a subquery
depends on a value in each row of the table named in the outer query.
Therefore, the subquery is evaluated repeatedly, once for each row
selected from the outer table.  This type of subquery is referred to
as a {\em correlated subquery}.

{\bf Example:}
Retrieve the name, department, and salary, of any employee whose
salary is above average for his or her department.

Enter this query:
\begin{verbatim}
     SELECT   POSSIBLE.ENAME, POSSIBLE.DEPT, POSSIBLE.SALARY
     FROM     EMPLOYEE POSSIBLE
     WHERE    SALARY >
       (SELECT   AVG (SALARY)
        FROM     EMPLOYEE AVERAGE
        WHERE    POSSIBLE.DEPT = AVERAGE.DEPT) ;
\end{verbatim}

{\bf Syntax Notes:}
\begin{itemize}
\item The outer \verb`SELECT`-\verb`FROM`-\verb`WHERE` block is the main query.

\item The inner \verb`SELECT`-\verb`FROM`-\verb`WHERE` block in parentheses is the
subquery.

\item POSSIBLE (following \verb`EMPLOYEE` in the outer query) and AVERAGE
(following \verb`EMPLOYEE` in the subquery) are alias table names for the
\verb`EMPLOYEE` table, so that the information may evaluated as though it
comes from two different tables.
\end{itemize}

It results in:

%\begin{screen}
\begin{verbatim}
  ENAME               DEPT   SALARY
  Krinski, Wanda      LIB    32500
  Brown, Penelope     MKT    37500
  Sanchez, Carla      MKT    35000
  Jones, David        RND    37500
\end{verbatim}
%\end{screen}

The column AVERAGE.DEPT correlates with POSSIBLE.DEPT in the main, or
outer, query.  In other words, the average salary for a department is
calculated in the subquery using the department of each employee from
the table in the main query (POSSIBLE).  The subquery computes the
average salary for this department and then compares it with a row in
the \verb`POSSIBLE` table.  If the salary in the \verb`POSSIBLE` table is greater
than the average salary for the department, then that employee's name,
department, and salary are displayed.

The process of the correlated subquery works in the following manner.
The department of the first row in POSSIBLE is used in the subquery to
compute an average salary.  Let's take Krinksi's row, whose department
is the corporate library (LIB).  In effect, the subquery is:
\begin{verbatim}
     SELECT   AVG (SALARY)
     FROM     EMPLOYEE AVERAGE
     WHERE    'LIB' = AVERAGE.DEPT ;
\end{verbatim}

LIB is the value from the first row in POSSIBLE, as alias for
\verb`EMPLOYEE`.

This pass through the subquery results in a value of \$27,250, the
average salary for the LIB dept.  In the outer query, Krinski's salary
of \$32,500 is compared with the average salary for LIB; since it is
greater, Krinski's name is displayed.

This process continues; next, Aster's row in POSSIBLE is evaluated,
where MKT is the department.  This time the subquery is evaluated as
follows:
\begin{verbatim}
     SELECT   AVG (SALARY)
     FROM     EMPLOYEE AVERAGE
     WHERE    'MKT' = AVERAGE.DEPT ;
\end{verbatim}

The results of this pass through the subquery is an average salary of
\$34,833 for MKT, the Product Marketing and Sales Department.  Since
Aster has a salary of \$32,000, a figure lower than the average, this
record is not displayed.

Every department in POSSIBLE is examined in a similar manner before
this subquery is completed.

\subsection{Subquery Using EXISTS}

There may be situations in which you are interested in retrieving
records where there exists at least one row that satisfies a
particular condition.  For example, the resume records stored in the
\verb`RESUME` table may include some individuals who are already employed at
Acme Industrial and so are entered in the \verb`EMPLOYEE` table.  If you
wanted to know which employees were seeking new jobs at the present
time, an existence test using the keyword \verb`EXISTS` can be used to answer
such a query.

This type of query is developed with a subquery.  The \verb`WHERE` clause of
the outer query is used to test the existence of rows that result from
a subquery.  The form of the \verb`WHERE` clause that is linked to the
subquery is:
\begin{verbatim}
     WHERE [NOT] EXISTS (subquery)
\end{verbatim}

This clause is satisfied if there is at least one row that would be
returned by the subquery.  If so, the subquery does not return any
values; it just sets an indicator value to true.  On the other hand,
if no elements satisfy the condition, or the set is empty, the
indicator value is false.

The subquery should return a single column only.

{\bf Example:}
Retrieve a list of Acme employees who have submitted resumes to
personnel for a different job placement.

Enter this query:
\begin{verbatim}
     SELECT   EID, ENAME
     FROM     EMPLOYEE
     WHERE    EXISTS
       (SELECT RNAME
        FROM   RESUME
        WHERE  EMPLOYEE.ENAME = RESUME.RNAME) ;
\end{verbatim}

The results are:

%\begin{screen}
\begin{verbatim}
  EID  ENAME
  107  Smith, Roberta
  113  Ferrer, Miguel
\end{verbatim}
%\end{screen}

In this query, the subquery cannot be evaluated completely before the
outer query is evaluated.  Instead, we have a correlated subquery.
For each row in \verb`EMPLOYEE`, a join of \verb`EMPLOYEE` and \verb`RESUME` tables is
performed (even though \verb`RESUME` is the only table that appears in the
subquery's \verb`FROM` clause) to determine if there is a resume name in
\verb`RESUME` that matches a name in \verb`EMPLOYEE`.

For example, for the first row in the \verb`EMPLOYEE` table ({\tt ENAME = 'Smith, Roberta'}) the subquery evaluates as ``true'' if at least one row in
the \verb`RESUME` table has {\tt RNAME = 'Smith, Roberta'}; otherwise, the
expression evaluates as ``false''.  Since there is a row in \verb`RESUME`
with {\tt RNAME = 'Smith, Roberta'}, the expression is true and Roberta
Smith's row is displayed.  Each row in \verb`EMPLOYEE` is evaluated in a
similar manner.

The following is an example of the interim join (in shortened form)
between the \verb`EMPLOYEE` and \verb`RESUME` Tables, for the above names which
satisfied the search requirement by appearing in both tables:

%\begin{screen}
\begin{verbatim}
  EMPLOYEE Table            RESUME Table
  EID ENAME          DEPT   RES_ID  RNAME           JOB       EXISTS
                                                              (subquery)
  107 Smith, Roberta RND    R406    Smith, Roberta  Engineer  TRUE
  113 Ferrer, Miguel CSS    R425    Ferrer, Miguel  Analyst   TRUE
\end{verbatim}
%\end{screen}

%% would this be possible? look down to next %%
%% John says this may be supported in future ... %%
%% exact syntax? might need string concatenation with + to work %%

Note in this example that there is no key ID field connecting the two
tables; therefore the character field for name is being used to join
the two tables, which might have been entered differently and
therefore is not an altogether reliable join.  This indicates that
such a search is an unusual rather than a usual action.

Such a search would be a good opportunity to use a Metamorph \verb`LIKE`
qualifier rather than a straight join on a column as above, where
\verb`ENAME` must match exactly \verb`RNAME`. A slightly more thorough way of
searching for names appearing in both tables which were not
necessarily intended to be matched exactly would use Metamorph's
approximate pattern matcher, indicated by a percent sign \verb`%`
preceding the name.  For example:

\begin{verbatim}
     SELECT   EID, ENAME
     FROM     EMPLOYEE
     WHERE    EXISTS
       (SELECT *
        FROM   RESUME
        WHERE  EMPLOYEE.ENAME LIKE '%' + RESUME.RNAME) ;
\end{verbatim}

In this example a name approximately like each \verb`RNAME` in the \verb`RESUME`
table would be compared to each \verb`ENAME` in the \verb`EMPLOYEE` table,
increasing the likelihood of a match. (String concatenation is used to
append the name found in the resume table to the percent sign
(\verb`%`) which signals the approximate pattern matcher XPM.)

%% wouldn't it be cool to do something like this? (above) %%
%% we can leave it in, no harm if doesn't work right away %%
%% shows off the potential for what could be done %%

Often, a query is formed to test if no rows are returned in a
subquery.  In this case, the following form of the existence test is
used:

\begin{verbatim}
     WHERE   NOT EXISTS (subquery)
\end{verbatim}

{\bf Example:}
List any authors of reports submitted to the online corporate library
who are not current employees of Acme Industrial.  To find this out we
would need to know which authors listed in the \verb`REPORT` table are not
entered as employees in the \verb`EMPLOYEE` table.

Use this query:
\begin{verbatim}
     SELECT   AUTHOR
     FROM     REPORT
     WHERE    NOT EXISTS
       (SELECT *
        FROM   EMPLOYEE
        WHERE  EMPLOYEE.ENAME = REPORT.AUTHOR) ;
\end{verbatim}
which would likely result in a list of former employees such as:

%\begin{screen}
\begin{verbatim}
  AUTHOR
  Acme, John Jacob Snr.
  Barrington, Cedrick II.
  Rockefeller, George G.
\end{verbatim}
%\end{screen}

Again, we have an example of a correlated subquery.  Below is
illustrated (in shortened form) how each row which satisfied the
search requirement above in REPORT is evaluated with the records in
\verb`EMPLOYEE` to determine which authors are not (or are no longer) Acme
employees.

%\begin{screen}
\begin{verbatim}
  REPORT Table                               EMPLOYEE Table  EXISTS
  TITLE              AUTHOR
  Company Origin     Acme, John Jacob Snr.                   FALSE
  Management Art     Barrington, Cedrick II.                 FALSE
  Financial Control  Rockefeller, George G.                  FALSE
\end{verbatim}
%\end{screen}

In this example each of the above authors from the REPORT Table are
tested for existence in the \verb`EMPLOYEE` Table.  When they are not found
to exist there it returns a value of FALSE.  Since the query condition
in the \verb`WHERE` clause is that it {\tt NOT EXISTS}, this changes the false
value to true, and these rows are displayed.

For each of the queries shown in this section, there are probably
several ways to obtain the same kind of result.  Some correlated
subqueries can also be expressed as joins.  These examples are given
not so much as the only definitive way to state these search requests,
but more so as to give a model for what kinds of things are possible.
%
%\subsection{Combining Query Results Using the UNION Operator}
%
%The UNION operator is used to merge the results of two or more queries
%into a single result.  Each query is connected by a UNION operator to
%produce the result.  In the process of merging the queries, duplicate
%rows are removed from the answer.  The general form of the operator
%is:
%\begin{verbatim}
%     SELECT   statement
%     UNION
%     SELECT   statement
%
%     [UNION           ]
%     [SELECT STATEMENT]
%     ...
%
%     [ORDER BY  integer [DESC] ] ;
%\end{verbatim}
%
%\subsubsection{Command Discussion}
%
%\begin{enumerate}
%\item The system first executes each query separately; the results
%from each query are merged with the results of the next query, until
%all the queries have been merged.  Any duplicate rows that occur are
%removed from the results.
%
%\item In order for queries to be merged, they must be {\em union
%compatible}; i.e., the data types and lengths of the corresponding
%items in each \verb`SELECT` clause must be identical.
%
%\item If ordering of the final, or merged, output is desired, the
%{\tt ORDER BY} clause is placed after the last query only and applies to the
%entire result.  Since the column name in each query may vary, the sort
%key must be specified as a number, not a column name.  (Refer {\em
%Sorting Your Results} in Chapter~\ref{chp:Quer}).
%\end{enumerate}
%
%{\bf Example:}
%Research and Development is hosting a Future Technology day.  The
%department head decides to invite all R\&D personnel as well as anyone
%who has submitted technology related reports to the online library.
%To create the list he must create a list of all staff in the R\&D
%department from the \verb`EMPLOYEE` table, and combine it with a list of
%technology related authors from the \verb`REPORT` table.
%
%This statement will create such a list:
%\begin{verbatim}
%     SELECT     ENAME
%     FROM       EMPLOYEE
%     WHERE      DEPT = 'RND'
%
%     UNION
%
%     SELECT     DISTINCT AUTHOR
%     FROM       REPORT
%     WHERE      TITLE LIKE '~technology'
%
%     ORDER BY   1 ;
%\end{verbatim}
%
%{\bf Syntax Notes:}
%\begin{itemize}
%\item The first \verb`SELECT`-\verb`FROM`-\verb`WHERE` statement is a query to find
%personnel who are in the RND (Research and Development) department.
%
%\item The second \verb`SELECT`-\verb`FROM`-\verb`WHERE` statement is a query to find titles
%of reports which contain words conceptually similar to the word
%technology.  The tilde `\verb`~`' preceding ``technology'' expands the
%\verb`LIKE` specification to include a set of words similar to
%``technology'', referred to as a concept set.  Metamorph query
%language is being used, as is covered in detail in
%Chapter~\ref{Chp:MMLike}.
%
%\item Authors are selected for matching titles.
%
%\item As author and employee names are both character fields entered
%last name first, they are {\em union compatible} and can be merged
%into a list.
%
%\item The `\verb`1`' following {\tt ORDER BY} indicates the output is
%ordered by the first item in the \verb`SELECT` clause.
%\end{itemize}
%
%Results of the query are:
%
%%\begin{screen}
%\begin{verbatim}
%  ENAME
%  Barrington, Kyle
%  Jackson, Herbert
%  Jones, David
%  Sanchez, Carla
%  Smith, Roberta
%\end{verbatim}
%%\end{screen}
%
%Texis executes the top query and finds the RND personnel in the
%\verb`EMPLOYEE` table.  Separately, authors are located who wrote technology
%related reports in the \verb`REPORT` table.  The results are merged, and any
%duplicates are removed.  Thus the above 5 names are the final output.
%
%For example, Carla Sanchez was located because she wrote two reports
%entitled, ``Marketing Mechanical Objects'' and ``Industrial Sales
%Techniques''.  These were a conceptual match on ``technology''.  The
%DISTINCT operator in the second \verb`SELECT` clause removes the duplications
%of this report, producing Carla's name only once in that list.
%
%When the author list is merged with the R\&D list, there may be
%additional duplications for selected authors who are also R\&D
%personnel.  The UNION operator removes any duplications between the
%two lists produced by the two select statements.

\section{Virtual Fields}

To improve the capabilities of Texis, especially with regard to Metamorph
searching multiple fields we implemented the concept of virtual fields.
This allows you to treat the concatentation of any number of text fields
as a single field.  As a single field you can create an index on the fields,
search the fields, and perform any other operation allowable on a field.
Concatenation is represented by the \verb`\` operator.  For example:

\begin{verbatim}
	SELECT TITLE
	FROM   PAPERS
	WHERE  ABSTRACT\BODY LIKE 'ink coloration';
\end{verbatim}

would display the title of all papers whose abstract or body matched the
query \verb`ink coloration`.  By itself this is helpful, but the real
change is that you could create an index on this virtual field as follows:

\begin{verbatim}
	CREATE METAMORPH INDEX IXMMABSBOD ON PAPERS(ABSTRACT\BODY);
\end{verbatim}

which could greatly improve the performance of this query.  You can create
any type of index on a virtual field, although it is important to remember
that for non Metamorph indices the sum of the fields should not exceed
2048 bytes.  If your keys are text fields this method allows you to create
a unique index across several fields.

\section{Column Aliasing}

Similar to the abililty to alias the name of a table in the from clause
it is also possible to alias column names.  An alias can have up to 35
characters (case is significant).

This has several possible uses.
One is simply to produce a more informative report, for example:

\begin{verbatim}
    SELECT COUNT(*) EMPLOYEES
    FROM   EMPLOYEES;
\end{verbatim}

might produce the following output

\begin{screen}
\begin{verbatim}
    EMPLOYEES
       42
\end{verbatim}
\end{screen}

Another important use is when using the create table as select statement.
This allows you to rename a field, or to name a calculated field.

\begin{verbatim}
    CREATE TABLE INVENTORY AS
    SELECT PROD_ID, SALES * 3 MAX_LEVEL, SALES MIN_LEVEL
    FROM   SALES;
\end{verbatim}

Would create a new table with three fields, PROD\_ID, MAX\_LEVEL,
and MIN\_LEVEL.


This chapter has illustrated various complex query constructions
possible with Texis, and has touched on the use of Metamorph in
conjunction with standard SQL queries.  The next chapter will explain
Metamorph query language in depth and give examples of its use in
locating relevant narrative text.
