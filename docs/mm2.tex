\chapter{Hits, Intersections and Sets}

\section{Definition and Background}

A ``{\em hit}'' is the term used to refer to a portion of text
Metamorph retrieves in response to a specified query or search
request.  We use the term ``hit'' to mean the program has found or
``hit'' on something it was looking for.  Hits can have varying
degrees of relevance or validity, based on the number of intersections
of concept matched within that delimited block of text.

An ``{\em intersection}'' is defined as an occurrence of at least one
member of each of two sets of strings located within a section of text
as demarcated by a specified beginning and ending delimiter.

A ``{\em set}'' is the group of possible strings a pattern matcher
will look for, as specified by the search item.  A set can be a list
of words and word forms, a range of characters or quantities, or some
other class of possible matches based on which pattern matcher
Metamorph uses to process that item.

Intersection quantity refers to the number of unions of sets existing
within the specified delimiters.  The maximum number of intersections
possible for any given query will be the maximum number of designated
sets minus one.

The number of intersections present in any given hit directly
determines how relevant that response will be to the stated query.
Therefore, the designated intersection quantity can modulate the
amount of abstraction or inference you wish to allow away from the
original concept as stated in your query.

Because intersection quantity can be adjusted by the user to determine
how tightly correlated the retrieved responses must be, the term
``correlation threshold level'' has in the past been used
interchangeably with ``intersection quantity'' in earlier writings and
versions of Metamorph.  However, intersection quantity is a more
precise lexical concept, and is therefore the preferred and more
accurate term.

Intersections, whether calculated automatically by Metamorph or set by
the user, signify the number of intersections of specified sets being
searched for, to be matched anywhere within the delimited block of
text; this could be within a sentence, paragraph or page or some other
designated amount of text.

Metamorph provides a template of variables which go into a search,
all of which can be stored so that one can analyze exactly
what produced the search results and adjust them as desired.  The
intersection quantity variable is the most important in terms of
designating how closely you wish to have the retrieved responses
(hits) correlated to your search query.


\section{Basic Operation}

A default Metamorph search will calculate the maximum number of
intersections possible in a hit based on your query.

Metamorph picks out the important elements of your question and
creates a series of sets of words and/or patterns; counting the number
of sets, it does a search for any hit which contains all those
elements.  If you ask:  ``\verb`Was there a power struggle?`'' the
program will look for hits containing 1 intersection of the 2 sets
``\verb`power`'' and ``\verb`struggle`''.  In this case the program is
executing the search at intersection quantity 1, as it is looking for
text containing 1 union of 2 sets.

When you first enter the query, Metamorph determines how many sets of
concepts it will be searching for.  Upon execution, the program will
attempt to find hits containing the maximum number of intersections of
something from each of those sets, within the defined delimiters of a
sentence.

A ``sentence'' is specified as the block of text from one sentence
type delimiter \verb`[.?!]` to the next sentence type delimiter
\verb`[.?!]`.  Any sentence found which contains that maximum number
of intersections will be considered a valid hit, and can be made
available for viewing, in the context of the full text of the file in
which it is located.


\section{Intersections and Search Logic}

A common and simple search would be to enter a few search items on the
query line, in hopes of locating a sentence which matches the idea of
what you have entered.  You want a sentence containing some
correlation to the ideas you have entered.  For example, enter on the
query line:

\begin{verbatim}
     property evaluation
\end{verbatim}

This should find a relevant response containing a connection to both
entered keywords.  The user would be happy to locate the sentence:

\begin{quote}
      The broker came to {\bf assess} the current market value of our {\bf house}.
\end{quote}

Here ``\verb`assess`'' is linked to ``\verb`evaluation`'', and
``\verb`house`'' is linked to ``\verb`property`'', within the bounds
of one sentence.  Therefore it is a hit, and can be brought up for
viewing.

Requiring an occurrence of 2 search items is referred to as 1
intersection.  The occurrence of 3 search items is 2 intersections.
The default search logic is to look for the maximum number of
intersections possible of all entered search items.  If you were
searching large quantities of text, you could add several qualifiers,
as follows:

\begin{verbatim}
     property tax market assessment
\end{verbatim}

The maximum number of intersections would be looked for; i.e., an
occurrence of ``\verb`property`'', ``\verb`tax`'', ``\verb`market`'',
and ``\verb`assessment`'', anywhere inside the text unit.  This can be
thought of as ``and'' logic.

To override the default ``and'' logic, you would enter the desired
number of intersections with `\verb`@#`':  as ``\verb`@0`'' (``at zero
intersections''), ``\verb`@1`'' (``at one intersection''),
``\verb`@2`'' (``at two intersections''), and so on.

To get ``or'' logic you would enter ``\verb`@0`'' on the query line
with your search items.  In this example:

\begin{verbatim}
     @0 property tax
\end{verbatim}

``\verb`@0`'' (read as:  ``at zero intersections'') means that zero
intersections are required; therefore you are specifying that you want
an occurrence of either ``\verb`property`'' or ``\verb`tax`'' anywhere
inside of the delimited text, or sentence.

To obtain different permutations of logic you can specify a number of
intersections greater than zero but less than an intersection of all
specified items.  For example:

\begin{verbatim}
     @1 property tax value
\end{verbatim}

requires one intersection of any two of the three specified items.
Therefore, the text unit will be retrieved if it contains an
occurrence of ``\verb`property`'' and ``\verb`tax`'',
``\verb`property`'' and ``\verb`value`'', or ``\verb`tax`'' and
``\verb`value`''.

Any search item not marked with a `\verb`+`' or a minus `\verb`-`' is
assumed to be equally weighted.  Each unmarked item could be preceded
with `\verb`=`' but it is not required on the query line as it is
understood.  Intersection quantities ``\verb`@#`'' apply to these
equally weighted sets not otherwise marked with `\verb`+`' or
`\verb`-`'.

Mandatory inclusion `\verb`+`' and mandatory exclusion `\verb`-`'
logic can be assigned, and can be used with combinatorial logic.  For
example, you might enter this query:

\begin{verbatim}
     +tax -federal @1 market assessment value property assets w/page
\end{verbatim}

This query means that within any page (``\verb`w/page`''), you must
include an occurrence of ``\verb`tax`'' (designated with the plus sign
`\verb`+`'), but must exclude the hit if it contains any reference to
``\verb`federal`'' (designated with the minus sign `\verb`-`').  The
``\verb`@1`'' means that you also want 1 intersection of any 2 of the
other 5 equally weighted (unmarked with `\verb`+`' or `\verb`-`')
query items:  ``\verb`market`'', ``\verb`assessment`'',
``\verb`value`'', ``\verb`property`'', and ``\verb`assets`''.

\section{Intersection Quantity Specification Further Explained}

If you want all possibilities you can find, you will find very
adequate answers to your search requests usually by setting the
quantity to 1 or 2. If you don't get answers at that quantity, drop it
lower.

With intersections at 0, the program will retrieve hits containing 0
set intersections; i.e., it will locate any occurrence of any set
element.  The denotation of 0 indicates no intersection of specified
sets is required.  In other words, you want Metamorph to locate any
hit in which there is any equivalence to any of the set elements in
your question.  This is comparable to an ``or'' search.

With intersections at 1, the program will retrieve hits containing an
intersection of any 2 specified set elements; for example, an
intersection of ``\verb`life`'' and ``\verb`death`''.  With
intersections at 2, the program will retrieve hits containing any 2
intersections of any 3 specified set elements; e.g., ``\verb`life`''
and ``\verb`death`'' plus ``\verb`death`'' and ``\verb`infinity`''.
With the intersection variable at 3, the program might locate an
occurrence of ``\verb`life`'' and ``\verb`death`'', ``\verb`death`''
and ``\verb`infinity`'', plus ``\verb`infinity`'' and
``\verb`money`''.  The higher the number of intersections being
sought, the more precise the retrieved response must be.

The lower you set the intersection quantity, the more hits are likely
to be found in relation to the question asked; but many of them are
likely to be quite abstract.  The higher you set the intersection
quantity, the fewer hits are likely to be found in relation to the
question asked; but those hits are likely to be more intelligently and
precisely related to the search query, as they require a higher number
of matching sets.

A smaller intersection quantity will also retrieve any larger
intersection quantity responses as a matter of course.  However, the
program only searches for the number of intersections of elements it
is directed to search for, so it will only highlight the requested
number of sets.

Conversely, the program cannot locate hits containing more
intersections than are possible.  If you have set intersections to 6,
but you enter a question containing only 3 set elements, realize that
one pass through the data will find no answers, as you have designated
more intersections than are possible to find.


\section{How `+' and `-' Logic Affects Intersection Settings}

The assignment of intersection quantity refers to the number of
intersections of sets being sought in the search.  This presumes that
all sets are equal to each other in value.  A search for:

\begin{verbatim}
      life  love  death
     =life =love =death  (same as above)
\end{verbatim}

presumes that each set has equal value; if something from each set is
found, it would reflect an intersection quantity of 2, as it contains
2 intersections of ``equal'' value.  In Metamorph terms, this is
referred to as ``set logic''.

When you assign a `\verb`+`' (``must include'') logic operator to a
set, this means that you must include something from that designated
set in any hit that is found.  When you assign a `\verb`-`' (``not'')
logic operator to a set, this means that you must exclude any hit
which contains any element from that designated set.

Examples of these might be:

\begin{verbatim}
     +spirit =life =love =death
               or
     -alien =life =love =death
\end{verbatim}

It doesn't matter whether the equal sign `\verb`=`' is entered or not.
It also doesn't matter what sequence these operand sets are entered
in, or how many of each.  The only rule is you cannot enter only
``not'' sets.

In the above two examples, the maximum intersection quantity is still
2. The ``must include'' (\verb`+`) set (``and logic'') and the ``must
exclude'' (\verb`-`) set (``not logic'') are outside the calculation
of maximum intersections possible, as both are special cases.  In a
search where you designate something like this:

\begin{verbatim}
     war peace -Reagan +diplomacy
\end{verbatim}

the maximum intersection quantity is only 1, referring to an
intersection of the ``\verb`war`'' and the ``\verb`peace`'' sets.
Therefore, if you wish to definitely find something related to
``\verb`diplomacy`'', but to exclude any references related to
``\verb`Reagan`'', and you want to intersect what is found with either
war or peace, set intersections to 0:

\begin{verbatim}
     @0 war peace -Reagan +diplomacy
\end{verbatim}

The 0 intersection setting applies to either the ``\verb`war`'' set or
the ``\verb`peace`'' set (the unmarked equally weighted sets), as both
``\verb`Reagan`'' and ``\verb`diplomacy`'' are special cases and not
included in the intersection quantity calculation.


\section{Modulating Set Size}

Metamorph can draw upon a vast built-in intelligence comprised of
equivalence sets, as contained in a 2MB+ collection of 250,000+ word
associations.  The user has the option of how much of this ``mind'' he
wants to draw upon at any given time, by regulating to what degree he
wishes to make use of the built-in connective structure in this
Equivalence File.

When enabled via \verb`keepeqvs`/\verb`Synonyms` (for the entire
query) or the \verb`~` operator (for a particular query term), each
query term is expanded by pulling from the Equivalence File the set of
known associations listed with each entry.  This lets you draw upon
the full power of all the associations in all of its equivalence sets.
Still, you have the option of quantitatively limiting or expanding
this set expansion, or selectively editing their content.

\subsection{Limiting Sets to Roots Only}

There may be times when you want to limit set size to root words only
from your search query.  This is the default in \verb`tsql`, Vortex
and the Search Appliance.  If using the MM3 API, or if equivalences
have been enabled with \verb`keepeqvs` (Vortex/\verb`tsql`) or
\verb`Synonyms` (Search Appliance), then equivalences can be excluded
for a particular word -- while still retaining morpheme processing -- by
preceding the word with a tilde (\verb`~`).  This should be done where
you wish to look for intersections of concepts while holding
abstraction to a minimum, and you do not want any automatic expansion
of those words into word lists.  If equivalences are not currently
enabled via \verb`keepeqvs`/\verb`Synonyms`, then the \verb`~` operator
has the opposite effect: it {\em enables} equivalences for that term
only.  Thus, it always toggles the pre-set behavior for its word.

  Regardless of settings, you can also give explicit equivalences
directly in the query, instead of the thesaurus-provided ones, using
parentheses; see p.~\pageref{`MetamorphParenSet'}.

Where you wish to use morpheme processing on root words only,
restricting concept expansion completely, turn off Equivalence File
access altogether by setting the global APICP flag ``\verb`keepeqvs`''
off.  Where ``\verb`keepeqvs`'' is set to off, the tilde (\verb`~`) is
not required (indeed, it would re-enable concept expansion).

Restricting set expansion is useful for proper nouns which you do not
want expanded into abstract concepts (e.g., \verb`President Bush`),
technical or legal terminology, or simply any precise discrete search.
Selectively cut off the set expansion by designating a tilde
(\verb`~`) preceding the word you are looking for.

Using REX syntax by preceding the word with a forward slash
(\verb`/`), can further delimit the pattern you are looking for,
though in a different manner.  Note however, that using non-SPM/PPM
pattern matchers such as REX may slow the query, as Metamorph indexes
cannot be used for such terms.

To check this out for yourself, in an application where Metamorph hit
markup has been set up, compare the results of the following queries
(assuming Vortex or \verb`tsql` defaults):

\begin{verbatim}
     Query 1:    President  Bush
     Query 2:   ~President ~Bush
     Query 3:   "President Bush"
     Query 4:   /President /Bush
     Query 5:   /\RPresident /\RBush
\end{verbatim}

\begin{description}

\item[Query 1] \verb`President Bush`:  In the first example you
would get any hit containing an occurrence of the word
``\verb`President`'' and the word ``\verb`Bush`'', including other
related word forms (suffixes etc.).  So you would get a hit like ``{\em {\bf
President} {\bf Bush} came to tea.}''; as well as ``{\em {\bf Bush}
attended a conference of corporation {\bf presidents}}.''  There are
no equivalences added to the ``\verb`President`'' set, or the
``\verb`Bush`'' set.

\item[Query 2] \verb`~President ~Bush`:  In the second example you have
elected to keep the full set size, so you would obtain references to
"\verb`President`" and "\verb`Bush`" while also allowing for other
abstractions.  Since the word ``\verb`chief`'' is associated with
``\verb`president`'', and the word ``\verb`jungle`'' is associated
with ``\verb`bush`'', you would retrieve a sentence such as, ``{\em We
met the {\bf chief} at his home deep in the Amazon {\bf jungle}.}''

\item[Query 3] \verb`"President Bush"`:  The third example calls for
``\verb`President`'' and ``\verb`Bush`'' as a two word phrase by
putting it in quotes, so that it will be treated as one set rather
than as two.  It has no equivalences, because the phrase
``\verb`President Bush`'' has no equivalences known by the Equivalence
File; you could add equivalences to that phrase if you wished by
editing the User Equiv File.  While you would retrieve the hit
``{\em {\bf President Bush} came to tea.}'', you would exclude the hit
``{\em {\bf Bush} attended a conference of corporation {\bf
presidents}.}'' You would get a hit like ``{\em We elected a new {\bf
president Bush}.}''

\item[Query 4] \verb`/President /Bush`:  In the fourth example you
have limited the root word set in a different way.  Signalling
REX with the forward slash `\verb`/`' means that you will use
REX to accomplish a string search on whatever comes after the
`\verb`/`'.  Therefore, you can find ``{\em Our {\bf president}'s name
is {\bf Bush}.}'' and ``{\em We planted those {\bf bush}es near the
{\bf President}'s house.}'' This search gets similar yet different
results than Example 1. Look at exactly what is highlighted by the
Metamorph hit markup to see the difference in what was located.

\item[Query 5] \verb`/\RPresident /\RBush`:  In the fifth example
there is better reason to use REX syntax, so that you can limit
the set even further by specifying proper nouns only.  The designation
`\verb`\R`' means to ``respect case'', and would retrieve the sentence
``{\em {\bf President} {\bf Bush} came to tea.}'', but would rule out
the sentence ``{\em {\bf Bush} attended a conference of corporation
{\bf president}s.}'' It would also rule out the hit ``{\em We elected
a new {\bf president} {\bf Bush}.}'', and ``{\em We planted those {\bf
bush}es near the {\bf President}'s house.}''

NOTE:  In the previous example, the ``respect case'' designation
(\verb`\R`) must follow a forward slash (\verb`/`) which indicates
that REX syntax follows.  Remember that words
in a query are not case sensitive unless you so designate,
using REX.  (See Chapter on REX syntax.)
\end{description}

\subsection{Expanding Sets by Use of See References}

You can exponentially increase the denseness of connectivity in the
equivalence file by turning on the option called ``See References''.
This greatly increases set size overall.

The concept of a ``See'' reference is the same as in a dictionary.
When you are looking up the word ``\verb`cat`'', you'll get a
description and a few definitions which try to outline exactly what
the concept of ``cat'' is all about.  Then it may say at the end,
``\verb`See also pet`''.

With See References set to on, the equivalence list associated with
the word ``\verb`cat`'' would be automatically expanded to include all
equivalences associated with the word ``\verb`pet`''.  Not all
equivalences have ``\verb`See also`'' notations, but with See
References on, all equivalences associated with any ``\verb`See
also`'' root word will be included as part of the original.  With See
References off, only the root word and its equivalences will be
included in that word set.

In the Equivalence File, See References are denoted with the at sign
(\verb`@`).  In this way, additional second level references can be
added or deleted from a root word entry.

For some of the more common words like ``\verb`take`'' and
``\verb`life`'' and such, allowing See References can make a set
surprisingly huge, so you should think twice before operating in this
mode.  It is really intended to expand the range of opportunity to the
user, so that there is more to pick and choose from when editing the
content of the word lists.

If you are using the default equivalence sets it is generally better
to run with the ``\verb`See`'' option ``\verb`off`''.  You would not
normally want to be doing searches where the word lists are too large;
we suggest limiting the set size to well under 100 in any case,
whether you are running with See References either on or off.

In a theoretical sense, given no limitations, using endless levels of
See References could eventually create a circle which covered the
whole spectrum of life, and led right back to the word one had started
with.  Therefore Metamorph has intentionally built-in limitations on
set size to prevent this connectivity of concept from becoming
redundant or out of hand.  Firstly, See References are restricted to
only one level of reference.  Secondly, if a word set approaches
around 256, Metamorph will automatically truncate it in the search
process, cutting off any words outside the numerical limit.

To make use of the See References option, set the corresponding
APICP flag ``\verb`see`''.

\section{Modulating Set Content}

The ability to dictate precise content of all word sets passed to the
Metamorph Search Engine is left open to the user on an optional basis.
We have tried to build into the program enough intelligence through
its Equivalence File so that you would never be required to pay any
attention to the lists of equivalences if you didn't want to.

Even if Metamorph has no known associations with a word it will still
process it with no difficulty, using its built-in morpheme processing
capability.  You'll find that more often than not you are not required
to teach Metamorph anything about your search vocabulary to get
satisfactory results, even when using technical terminology.

When you want to control exactly what associations are made with any
or all the words or expressions in your searches, you can do so by
editing the equivalence set associated with any word already known by
the Equivalence File, or by creating associations for a new or created
word not yet known.  This is covered in detail under the Chapter on
{\em Thesaurus Customization}.


\section{Modulating Hit Size by Adjusting Delimiters}

\subsection{Discussion}

Delimiters can be defined as the beginning and ending patterns in the
text which define the text unit inside of which your query items will
be located.  Concept proximity is adjusted through delimiters.

If you look for the words ``\verb`bear`'' and ``\verb`woods`'' within
a sentence, the result will be a tight match to your query.  Looking
for ``\verb`bear`'' and ``\verb`woods`'' inside a paragraph is less
restrictive.  Requiring only that ``\verb`bear`'' and ``\verb`woods`''
occur inside the same section might or might not have much to do with
what you are looking for.  Knowing that, you would probably add more
qualifying search items when searching by section; e.g.,
``\verb`bear`'' ``\verb`woods`'' ``\verb`winchester`''
``\verb`rifle`''.  In short, the relevance of your search results will
differ greatly based on the type of text unit by which you are
searching.

Beginning and ending delimiters are defined by repeating patterns in
the text, technically known as regular expressions.  A sentence can be
identified by its ending punctuation (period, question mark, or
exclamation point); a paragraph is often identified by a few new lines
in a row; a page is often identified by the presence of a page
formatting character.

In a Metamorph search, the entered query concepts will be searched for
within the bounds of two such expressions, called delimiters.  These
regular expressions are defined with REX syntax for Regular
Expressions.  If you know how to write a regular expression using
REX you can enter delimiters of your own design.  In lieu of
that you can rely upon the defaults that have been provided.

Since the most common types of search are by line, sentence,
paragraph, page, or whole document, these expressions have been
written into the Metamorph Query Language so that you can signal their
use dynamically from within any query, using English.  The expression
``\verb`w/delim`'' means ``within a \ldots'' where 4 characters
following ``\verb`w/`'' are an abbreviation for a commonly delimited
unit of text.  If you want to search by paragraph, you would add
``\verb`w/para`'' as an item on the query line.

For example:

\begin{verbatim}
     power struggle w/para
\end{verbatim}

Such a search will look within any paragraph for an occurrence of the
concept ``\verb`power`'' and the concept ``\verb`struggle`''.

You can designate a quantitative proximity, using the same syntax, by
stating how many characters you want before and after the located
search items.  For example:  \begin{verbatim} power struggle w/150
\end{verbatim} Such a search will look within a 300 character range
(150 before, 150 after the first item located) for an occurrence of
the concept \verb`power` and the concept \verb`struggle`.  This is
useful for text which doesn't follow any particular text pattern, such
as source code.

You can also write your own expressions, useful for section heads
and/or tails.  To enter delimiters of your own design, create a
REX expression first which works, and enter it following the
``\verb`w/`''.  For example:

\begin{verbatim}
     power struggle w/\n\RSECTION
\end{verbatim}

This search uses the pattern ``\verb`SECTION`'' where it begins on a
new line and is in Caps only, as both beginning and ending delimiters.
Thus an occurrence of the set ``\verb`power`'' and the set
``\verb`struggle`'' need only occur within the same section as so
demarcated, which might be several pages long.

You can figure out such useful section delimiter expressions when
setting up an application, and use the corresponding APICP flag
\verb`sdexp` (start delimiter expression) and \verb`edexp` (end
delimiter expression) in a Vortex script to make use of them.

The following examples of queries would dictate the proximity of the
search items ``\verb`power`'' and ``\verb`struggle`'', by specifying
the desired delimiters.  Noting the highlighted search items, you
might try these searches on the demo text files to see the difference
in what is retrieved.

\begin{verbatim}
  power struggle w/line   within a line (1 new line)
  power struggle w/sent   within a sentence (ending punctuation)
  power struggle w/para   within a paragraph (a new line + some space)
  power struggle w/page   within a page (where format character exists)
  power struggle w/all    within whole document (all of the record)

  power struggle w/500    within a window of 500 characters forward
                          and backwards

  power struggle w/$$$    within user designed expression for a section;
                          where what follows the slash `/' is assumed to
                          be a REX expression.  (In this case, the
                          expression means 3 new lines in a row.)
\end{verbatim}

More often than not the beginning and ending delimiters are the same.
Therefore if you do not specify an ending delimiter (as in the above
example), it will be assumed that the one specified is to be used for
both.  If two expressions are specified, the first will be beginning,
the second will be ending.  Specifying both would be required most
frequently where special types of messages or sections are used which
follow a prescribed format.

Another factor to consider is whether you want the expression defining
the text unit to be included inside that text unit or not.  For
example, the ending delimiter for a sentence (ending punctuation from
the located sentence) obviously belongs with the hit.  However, the
beginning delimiter (ending punctuation from the previous sentence) is
really the end of the last sentence, and therefore should be excluded.

Inclusion or exclusion of beginning and ending delimiters with the hit
has been thought out for the defaults provided.
However, if you are designing your own beginning and ending
expressions, you may wish to so specify.

\subsection{Delimiter Syntax Summary}

\begin{verbatim}
        w/{abbreviation}
     or
        w/{number}
     or
        w/{expression}
     or
        W/{expression}
     or
        w/{expression} W/{expression}
     or
        W/{expression} w/{expression}
     or
        w/{expression} w/{expression}
     or
        W/{expression} W/{expression}
\end{verbatim}

\subsection{Rules of Delimiter Syntax}

\begin{itemize}

\item The above can be anywhere on a query line, and is interpreted as
``within \{the following delimiters\}''.

\item Accepted built-in abbreviations following the slash `\verb`/`'
are:

\begin{tabbing}
\verb`[^\digit\upper][.?!][\space'"]`xxx \= Meaning \kill
ABBREV       \> MEANING \\
\verb`line`  \> within a line \\
\verb`sent`  \> within a sentence \\
\verb`para`  \> within a paragraph \\
\verb`page`  \> within a page \\
\verb`all`   \> within a whole record
\end{tabbing}

\begin{tabbing}
\verb`[^\digit\upper][.?!][\space'"]`xxx \= Meaning \kill
REX EXPRESSION                           \> MEANING   \\
\verb`$`                                 \> 1 new line      \\
\verb`[^\digit\upper][.?!][\space'"]`    \> not a digit or upper case letter, then \\
                                         \> a period, question, or exclamation point, then \\
                                         \> any space character, single or double quote  \\
\verb`\x0a=\space+`                      \> a new line + some space  \\
\verb`\x0c`                              \> form feed for printer output \\
\verb`""`                                \> both start and end delims are empty
\end{tabbing}

\item A number following a slash `\verb`/`' means the number of
characters before and after the first search item found.  Therefore
``\verb`w/250`'' means ``within a proximity of 250 characters''.  When
the first occurrence of a valid search item is found, a window of 250
characters in either direction will be used to determine if it is a
valid hit.  The implied REX expression is:  ``\verb`.{250}`''
meaning ``250 of any character''.

\item If what follows the slash `\verb`/`' is not recognized as a
built-in, it is assumed that what follows is a REX expression.

\item If one expression only is present, it will be used for both
beginning and ending delimiter.  If two expressions are present, the
first is the beginning delimiter, the second the ending delimiter.
The exception is within-$N$ (e.g. ``\verb`w/250`''), which always
specifies both start and end delimiters, overriding any preceding
``\verb`w/`''.

\item The use of a small `\verb`w`' means to exclude the delimiters
from the hit.

\item The use of a capital `\verb`W`' means to include the delimiters
in the hit.

\item Designate small `\verb`w`' and capital `\verb`W`' to exclude
beginning delimiter, and include ending delimiter, or vice versa
Note that for within-$N$ queries (e.g. ``\verb`w/250`''), the
``delimiter'' is effectively always included in the hit, regardless
of the case of the \verb`w`.

\item If the same expression is to be used, the expression need not be
repeated.  Example:  ``\verb`w/[.?!] W/`'' means to use an ending
punctuation character as both beginning and end delimiter, but to
exclude the beginning delimiter from the hit, and include the end
delimiter in the hit.
\end{itemize}
