% ----------------------------------------------------------------------------
\NAME{{\tt tsql} - Texis Interactive SQL}

\SYNOPSIS
\begin{verbatim}
tsql [-a command] [-c [-w width]] [-l rows] [-hqm?] [-d database]
[-u username] [-p password] [-i file] ["SQL statement"]
\end{verbatim}

\DESCRIPTION

{\tt Tsql} is the main program provided for interactive use of a Texis
database.  You should either be in the database directory, or else
specify it on the command line as the {\tt -d <database>} option.

If a query is present on the command line than {\tt tsql} will execute
that statement, and display the results on stdout (the screen).  If no
query is present then queries will be accepted on stdin (the
keyboard).  Queries on stdin must be terminated by a semicolon.  To
exit from {\tt tsql} you should produce EOF on it's stdin.  On Unix
systems this is usually done with Control-D, and on Windows with
Control-Z.

{\tt Tsql} also provides facilities for doing many administrative
tasks.

{\bf Options}

To aid in the use of {\tt tsql}  from scripts or as parts of a chain
of commands there are many options available.  The options are
\begin{description}
\item[-h]       Don't display the column headers.
\item[-q]       Don't display the SQL prompt and copyright statement.
\item[-c]       Change the format to one field per line.
\item[-w width] Make the field headings width characters long.  0 means
                use the longest heading's width.
\item[-l rows]  Limit output to the number of rows specified.
\item[-u username]  Login as username.  If the -p option is not specified
                    then you will be prompted for a password.  The user
                    must have been previously added to the system.  This
                    forces usage of the Texis permission scheme.  If this
                    is not specified then the name defaults to PUBLIC.
                    See Chapter~\ref{Chp:Sec} for more
                    information.
\item[-p password]  Use password to log in.
\item[-P password]  \_SYSTEM password to log in for admin.
\item[-d database]  Specify the location of the database.
\item[-m]           Create the named database. See creatdb.
\item[-i file]      Read SQL commands from file instead of stdin.
                    Commands read this way will echo to stdout, whereas
                    commands read from input redirection would not.
\item[-r]           Read the default profile file.
\item[-R profile]   Read the specified profile file.
\item[-f]           Specify a format.  One or two characters can follow with
                    the following meanings.
                    \begin{description}
                    \item[t]  same as \verb|-c| option
                    \item[c]  default behaviour
                    \item[any other character] is a field separator character.
                    This can be followed by q to suppress the quotes around
		    the fields.
                    To get quoted comma separated values you would use
                    \verb|-f ,|
                    \end{description}
\item[-?]           Print a command summary.
\end{description}

% ----------------------------------------------------------------------------
