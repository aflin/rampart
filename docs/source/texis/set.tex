
% $Id$
%
%  Use:        For:               PDF render:   Online render:
%  ----        ----               ----------    --------------
%  \verb`...`  Keywords           Fixed-font    <code> fixed-font red-on-pink
%  {\tt ... }  User input         Fixed-font    <tt> fixed-font
%  {\bf ...}   Settings/sections  Bold          <b> bold

% ============================================================================
\section{Server Properties}
\label{ServerProperties}

  There are a number of properties that are settable in the SQL
Engine.  They do not need to be changed unless the behavior of the
system must be modified.  The properties are set using the following
SQL syntax:

\begin{verbatim}
    SET property = value;
\end{verbatim}

The \verb`value` can be one of three types depending on the property:
numeric, boolean or string.  A boolean value is either an integer--0
is false, anything else is true--or one of the following strings:
``\verb`on`'', ``\verb`off`'', ``\verb`true`'', ``\verb`false`'',
``\verb`yes`'' or ``\verb`no`''.

The settings are grouped as follows:

% WTF properties:
% alequivs
% alintersects
% allinear
% allowramtableblob
% alnot
% alphrase
% alpostproc
% alwild
% alwithin
% btreeoptimizeoff
% btreeoptimizeon
% dedupmultiitemresults
% defsuffrm
% denymode
% eastpositive
% enablesubsetintersect
% eqprefix
% exactphrase
% findselloopcheck
% fldmathverbosemaxvaluesize
% inc_edexp
% inc_sdexp
% indexmmapbufsize
% indexreadbufsize
% indexwritebufsize
% kdbfoptimizeoff
% kdbfoptimizeon
% kdbfverify
% keepeqvs
% likerpercent
% listindextmp
% locksleepdecrement
% locksleepincrement
% max_index_text
% maxrows
% max_rows
% nolocking
% olddelim
% qmaxsets
% qmaxsetwords
% qmaxterms
% qmaxwords
% qminprelen
% qminwordlen
% querysettings
% reqedelim
% reqsdelim
% showplan
% strlstrelopvarcharpromoteviacreate
% tablereadbufsize
% traceddcache
% traceddcachetablename
% traceidx
% traceindex
% tracelocksdatabase
% tracelockstable
% tracerppm
% ueqprefix
% usestringcomparemodeforstrlst
% varchar2strlstsep
% verbose
% verifysingle

% WTF optimize settings:
% sortedvarfls
% pointersintostrlst
% linearrankindexexps
% likeandnoinv
% likeand
% indexdatagroupby
% fastmmindexupdate
% indexonly
% auxdatalen
% copy

% ----------------------------------------------------------------------------
\subsection{Search and optimization parameters}

These settings affect the way that Texis will process the search.  They
include settings which change the meaning of the search, as well as how
the search is performed.

\begin{description}
\item[defaultlike]

  Defines which sort of search should occur when a \verb`like` or
\verb`contains` operator is in the query.  The default setting of
``\verb`like`'' behaves in the normal manner.  Other settings that can
be set are ``\verb`like3`'', ``\verb`likep`'', ``\verb`liker`'' and
``\verb`matches`''.  In each case the \verb`like` operator will act as
if the specified operator had been used instead.

\item[matchmode]
  Changes the behavior of the \verb`matches` clause.  The default
behavior is to use underscore and percent as the single and
multi-character character wildcards.  Setting \verb`matchmode` to 1 will
change the wildcards to question-mark and asterisk.

\item[predopttype]
  The Texis engine can reorder the \verb`where` clause in an attempt
to make it faster to evaluate.  There are a number of ways this can be
done; the \verb`predopttpye` property controls the way it reorders.
The values are 0 to not reorder, 1 to evaluate \verb`and` first, 2 to
evaluate \verb`or` first.  The default is 0.

\item[ignorecase]
{\bf Note: } Deprecated; see \verb`stringcomparemode` setting which
supercedes this.
Setting \verb`ignorecase` to true will cause string comparisons
(equals, sorting, etc.) in the SQL engine to ignore case,
e.g. ``\verb`A`'' will compare identical to ``\verb`a`''.  (This is
distinct from {\em text} comparisons, e.g.  the \verb`LIKE` operator,
which ignore case by default and are unaffected by \verb`ignorecase`.)
{\bf Note:} This setting will also affect any indices that are built;
the value set at index creation will be saved with the index and used
whenever that index is used.  {\bf Note:} In versions prior to version
5.01.1208300000 20080415, the value of \verb`ignorecase` {\em must} be
explicitly set the same when an index is created, when it or its table
is updated and when it is used in a search, or incorrect results
and/or corrupt indexes may occur.  In later versions, this is not
necessary; the saved-at-index-creation value will automatically be
used.
In version 6 and later, this setting toggles the \verb`ignorecase`
flag of the \verb`stringcomparemode` setting, which supercedes it.

\item[textsearchmode]
  Sets the APICP \verb`textsearchmode` property; see Vortex manual
for details and important caveats.  Added in version 6.

\item[stringcomparemode]
  Sets the APICP \verb`stringcomparemode` property; see Vortex manual
for details and important caveats.  Added in version 6.

\item[tracemetamorph]
  Sets the \verb`tracemetamorph` debug property; see Vortex manual
for details.  Added in version 7.00.1375225000 20130730.

\item[tracerowfields]
  Sets the \verb`tracerowfields` debug property; see Vortex manual
for details.  Added in version 7.02.1406754000 20140730.

\item[tracekdbf]
  Sets the \verb`tracekdbf` debug property; see Vortex manual for details.

\item[tracekdbffile]
  Sets the \verb`tracekdbffile` debug property; see Vortex manual for details.

\item[kdbfiostats]
  Sets the \verb`kdbfiostats` debug property; see Vortex manual for details.

\item[btreecachesize]
  Index pages are cached in memory while the index is used.  The size
of the memory cache can be adjusted to improve performance.  The
default is 20, which means that 20 index pages can be cached.  This
can be increased to allow more pages to be cached in memory.  This
will only help performance if the pages will be accessed in random
order, more than 20 will be accessed, and the same page is likely to
be accessed at different times.  This is most likely to occur in a
join, when a large number of keys are looked up in the index.
Increasing the size of the cache when not needed is likely to hurt
performance, due to the extra overhead of managing a larger cache.
The cache size should not be decreased below the default of 20, to
allow room for all pages which might need to be accessed at the same
time.

\item[ramrows]
  When ordering large result sets, the data is initially ordered in
memory, but if more than \verb`ramrows` records are being ordered the
disk will be used to conserve memory.  This does slow down performance
however.  The default is 10000 rows.  Setting \verb`ramrows` to 0 will
keep the data in memory.

\item[ramlimit] \verb`ramlimit` is an alternative to \verb`ramrows`.
Instead of limiting the number of records, the number of bytes of data
in memory is capped.  By default it is 0, which is unlimited.  If both
\verb`ramlimit` and \verb`ramrows` are set then the first limit to be
met will trigger the use of disk.

\item[bubble] Normally Texis will bubble results up from the index to
the user.  That is a matching record will be found in the index, returned
to the user, then the next record found in the index, and so forth till
the end of the query.  This normally generates the first results as quickly
as possible.  By setting \verb`bubble` to 0 the entire set of matching
record handles will be read from the index first, and then each record
processed from this list.

\item[optimize,nooptimize] Enable or disable optimizations.  The argument
should be a comma separated list of optimizations that you want to enable
or disable.  The available optimizations are:
\begin{description}
\item[join]  Optimize join table order.  The default is enabled.
When enabled Texis will arrange the order of the tables in the \verb`FROM`
clause to improve the performance of the join.  This can be disabled if
you believe that Texis is optimizing incorrectly.  If it is disabled then
Texis will process the tables in the left to right order, with the first
table specified being the driving table.  Added in version 02.06.927235551.
\item[compoundindex]  Allow the use of compound indexes to resolve searches.
For example if you create an index on table (field1, field2), and then
search where field1 = value and field2 = value, it will use the index to
resolve both portions of this.  When disabled it would only look for field1
in the index.
Added in version 02.06.929026214.
\item[countstar]  Use any regular index to determine the number of records
in the table.  If disabled Texis will read each record in the table to
count them.
Added in version 02.06.929026214.
\item[minimallocking]  Controls whether the table will be locked when doing
reads of records pointed to by the index used for the query.  This is enabled
by default, which means that read locks will not be used.  This is the optimal
setting for databases which are mostly read, with few writes and small records.
Added in version 03.00
\item[groupby]  This setting is enabled by default and will cause the data
to be read only once to perform a group by operation.  The query should
produce indentical results whether this is enabled or disabled, with the
performance being the only difference.
Added in version 03.00
\item[faststats]  When enabled, which is the default, and when the
appopriate indexes exist Texis will try and resolve aggregate functions
directly from the index that was used to perform the \verb`WHERE` clause.    
Added in version 03.00
\item[readlock]  When enabled, which is the default, Texis will use readlocks
more efficiently if there are records that are scanned, but don't match the
query.  Texis will hold the read lock until a matching record is found,
rather than getting and releasing a read lock for every record read.  If you
are suffering from lock contention problems, with writes waiting, then this
can be disabled, which will allow more opportunity for the write locks to
be granted.  This is not normally suggested, as the work required to grant
and release the locks would typically  negate the benefit.
Added in version 03.00
\item[analyze]  When enabled, which is the default, Texis will analyze the
query for which fields are needed.  This can allow for more efficient query
processing in most cases.  If you are executing a lot of different SQL
statements that are not helped by the analysis you can disable this.
Added in version 03.00
\item[skipahead]  When enabled, which is the default, Texis will skipahead
as efficiently as possible, typically used with the SKIP parameter in
Vortex.  If disabled Texis will perform full processing on each skipped
record, and discard the record.
Added in version 03.00
\item[likewithnots]  When enabled (default), \verb`LIKE`/\verb`LIKEP`-type
searches with NOT sets (negated terms) are optimized for speed.
Added in version 4.02.1041535107 Jan  2 2003.

\item[shortcuts] When enabled (default), a fully-indexed
  \verb`LIKE`/\verb`LIKEIN` clause \verb`OR`ed with another
  fully-indexed \verb`LIKE`/\verb`LIKEIN` should not cause an
  unnecessary post-process for the \verb`LIKE`s (and entire query).
  Added in version 4.03.1061229000 20030818 as \verb`optimization18`;
  in version 7.06.1475000000 20160927, alias \verb`shortcuts` added.

\item[likehandled] When enabled (default), a fully-indexed
  \verb`LIKE`/\verb`LIKEIN` clause \verb`OR`ed with another
  fully-indexed non-\verb`LIKE`/\verb`LIKEIN` clause should not cause
  an unnecessary post-process for the \verb`LIKE` (and entire query).

  Also, linear and post-process \verb`LIKE`/\verb`LIKEIN` operations
  caused not by the Metamorph query itself, but by the presence of
  another \verb`OR`ed/\verb`AND`ed clause, do not check
  \verb`allinear` nor \verb`alpostproc` when this optimization is
  disabled (i.e. they will perform the linear or post-process
  regardless of settings, silently).  E.g. fully-indexed \verb`LIKE`
  \verb`OR`ed with linear clause, or two fully-indexed \verb`LIKE`s
  \verb`AND`ed (where the first's results are under
  \verb`maxlinearrows`), could cause linear search or post-processing,
  respectively, of an otherwise fully-indexable Metamorph query.

  Added in version 7.06.1475014000 20160927.

\item[indexbatchbuild] When enabled, indexes are built as a batch,
i.e. the table is read-locked continuously.  When disabled (the
default), the table is read-locked intermittently if possible
(e.g. Metamorph index), allowing table modifications to proceed even
during index creation.  A continuous read lock allows greater read
buffering of the table, possibly increasing index build speed
(especially on platforms with slow large-file \verb`lseek` behavior),
at the expense of delaying table updates until after the index is
nearly built, which may be quite some time.  Note that non-Metamorph
indexes are {\em always} built with a continuous read lock --
regardless of this setting -- due to the nature of the index.  Added
in version 5.01.1177455498 20070424.

\item[indexdataonlycheckpredicates] When enabled (the default), allows
  the index-data-only optimization\footnote{The index-data-only
    optimization allows Texis to not only use the index to resolve the
    {\tt WHERE} clause, but also the {\tt SELECT} clause in certain
    circumstances, potentially avoiding a read of the table altogether
    and speeding up results.  One of the prerequisites for this
    optimization is that the {\tt SELECT} clause only refer to columns
    available in the index.} to proceed even if the {\tt SELECT}
  columns are renamed or altered in expressions.  Previously, the
  columns had to be selected as-is with no renaming or expressions.
  Added in version 7.00.1369437000 20130524.

\item[indexvirtualfields] When enabled (the default), attempts to
  reduce memory usage when indexing virtual fields (especially with
  large rows) by freeing certain buffers when no longer needed.
  Currently only applies to Metamorph and Metamorph inverted indexes.
  Added in version 6.00.1322890000 20111203.

\end{description}
Example:  \verb`set nooptimize='minimallocking'`

\item[options,nooptions] Enable or disable certain options.  The
  argument should be a comma separated list of options to enable or
  disable.  All options are off by default.  The available options are:

  \begin{description}
    \item[triggers] When on, {\em disable} the creation of triggers.
    \item[indexcache] Cache certain Metamorph index search results, so
      that an immediately following Metamorph query with the same
      \verb`WHERE` clause might be able to re-use the index results
      without re-searching the index.  E.g. may speed up a
      \verb`SELECT field1, field2, ...` Metamorph query that follows
      a \verb`SELECT count(*)` query with the same \verb`WHERE` clause.
    \item[ignoremissingfields] Ignore missing fields during an
      \verb`INSERT` or \verb`UPDATE`, i.e. do not issue a message and
      fail the query if attempting to insert a non-existent field.
      This may be useful if a SQL \verb`INSERT` statement is to be
      used against a table where some fields are optional and may not
      exist.
  \end{description}

Example:  \verb`set options='indexcache'`

\item[ignorenewlist] When processing a Metamorph query you can instruct
Texis to ignore the unoptimized portion of a Metamorph index by issuing
the SQL \verb`set ignorenewlist = 1;`.  If you have a continually
changing dataset, and the index is frequently updated then the default
of processing the unoptimized portion is probably correct.  If the data
tends to change in large batches, followed by a reoptimization of the
index then the large batch can cause significant processing overhead.
In that case it may be wise to enable the \verb'ignorenewlist' option.
If the option is enable then records that have been updated in the batch
will not be found with Metamorph queries until the index has been optimized.
Added in version 02.06.934400000.

\item[indexwithin] How to use the Metamorph index when processing
  ``within $N$'' ({\tt w/$N$}) \verb`LIKE`-type queries.  It is an
  integer combination of bit flags:
  \begin{description}
    \item[0x01]: Use index for {\tt w/$N$} searches
                  when \verb`withinmode` is ``\verb`char [span]`''
    \item[0x02]: Use index for {\tt w/$N$} searches
                  when \verb`withinmode` is ``\verb`word [span]`''
    \item[0x04]: Optimize within-chars window down
    \item[0x08]: Do not scale up intervening (non-query) words part of
      window to account for words matching multiple index expressions,
      which rarely occur; this reduces false (too wide) hits from the
      index.  Also do not require post-processing if multiple index
      expressions.  In rare cases valid hits may be missed if an
      intervening word does index-match multiply; the {\tt $N$} value
      can simply be increased in the query to return these.
  \end{description}
  The default is 0xf in version 7.06.1525203000 20180501 and later,
  when support for 0x8 was also added.  In version 5.01.1153865548
  20060725 up to then, the default was 0x7.  The setting was added in
  version 4.04.1075255999 20040127 with a default of 0.

\item[wildoneword] 
  Whether wildcard expressions in Metamorph queries span a single word
  only, i.e. for multi-substring wildcards.  If 0 (false), the query
  ``\verb`st*ion`'' matches ``\verb`stallion`'' as well as
  ``{\tt stuff an onion}''.  If 1 (true), then ``\verb`st*ion`'' only matches
  ``\verb`stallion`'', and linear-dictionary index searches are
  possible (if enabled), because there are no multi-word matches to
  (erroneously) miss.  {\bf Note:} prior to version 5.01.1208472000
  20080417, this setting did not apply to linear searches; linear or
  post-process searches may have experienced different behavior.
  The default is 1 in version 6 and later, 0 in version 5 and earlier.
  Added in version 4.03.1058230349 20030714.

\item[wildsufmatch]
  Whether wildcard expressions in Metamorph queries suffix-match their
  trailing substrings to the end of words.  If 0 (false), the query
  ``\verb`*so`'' matches ``\verb`also`'' as well as
  ``\verb`absolute`''.  If 1 (true), then ``\verb`*so`'' only matches
  ``\verb`also`''.  Affects what terms are matched during
  linear-dictionary index searches.  {\bf Note:} prior to version
  5.01.1208472000 20080417, this setting did not apply to linear
  searches; linear or post-process searches may have experienced
  different behavior.  The default is 1 in version 6 and later, 0 in
  version 5 and earlier.  Added in version 4.03.1058230349 20030714.

\item[wildsingle]
  An alias for setting \verb`wildoneword` and \verb`wildsufmatch`
  together, which is usually desired.  Added in version 4.03.1058230349
  20030714.

\item[allineardict]
  Whether to allow linear-dictionary Metamorph index searches.
  Normally a Metamorph query term is either binary-index searchable
  (fastest), or else must be linear-table searched (slowest).
  However, certain terms, while not binary-index searchable, can be
  linear-dictionary searched in the index, which is slower than
  binary-index, yet faster than linear-table search.  Examples include
  leading-prefix wildcards such as ``\verb`*tion`''.  The default is 0
  (false), since query protection is enabled by default.  Note that
  \verb`wildsingle` should typically be set true so that wildcard
  syntax is more likely to be linear-dictionary searchable.  Added in
  version 4.03.1058230349 20030714.

\item[indexminsublen]
  The minimum number of characters that a Metamorph index word expression
  must match in a query term, in order for the term to utilize the index.
  A term with fewer than \verb`indexminsublen` indexable characters
  is assumed to potentially match too many words in the index for
  an index search to be more worthwhile/faster than a linear-table search.

  For binary-index searchable terms, \verb`indexminsublen` is tested
  against the minimum prefix length; e.g. for query ``\verb`test.#@`''
  the length tested is 4 (assuming default index word expression of
  ``\verb`\alnum{2,99}`'').  For linear-dictionary index searches, the
  length tested is the total of all non-wildcard characters; e.g. for
  query ``\verb`ab*cd*ef`'' the length tested is 6.

  The default for \verb`indexminsublen` is 2.  Added in version
  4.03.1058230349 20030714.  Note that the query -- regardless of
  index or linear search -- must also pass the \verb`qminprelen`
  setting.

\item[dropwordmode]
  How to remove words from a query set when too many are present
  (\verb`qmaxsetwords` or \verb`qmaxwords` exceeded) in an index
  search, e.g. for a wildcard term.  The possible values are 0 to
  retain suffixes and most common words up to the word limit, or 1 to
  drop the entire term.  The default is 0.
  Added in version 3.00.947633136 20000111.

\item[metamorphstrlstmode]
\label{`metamorphstrlstmode'}
  How to convert a \verb`strlst` Metamorph query (perhaps generated by
  Vortex \verb`arrayconvert`) to a regular string Metamorph query.
  For example, for the \verb`strlst` query composed of the 3 strings
  ``\verb`one`'', ``\verb`two`'', and ``\verb`bear arms`'', the
  various modes would convert as follows:

  \begin{itemize}
    \item \verb`allwords` \\
      Space-separate each string, e.g. ``{\tt one two bear arms}''.
    \item \verb`anywords` \\
      Space-separate each string and append ``\verb`@0''', e.g.
       ``{\tt one two bear arms @0}''.
    \item \verb`allphrases` \\
      Space-separate and double-quote each string, e.g.
      ``{\tt "one" "two" "bear arms"}''.
    \item \verb`anywords` \\
      Space-separate and double-quote each string, and append ``\verb`@0''',
      e.g. ``{\tt "one" "two" "bear arms" @0}''.
    \item \verb`equivlist` \\
      Make the string list into a parenthetical comma-separated list,
      e.g. ``{\tt (one,two,bear arms)}''.
  \end{itemize}

  The default is \verb`equivlist`.  Added in version 5.01.1225240000
  20081028.  See also the \verb`varchartostrlstsep` setting
  (p.~\pageref{`varchartostrlstsep'}), which affects conversion of
  \verb`varchar` to \verb`strlst` in other contexts.

\item[compatibilityversion]
\label{SqlPropertyCompatibilityVersion}

  Sets the Texis compatibility version -- the version to attempt to
  behave as -- to the given string, which is a Texis version of the
  form ``$major$[.$minor$[.$release$]]'', where $major$ is a major
  version integer, $minor$ is a minor version integer, and $release$
  is a release integer.  Added in version 7.  See the
  \verb`<vxcp compatibilityversion>` setting in Vortex for details.
  See also the {\tt [Texis] Compatibility Version} setting
  (p.~\pageref{TexisIniTexisCompatibilityVersion}) in {\tt texis.ini},
  which the \verb`compatibilityversion` setting defaults to.

\item[failifincompatible]

  Whenever set nonzero/true, and the most recent
  \verb`compatibilityversion` setting attempt failed, then all future
  SQL statements will fail with an error message.  Since there is no
  conditional (``if'') statement in SQL, this allows a SQL script to
  essentially abort if it tries to set a Texis compatibility version
  that is unsupported, rather than continue with possibly undesired
  side effects.  Added in version 7.  See also
  \verb`<vxcp compatibilityversion>` in Vortex, which obviates the
  need for this setting, as it has a checkable error return.

\item[groupbymem]

  When set nonzero/true (the default), try to minimize memory usage
  during \verb`GROUP BY`/\verb`DISTINCT` operations (e.g. when using an
  index and sorting is not needed).  Added in version 7.00.1370039228
  20130531.

\item[legacyversion7orderbyrank]
\label{SqlPropertyLegacyVersion7OrderByRank}

  If on, an {\tt ORDER BY \$rank} (or {\tt \$rank}-containing
  expression) uses legacy version 7 behavior, i.e. typically orders in
  numerically descending order, but may change to ascending (and have
  other idiosyncrasies) depending on index, expression and \verb`DESC`
  flag use.  If disabled, such {\tt ORDER BY}s are consistent with
  others: numerically ascending unless \verb`DESC` flag given (which
  would typically be given, to maintain descending-numerical-rank
  order).

  The default is the value of the {\tt [Texis] Legacy Version 7 Order
    By Rank} setting (p.~\pageref{TexisIniLegacyVersion7OrderByRank})
  in {\tt conf/texis.ini}, which is off by default with
  \verb`compatibilityversion` 8 and later, on in earlier
  versions (\verb`compatibilityversion` defaults to Texis Version).
  Added in version 7.06.1508871000 20171024.

  Note that this setting may be removed in a future release, as its
  enabled behavior is deprecated.  Its existence is only to ease
  transition of old code when upgrading to Texis version 8, and thus
  should only be used temporarily.  Old code should be updated to
  reflect version 8 default behavior -- and this setting removed --
  soon after upgrading.
  
\end{description}

% ----------------------------------------------------------------------------
\subsection{Metamorph parameters}

These settings affect the way that text searches are performed.  They are
equivalent to changing the corresponding parameter in the profile, or by
calling the Metamorph API function to set them (if there is an equivalent).
They are:

\begin{description}
\item[minwordlen] The smallest a word can get due to suffix and prefix
removal.  Removal of trailing vowel or double consonant can make it a
letter shorter than this.  Default 255.

\item[keepnoise] Whether noise words should be stripped from the query
and index.  Default off.

\item[suffixproc] Whether suffixes should be stripped from the words to
find a match.  Default on.

\item[prefixproc] Whether prefixes should be stripped from the words to
find a match.  Turning this on is not suggested when using a Metamorph
index.  Default off.


\item[rebuild] Make sure that the word found can be built from the root
and appropriate suffixes and prefixes.  This increases the accuracy of
the search.  Default on.

\item[useequiv] Perform thesaurus lookup.  If this is on then the word
and all equivalences will be searched for.  If it is off then only the
query word is searched for.  Default off.  Aka {\bf keepeqvs}
in version 5.01.1171414736 20070213 and later.

\item[inc\_sdexp]
  Include the start delimiter as part of the hit.  This is not
generally useful in Texis unless hit offset information is being
retrieved.  Default off.

\item[inc\_edexp]
  Include the end delimiter as part of the hit.  This is not
generally useful in Texis unless hit offset information is being
retrieved.  Default on.

\item[sdexp]  Start delimiter to use: a regular expression to match
the start of a hit.  The default is no delimiter.

\item[edexp]  End delimiter to use: a regular expression to match
the start of a hit.  The default is no delimiter.

\item[intersects] Default number of intersections in Metamorph
  queries; overridden by the \verb`@` operator.  Added in version
  7.06.1530212000 20180628.

\item[hyphenphrase]  Controls whether a hyphen between words searches
for the phrase of the two words next to each other, or searches for
the hyphen literally.  The default value of 1 will search for the two
words as a phrase.  Setting it to 0 will search for a single term
including the hyphen.  If you anticipate setting hyphenphrase to 0 then
you should modify the index word expression to include hyphens.

\item[wordc] For language or wildcard query terms during linear
  (non-index) searches, this defines which characters in the document
  consitute a word.  When a match is found for language/wildcard
  terms, the hit is expanded to include all surrounding word
  characters, as defined by this setting.  The resulting expansion
  must then match the query term for the hit to be valid.  (This
  prevents the query ``\verb`pond`'' from inadvertently matching the
  text ``\verb`correspondence`'', for example.)  The value is
  specified as a REX character set.  The default setting is
  \verb|[\alpha\']| which corresponds to all letters and apostrophe.
  For example, to exclude apostrophe and include digits use:
  \verb|set wordc='[\alnum]'| Added in version 3.00.942260000.  Note
  that this setting is for linear searches: what constitutes a word
  for Metamorph {\em index} searches is controlled by the index
  expressions ({\bf addexp} property, p.~\pageref{addexpSqlProperty}).
  Also note that non-language, non-wildcard query terms (e.g.
  \verb`123` with default settings) are not word-expanded.

\item[langc] Defines which characters make a query term a language
  term.  A language term will have prefix/suffix processing applied
  (if enabled), as well as force the use of {\bf wordc} to qualify the
  hit (during linear searches).  Normally {\bf langc} should be set
  the same as {\bf wordc} with the addition of the phrase characters
  space and hyphen.  The default is \verb|[\alpha\' \-]| Added in
  version 3.00.942260000.

\item[withinmode]
    A space- or comma-separated unit and optional type for the
    ``within-$N$'' operator (e.g. \verb`w/5`).  The unit is one of:
    \begin{itemize}
      \item \verb`char` for within-$N$ characters
      \item \verb`word` for within-$N$ words
    \end{itemize}
    The optional type determines what distance the operator measures.
    It is one of the following:
    \begin{itemize}
      \item \verb`radius` (the default if no type is specified when
        set) indicates all sets must be within a radius $N$ of an
        ``anchor'' set, i.e. there is a set in the match such that all
        other sets are within $N$ units right of its right edge or $N$
        units left of its left edge.
      \item \verb`span` indicates all sets must be within an $N$-unit
        span
    \end{itemize}
    Added in version 4.04.1077930936 20040227.  The optional type was
    added in version 5.01.1258712000 20091120; previously the only
    type was implicitly \verb`radius`.  In version 5 and earlier the
    default setting was \verb`char` (i.e. {\tt char radius}); in
    version 6 and later the default is {\tt word span}.

\item[phrasewordproc]
    Which words of a phrase to do suffix/wildcard processing on.  The
    possible values are \verb`mono` to treat the phrase as a
    monolithic word (i.e. only last word processed, but entire phrase
    counts towards {\bf minwordlen}); \verb`none` for no
    suffix/wildcard processing on phrases; or \verb`last` to process just
    the last word.
%    ; or \verb`all` to process all words.
%
%    For example, changing the mode to \verb`all` would allow the phrase
%    query \verb`"bear arms"` to match ``\verb`bearing arm` (with suffix
%    processing on and \verb`minwordlen` 3).  With the default mode,
%    it would match ``\verb`bear arm`'' but not ``\verb`bearing arm`''.
    Note that a phrase is multi-word, i.e. a single word in double-quotes
    is not considered a phrase, and thus {\bf phrasewordproc} does not apply.
    Added in version 4.03.1082000000 20040414.  Mode \verb`none`
    supported in version 5.01.1127760000 20050926.

\item[mdparmodifyterms]
  If nonzero, allows the Metamorph query parser to modify search terms
by compression of whitespace and quoting/unquoting.  This is for
back-compatibility with earlier versions; enabling it will break the
information from bit 4 of \verb`mminfo()` (query offset/lengths of
sets).  Added in version 5.01.1220640000 20080905.

\end{description}

% -----------------------------------------------------------------------------
\subsection{Rank knobs}
\label{rankknobs}

  The following properties affect the document ranks from \verb`likep`
and \verb`like` queries, and hence the order of returned documents for
\verb`likep`.  Each property controls a factor used in the rank.  The
property's value is the relative importance of that factor in
computing the rank.  The properties are settable from 0 (factor has no
effect at all) to 1000 (factor has maximum relative importance).

  It is important to note that these property weights are relative to
the sum of all weights.  For example, if \verb`likepleadbias` is
set to 1000 and the remaining properties to 0, then a hit's rank will
be based solely on lead bias.  If \verb`likepproximity` is then set to
1000 as well, then lead bias and proximity each determine 50\% of the
rank.

\begin{description}
\item[likepproximity]
  Controls how important proximity of terms is.  The closer the hit's
terms are grouped together, the better the rank.  The default weight
is 500.

\item[likepleadbias]
  Controls how important closeness to document start is.  Hits closer
to the top of the document are considered better.  The default weight
is 500.

\item[likeporder]
  Controls how important word order is: hits with terms in the same
order as the query are considered better.  For example, if searching
for ``{\tt bear arms}'', then the hit ``{\tt arm bears}'', while
matching both terms, is probably not as good as an in-order match.
The default weight is 500.

\item[likepdocfreq]
  Controls how important frequency in document is.  The more
occurrences of a term in a document, the better its rank, up to a point.
The default weight is 500.

\item[likeptblfreq]
  Controls how important frequency in the table is.  The more a term
occurs in the table being searched, the {\em worse} its rank.  Terms
that occur in many documents are usually less relevant than rare
terms.  For example, in a web-walk database the word ``\verb`HTML`''
is likely to occur in most documents: it thus has little use in
finding a specific document.  The default weight is 500.

\end{description}

% -----------------------------------------------------------------------------
\subsection{Other ranking properties}
\label{otherrank}

  These properties affect how \verb`LIKEP` and some \verb`LIKE`
queries are processed.

\begin{description}
\item[likeprows] Only the top \verb`likeprows` relevant documents are
  returned by a \verb`LIKEP` query (default 100).  This is an
  arbitrary cut-off beyond which most results would be increasingly
  useless.  It also speeds up the query process, because fewer rows
  need to be sorted during ranking.  By altering \verb`likeprows` this
  threshold can be changed, e.g. to return more results to the user
  (at the potential cost of more search time).  Setting this to 0 will
  return all relevant documents (no limit).

Note that in some circumstances, a \verb`LIKEP` query might return
more than \verb`likeprows` results, if for example later processing
requires examination of all \verb`LIKEP`-matching rows (e.g. certain
\verb`AND` queries).  Thus a SQL statement containing \verb`LIKEP` may
or may not be limited to \verb`likeprows` results, depending on other
clauses, indexes, etc.

% wtf not implemented by new MM index:
%\item[likeptime]
%  This is the amount of time in seconds to be dedicated to the ranking
%portion of a \verb`LIKEP` query.  The default (0) is to take as much
%time as needed to yield the number of rows indicated by
%\verb`likeprows`.  The time is checked after each record is processed,
%so the actual time taken may be slightly longer than the set time.
%Also at least one row is always returned.

\item[likepmode]
  Sets the mode for \verb`LIKEP` queries.  This can be either 0, for
early, or 1 for late.  The default is 1, which is the correct setting
for almost all cases.  Does not apply to most Metamorph index searches.

\item[likepallmatch] Setting this to 1 forces \verb`LIKEP` to only
  consider those documents containing {\em all} (non-negated) query
  terms as matches (i.e. just as \verb`LIKE` does).  By default, since
  \verb`LIKEP` is a ranking operator it returns the best results even
  if only some of the set-logic terms (non-\verb`+` or \verb`-`
  prefix) can be found.  (Note that required terms -- prefixed with a
  \verb`+` -- are always required in a hit regardless of this setting.
  Also note that if {\tt likepobeyintersects} is true, an {\tt @}
  operator value in the query will override this setting.)

\item[likepobeyintersects] Setting this to 1 forces \verb`LIKEP` to
  obey the intersects operator ({\tt @}) in queries (even when {\tt
    likepallmatch} is true).  By default \verb`LIKEP` does not use it,
  because it is a ranking operator.  Setting both \verb`likepallmatch`
  and \verb`likepobeyintersects` to 1 will make \verb`LIKEP` respect
  queries the same as \verb`LIKE`. (Note: \verb`apicp`
  \verb`alintersects` may have to be enabled in Vortex as well.)

\item[likepinfthresh] This controls the ``infinity'' threshold in
  \verb`LIKE` and \verb`LIKEP` queries: if the estimated number of
  matching rows for a set is greater than this, the set is considered
  infinitely-occurring.  If all the search terms found in a given
  document are such infinite sets, the document is given an estimated
  rank.  This saves time ranking irrelevant but often-occurring
  matches, at the possible expense of rank position.  The default is
  0, which means infinite (no infinite sets; rank all documents).

\item[likepindexthresh]
  Controls the maximum number of matching documents to examine
(default infinite) for \verb`LIKEP` and \verb`LIKE`.  After this many
matches have been found, stop and return the results obtained so far,
even if more hits exist.  Typically this would be set to a high
threshold (e.g. 100000): a query that returns more than that many hits
is probably not specific enough to produce useful results, so save
time and don't process the remaining hits.  (It's also a good bet that
something useful was already found in the initial results.)  This
helps keep such noisy queries from loading a server, by stopping
processing on them early.  A more specific query that returns fewer
hits will fall under this threshold, so all matches will be considered
for ranking.

  Note that setting \verb`likepindexthresh` is a tradeoff between
speed and accuracy: the lower the setting, the faster queries can be
processed, but the more queries may be dropping potentially
high-ranking hits.
\end{description}

% -----------------------------------------------------------------------------
\subsection{Indexing properties}
\label{sec:indexprop}

\begin{description}

\item[indexspace]
\label{indexspace}
A directory in which to store the index files.  The
default is the empty string, which means use the database directory.
This can be used to put the indexes onto another disk to balance load
or for space reasons.  If \verb`indexspace` is set to a non-default
value when a Metamorph index is being updated, the new index will be
stored in the new location.

\item[indexblock] When a Metamorph index is created on an indirect field,
the indirect files are read in blocks.  This property allows the size
of the block used to be redefined.

\item[indexmem]
\label{indexmem}
  When indexes are created Texis will use memory to speed up the
process.  This setting allows the amount of memory used to be
adjusted.  The default is to use 40\% of physical memory, if it
can be determined, and to use 16MB if not.  If the value set is less
than 100 then it is treated as a percentage of physical memory.  It
the number is greater than 100 then it is treated as the number of
bytes of memory to use.  Setting this value too high can cause
excessive swapping, while setting it too low causes unneeded extra
merges to disk.

\item[indexmeter]
\label{indexmeter}
  Whether to print a progress meter during index creation/update.
The default is 0 or \verb`'none'`, which suppresses the meter.
A value of 1 or \verb`'simple'` prints a simple hash-mark meter
(with no tty control codes; suitable for redirection to a file and
reading by other processes).  A value of 2 or \verb`'percent'` or
\verb`'pct'` prints a hash-mark meter with a more detailed percentage
value (suitable for large indexes).  Added in
version 4.00.998688241 Aug 24 2001.

\item[meter]

  A semicolon-separated list of processes to print a progress meter
for.  Syntax:

\verb`     `{\tt \{$process$[=$type$]\}|$type$ [; ...]}

A $process$ is one of \verb`index`, \verb`compact`, or the catch-all
alias \verb`all`.  A $type$ is a progress meter type, one of
\verb`none`, \verb`simple`, \verb`percent`, \verb`on` (same as
\verb`simple`) or \verb`off` (same as \verb`none`).  The default
$type$ if not given is \verb`on`.  E.g. to show a progress meter for
all meterable processes, simply set \verb`meter` to \verb`on`.
Added in version 6.00.1290500000 20101123.

\item[addexp] An additional REX expression to match words to be
\label{addexpSqlProperty}
  indexed in a Metamorph index.  This is useful if there are
  non-English words to be searched for, such as part numbers.  When an
  index is first created, the expressions used are stored with it so
  they will be updated properly.  The default expression is
  \verb`\alnum{2,99}`.  {\bf Note:} Only the expressions set when the
  index is initially created (i.e. the first {\tt CREATE METAMORPH
  ...} statement -- later statements are index updates) are saved.
  Expressions set during an update (issuance of ``{\tt create
  metamorph [inverted] index}'' on an existent index) will {\em not}
  be added.

\item[delexp] This removes an index word expression from the list.
Expressions can be removed either by number (starting with 0) or by
expression.

\item[lstexp] Lists the current index word expressions.  The
value specified is ignored (but required syntactically).

\item[addindextmp] Add a directory to the list of directories to use
for temporary files while creating the index.  If temporary files are
needed while creating a Metamorph index they will be created in one of
these directories, the one with the most space at the time of
creation.  If no \verb|addindextmp| dirs are specified, the default
list is the index's destination dir (e.g. database or \verb|indexspace|),
and the environment variables \verb|TMP| and \verb|TMPDIR|.

\item[delindextmp]  Remove a directory from the list of directories to
use for temporary files while creating a Metamorph index.

\item[lstindextmp]  List the directories used for temporary files while
creating Metamorph indices.  Aka \verb`listindextmp`.

% wtf not used by new MM index?
%\item[infthresh]  This sets a threshold for Metamorph indexes.  If more
%than this number of documents match a given term then that term is deemed
%to be omnipresent, and further processing on that term is stopped.  This
%can be overridden by \verb`infpercent`.  Default -1 (disabled).

% wtf not used by new MM index?
%\item[infpercent] Similar to \verb`infthresh`, except the number of
%documents is specified as a percentage of the total, as opposed to an
%absolute number.  Default -1 (disabled).

\item[indexvalues]
\label{indexvalues}
  Controls how a regular (B-tree) index stores table values.  If set
  to {\tt splitstrlst} (the default), then \verb`strlst`-type fields
  are split, i.e. a separate (item,recid) tuple is stored for {\em
  each} (\verb`varchar`) item in the \verb`strlst`, rather than just
  one for the whole (strlst,recid) tuple.  This allows the index to be
  used for some set-like operators that look at individual items in a
  \verb`strlst`, such as most \verb`IN`, \verb`SUBSET`
  (p.~\pageref{SubsetOperator}) and \verb`INTERSECT`
  (p.~\pageref{IntersectOperator}) queries.

  If \verb`indexvalues` is set to \verb`all` -- or the index is not on
  a \verb`strlst` field, or is on multiple fields -- such splitting
  does not occur, and the index can generally not be used for set-like
  queries (with some exceptions; see p.~\pageref{SubsetIndexUsage} for
  details).

  Note that if index values are split (i.e. \verb`splitstrlst` set and
  index is one field which is \verb`strlst`), table rows with an empty
  (zero-items) \verb`strlst` value will not be stored in the index.
  This means that queries that require searching for or listing
  empty-\verb`strlst` table values cannot use such an index.  For
  example, a subset query with a non-empty parameter on the right side
  and a \verb`strlst` table column on the left side will not be able
  to return empty-\verb`strlst` rows when using an index, even though
  they match.  Also, subset queries with an empty-\verb`strlst` or
  empty-\verb`varchar` parameter (left or right side) must use an
  \verb`indexvalues=all` index instead.  Thus if empty-\verb`strlst`
  subset query parameters are a possibility, both types of index
  (\verb`splitstrlst` and \verb`all`) should be created.

  As with \verb`stringcomparemode`, only the creation-time
  \verb`indexvalues` value is ever used by an index, not the current
  value, and the optimizer will attempt to choose the best index at
  search time.  The \verb`indexvalues` setting was added in Texis
  version 7; previous versions effectively had \verb`indexvalues` set
  to \verb`splitstrlst`.  {\bf Caveat:} A version 6 Texis will issue
  an error when encountering an {\tt indexvalues=all} index (as it is
  unimplemented in version 6), and will refuse to modify the index or
  the table it is on.  {\bf A version 5 or earlier Texis, however, may
  silently corrupt an {\tt indexvalues=all} index during table
  modifications.}

\item[btreethreshold] This sets a limit as to how much of an index should
be used.  If a particular portion of the query matches more than the given
percent of the rows the index will not be used.  It is often more efficient
to try and find another index rather than use an index for a very frequent
term.  The default is set to 50, so if more than half the records match,
the index will not be used.  This only applies to ordinary indices.
% infthresh/infpercent for old Metamorph:
% See \verb`infthresh` and \verb`infpercent` for control of Metamorph indices.

\item[btreelog] Whether to log operations on a particular B-tree, for
  debugging.  Generally enabled only at the request of tech support.
  The value syntax is:
\begin{quote}
$[$\verb`on=`$|$\verb`off=`$][$\verb`/dir/`$]$\verb`file`$[$\verb`.btr`$]$
\end{quote}
Prefixing \verb`on=` or \verb`off=` turns logging on or off,
respectively; the default (if no prefix) is on.  Logging applies to
the named B-tree file; if a relative path is given, logging applies to
the named B-tree in any database accessed.

  The logging status is also saved in the B-tree file itself, if the
index is opened for writing (e.g. at create or update).  This means
that once logging is enabled and saved, {\em every} process that
accesses the B-tree will log operations, not just ones that have
\verb`btreelog` explicitly set.  This is critical for debugging, as
every operation must be logged.  Thus, \verb`btreelog` can just be set
once (e.g. at index create), without having to modify (and track down)
every script that might use the B-tree.  Logging can be disabled
later, by setting ``\verb`off=file`'' and accessing the index for an
update.

  Operations are logged to a text file with the same name as the
B-tree, but ending in ``\verb`.log`'' instead of ``\verb`.btr`''.
The columns in the log file are as follows; most are for tech
support analysis, and note that they may change in a future Texis release:
\begin{itemize}
  \item {\bf Date} Date
  \item {\bf Time} Time (including microseconds)
  \item {\bf Script and line} Vortex script and line number, if known
  \item {\bf PID} Process ID
  \item {\bf DBTBL handle} \verb`DBTBL` handle
  \item {\bf Read locks} Number of read locks (\verb`DBTBL.nireadl`)
  \item {\bf Write locks} Number of write locks (\verb`DBTBL.niwrite`)
  \item {\bf B-tree handle} \verb`BTREE` handle
  \item {\bf Action} What action was taken:
    \begin{itemize}
      \item \verb`open` B-tree open: {\bf Recid} is root page offset
      \item \verb`create` B-tree create
      \item \verb`close` B-tree close
      \item \verb`RDroot` Read root page
      \item \verb`dump` B-tree dump
      \item \verb`WRhdr` Write B-tree header: {\bf Recid} is root page offset
      \item \verb`WRdd` Write data dictionary: {\bf Recid} is \verb`DD` offset.
                       (Read \verb`DD` at open is not logged.)
      \item \verb`delete` Delete key: {\bf Recid} is for the key
      \item \verb`append` Append key
      \item \verb`insert` Insert key
      \item \verb`search` Search for key
      \item \verb`RDpage` Read page: {\bf Recid} is for the page
      \item \verb`WRpage` Write page
      \item \verb`CRpage` Create page
      \item \verb`FRpage` Free page
      \item \verb`FRdbf` Free DBF block
    \end{itemize}
  \item {\bf Result} Result of action:
    \begin{itemize}
      \item \verb`ok` Success
      \item \verb`fail` Failure
      \item \verb`dup` Duplicate (e.g. duplicate insert into unique B-tree)
      \item \verb`hit` Search found the key
      \item \verb`miss` Search did not find the key
    \end{itemize}
  \item {\bf Search mode} Search mode:
    \begin{itemize}
      \item \verb`B` Find before
      \item \verb`F` Find
      \item \verb`A` Find after
    \end{itemize}
  \item {\bf Index guarantee} \verb`DBTBL.indguar` flag (\verb`1` if no
    post-process needed)
  \item {\bf Index type} Index type:
    \begin{itemize}
      \item \verb`N` \verb`DBIDX_NATIVE` (bubble-up)
      \item \verb`M` \verb`DBIDX_MEMORY` (RAM B-tree)
      \item \verb`C` \verb`DBIDX_CACHE` (RAM cache)
    \end{itemize}
  \item {\bf Recid} Record id; see notes for {\bf Action} column
  \item {\bf Key size} Key size (in bytes)
  \item {\bf Key flags} Flags for each key value, separated by commas:
    \begin{itemize}
      \item \verb`D` \verb`OF_DESCENDING`
      \item \verb`I` \verb`OF_IGN_CASE`
      \item \verb`X` \verb`OF_DONT_CARE`
      \item \verb`E` \verb`OF_PREFER_END`
      \item \verb`S` \verb`OF_PREFER_START`
    \end{itemize}
  \item {\bf Key} Key, i.e. value being inserted, deleted etc.;
    multiple values separated with commas
\end{itemize}

Unavailable or not-applicable fields are logged with a dash.  Note
that enabling logging can produce a large log file quickly; free disk
space should be monitored.  The \verb`btreelog` setting was added in
version 5.01.1134028000 20051208.

\item[btreedump] Dump B-tree indexes, for debugging.  Generally enabled
only at the request of tech support.  The value is an integer whose
bits are defined as follows:

  Bits 0-15 define what to dump.  Files are created that are named after
the B-tree, with a different extension:
\begin{itemize}
  \item[Bit] 0: Issue a \verb`putmsg` about where dump file(s) are
  \item[Bit] 1: \verb`.btree` file: Copy of in-mem \verb`BTREE` struct
  \item[Bit] 2: \verb`.btrcopy` file: Copy of \verb`.btr` file
  \item[Bit] 3: \verb`.cache` file: Page cache from \verb`BCACHE`, \verb`BPAGE`
  \item[Bit] 4: \verb`.his` file: History from \verb`BTRL`
  \item[Bit] 5: \verb`.core` file: \verb`fork()` and dump core
\end{itemize}

Bits 16+ define when to dump:
\begin{itemize}
  \item[Bit] 16: At ``{\tt Cannot insert value}'' messages
  \item[Bit] 17: At ``{\tt Cannot delete value}'' messages
  \item[Bit] 18: At ``{\tt Trying to insert duplicate value}'' messages
\end{itemize}

The files are for tech support analysis.  Formats and bits subject to
change in future Texis releases.  The \verb`btreedump` setting was added
in version 5.01.1131587000 20051109.

\item[maxlinearrows] This set the maximum number of records that should
be searched linearly.  If using the indices to date yield a result set
larger than \verb`maxlinearrows` then the program will try to find more
indices to use.  Once the result set is smaller than \verb`maxlinearrows`,
or all possible indices are exhausted, the records will be processed.  The
default is 1000.

\item[likerrows] How many rows a single term can appear in, and still be
returned by \verb`liker`.  When searching for multiple terms with \verb`liker`
and \verb`likep` one does not always want documents only containing a
very frequent term to be displayed.  This sets the limit of what is
considered frequent.  The default is 1000.

\item[indexaccess] If this option is turned on then data from an index
can be selected as if it were a table.  When selecting from an
ordinary (B-tree) index, the fields that the index was created on will
be listed.  When selecting from a Metamorph index a list of words
(\verb`Word` column`), count of rows containing each word
(\verb`RowCount`), and -- for Metamorph inverted indexes -- count of
all hits in all rows (\verb`OccurrenceCount`) for each word will be
returned.

\item[indexchunk] In versions of Texis after October 1998, the
\verb`indexchunk` setting is deprecated and unused.  In prior releases,
when creating a Metamorph index temporary files are
used which in the worst case can grow to twice the size of the data
being indexed.  This process can be broken into stages, such that
after indexing a certain amount of data the temporary files are
processed, to generate a partial index, and then the process repeats
for the rest of the data.  By default the amount of free disk space is
checked on startup, and used to calculate when it will need to perform
the processing step.  If the system does not report free disk space
accurately, or to free more disk space, this value can be changed.
The default is 0, which automatically calculates a value.  Otherwise
it is set to the number of bytes of data to index before processing
the temporary files.  Lower values conserve disk space, at the expense
of more time to process intermediate files.

\item[cleanupwait] {\em Windows/NT specific} After updating a
Metamorph index the database will wait this long before trying to
remove the old copy of the index.  This is to allow any other process
currently using the index time to stop using the index, so it can be
removed.  The default is twenty seconds.  If a whole batch of
Metamorph indices are being updated right after another, it may be
useful to set this to 0 for all but the last index, as an attempt will
be made to remove all old indices after every index update.

\item[dbcleanupverbose] Integer whose bit flags control some tracing
messages about database cleanup housekeeping (e.g. removal of unneeded
temporary or deleted indexes and tables).  A bit-wise OR of the
following values:

\begin{itemize}
  \item \verb`0x01`: Report successful removal of temporary/deleted
    indexes/tables.
  \item \verb`0x02`: Report failed removal of such indexes/tables.
  \item \verb`0x04`: Report on in-use checks of temporary indexes/tables.
\end{itemize}

The default is 0 (i.e. no messages).  Note that these cleanup actions
may also be handled by the Database Monitor; see also the {\tt
[Monitor] DB Cleanup Verbose} setting in {\tt conf/texis.ini}.  Added
in version 6.00.1339712000 20120614.

\item[indextrace] For debugging: trace index usage, especially during
searches, issuing informational \verb`putmsg`s.  Greater values
produce more messages.  Note that the meaning of values, as well as
the messages printed, are subject to change without notice.
Aka \verb`traceindex`, \verb`traceidx`.  Added
in version 3.00.942186316 19991109.

\item[tracerecid] For debugging: trace index usage for this particular
recid.  Added in version 3.01.945660772 19991219.

\item[indexdump] For debugging: dump index recids during search/usage.
Value is a bitwise OR of the following flags:
\begin{description}
\item[Bit 0] for new list
\item[Bit 1] for delete list
\item[Bit 2] for token file
\item[Bit 3] for overall counts too
\end{description}
The default is 0.

\item[indexmmap] Whether to use memory-mapping to access Metamorph
index files, instead of \verb`read()`.  The value is a bitwise OR
of the following flags:
\begin{description}
\item[Bit 0] for token file
\item[Bit 1] for \verb`.dat` file
\end{description}
The default is 1 (i.e. for token file only).  Note that memory-mapping
may not be supported on all platforms.

\item[indexreadbufsz] Read buffer size, when reading (not
memory-mapping) Metamorh index \verb`.tok` and \verb`.dat` files.  The
default is 64KB; suffixes like ``\verb`KB`'' are respected.  During
search, actual read block size could be less (if predicted) or more
(if blocks merged).  Also used during index create/update.  Decreasing
this size when creating large indexes can save memory (due to the
large number of intermediate files), at the potential expense of time.
Aka \verb`indexreadbufsize`.  Added in version 4.00.1006398833
20011121.

\item[indexwritebufsz] Write buffer size for creating Metamorph
indexes.  The default is 128KB; suffixes like ``\verb`KB`'' are
respected.  Aka \verb`indexwritebufsize`.  Added in version
4.00.1007509154 20011204.

\item[indexmmapbufsz] Memory-map buffer size for Metamorph indexes.
During search, it is used for the \verb`.dat` file, if it is
memory-mapped (see \verb`indexmmap`); it is ignored for the
\verb`.tok` file since the latter is heavily used and thus fully
mapped (if \verb`indexmmap` permits it).  During index update,
\verb`indexmmapbufsz` is used for the \verb`.dat` file, if it is
memory-mapped; the \verb`.tok` file will be entirely memory-mapped if
it is smaller than this size, else it is read.  Aka
\verb`indexmmapbufsize`.  The default is 0, which uses 25\% of RAM.
Added in version 3.01.959984092 20000602.  In version 4.00.1007509154
20011204 and later, ``\verb`KB`'' etc. suffixes are allowed.

\item[indexslurp] Whether to enable index ``slurp'' optimization
during Metamorph index create/update, where possible.  Optimization is
always possible for index create; during index update, it is possible
if the new insert/update recids all occur after the original recids
(e.g. the table is insert-only, or all updates created a new block).
Optimization saves about 20\% of index create/update time by merging
piles an entire word at a time, instead of word/token at a time.  The
default is 1 (enabled); set to 0 to disable.  Added in version
4.00.1004391616 20011029.

\item[indexappend] Whether to enable index ``append'' optimization
during Metamorph index update, where possible.  Optimization is
possible if the new insert recids all occur after the original recids,
and there were no deletes/updates (e.g. the table is insert-only); it
is irrelevant during index create.  Optimization saves index build
time by avoiding original token translation if not needed.  The
default is 1 (enabled); set to 0 to disable.  Added in version
4.00.1006312820 20011120.

\item[indexwritesplit] Whether to enable index ``write-split''
optimization during Metamorph index create/update.  Optimization saves
memory by splitting the writes for (potentially large) \verb`.dat`
blocks into multiple calls, thus needing less buffer space.  The
default is 1 (enabled); set to 0 to disable.  Added in version
4.00.1015532186 20020307.

\item[indexbtreeexclusive] Whether to optimize access to certain
index B-trees during exclusive access.  The optimization may reduce
seeks and reads, which may lead to increased index creation speed on
platforms with slow large-file \verb`lseek` behavior.  The default is
1 (enabled); set to 0 to disable.  Added in version 5.01.1177548533
20070425.

\item[mergeflush] Whether to enable index ``merge-flush'' optimization
during Metamorph index create/update.  Optimization saves time by
flushing in-memory index piles to disk just before final merge;
generally saves time where \verb`indexslurp` is not possible.  The
default is 1 (enabled); set to 0 to disable.  Added in version
4.00.1011143988 20020115.

\item[indexversion]
\item{indexversion}
Which version of Metamorph index to produce or
update, when creating or updating Metamorph indexes.  The supported
values are 0 through 3; the default is 2.  Setting version 0 sets the
default index version for that Texis release.  Note that old versions
of Texis may not support version 3 indexes.  Version 3 indexes may use
less disk space than version 2, but are considered experimental.
Added in version 3.00.954374722 20000329.

\item[indexmaxsingle]
\label{indexmaxsingle}
For Metamorph indexes; the maximum number of
locations that a single-recid dictionary word may have and still be
stored solely in the \verb`.btr` B-tree file (without needing a
\verb`.dat` entry).  Single-recid-occurence words usually have their
data stored solely in the B-tree to save a \verb`.dat` access at
search time.  However, if the word occurs many times in that single
recid, the data (for a Metamorph inverted index) may be large enough
to bloat the B-tree and thus negate the savings, so if the
single-recid word occurs more than \verb`indexmaxsingle` times, it is
stored in the \verb`.dat`.  The default is 8.

\item[uniqnewlist] Whether/how to unique the new list during
Metamorph index searches.  Works around a potential bug in old
versions of Texis; not generally set.  The possible values are:
\begin{description}
  \item[0]:  do not unique at all
  \item[1]:  unique auxillary/compound index new list only
  \item[2]:  unique all new lists
  \item[3]:  unique all new lists and report first few duplicates
\end{description}
The default is 0.

\item[tablereadbufsz] Size of read buffer for tables, used when it is
possible to buffer table reads (e.g. during some index creations).  The
default is 16KB.  When setting, suffixes such as ``\verb`KB`'' etc.
are supported.  Set to 0 to disable read buffering.  Added in version
5.01.1177700467 20070427.  Aka \verb`tablereadbufsize`.

\end{description}

% -----------------------------------------------------------------------------
\subsection{Locking properties}

These properties affect the way that locking occurs in the database engine.
Setting these properties without understanding the consequences can lead
to inaccurate results, and even corrupt tables.

\begin{description}
\item[singleuser] This will turn off locking completely.  {\em This should
be used with extreme caution}.  The times when it is safe to use this
option are if the database is read-only, or if there is only one
connection to the database.  Default off.  This replaces the prior
setting of \verb`nolocking`.

\item[lockmode] This can be set to either manual or automatic.  In
manual mode the person writing the program is responsible for getting
and releasing locks.  In automatic mode Texis will do this itself.
Manual mode can reduce the number of locks required, or implement
specific application logic.  In manual mode care must be taken that
reads and writes can not occur at the same time.  The two modes can
co-exist, in that one process can have manual mode, and the other
automatic.  Default automatic.

\item[locksleepmethod] Determines whether to use a portable or OS
specific method of sleeping while waiting for a lock.  By default the
OS specific method is used.  This should not need to be changed.

\item[locksleeptime] How long to wait between attempts to check the
lock.  If this value is too small locks will be checked too often,
wasting CPU time.  If it is too high then the process might be
sleeping when there is no lock, delaying database access.  Generally
the busier the system the higher this setting should be.  It is
measured in thousandths of a second.  The default is 20.

\item[locksleepmaxtime] 
The lock sleep time automatically increments the if unable to get a
lock to allow other processes an opportunity to get the CPU.  This
sets a limit on how lock to sleep.
It is measured in thousandths of a second.  The default is 100.
Added in version 4.00.1016570000.

\item[fairlock] Whether to be fair or not.  A process which is running
in fair mode will not obtain a lock if the lock which has been waiting
longest would conflict.  A process which is not in fair mode will
obtain the lock as soon as it can.  This can cause a process to wait
forever for a lock.  This typically happens if there are lots of
processes reading the table, and one trying to write.  Setting
\verb`fairlock` to true will guarantee that the writer can obtain the
lock as long as the readers are getting and releasing locks.  Without
\verb`fairlock` there is no such guarantee, however the readers will
see better performance as they will rarely if ever wait for the
writer.  This flag only affects the process which sets the flag.  It
is not possible to force another process to be fair.  The default is
that it operates in fair mode.

\item[lockverbose] How verbose the lock code should be.  The default
minimum level of 0 will report all serious problems in the lock
manager, as they are detected and corrected.  A verbosity level of 1
will also display messages about less serious problems, such as
processes that have exited without closing the lock structure.  Level
2 will also show when a lock can not be immediately obtained.  Level 3
will show every lock as it is released.  In version 5.01.1160010000 20061004
and later, the level can be bitwise OR'd with 0x10 and/or 0x20
to report system calls before and after (respectively).  Levels 1 and
above should generally only be used for debugging.
In version 7.07.1565800000 20190814 and later, 0x40 and 0x80 may be
set to report before and after semaphore locking/unlocking.

\item[debugbreak] Stop in debugger when set.  Internal/debug use available
in some versions.  Added in version 4.02.1045505248 Feb 17 2003.

\item[debugmalloc] Integer; controls debug malloc library.  Internal/debug
use in some versions.  Added in version 4.03.1050682062 Apr 18 2003.

\end{description}

% -----------------------------------------------------------------------------
\subsection{Miscellaneous Properties}

These properties do not fit nicely into a group, and are presented here.

\begin{description}
\item[tablespace] Similar to \verb`indexspace` above.  Sets a
directory into which tables created will be placed.  This property
does not stay set across invocations.  Default is empty string, which
means the database directory.

\item[datefmt] This is a \verb`strftime` format used to format dates
for conversion to character format.  This will affect \verb`tsql`, as
well as attempts to retrieve dates in ASCII format.  Although the
features supported by different operating systems will vary, some of
the more common format codes are:

\begin{itemize}
\item[\verb`\%\%`] Output \verb`%`
\item[\verb`\%a`] abbreviated weekday name
\item[\verb`\%A`] full weekday name
\item[\verb`\%b`] abbreviated month name
\item[\verb`\%B`] full month name
\item[\verb`\%c`] local date and time representation
\item[\verb`\%d`] day of month (01 - 31)
\item[\verb`\%D`] date as \verb`%m/%d/%y`
\item[\verb`\%e`] day of month ( 1 - 31)
\item[\verb`\%H`] Hour (00 - 23)
\item[\verb`\%I`] Hour (01 - 12)
\item[\verb`\%j`] day of year (001 - 366)
\item[\verb`\%m`] month (01 - 12)
\item[\verb`\%M`] Minute (00 - 59)
\item[\verb`\%p`] AM/PM
\item[\verb`\%S`] Seconds (00 - 59)
\item[\verb`\%U`] Week number (beginning Sunday) (00-53)
\item[\verb`\%w`] Week day (0-6) (0 is Sunday)
\item[\verb`\%W`] Week number (beginning Monday) (00-53)
\item[\verb`\%x`] local date representation
\item[\verb`\%X`] local time representation
\item[\verb`\%y`] two digit year (00 - 99)
\item[\verb`\%Y`] Year with century
\item[\verb`\%Z`] Time zone name
\end{itemize}
Default \verb`%Y-%m-%d %H:%M:%S`, which can be
restored by setting datefmt to an empty string.
Note that in version 6.00.1300386000 20110317 and later, the
\verb`stringformat()` SQL function can be used to format dates
(and other values) without needing to set a global property.

\item[timezone] Change the default timezone that Texis will use.  This
should be formatted as for the TZ environment variable.  For example for
US Eastern time you should set timezone to \verb`EST5EDT`.  Some systems
may allow alternate representations, such as \verb`US/Eastern`, and if
your operating system accepts them, so will Texis.

\item[locale] Can be used to change the locale that Texis uses.  This
  will impact the display of dates if using names, as well as the
  meaning of the character classes in REX expressions, so
  \verb`\alpha` will be correct.  Also with the correct locale set
  (and OS support), Metamorph will work case insensitively correctly
  (with mono-byte character sets and Texis version 5 or earlier; see
  \verb`textsearchmode` for UTF-8/Unicode and version 6 or later
  support).

\item[indirectcompat]  Setting this to 1 sets compatibility with early
versions of Texis as far as display of indirects go.  If set to 1 a
trailing \verb`@` is added to the end of the filename.  Default 0.

\item[indirectspace]  Controls where indirects are created.  The default
location is a directory called indirects in the database directory.  Texis
will automatically create a directory structure under that directory to
allow for efficient indirect access.  At the top level there will be 16
directories, 0 through 9 and a through f.  When you create the directory
for indirects you can precreate these directories, or use them as mount
points.  You should make sure that the Texis user has permissions to the
directories.
Added in version 03.00.940520000

\item[triggermode] This setting changes the way that the command is treated
when creating a trigger.  The default behavior is that the command will be
executed with an extra arg, which is the filename of the table containing
the records.  If \verb`triggermode` is set to 1 then the strings \verb`$db`
and \verb`$table` are replaced by the database and table in that database
containing the records.  This allows any program which can access the
database to retrieve the values in the table without custom coding.

\item[paramchk] Enables or disables the checking of parameters in the
SQL statement.  By default it is enabled, which will cause any unset
parameters to cause an error.  If paramchk is set to 0 then unset
parameters will not cause an error, and will be ignored.  This lets a
single complex query be given, yet parameter values need only be
supplied for those clauses that should take effect on the query.

\item[message,nomessage] Enable or disable messages from the SQL engine.
The argument should be a comma separated list of messages that you want to
enable or disable.  The known messages are:
\begin{description}
\item[duplicate]  Message {\tt Trying to insert duplicate value () in index}
when an attempt is made to insert a record which has a duplicate value and
a unique index exists.  The default is enabled.
\end{description}

\item[varchartostrlstsep]
\label{`varchartostrlstsep'}
  The separator character or mode to use when converting a
  \verb`varchar` string into a \verb`strlst` list of strings in Texis.
  The default is set by the \verb`conf/texis.ini` setting {\tt [Texis]
    Varchar To Strlst Sep} (p.~\pageref{tcVarcharToStrlstSep}); if
  that is not set, the ``factory'' built-in default is \verb`create`
  in version 7 (or \verb`compatibilityversion` 7) and later, or
  \verb`lastchar` in version 6 (or \verb`compatibilityversion` 6) and
  earlier.

  A value of \verb`create` indicates that the separator is to be
  created: the entire string is taken intact as the sole item for the
  resulting \verb`strlst`,\footnote{In version 7 (or
    \verb`compatibilityversion` 7) and later, note that in {\tt
      create} mode, an empty source string will result in an empty
    (zero-items) {\tt strlst}: this helps maintain consistency of
    empty-string meaning empty-set for {\tt strlst}, as is true in
    other contexts.  In version 6 and earlier an empty source string
    produced a one-empty-string-item {\tt strlst} in {\tt create}
    mode.} and a separator is created that is not present in the
  string (to aid re-conversion to \verb`varchar`).  This can be used
  in conjunction with Vortex's {\tt <sqlcp arrayconvert>} setting to
  ensure that single-value as well as multi-value Vortex variables are
  converted consistently when inserted into a \verb`strlst` column:
  single-value vars by \verb`varchartostrlstsep`, multi-value by
  \verb`arrayconvert`.

  The value \verb`lastchar` indicates that the last character in the
  source string should be the separator; e.g. ``{\tt a,b,c,}'' would
  be split on the comma and result in a \verb`strlst` of 3 values:
  ``{\tt a}'', ``{\tt b}'' and ``{\tt c}''.

  \verb`varchartostrlstsep` may also be a single byte character, in
  which case that character is used as the separator.  This is useful
  for converting CSV-type strings e.g. ``{\tt a,b,c}'' without having to
  modify the string and append the separator character first (i.e. for
  {\tt lastchar} mode).

  \verb`varchartostrlstsep` may also be set to \verb`default` to
  restore the default (\verb`conf/texis.ini`) setting.  It may also be
  set to \verb`builtindefault` to restore the ``factory'' built-in
  default (which changes under \verb`compatibilityversion`, see
  above); these values were added in version 5.01.1231553000 20090109.
  If no \verb`conf/texis.ini` value is set, \verb`default` is the same
  as \verb`builtindefault`.

  \verb`varchartostrlstsep` was added in version 5.01.1226978000
  20081117.  See also the \verb`metamorphstrlstmode` setting
  (p.~\pageref{`metamorphstrlstmode'}), which affects conversion of
  \verb`strlst` values into Metamorph queries; and the \verb`convert`
  SQL function (p.~\pageref{convertSqlFunction}), which in Texis
  version 7 and later can take a \verb`varchartostrlstsep` mode
  argument.  The \verb`compatibilityversion` property
  (p.~\pageref{SqlPropertyCompatibilityVersion}), when set, affects
  \verb`varchartostrlstsep` as well.

\item[multivaluetomultirow]
\label{multivaluetomultirow}
  Whether to split multi-value fields (e.g. \verb`strlst`) into
  multiple rows (e.g. of \verb`varchar`) when appropriate, i.e. during
  {\tt GROUP BY} or {\tt DISTINCT} on such a field.  If nonzero/true,
  a {\tt GROUP BY} or {\tt DISTINCT} on a \verb`strlst` field will
  split the field into its \verb`varchar` members for processing.  For
  example, consider the following table:
  \begin{verbatim}
    create table test(Colors strlst);
    insert into test(Colors)
      values(convert('red,green,blue,', 'strlst', 'lastchar'));
    insert into test(Colors)
      values(convert('blue,orange,green,', 'strlst', 'lastchar'));
  \end{verbatim}
  With \verb`multivaluetomultirow` set true, the statement:
  \begin{verbatim}
    select count(Colors) Count, Colors from test group by Colors;
  \end{verbatim}
  generates the following output:
  \begin{verbatim}
          Count       Colors
    ------------+------------+
               2 blue
               2 green
               1 orange
               1 red
  \end{verbatim}
  Note that the \verb`strlst` values have been split, allowing the two
  \verb`blue` and \verb`green` values to be counted individually.
  This also results in the returned \verb`Colors` type being
  \verb`varchar` instead of its declared \verb`strlst`, and the sum of
  \verb`Count` values being greater than the number of rows in the
  table.  Note also that merely \verb`SELECT`ing a \verb`strlst` will
  not cause it to be split: it must be specified in the {\tt GROUP BY}
  or {\tt DISTINCT} clause.

  The \verb`multivaluetomultirow` was added in version 5.01.1243980000
  20090602.  It currently only applies to \verb`strlst` values and
  only to single-column {\tt GROUP BY} or {\tt DISTINCT} clauses.  A
  system-wide default for this SQL setting can be set in {\tt
    conf/texis.ini} with the {\tt [Texis] Multi Value To Multi Row}
  setting.  If unset, it defaults to true through version 6 (or
  \verb`compatibilityversion` 6), and false in version 7 and later
  (because in general {\tt GROUP BY}/{\tt DISTINCT} are expected to
  return true table rows for results).  The
  \verb`compatibilityversion` property
  (p.~\pageref{SqlPropertyCompatibilityVersion}), when set, affects
  this property as well.

\item[inmode]
\label{InmodeProperty}
  How the {\tt IN} operator should behave.  If set to \verb`subset`,
  {\tt IN} behaves like the {\tt SUBSET} operator
  (p.~\pageref{SubsetOperator}).  If set to \verb`intersect`, {\tt IN}
  behaves like the {\tt INTERSECT} operator
  (p.~\pageref{IntersectOperator}).  Added in version 7, where the
  default is \verb`subset`.  Note that in version 6 (or
  \verb`compatibilityversion` 6) and earlier, {\tt IN} always behaved
  in an {\tt INTERSECT}-like manner.  The \verb`compatibilityversion`
  property (p.~\pageref{SqlPropertyCompatibilityVersion}), when set,
  affects this property as well.

\item[hexifybytes]
\label{hexifybytesProperty}

  Whether conversion of \verb`byte` to \verb`char` (or vice-versa)
  should encode to (or decode from) hexadecimal.  In Texis version 6
  (or \verb`compatibilityversion` 6) and earlier, this always
  occurred.  In Texis version 7 (or \verb`compatibilityversion` 7) and
  later, it is controllable with the \verb`hexifybytes` SQL property:
  0 for off/as-is, 1 for hexadecimal conversion.  This property is on
  by default in \verb`tsql` (i.e. hex conversion ala version 6 and
  earlier), so that \verb`SELECT`ing from certain system tables that
  contain \verb`byte` columns will still be readable from the command
  line.  However, the property is off by default in version 7 and
  later non-\verb`tsql` programs (such as Vortex), to avoid the hassle
  of hex conversion when raw binary data is needed (e.g. images), and
  because Vortex etc. have more tools for dealing with binary data,
  obviating the need for hex conversion.  (The \verb`hextobin()` and
  \verb`bintohex()` SQL functions may also be useful,
  p.~\pageref{bintohexSqlFunction}.)  The \verb`hexifybytes` property
  was added in version 7.  It is also settable in the
  \verb`conf/texis.ini` config file (p.~\pageref{tcHexifyBytes}).  The
  \verb`compatibilityversion` property
  (p.~\pageref{SqlPropertyCompatibilityVersion}), when set, affects
  this property as well.

\item[unalignedbufferwarning]

  Whether to issue ``{\tt Unaligned buffer}'' warning messages when
  unaligned buffers are encountered in certain situations.  Messages
  are issued if this setting is true/nonzero (the default).  Added in
  version 7.00.1366400000 20130419.

\item[unneededrexescapewarning]
\label{UnneededRexEscapeWarningSqlProperty}

  Whether to issue ``{\tt REX: Unneeded escape sequence ...}''
  warnings when a REX expression uses certain unneeded escapes.  An
  unneeded escape is when a character is escaped that has no special
  meaning in the current context in REX, either alone or escaped.
  Such escapes are interpreted as just the literal character alone
  (respect-case); e.g ``\verb`\w`'' has no special meaning in REX, and
  is taken as ``\verb`w`''.

  While such escapes have no meaning currently, some may take on a
  specific new meaning in a future Texis release, if REX syntax is
  expanded.  Thus using them in an expression now may unexpectedly
  (and silently) result in their behavior changing after a Texis
  update; hence the warning message.  Expressions using such escapes
  should thus have them changed to the unescaped literal character.

  If updating the code is not feasible, the warning may be silenced by
  setting \verb`unneededrexescapewarning` to 0 -- at the risk of
  silent behavior change at an upgrade.  Added in version
  7.06.1465574000 20160610.  Overrides {\tt [Texis] Unneeded REX
    Escape Warning} setting
  (p.~\pageref{TexisIniUnneededREXEscapeWarning}) in {\tt
    conf/texis.ini}.

\item[nulloutputstring]

  The string value to output for SQL NULL values.  The default
  is ``\verb`NULL`''.  Note that this is different than the output
  string for zero-integer \verb`date` values, which are also shown
  as ``\verb`NULL`''.  Added in version 7.02.1405382000 20140714.

\item[validatebtrees]

  Bit flags for additional consistency checks on B-trees.  Added in
  version 7.04.1449078000 20151202.  Overrides
  {\tt [Texis] Validate Btrees} setting
  (p.~\pageref{TexisIniTexisValidateBtrees}) in \verb`conf/texis.ini`.

\end{description}
