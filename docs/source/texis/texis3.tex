\chapter{Intelligent Text Search Queries}{\label{Chp:MMLike}}

This manual has concentrated so far on the manipulation of fields
within a relational database.  As a text information server Texis
provides all the capabilities one would expect from a traditional
RDBMS.

Texis was also designed to incorporate large quantities of narrative
full text stored within the database.  This is evidenced by its data
types as presented in Chapter~\ref{chp:TabDef}, {\em Table
Definition}.  When textual content becomes the focus of the search,
the emphasis shifts from one of document management to one of
research.  Texis has embodied within it special programs geared to
accomplish this.

Metamorph was originally a stand-alone program designed to meet
intelligent full text retrieval needs on full text files.  Since 1986
it has been used in a variety of environments where concept oriented
text searching is desired.  Now within the \verb`LIKE` clause, all of what is
possible on full text information with Metamorph is possible with
Texis, within the framework of a relational database.

Metamorph is covered in a complete sense in its own section in this
manual and can be studied of itself.  Please refer to the {\em Metamorph
Intelligent Text Query Language} sections for a full understanding
of all Metamorph's theory and application.

This chapter deals with the use of Metamorph within the \verb`LIKE` portion
of the \verb`WHERE` clause.  Texis can accomplish any Metamorph search
through the construction of a \verb`SELECT`-\verb`FROM`-\verb`WHERE` block, where the
Metamorph query is enclosed in single quotes \verb`'query'` following
\verb`LIKE`.

\section{Types of Text Query}

There are two primary types of text query that can be performed by Texis.
The first form is used to find those records which match the query, and is
used when you only care if the record does or does not match.  The second
form is used when you want to rank the results and produce the best answers
first.

The first type of query is done with \verb`LIKE`, or \verb`LIKE3` to avoid post processing
if the index can not fully resolve the query.  The ranking queries are done
with \verb`LIKEP` or \verb`LIKER`.  \verb`LIKER` is a faster, and less precise ranking figure
than the one returned by \verb`LIKEP`.  The ranking takes into account how many of
the query terms are present, as well as their weight, and for \verb`LIKEP`, how
close together the words are, and where in the document they occur.  Most
queries will use \verb`LIKE` of \verb`LIKEP`, with \verb`LIKE3` and \verb`LIKER` used in special
circumstances when you want to avoid the post-processing that would fully
resolve the query.

There are also two forms of Metamorph index, the Metamorph inverted index
and the Metamorph index.  The inverted form contains additional information
which allows phrases to be resolved and \verb`LIKEP` rankings to be calculated
entirely using the index.  This improved functionality comes at a cost in
terms of space.

If your queries are single word \verb`LIKE` queries then the Metamorph
index has all the information needed, so the additional overhead of
the inverted index is not needed.

\section{Definition of Metamorph Terms}

\begin{description}
\item[Query:] A Metamorph Query is the question or statement of search
items to be matched in the text within specified beginning and ending
delimiters.  A Query is comprised of one or more search items which
can be of different types, and a ``within delimiter'' specification
which is either understood or expressed.  In Texis the Metamorph Query
refers to what is contained within single quotes \verb`'query'`
following \verb`LIKE` in the \verb`WHERE` clause.

\item[Hit:] A Hit is the text Metamorph retrieves in response to a
query, whose meaning matches the Query to the degree specified.

\item[Search Item:] A Search Item is an English word or a special
expression inside a Metamorph Query.  A word is automatically
processed using certain linguistic rules.  Special searches are
signaled with a special character leading the item, and are governed
respectively by the rules of the pattern matcher invoked.

\item[Set:] A Set is the group of possible strings a pattern matcher
will look for, as specified by the Search Item.  A Set can be a list
of words and word forms, a range of characters or quantities, or some
other class of possible matches based on which pattern matcher
Metamorph uses to process that item.

\item[Intersection:] A portion of text where at least one member of
two Sets each is matched.

\item[Delimiters:] Delimiters are repeating patterns in the text which
define the bounds within which search items are found in proximity to
each other.  These patterns are specified as regular expressions.  A
within operator is used to specify delimiters in a Metamorph Query.

\item[Intersection Quantity:] The number of unions of sets existing
within the specified Delimiters.  The maximum number of Intersections
possible for any given Query is the maximum number of designated Sets
minus one.
\end{description}

Hits can have varying degrees of relevance based on the number of set
intersections occurring within the delimited block of text, definition
of proximity bounds, and weighting of search items for inclusion or
exclusion.

Intersection quantity, Delimiter bounds, and Logic weighting can be
adjusted by the user as part of Metamorph Query specification.

% NOTE: this section title is link to by name; see symlinks.txt and Makefile:
\section{Adjusting Linguistic Controls}

Concept sets can be edited to include special vocabulary, acronyms,
and slang.  There is sufficient vocabulary intelligence off the shelf
so that editing is not required to make good use of the program
immediately upon installation.  However, such customization is
encouraged to keep online research in rapport with users' needs,
especially as search routines and vocabulary evolve.

A word need not be ``known'' by Metamorph for it to be processed.  The
fact of a word having associations stored in the Thesaurus makes
abstraction of concept possible, but is not required to match word
forms.  Such word stemming knowledge is inherent.  And, any string of
characters can be matched exactly as entered.

You can edit the special word lists Metamorph uses to process English
if you wish.  As it may not be immediately apparent to what degree
these word lists may affect general searching, it is cautioned that
such editing be used sparingly and with the wisdom of experience.
Even so, what Metamorph deems to be Noise, Prefixes, and Suffixes is
all under user control.

See the Metamorph portion of this manual for a complete explanation
of all these terms and other background information.

% NOTE: this section title is link to by name; see symlinks.txt and Makefile:
\section{Constructing a Metamorph Query}

The following types of searches are all possible within a Metamorph
Query, as contained in single quotes following the \verb`LIKE` portion of the
\verb`WHERE` clause, in any \verb`SELECT`-\verb`FROM`-\verb`WHERE` block.

\subsection{Keyword Search}

Your search can be as simple as a single word or string.  If you want
references to do with dogs, type in the word ``dog''.

{\bf Example:}
Let's say Corporate Legal Support maintains a table called CODES which
includes full text of the local ordinances of the town in which Acme
Industrial has its headquarters.  The full text field of each
ordinance is stored in a column called BODY.

To find ordinances containing references to dogs, the \verb`WHERE` clause
takes this form:

\begin{verbatim}
     WHERE column-name LIKE 'metamorph-query'
\end{verbatim}

% MAW 09-18-94 - add this paragraph
You can put any Metamorph query in the quotes (\verb`'metamorph-query'`)
although you would need to escape a literal \verb`'` with another
\verb`'` by typing \verb`''`, if you want the character \verb`'` to be
part of the query.

Using Metamorph's defaults, a sentence in the body of the ordinance
text will be sought which contains a match to the query.  Whatever is
dictated in the \verb`SELECT` portion of the statement is what will be
displayed.  All outer logic applies, so that multiple queries can be
sought through use of {\tt AND}, {\tt OR}, {\tt NOT}, and so on.  See
Chapter~\ref{chp:Quer}, especially the section, {\em Additional
Comparison Operators}, for a complete understanding of how the \verb`LIKE`
clause fits into a \verb`SELECT`-\verb`FROM`-\verb`WHERE` statement.

In this example, the \verb`WHERE` clause would look like this:

\begin{verbatim}
     WHERE BODY LIKE 'dog'
\end{verbatim}

When Texis executes the search, ordinances whose bodies contain
matching sentences would be retrieved.  An example of a qualifying
sentence would be:

\begin{verbatim}
     DOG:  any member of the canine family.
\end{verbatim}

And this sentence:

\begin{verbatim}
     It shall be unlawful and a nuisance for any DOG owner to
     permit a dog to be free of restraint in the city.
\end{verbatim}

An English word entered in a Metamorph Query retrieves occurrences of
forms of that word in both lower and upper case, regardless of how it
was entered; i.e., the default keyword search is case insensitive.

Each matched sentence is called a {\em HIT}.  Metamorph locates all such
{\em hits} containing ``dog'' and any other ``dog'' word forms adhering to
the linguistic settings in place.  There would normally be quite a few
hits for a common keyword query like this.

\subsection{Refining a Query}

To refine a query, thereby further qualifying what is judged a hit,
add any other keywords or concepts which should appear within the same
concept grouping.

{\bf Example:}
\begin{verbatim}
     WHERE BODY LIKE 'dog fine'
\end{verbatim}

Fewer hits will be retrieved than when only one search item is entered
(i.e., ``dog''), as you are requiring both ``dog'' and ``fine'' to
occur in the same sentence.  This sentence would qualify:

\begin{verbatim}
     The owner of any DOG who permits such a dog to be free of
     restraint in violation of Section 4.2 of this article shall
     pay a FINE of not less than twenty-five dollars.
\end{verbatim}

You may enter as many query items as you wish, to qualify the hits to
be found.

{\bf Example:}
\begin{verbatim}
     WHERE BODY LIKE 'dog owner vaccination city'
\end{verbatim}

Such a query locates this sentence:

\begin{verbatim}
     Every veterinarian who VACCINATES any cat or DOG within the
     CITY limits shall issue a certification of vaccination to
     such OWNER.
\end{verbatim}

You needn't sift through references which seem too broad or too
numerous.  Refine your query so it produces only what you judge to be
relevant to the goal of your search.

\subsection{Adjusting Proximity Range by Specifying Delimiters}

By default Texis considers the entire field to be a hit when the
full text is retrieved.

If you want your search items to occur within a more tightly
constrained proximity range this can be adjusted.  If you are using
Vortex you will need to allow within operators which are disabled by
default due to the extra processing required.

% MAW 09-18-94 - add w/all w/#
% MAW 10-07-99 - change default from w/sent to w/all (only 3 years late)
Add a ``within'' operator to your query syntax;
``\verb`w/line`'' indicates a line;
``\verb`w/para`'' indicates a paragraph;
``\verb`w/sent`'' indicates a sentence;
 ``\verb`w/all`'' incdicates the entire field;
``\verb`w/#`'' indicates \verb`#` characters.
The default proximity is ``\verb`w/all`''.

{\bf Example:}
Using the legal ordinance text, we are searching the full text bodies
of those ordinances for controls issued about dogs.  The following
query uses sentence proximity to qualify its hits.

\begin{verbatim}
     WHERE BODY LIKE 'dog control w/sent'
\end{verbatim}

This sentence qualifies as a hit because ``control'' and ``dogs'' are
in the same sentence.

\begin{verbatim}
     Ordinances provide that the animal CONTROL officer takes
     possession of DOGS which are free of restraint.
\end{verbatim}

Add a within operator to the Metamorph query to indicate both stated
search items must occur within a single line of text, rather than
within a sentence.

\begin{verbatim}
     WHERE BODY LIKE 'dog control w/line'
\end{verbatim}

The retrieved concept group has changed from a sentence to a line, so
``dog'' and ``control'' must occur in closer proximity to each other.
Now the line, rather than the sentence, is the hit.

\begin{verbatim}
     CONTROL officer takes possession of DOGS
\end{verbatim}

Expanding the proximity range to a paragraph broadens the allowed
distance between located search words.

\begin{verbatim}
     WHERE BODY LIKE 'dog control w/para'
\end{verbatim}

The same query with a different ``within'' operator now locates this
whole paragraph as the hit:

\begin{verbatim}
     The mayor, subject to the approval of the city council,
     shall appoint an animal CONTROL officer who is qualified to
     perform the duties of an animal control officer under the
     laws of this state and the ordinances of the city.  This
     officer shall take possession of any DOG which is free of
     restraint in the city.
\end{verbatim}

The words ``control'' and ``dog'' span different lines and different
sentences, but are within the same paragraph.

These ``within'' operators for designating proximity are also referred
to as delimiters.  Any delimiter can be designed by creating a regular
expression using REX syntax which follows the ``\verb`w/`''.  Anything
following ``\verb`w/`'' that is not one of the previously defined
special delimiters is assumed to be a REX expression.  For example:

\begin{verbatim}
     WHERE BODY LIKE 'dog control w/\RSECTION'
\end{verbatim}

What follows the `\verb`w/`' now is a user designed REX expression for
sections. This would work on text which contained capitalized headers
leading with ``\verb`SECTION`'' at the beginning of each such section
of text.

Delimiters can also be expressed as a number of characters forward and
backwards from the located search items.  For example:

\begin{verbatim}
     WHERE BODY LIKE 'dog control w/500'
\end{verbatim}

In this example ``dog'' and ``control'' must occur within a window of
500 characters forwards and backwards from the first item located.

More often than not the beginning and ending delimiters are the same.
Therefore if you do not specify an ending delimiter (as in the above
example), it will be assumed that the one specified is to be used for
both.  If two expressions are specified, the first will be beginning,
the second will be ending.  Specifying both would be required most
frequently where special types of messages or sections are used which
follow a prescribed format.

Another factor to consider is whether you want the expression defining
the text unit to be included inside that text unit or not.  For
example, the ending delimiter for a sentence obviously belongs with
the hit.  However, the beginning delimiter is really the end of the
last sentence, and therefore should be excluded.

Inclusion or exclusion of beginning and ending delimiters with the hit
has been thought out for the defaults provided with the program.
However, if you are designing your own beginning and ending
expressions, you may wish to specify this.

\subsubsection{Delimiter Syntax Summary}
\begin{verbatim}
        w/{abbreviation}
     or
        w/{number}
     or
        w/{expression}
     or
        W/{expression}
     or
        w/{expression} W/{expression}
     or
        W/{expression} w/{expression}
     or
        w/{expression} w/{expression}
     or
        W/{expression} W/{expression}
\end{verbatim}

\subsubsection{Rules of Delimiter Syntax}

\begin{itemize}

\item The above can be anywhere in a Metamorph query, and is
interpreted as ``within \{the following delimiters\}''.

\item Accepted built-in abbreviations following the slash `\verb`/`'
are:

\begin{table}[h]
\caption{Metamorph delimiter abbreviations}{\label{tab:within}}
\begin{tabbing}
\verb`[^\digit\upper][.?!][\space'"]`xxx \= Meaning \kill
Abbreviation \> Meaning \\
\verb`line`  \> within a line \\
\verb`sent`  \> within a sentence \\
\verb`para`  \> within a paragraph \\
\verb`page`  \> within a page  \\
\verb`all`   \> within a field \\
\verb`NUMBER`\> within NUMBER characters \\
\end{tabbing}

\begin{tabbing}
\verb`[^\digit\upper][.?!][\space'"]`xxx \= Meaning \kill
REX Expression                           \> Meaning   \\
\verb`$`                                 \> 1 new line      \\
\verb`[^\digit\upper][.?!][\space'"]`    \> not a digit or upper case letter, then \\
                                         \> a period, question, or exclamation point, then \\
                                         \> any space character, single or double quote  \\
\verb`\x0a=\space+`                      \> a new line + some space  \\
\verb`\x0c`                              \> form feed for printer output \\
\end{tabbing}
\end{table}

\item A number following a slash `\verb`/`' means the number of
characters before and after the first search item found.  Therefore
``\verb`w/250`'' means ``within a proximity of 250 characters''.  When
the first occurrence of a valid search item is found, a window of 250
characters in either direction will be used to determine if it is a
valid hit.  The implied REX expression is:  ``\verb`.{,250}`''
meaning ``250 of any character''.

\item If what follows the slash `\verb`/`' is not recognized as a
built-in, it is assumed that what follows is a REX expression.

\item If one expression only is present, it will be used for both
beginning and ending delimiter.  If two expressions are present, the
first is the beginning delimiter, the second the ending delimiter.
The exception is within-$N$ (e.g. ``\verb`w/250`''), which always
specifies both start and end delimiters, overriding any preceding
``\verb`w/`''.

\item The use of a small `\verb`w`' means to exclude the delimiters
from the hit.

\item The use of a capital `\verb`W`' means to include the delimiters
in the hit.

\item Designate small `\verb`w`' and capital `\verb`W`' to exclude
beginning delimiter, and include ending delimiter, or vice versa.
Note that for within-$N$ queries (e.g. ``\verb`w/250`''), the
``delimiter'' is effectively always included in the hit, regardless
of the case of the \verb`w`.

\item If the same expression is to be used, the expression need not be
repeated.  Example:  ``\verb`w/[.?!] W/`'' means to use an ending
punctuation character as both beginning and end delimiter, but to
exclude the beginning delimiter from the hit, and include the end
delimiter in the hit.
\end{itemize}

\subsection{Using Set Logic to Weight Search Items}

\subsubsection{Set Logic and Intersections Defined}

Any search item entered in a query can be weighted for determination
as to what qualifies as a hit.

All search items indicate to the program a set of possibilities to be
found.  A keyword is a set of valid derivations of that word's root
(morpheme).  A concept set includes a list of equivalent meaning
words.  A special expression includes a range of strings that could be
matched.

Therefore, whatever weighting applies to a search item applies to the
whole set, and is referred to as ``set logic''.

The most usual logic in use is ``AND'' logic.  Where no other
weighting is given, it is understood that all entered search items
have equal weight, and you want each one to occur in the targeted hit.

Here is an example of a typical query, where no special weighting has
been assigned:

\begin{verbatim}
     WHERE BODY LIKE 'mayor powers duties city'
\end{verbatim}

The query equally weights each item, and searches for a sentence
containing ``mayor'' and ``powers'' and ``duties'' and ``city''
anywhere within it, finding this sentence:

\begin{verbatim}
     In the case of absence from the CITY or the failure,
     inability or refusal of both the MAYOR and mayor pro tempore
     to perform the DUTIES of mayor, the city council may elect
     an acting mayor pro tempore, who shall serve as mayor with
     all the POWERS, privileges, and duties.
\end{verbatim}

Only those words required to qualify the sentence as a hit are located
by the program, for maximum search efficiency.

In this example, there are several occurrences of the search items
``mayor'', ``duties'', and ``city''.  It was only necessary to locate
each item once to confirm validity of the hit.  Such words may be
found by the search program in any order.

The existence of more than one matched search item in a hit is called
an intersection.  Specifying two keywords in a query indicates you
want both keywords to occur, or intersect, in the sentence.

A 2 item search is common, and can be thought of as 1 intersection of
2 sets.

{\bf Example:}
\begin{verbatim}
     WHERE BODY LIKE '~alcohol ~consumption'
\end{verbatim}

In the above example, the tilde (\verb`~`) preceding ``alcohol'' and
preceding ``consumption'' enables concept expansion on both words,
thereby including the set of associations listed for each word in the
Thesaurus.

Where something from the concept set ``alcohol'' and something from
the concept set ``consumption'' meet within a sentence, there is a
hit.  This default set logic finds a 1 intersection sentence:

\begin{verbatim}
     It shall be unlawful to USE the city swimming pool or enter
     the enclosure in which it is located when a person is
     INTOXICATED or under the influence of illegal drugs.
\end{verbatim}

``Use'' is in the ``consumption'' concept set; ``intoxicated'' is in
the ``alcohol'' concept set.

These two sets have herein intersected, forcing the context of the set
members to be relevant to the entered query.

\subsubsection{Maximum Intersections Possible (``AND'')}

Adding a search item dictates stricter relevance requirements.  Here,
a sentence has to contain 2 intersections of 3 search items to be
deemed a valid hit.

{\bf Example:}
\begin{verbatim}
     WHERE BODY LIKE '~alcohol ~sweets ~consumption'
\end{verbatim}

Such a 2 intersection search finds this hit:

\begin{verbatim}
     any public sleeping or EATING place, or any place or vehicle
     where food or DRINK is manufactured, prepared, stored, or
     any manufacturer or vendor of CANDIES or manufactured
     sweets.
\end{verbatim}

Default intersection logic is to find the maximum number of set
intersections possible in the stated query; that is, an ``and'' search
where an intersection of all search items is required.

\subsubsection{Specifying Fewer Intersections}

The casual user can use the ``AND'' logic default.  Even so, here is
another way to write the above 2 intersection query, where the number
of desired intersections (2) is preceded by the at sign (\verb`@`):

{\bf Example:}
\begin{verbatim}
     WHERE BODY LIKE '~alcohol ~sweets ~consumption @2'
\end{verbatim}

The ``\verb`@2`'' designation is redundant as it is understood by the
program to be the default maximum number of intersections possible,
but it would yield the same results.

It is possible to find different permutations of which items must
occur inside a hit.  Even where the maximum number of intersections
possible is being sought, this is still seen as a permutation, and is
referred to as {\em permuted} logic.

The meaning of ``{\em permuted}'' takes on more significance when
fewer intersections of items are desired.

If you wanted only one intersection of these three items, it would
create an interesting range of possibilities.  You might find an
intersection of any of the following combinations:

\begin{verbatim}
     alcohol (AND) sweets
     alcohol (AND) consumption
     sweets  (AND) consumption
\end{verbatim}

Specify one intersection only (\verb`@1`), while listing the 3
possible items.

{\bf Example:}
\begin{verbatim}
     WHERE BODY LIKE '~alcohol ~sweets ~consumption @1'
\end{verbatim}

This 1 intersection search finds the following, where any 2
occurrences from the 3 specified sets occur within the hit.  Hits for
a higher intersection number (\verb`@2`) as shown above also appear.

\begin{verbatim}
  ~consumption (and) ~alcohol @1
     It shall be unlawful to USE the city swimming pool or enter
     the enclosure in which it is located when a person is
     INTOXICATED or under the influence of illegal drugs.

  ~consumption (and) ~sweets @1
     any public sleeping or EATING place, or any place or vehicle
     where food or drink is manufactured, prepared, stored, or
     any manufacturer or vendor of CANDIES or manufactured
     sweets.

  ~sweets (and) ~alcohol @1
     subject to inspection are:  Bakery or CONFECTIONERY shop
     (retail), Beverage sale ALCOHOLIC ...

  ~consumption (and) ~alcohol @1
     A new EATING and DRINKING establishment shall be one which
     is newly erected or constructed at a given location.

  ~alcohol (and) ~consumption @1
     involving the sale of spirituous, vinous, or malt LIQUORS,
     including beer in unbroken packages for off-premises
     CONSUMPTION.
\end{verbatim}

The ``\verb`@#`'' intersection quantity designation is not position
dependent; it can be entered anywhere in the Metamorph query.

Any number of intersections may be specified, provided that number
does not exceed the number of intersections possible for the entered
number of search items.

\subsubsection{Specifying No Intersections (``OR'')}

Using this intersection quantity model, what is commonly understood to
be an ``OR'' search is any search which requires no (zero)
intersections at all.  In an ``or'' search, any occurrence of any item
listed qualifies as a hit; the item need not intersect with any other
item.

Designate an ``or'' search using the same intersection quantity
syntax, where zero (\verb`0`) indicates no intersections are required
(\verb`@0`):

{\bf Example:}
\begin{verbatim}
     WHERE BODY LIKE '~alcohol ~sweets ~consumption @0'
\end{verbatim}

In addition to the hits listed above for a higher number of
intersections, the following 0 intersection hits would be found, due
to the presence of only one item (a or b or c) required:

\begin{verbatim}
  ~alcohol @0
     Every person licensed to sell LIQUOR, wine or beer or mixed
     beverages in the city under the Alcoholic Beverage Code
     shall ...

  ~sweets @0
     An establishment preparing and selling at retail on the
     premises, cakes, pastry, CANDIES, breads and similar food
     items.

  ~consumption @0
     To regulate the disposal and prohibit the BURNING of garbage
     and trash; ...
\end{verbatim}

All such items are considered {\em permuted}, at intersection number
zero (0).

\subsubsection{Weighting Items for Precedence (+)}

Intersection logic treats all search items as equal to each other,
regardless of the number of understood or specified intersections.
You can indicate a precedence for a particular search item which falls
outside the intersection quantity setting.

A common example is where you are interested chiefly in one subject,
but you want to see occurrences of that subject in proximity to one or
more of several specified choices.  This would be an ``or'' search in
conjunction with one item marked for precedence.  You definitely want
A, along with either B, or C, or D.

Use the plus sign (\verb`+`) to mark search items for mandatory
inclusion.  Use \verb`@0` to signify no intersections are required of
the unmarked permuted items.  The number of intersections required as
specified by `\verb`@#`' will apply to those permuted items remaining.

{\bf Example:}
\begin{verbatim}
     WHERE BODY LIKE '+license plumbing alcohol taxes @0'
\end{verbatim}

This search requires (\verb`+`) the occurrence of ``license'', which
must be found in the same sentence with either ``plumbing'',
``alcohol'', or ``taxes''.

The 0 intersection designation applies only to the unmarked permuted
sets.  Since ``license'' is weighted with a plus (\verb`+`), the
``\verb`@0`'' designation applies to the other search items only.

This query finds the following hits:

\begin{verbatim}
  +license (and) @0 alcohol
     Every person licensed to sell liquor, wine or beer or mixed
     beverages in the city under the ALCOHOLIC Beverage Code
     shall pay to the city a LICENSE fee equal to the maximum
     allowed as provided for in the Alcoholic Beverage Code.

  +license (and) @0 plumbing
     Before any person, firm or corporation shall engage in the
     PLUMBING business within the city, he shall be qualified as
     set forth herein, and a LICENSE shall be obtained from the
     State Board of Plumbing Examiners as required.

  +license (and) @0 taxes
     The city may assess, levy and collect any and all character
     of TAXES for general and special purposes on all subjects or
     objects, including occupation taxes, LICENSE taxes and
     excise taxes.
\end{verbatim}

More than one search item may be marked with a plus (\verb`+`) for
inclusion, and any valid intersection quantity (\verb`@#`) may be used
to refer to the other unmarked items.  Any search item, including
phrases and special expressions, may be weighted for precedence in
this fashion.

\subsubsection{Marking Items for Exclusion (``NOT'') (-)}

You can exclude a hit due to the presence of one or more search items.
Such mandatory exclusion logic for a particular search item falls
outside the intersection quantity setting, as does inclusion, and
applies to the whole set in the same manner.  This is sometimes
thought of as ``NOT'' logic, designated with a minus sign (\verb`-`).

A common example is where one item is very frequently used in the
text, so you wish to rule out any hits where it occurs.  You want an
intersection of A and B, but not if C is present.

Use the minus sign (\verb`-`) to mark search items for exclusion.
Default or specified intersection quantities apply to items not marked
with a plus (\verb`+`) or minus (\verb`-`).  The number of
intersections required will apply to the remaining permuted items.

{\bf Example:}
\begin{verbatim}
     WHERE BODY LIKE 'license ~alcohol -drink'
\end{verbatim}

This search has the goal of finding licensing issues surrounding
alcohol.  However, the presence of the word ``drink'' might
incorrectly produce references about restaurants that do not serve
alcohol.

Excluding the hit if it contains ``drink'' retrieves these hits:

\begin{verbatim}
 license (and) ~alcohol -drink
     Every person licensed to sell LIQUOR, wine or beer or mixed
     beverages in the city shall pay to the city a LICENSE fee
     equal to the maximum allowed ...

 license (and) ~alcohol -drink
     State law reference(s)--Authority to levy and collect
     LICENSE fee, V.T.C.A., ALCOHOLIC Beverage Code 11.38, 61.36.
\end{verbatim}

But excludes this hit:

\begin{verbatim}
 license (and) ~alcohol -drink  {Excluded Hit}
     The city council shall have the power to regulate, LICENSE
     and inspect persons, firms, or associations operating,
     managing, or conducting any place where food or DRINK is
     manufactured, prepared, or otherwise handled within city
     limits.
\end{verbatim}

More than one search item may be marked with a minus (\verb`-`) for
exclusion, along with items marked with plus (\verb`+`) for inclusion,
and any valid intersection quantity (\verb`@#`) specification.  Any
search item, including phrases and special expressions, may be marked
for exclusion in this fashion.

\subsubsection{Combinatorial Logic}

Weighting search items for inclusion (\verb`+`) or exclusion
(\verb`-`) along with an intersection specification which is less than
the maximum quantity possible (the default ``AND'' search) can be used
in any combination.

Default logic is adequate for the casual user to get very satisfactory
search results with no special markings, so this extra syntax need not
be learned.  However, such syntax is available to the more exacting
user if desired.

A rather complicated but precise query might make use of weighting for
inclusion, exclusion, and also a specified intersection quantity, as
follows.

{\bf Example:}
\begin{verbatim}
     WHERE BODY LIKE '+officer @1 law power pay duties -mayor'
\end{verbatim}

The above query makes these requirements:

\begin{itemize}
\item ``Officer'' must be present, plus \ldots

\item 1 intersection of any 2 of the unmarked search items ``law'',
``power'', ``pay'', ``duties'', but \ldots

\item Not if ``mayor'' is present (i.e., exclude it).
\end{itemize}

This query retrieves the following hits, while excluding hits
containing ``mayor''.

\begin{verbatim}
  power, duties +officer (but not) -mayor
     The city council shall have POWER from time to time to
     require other further DUTIES of all OFFICERS whose duties
     are herein prescribed.

  law, duties +officer (but not) -mayor
     Proof shall be made before some OFFICER authorized by the
     LAW to administer oaths, and filed with the person
     performing the DUTIES of city secretary.

  law, power +officer (but not) -mayor
     In case of any irreconcilable conflict between the
     provisions of this Charter and any superior LAW, the POWERS
     of the city and its OFFICERS shall be defined in such
     superior laws.

  duties, law +officer (but not) -mayor
     The plan must be designed to enable the records management
     OFFICER to carry out his or her DUTIES prescribed by state
     LAW and this article effectively.

  pay, duties +officer (but not) -mayor
     PAYMENT of firefighting OFFICIAL performing DUTIES outside
     of territorial limits of city.

  power, duties +officer (but not) -mayor
     POWERS and DUTIES of OFFICIAL assigned to assist in the
     city.
\end{verbatim}

Any search item, including keywords, wildcards, concept searches,
phrases, and special expressions, can be weighted for inclusion,
exclusion, and combinatorial set logic.

\subsubsection{Combinatorial Logic and {\tt LIKER}}

When using \verb`LIKER` the default weighting of the terms has two factors.
The first is based on the words location in the query, with the most
important term first.  The second is based on word frequency in the
document set, where common words have a lesser importance attached
to them.  The logic operators (+ and -) remove the second factor.
Additionally the not operator (-) negates the first factor.

\subsubsection{Metamorph Logic Rules Summary}

Logic operators apply to any entered search item.

\begin{enumerate}
\item Precedence (mandatory inclusion) is expressed by a plus sign
(\verb`+`) preceding the concept or expression.  A plus (\verb`+`)
item is ``Required''.

\item Exclusion (``not'' logic) is expressed by a minus sign
(\verb`-`) preceding the concept or expression.  A minus (\verb`-`)
item is ``Excluded''.

\item Equivalence (``and'' logic) may be expressed by an equal sign
(\verb`=`) preceding the concept or expression.  An equal sign is
assumed where no other logic operator is assigned.  An equal
(\verb`=`) item is ``Permuted''.

\item Search items not marked with (\verb`+`) or (\verb`-`) are
considered to be equally weighted.  Intersection quantity logic
(\verb`@#`) applies to these unmarked (\verb`=`) permuted sets only.

\item Where search items are not otherwise marked, the default set
logic in use is ``And'' logic.  The maximum number of intersections
possible is sought.

\item Designate ``Or'' logic with zero intersections (\verb`@0`),
applying to any unmarked permuted search items.

\item Logic operators and intersection quantity settings can be used
in combination with each other, referred to as combinatorial logic.
\end{enumerate}

Metamorph's use of logic should not be confused with Boolean
operators.  Metamorph deals with these logic operators as sets rather
than single strings, a different methodology.

\section{Other Metamorph Features}

Metamorph contains many special kinds of searches.  Again, any
Metamorph search can be constructed as part of the \verb`LIKE` clause.
See the Metamorph section of this manual for a complete treatment
of this subject.
