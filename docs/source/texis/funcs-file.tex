% ----------------------------------------------------------------------------
\subsection{File functions}

% - - - - - - - - - - - - - - - - - - - - - - - - - - - - - - - - - - - - - -
\subsubsection{fromfile, fromfiletext}

The \verb`fromfile` and \verb`fromfiletext` functions read a file.
The syntax is
\begin{verbatim}
   fromfile(filename[, offset[, length]])
   fromfiletext(filename[, offset[, length]])
\end{verbatim}

These functions take one required, and two optional arguments.  The
first argument is the filename.  The second argument is an offset into
the file, and the third argument is the length of data to read.  If the
second argument is omitted then the file will be read from the
beginning.  If the third argument is omitted then the file will be read
to the end.  The result is the contents of the file.  This can be used
to load data into a table.  For example if you have an indirect field
and you wish to see the contents of the file you can issue SQL similar
to the following.

The difference between the two functions is the type of data that is
returned.  \verb`fromfile` will return varbyte data, and
\verb`fromfiletext` will return varchar data.  If you are using the
functions to insert data into a field you should make sure that you
use the appropriate function for the type of field you are inserting
into.

\begin{verbatim}
     SELECT  FILENAME, fromfiletext(FILENAME)
     FROM    DOCUMENTS
     WHERE   DOCID = 'JT09113' ;
\end{verbatim}

The results are:

\begin{screen}
\begin{verbatim}
  FILENAME            fromfiletext(FILENAME)
  /docs/JT09113.txt   This is the text contained in the document
  that has an id of JT09113.
\end{verbatim}
\end{screen}

% - - - - - - - - - - - - - - - - - - - - - - - - - - - - - - - - - - - - - -
\subsubsection{totext}

Converts data or file to text.  The syntax is
\begin{verbatim}
   totext(filename[, args])
   totext(data[, args])
\end{verbatim}

This function will convert the contents of a file, if the argument given is
an indirect, or else the result of the expression, and convert it to text.
It does this by calling the program \verb'anytotx', which must be in the
path.  The \verb`anytotx` program (obtained from Thunderstone) will handle
\verb`PDF` as well as many other file formats.

As of version 2.06.935767000 the \verb`totext` command will take an optional
second argument which contains arguments to the \verb`anytotx` program.
See the documentation for \verb`anytotx` for details on its arguments.

\begin{verbatim}
     SELECT  FILENAME, totext(FILENAME)
     FROM    DOCUMENTS
     WHERE   DOCID = 'JT09113' ;
\end{verbatim}

The results are:

\begin{screen}
\begin{verbatim}
  FILENAME            totext(FILENAME)
  /docs/JT09113.pdf   This is the text contained in the document
  that has an id of JT09113.
\end{verbatim}
\end{screen}

% - - - - - - - - - - - - - - - - - - - - - - - - - - - - - - - - - - - - - -
\subsubsection{toind}

Create a Texis managed indirect file.  The syntax is
\begin{verbatim}
   toind(data)
\end{verbatim}

This function takes the argument, stores it into a file, and returns the
filename as an \verb`indirect` type.  This is most often used in combination
with \verb`fromfile` to create a Texis managed file.  For example:

\begin{verbatim}
     INSERT  INTO DOCUMENTS
     VALUES('JT09114', toind(fromfile('srcfile')))
\end{verbatim}

The database will now contain a pointer to a copy of \verb|srcfile|, which
will remain searchable even if the original is changed or removed.  An
important point to note is that any changes to \verb|srcfile| will not be
reflected in the database, unless the table row's \verb`indirect` column
is modified (even to the save value, this just tells Texis to re-index it).

% - - - - - - - - - - - - - - - - - - - - - - - - - - - - - - - - - - - - - -
\subsubsection{canonpath}

  Returns canonical version of a file path, i.e. fully-qualified and
without symbolic links:

\begin{verbatim}
  canonpath(path[, flags])
\end{verbatim}

The optional \verb`flags` is a set of bit flags: bit 0 set if error
messages should be issued, bit 1 set if the return value should be
empty instead of \verb`path` on error.  Added in version
5.01.1139446515 20060208.

% - - - - - - - - - - - - - - - - - - - - - - - - - - - - - - - - - - - - - -
\subsubsection{pathcmp}

  File path comparison function; like C function \verb`strcmp()` but
for paths:

\begin{verbatim}
  pathcmp(pathA, pathB)
\end{verbatim}

Returns an integer indicating the sort order of \verb`pathA` relative
to \verb`pathB`: 0 if \verb`pathA` is the same as \verb`pathB`, less
than 0 if \verb`pathA` is less than \verb`pathB`, greater than 0 if
\verb`pathA` is greater than \verb`pathB`.  Paths are compared
case-insensitively if and only if the OS is case-insensitive for
paths, and OS-specific alternate directory separators are considered
the same (e.g. ``\verb`\`'' and ``\verb`/`'' in Windows).  Multiple
consecutive directory separators are considered the same as one.  A
trailing directory separator (if not also a leading separator) is
ignored.  Directory separators sort lexically before any other
character.

  Note that the paths are only compared lexically: no attempt is made
to resolve symbolic links, ``{\tt ..}'' path components, etc.  Note
also that no inference should be made about the magnitude of negative
or positive return values: greater magnitude does not necessarily
indicate greater lexical ``separation'', nor should it be assumed that
comparing the same two paths will always yield the same-magnitude
value in future versions.  Only the sign of the return value is
significant.  Added in version 5.01.1139446515 20060208.

% - - - - - - - - - - - - - - - - - - - - - - - - - - - - - - - - - - - - - -
\subsubsection{basename}

  Returns the base filename of a given file path.

\begin{verbatim}
  basename(path)
\end{verbatim}

The basename is the contents of \verb`path` after the last path separator.
No filesystem checks are performed, as this is a text/parsing function;
thus ``\verb`.`'' and ``\verb`..`'' are not significant.
Added in version 7.00.1352510000 20121109.

% - - - - - - - - - - - - - - - - - - - - - - - - - - - - - - - - - - - - - -
\subsubsection{dirname}

  Returns the directory part of a given file path.

\begin{verbatim}
  dirname(path)
\end{verbatim}

The directory is the contents of \verb`path` before the last path separator
(unless it is significant -- e.g. for the root directory -- in which case
it is retained).  Added in version 7.00.1352510000 20121109.
No filesystem checks are performed, as this is a text/parsing function;
thus ``\verb`.`'' and ``\verb`..`'' are not significant.

% - - - - - - - - - - - - - - - - - - - - - - - - - - - - - - - - - - - - - -
\subsubsection{fileext}

  Returns the file extension of a given file path.

\begin{verbatim}
  fileext(path)
\end{verbatim}

  The file extension starts with and includes a dot.  The file extension
is only considered present in the basename of the path, i.e. after the
last path separator.  Added in version 7.00.1352510000 20121109.

% - - - - - - - - - - - - - - - - - - - - - - - - - - - - - - - - - - - - - -
\subsubsection{joinpath}

Joins one or more file/directory path arguments into a merged path,
inserting/removing a path separator between arguments as needed.
Takes one to 5 path component arguments.  E.g.:

\begin{verbatim}
  joinpath('one', 'two/', '/three/four', 'five')
\end{verbatim}

yields

\begin{verbatim}
  one/two/three/four/five
\end{verbatim}

Added in version 7.00.1352770000 20121112.  Redundant path separators
internal to an argument are not removed, nor are ``{\tt .}'' and ``{\tt
  ..}'' path components removed.  Prior to version 7.07.1550082000
20190213 redundant path separators between arguments were not
removed.

% - - - - - - - - - - - - - - - - - - - - - - - - - - - - - - - - - - - - - -
\subsubsection{joinpathabsolute}

Like \verb`joinpath`, except that a second or later argument that is
an absolute path will overwrite the previously-merged path.  E.g.:

\begin{verbatim}
  joinpathabsolute('one', 'two', '/three/four', 'five')
\end{verbatim}

yields

\begin{verbatim}
  /three/four/five
\end{verbatim}

Under Windows, partially absolute path arguments -- e.g. ``{\tt
  /dir}'' or ``{\tt C:dir}'' where the drive or dir is still relative
-- are considered absolute for the sake of overwriting the merge.

Added in version 7.00.1352770000 20121112.  Redundant path separators
internal to an argument are not removed, nor are ``{\tt .}'' and
``{\tt ..}'' path components removed.  Prior to version
7.07.1550082000 20190213 partially absolute arguments were not
considered absolute.

% ----------------------------------------------------------------------------
