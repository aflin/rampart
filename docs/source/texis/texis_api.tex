%% MAW 09-18-94 removed blob refs, brought rest up to date
%**********************************************************************/

%\documentstyle[epiman,12pt]{book}
%\begin{document}           % End of preamble and beginning of text.

%\title{Network Texis API Documentation}
%\tableofcontents
%\PART{The Texis API}
%\chapter{EMBEDDING Texis}
\chapter{The Texis Client API}{\label{Part:V:Chp:tcapi}}{\label{Part:V:Chp:Embed}}
\section{Overview}
This chapter provides a reference to the functions that you will need to
use to write your own custom application that talks to Texis.

\NAME{Texis Client Functions}
\SYNOPSIS

{\bf SQL interface}
\begin{verbatim}
TEXIS *texis_open(char *database, char *user, char *password);
TEXIS *texis_open_options(void *, void *, void *, char *database, char *user,
                          char *group, char *password);
int texis_set(TEXIS *, char *, char *);
int texis_prepare(TEXIS *, char *sqlQuery);
int texis_execute(TEXIS *);
int texis_param(TEXIS *, int paramNum, void *data, long *dataLen,
                int cType, int sqlType);
int texis_resetparams(TEXIS *);
int texis_cancel(TEXIS *);
FLDLST *texis_fetch(TEXIS *, int stringsFrom);
int texis_flush(TEXIS *tx);
int texis_flush_scroll(TEXIS *tx, int nrows);

char **texis_getresultnames(TEXIS *tx);
int *texis_getresultsizes(TEXIS *tx);
int texis_getrowcount(TEXIS *tx);
int texis_getCountInfo(TEXIS *tx, TXCOUNTINFO *countInfo);
TEXIS *texis_close(TEXIS *);
FLDLST *freefldlst(FLDLST *fl);

\end{verbatim}

\DESCRIPTION

The Texis API provides the client program with an easy to use interface to
any Texis server.

The programmer is {\em strongly} encouraged to play with the example
programs provided with the package before attempting their own
implementation.

\begin{description}
\item[SQL interface]

    This group of operations is for passing a user's query to the
    database for processing and subsequently obtaining the results.

\end{description}


%**********************************************************************/
\NAME{FLDLST - Struct containing results}

\SYNOPSIS
\begin{verbatim}
  typedef struct FLDLST
  {
     int      n;                 /* number of items in arrays */
     int      type [FLDLSTMAX];  /* data type */
     void    *data [FLDLSTMAX];  /* pointer to data */
     int      ndata[FLDLSTMAX];  /* number of els in data */
     char    *name [FLDLSTMAX];  /* name of field */
     int      ondata[FLDLSTMAX]; /* 2002-07-05 number of els in schema */
  } FLDLST;
\end{verbatim}

A \verb`FLDLST` is a structure containing parallel arrays of information
about the selected fields and a count of those fields. \verb`FLDLST` members:

\verb`int n` --- The number of fields in the following lists.

\verb`int *type` --- An array of types of the fields.  Each element will
be one of the \verb`FTN_xxxx` macros.

\verb`void **data` --- An array of data pointers.  Each element will
point to the data for the field. The data for a field is an
array of \verb`type`s.

\verb`int *ndata` ---  An array of data counts. Each element says how
many elements are in the \verb`data` array for this field.

\verb`char **name` --- An array of strings containing the names of the
fields.

\verb`SRCHLST *sl` --- An array of SRCHLSTs containing information
Metamorph searches, if any. Filled in, on request only,
by \verb`n_fillsrchlst()`.

\verb`MMOFFS mmoff` --- An array of Metamorph subhit offsets and lengths
from Metamorph searches, if any. Filled in, on request only,
by \verb`n_fillsrchlst()`.

Possible types in \verb`FLDLST->type` array:

\verb`FTN_BYTE` --- An 8 bit \verb`byte`.

\verb`FTN_CHAR` --- A \verb`char`.

\verb`FTN_DOUBLE` --- A \verb`double`.

\verb`FTN_DATE` --- A \verb`long` in the same form as that from the
\verb`time()` system call.

\verb`FTN_DWORD` --- A 32 bit \verb`dword`.

\verb`FTN_FLOAT` --- A \verb`float`.

\verb`FTN_INT` --- An \verb`int`.

\verb`FTN_INTEGER` --- An \verb`int`.

\verb`FTN_LONG` --- A \verb`long`.

\verb`FTN_INT64` --- An \verb`int64`.

\verb`FTN_UINT64` --- A \verb`uint64`.

\verb`FTN_SHORT` --- A \verb`short`.

\verb`FTN_SMALLINT` --- A \verb`short`.

\verb`FTN_WORD` --- A 16 bit \verb`word`.

\verb`FTN_INDIRECT` --- A \verb`char` string URL indicating the file that
the data for this field is stored in.

\verb`FTN_COUNTER` --- A \verb`ft_counter` structure containing a unique serial
number.

\verb`FTN_STRLST` --- A delimited list of strings.

The \verb`type` may also be \verb`|`ed with \verb`FTN_NotNullableFlag`
(formerly \verb`DDNULLBIT`) and/or \verb`DDVARBIT`.
If \verb`FTN_NotNullableFlag` is set, it indicates that the field is not
allowed to be NULL. \verb`DDVARBIT` indicated that the field is
variable length instead of fixed length. When handling result rows
these bits can generally be ignored.

%**********************************************************************/

\NAME{TXCOUNTINFO - Struct containing idnex count info}

\SYNOPSIS
\begin{verbatim}
  typedef struct TXCOUNTINFO
  {
    EPI_HUGEINT	rowsMatchedMin;
    EPI_HUGEINT	rowsMatchedMax;
    EPI_HUGEINT	rowsReturnedMin;
    EPI_HUGEINT	rowsReturnedMax;
    EPI_HUGEINT	indexCount;
  } TXCOUNTINFO;
\end{verbatim}

\DESCRIPTION
The \verb`TXCOUNTINFO` struct contains information about the min/max
number of table rows matching the query.
%
The \verb`rowsMatchedMin`
and \verb`rowsMatchedMax` members are before
\begin{itemize}
  \item GROUP BY
  \item likeprows
  \item aggregates (count(*))
  \item multivaluetomultirow
\end{itemize}

If the number is unknown the result will be less than 0.  If
\verb`rowsMatchedMin` and \verb`rowsMatchedMax` are different then
the exact count is unknown.

The \verb`rowsReturnedMin`
and \verb`rowsReturnedMax` members indicate the number of rows that
would be returned by \verb`texis_fetch` and are after
\begin{itemize}
  \item GROUP BY
  \item likeprows
  \item aggregates (count(*))
  \item multivaluetomultirow
\end{itemize}

If the number is unknown the result will be less than 0.  If
\verb`rowsReturnedMin` and \verb`rowsReturnedMax` are different then
the exact count is unknown.


%**********************************************************************/

\NAME{texis\_getCountInfo() - Hit count information}

\SYNOPSIS
\begin{verbatim}
#include "texisapi.h"
int texis_getCountInfo(TEXIS *tx, TXCOUNTINFO *countinfo);
\end{verbatim}

\DESCRIPTION

This function will return the count of rows to be read from the index
for the prepared statement, if available.

\CAVEATS

This function populates a \verb`TXCOUNTINFO` structure with information
about the number of rows expected to be returned.  A range of rows may be
provided if rows may be filtered as rows are returned, for example a
combination of indexed and unindexed clauses.

The lower and upper bounds of the array will be updated as rows are returned
and they may converge to the same number.

If the bound is negative that means it is unknown, for example if no index
is being used.

%**********************************************************************/

\NAME{texis\_open(), texis\_dup(), texis\_close() - SQL interface}
\SYNOPSIS
\begin{verbatim}
TEXIS *texis_open(char *database, char *user, char *password);
TEXIS *texis_dup(TEXIS *tx);
TEXIS *texis_close(TEXIS *tx);
\end{verbatim}

\DESCRIPTION

\verb`texis_open()` opens the database. It returns a valid TEXIS pointer
on success or \verb`TEXISPN` on failure. \verb`TEXISPN` is an alias for
\verb`(TEXIS *)NULL`.

\verb`texis_dup()` creates a new TEXIS pointer to the same database as a
currently valid handle.  This saves much of the overhead of opening
a new connection to the database.  The returned handle is a clean
TEXIS handle, and will not have a copy of the SQL statement from the
copied handle.
It returns a valid TEXIS pointer on success or
\verb`TEXISPN` on failure. \verb`TEXISPN` is an alias for \verb`(TEXIS *)NULL`.

\verb`texis_close()` closes the previously opened database.  It always
returns \verb`TEXISPN`.

SQL statements are setup and executed with \verb`texis_prepare()`,
\verb`texis_execute()`, and \verb`texis_fetch()`.

\SEE
\begin{verbatim}
texis_prepare(), texis_execute(), and texis_fetch()
\end{verbatim}

%**********************************************************************/

\NAME{texis\_set() - Set properties}
\SYNOPSIS
\begin{verbatim}
int texis_set(TEXIS *, char *property, char *value);
\end{verbatim}

\DESCRIPTION

Equivalent to the the SQL {\tt SET} statement for setting server properties
(p.~\pageref{ServerProperties})

%**********************************************************************/

\NAME{texis\_prepare(), texis\_execute(), texis\_prepexec() - SQL interface}
\SYNOPSIS
\begin{verbatim}
int texis_prepare(TEXIS *, char *sqlQuery);
int texis_prepexec(TEXIS *, char *sqlQuery);
int texis_execute(TEXIS *);
\end{verbatim}

\DESCRIPTION

These functions perform SQL statement setup and execution for databases
opened with \verb`texis_open()`. \verb`texis_prepare()` takes a \verb`TEXIS`
pointer from \verb`texis_open()` and a SQL statement.

These functions provide an efficient way to perform the same
SQL statement multiple times with varying parameter data.
\verb`texis_prepare()` will return 1 on success, 0 on failure.

\verb`texis_execute()` and \verb`texis_fetch()` or \verb`texis_flush()` would
then be called to handle the results of the statement as with \verb`texis_prepare()`.

Once a SQL statement is prepared with \verb`texis_prepare()` it may be
executed multiple times with \verb`texis_execute()`. Typically the parameter
data is changed between executions using the \verb`texis_param()` function.

\verb`texis_execute()` will start to execute the statement prepared with
\verb`texis_prepare()`.  \verb`texis_execute()` will return 1
on success, 0 on failure.

Parameters should be set between \verb`texis_prepare()` and
\verb`texis_execute()`.  If there are no parameters you can also use
\verb`texis_prepexec()`, which will call \verb`texis_prepare()` and
\verb`texis_execute()` in one call.

\EXAMPLE
\begin{verbatim}
SERVER *se;
TEXIS     *tx;
long    date;
char   *title;
char   *article;
int     tlen, alen, dlen;

   ...
   if(!texis_prepare(tx,"insert into docs values(counter,?,?,?);"))
      { puts("texis_prepare Failed"); return(0); }
   for( each record to insert )
   {
      ...
      date=...
      dlen=sizeof(date);
      title=...
      tlen=strlen(title);
      article=...
      alen=strlen(article);
      if(!texis_param(tx,1,&date  ,&dlen,SQL_C_LONG,SQL_DATE       ) ||
         !texis_param(tx,2,title  ,&tlen,SQL_C_CHAR,SQL_LONGVARCHAR) ||
         !texis_param(tx,3,article,&alen,SQL_C_CHAR,SQL_LONGVARCHAR));
         { puts("texis_param Failed"); return(0); }
      if(!texis_execute(tx))
         { puts("texis_execute Failed"); return(0); }
      texis_flush(tx);
   }
   ...
\end{verbatim}

\EXAMPLE
\begin{verbatim}
TEXIS     *tx;
char   *query;
int     qlen;
FLDLST *fl;

   ...
   if(!texis_prepare(tx,
         "select id,Title from docs where Article like ?;"))
      { puts("texis_prepare Failed"); return(0); }
   for( each Article query to execute )
   {
      query=...
      qlen=strlen(query);
      if(!texis_param(tx,1,query,&qlen,SQL_C_CHAR,SQL_LONGVARCHAR))
         { puts("texis_param Failed"); return(0); }
      if(!texis_execute(tx))
         { puts("texis_execute Failed"); return(0); }
      while((fl=texis_fetch(tx,-1))!=FLDLSTPN)
      {
         ...
      }
   }
   ...
\end{verbatim}

\SEE
\begin{verbatim}
texis_open(), texis_fetch()
\end{verbatim}

%**********************************************************************/

\NAME{texis\_fetch() - SQL interface}
\SYNOPSIS
\begin{verbatim}
FLDLST *texis_fetch(TEXIS *tx, int stringsFrom);
\end{verbatim}

\DESCRIPTION

\verb`texis_fetch()` returns a \verb`FLDLST` pointer to the result
set.  The \verb`stringsFrom` paramter lets you control if results are
automatically converted to text.  An argument value of -1 will not
convert any result, other numbers will start converting from that result
column, so passing 0 will cause all fields to be converted.

Continue calling \verb`texis_fetch()` to get subsequent result rows.
\verb`texis_fetch()` will return \verb`FLDLSTPN` when there are no more
result rows.

\EXAMPLE
\begin{verbatim}
TEXIS     *tx;
FLDLST *fl;

   ...
   if((tx=texis_open())!=TEXISPN)      /* initialize database connection */
   {
      ...
                                                     /* setup query */
      if(texis_prepare(tx,
                 "select NAME from SYSTABLES where CREATOR!='texis';"
                )!=TEXISPN)
         while((fl=texis_fetch(tx))!=FLDLSTPN)/* get next result row */
         {
            dispfields(fl);            /* display all of the fields */
         }
      ...
      texis_close(tx);                   /* close database connection */
   }
   ...
\end{verbatim}

\SEE
\begin{verbatim}
texis_prepare(), texis_execute()
\end{verbatim}

%**********************************************************************/

\NAME{texis\_param(), texis\_resetparams() - SQL interface}
\SYNOPSIS
\begin{verbatim}
int texis_param(tx, ipar, buf, len, ctype, sqltype)
TEXIS   *tx;
int     ipar;
void    *buf;
int     *len;
int     ctype;
int     sqltype;

int texis_resetparams(tx)
TEXIS	*tx;
\end{verbatim}

\DESCRIPTION

These functions allow you to pass arbitrarily large or complex data into
a SQL statement. Sometimes there is data that won't work in the confines
of the simple C string that comprises an SQL statement. Large text fields
or binary data for example.

Call \verb`texis_param()` to setup parameters after \verb`texis_prepare()` and
before \verb`texis_execute()`.  If you have a statement you have already
executed once, and you want to execute again with different data, which
may have parameters unset which were previously set you can call
\verb`texis_resetparams()`.  This is not neccessary if you will explicitly
set all the parameters.
Place a question mark (\verb`?`) in the SQL statement where you would
otherwise place the data.

These are the parameters:
\begin{description}
\item[TEXIS     *tx]      The prepared SQL statement.
\item[int     ipar]    The number of the parameter, starting at 1.
\item[void   *buf]     A pointer to the data to be transmitted.
\item[int    *len]     A pointer to the length of the data.  This can be
                       \verb`(int *)NULL` to use the default length, which
                       assumes a \verb`'\0'` terminated string for
                       character data.
\item[int     ctype]   The type of the data.  For text use SQL\_C\_CHAR.
\item[int     sqltype] The type of the field being inserted into.  For
                       varchar use SQL\_LONGVARCHAR.
\end{description}

\begin{tabular}{|l|l|l|l|}\hline
Field type & sqltype          & ctype           & C type     \\ \hline
varchar    & SQL\_LONGVARCHAR & SQL\_C\_CHAR    & char       \\
varbyte    & SQL\_BINARY      & SQL\_C\_BINARY  & byte       \\
date       & SQL\_DATE        & SQL\_C\_LONG    & long       \\
integer    & SQL\_INTEGER     & SQL\_C\_INTEGER & long       \\
smallint   & SQL\_SMALLINT    & SQL\_C\_SHORT   & short      \\
float      & SQL\_FLOAT       & SQL\_C\_FLOAT   & float      \\
double     & SQL\_DOUBLE      & SQL\_C\_DOUBLE  & double     \\
varind     & SQL\_LONGVARCHAR & SQL\_C\_CHAR    & char       \\
counter    & SQL\_COUNTER     & SQL\_C\_COUNTER & ft\_counter \\
\hline
\end{tabular}


\EXAMPLE
\begin{verbatim}
TEXIS  *tx;
char   *description;
char   *article;
int     lend, lena;

   if(!texis_prepare(tx,"insert into docs values(?,?);"))
      { puts("texis_prepare Failed"); return(0); }
   ...
   description="a really large article";
   lend=strlen(description);
   article=...
   lena=strlen(article);
   if(!texis_param(tx,1,description,&lend,SQL_C_CHAR,SQL_LONGVARCHAR)||
      !texis_param(tx,2,article    ,&lena,SQL_C_CHAR,SQL_LONGVARCHAR));
      { puts("texis_param Failed"); return(0); }
   if(!texis_execute(tx))
      { puts("texis_execute Failed"); return(0); }
   ...
\end{verbatim}

%**********************************************************************/

\NAME{texis\_cancel() - SQL interface}
\SYNOPSIS
\begin{verbatim}
int texis_cancel(tx);
TEXIS      *tx;
\end{verbatim}

\DESCRIPTION

This function is used to cancel a TEXIS statement.  It can be called in a
signal handler or thread to flag that the currently executing call on the
TEXIS handle should terminate at the next possible opportunity.

Once the thread has returned the TEXIS handle should be closed.

%**********************************************************************/

\NAME{texis\_flush() - SQL interface}
\SYNOPSIS
\begin{verbatim}
int texis_flush(tx);
TEXIS      *tx;
\end{verbatim}

\DESCRIPTION

This function flushes any remaining results from the current SQL
statement. Execution of the statement is finished however. This
is useful for ignoring the output of \verb`INSERT` and \verb`DELETE` statements.

\EXAMPLE
\begin{verbatim}
SERVER *se;

   ...
                                                     /* setup query */
   if(texis_prepare(tx,
              "delete from customer where lastorder<'1990-01-01';"
             )!=TEXISPN)
      texis_flush(tx);                        /* ignore result set */
   ...
\end{verbatim}


%**********************************************************************/

\NAME{texis\_flush\_scroll() - SQL interface}
\SYNOPSIS
\begin{verbatim}
int texis_flush_scroll(tx,nrows);
TEXIS      *tx;
int     nrows;
\end{verbatim}

\DESCRIPTION

This function flushes up to nrows results from the specified SQL
statement. This is useful for skipping a number of records from
a \verb`SELECT`.

The return will be the number of records actually flushed if
successful, or if an error occurred then the return will be a
negative number of (-1 - rowsflushed).  Reaching the end of the
results is not considered an error, and will result in a return
less than nrows.

\EXAMPLE
\begin{verbatim}
SERVER *se;

   ...
                                                     /* setup query */
   if(texis_prepare(tx,
              "select name from customer where lastorder<'1990-01-01';"
             )!=TEXISPN)
      texis_flush_scroll(tx,10);           /* ignore first 10 results */
   ...
\end{verbatim}


%**********************************************************************/

\newpage\section{Modifying the server}
% MAW 09-19-94 wtf try modify server/tsql then check doc

\NAME{adduserfuncs - Adding to the server}

\SYNOPSIS
\begin{verbatim}
#include <sys/types.h>
#include "dbquery.h"

void adduserfuncs(fo)
FLDOP *fo;
\end{verbatim}

\DESCRIPTION

This section describes how to add user defined types and functions to the
texis server. The file {\tt aufex.c} in the api directory provides an
outline of how to do this.  The server calls the function {\tt
adduserfuncs} which gives you the opportunity to add data types, operators
and functions to the server.  There are three functions that are used for
adding to the server. They are {\tt dbaddtype, foaddfuncs} and {\tt
fosetop.}

Once you have created your own version of {\tt aufex.c} you need to
compile and link this file with the rest of the daemon objects.  You must
make sure that this file is linked before the libraries to make sure your
function gets called.  An example makefile is also included in the api
directory which shows the needed objects and include paths to make a new
daemon.  After a new daemon has been created, make sure that it is running
and not the standard daemon.  See the documentation on {\tt texisd} to see
how to invoke it.

\NAME{dbaddtype - Add a datatype}

\SYNOPSIS
\begin{verbatim}
int dbaddtype(name, type, size)
char *name;
int type;
int size;
\end{verbatim}

\DESCRIPTION
{\bf Parameters}
\begin{description}
\item[name] the new name for the type.  It should not start with the
string ``var'', as that is reserved for declaring the variable size
form of the datatype.
\item[type] an integer which is used to refer to the type in functions
etc.  For a user added type this number should be between 32 and 63
inclusive.  This number should be unique.
\item[size] the size of one element of the datatype.  When one item of
this type is written to the database, at least size bytes will be
transferred.
\end{description}

The function will return 0 if successful, and -1 if there is no room
for more datatypes, or if a type with a different name already exists
with the same type number.


\EXAMPLE
\begin{verbatim}
typedef struct tagTIMESTAMP
{
   short           year;
   unsigned short  month;
   unsigned short  day;
   unsigned short  hour;
   unsigned short  minute;
   unsigned short  second;
   unsigned long   fraction;
} TIMESTAMP;

dbaddtype("timestamp", 32, sizeof(TIMESTAMP);
\end{verbatim}

\NAME{foaddfuncs - Add functions}

\SYNOPSIS
\begin{verbatim}
#include <sys/types.h>
#include "dbquery.h"

int foaddfuncs(fo, ff, n)
FLDOP *fo;
FLDFUNC *ff;
int n;
\end{verbatim}

\DESCRIPTION

{\tt Foaddfuncs} adds a function to the math unit of Texis.  The function
can take up to five arguments, and returns a single argument.  The function
will be called with pointers to {\tt FLD} structures.  The return values
should be stuffed into the pointer to the first argument.

The math unit takes care of all the required stack manipulation, so no
stack manipulation is required in the function. The math unit will
always pass the maximum number of arguments to the function.

{\tt Foaddfuncs} takes an array of function descriptions as one of its
arguments.  The functions description function looks like
\begin{verbatim}
struct {
   char *name;                                  /* name of function */
   int (*func)();                                        /* handler */
   int   minargs;                 /* minimum # of arguments allowed */
   int   maxargs;                 /* maximum # of arguments allowed */
   int   rettype;                                    /* return type */
   int   types[MAXFLDARGS];   /* argument types, 0 means don't care */
} FLDFUNC;
\end{verbatim}

{\bf Parameters}
\begin{description}
\item[fo] The math unit to add to.
\item[ff] Array of function descriptions to be added.
\item[n] The number of functions being added.
\end{description}

\EXAMPLE
\begin{verbatim}
int
fsqr(f)
FLD *f;
{
   int     x;
   size_t  sz;

   x = *(ft_int *)getfld(f, &sz);      /* Get the number */
   x = x * x ;                              /* Square it */
   putfld(f, x, 1);                    /* Put the result */
   return 0;
}

static FLDFUNC  dbfldfuncs[]=
{
   {"sqr",   fsqr, 1, 1, FTN_INT, FTN_INT, 0, 0, 0, 0 },
};
#define NFLDFUNCS (sizeof(dbfldfuncs)/sizeof(dbfldfuncs[0]))

foaddfuncs(fo, dbfldfuncs, NFLDFUNCS);
\end{verbatim}

This will add a function to square an integer.

\NAME{fosetop - Add an operator}

\SYNOPSIS
\begin{verbatim}
#include <sys/types.h>
#include "dbquery.h"

int fosetop(fo, type1, type2, func, ofunc)
FLDOP *fo;
int type1;
int type2;
fop_type func;
fop_type *ofunc;
\end{verbatim}

\DESCRIPTION

{\tt Fosetop} changes a binary operator in the math unit of Texis.  The
function {\tt func} will be called for all operations on (type1, type2). If
the function does not know how to handle the specific operation, and ofunc
is not NULL, ofunc can be called to hande the operation.

The function being called should look like
\begin{verbatim}
int
func(f1, f2, f3, op)
FLD *f1;
FLD *f2;
FLD *f3;
int op;
{
}
\end{verbatim}

{\bf Parameters}
\begin{description}
\item[fo] The math unit to change.
\item[type1] The type of the first operand.
\item[type2] The type of the second operand.
\item[func] Pointer to the function to perform the operation.
\item[ofunc] Pointer to pointer to function.  The pointer to function will
be stuffed with the old function to perform the operation.  This can be
used to add some cases, and keep the old functionality in others.  The
return value should be 0 for success.  The result of the operation should
be put in f3.

\end{description}

\EXAMPLE
\begin{verbatim}
#include <sys/types.h>
#include "dbquery.h"

fop_type o_ftich;          /* the pointer to the previous handler */

int
n_ftich(f1, f2, f3, op)
FLD     *f1;
FLD     *f2;
FLD     *f2;
int     op;
{
        TIMESTAMP       *ts;
        double          d;
        int             n;

        if (op != FOP_ASN)      /* We only know about assignment. */
                if (o_ftich)
                        return ((*o_ftich)(f1, f2, f3, op));
                else
                        return FOP_EINVAL;

        /* Set up the return field. */
        f3->type = FTN_TIMESTAMP;
        f3->elsz = sizeof(TIMESTAMP);
        f3->size = sizeof(TIMESTAMP);
        f3->n = 1;
        if(sizeof(TIMESTAMP) > f3->alloced)
        {
                void *v;

                v = malloc(sizeof(TIMESTAMP));
                setfld(f2, v, sizeof(TIMESTAMP));
                f3->alloced = sizeof(TIMESTAMP);
        }

/* First 0 out all the elements */
        ts = getfld(f1, NULL);
        ts->year = 0;
        ts->month = 0;
        ts->day = 0;
        ts->hour = 0;
        ts->minute = 0;
        ts->second = 0;
        ts->fraction = 0;

/* Now read in the values */
        n = sscanf(getfld(f2, NULL), "%hu/%hu/%hd %hu:%hu:%hu%lf",
                &ts->month, &ts->day, &ts->year,
                &ts->hour, &ts->minute, &ts->second, &d);

/* Convert any fractional seconds into the appropriate number
   of billionths of a second.
*/
        if (n == 7)
                ts->fraction = d * 1000000000 ;
}

 .
 .
 .
        fosetop(fo, FTN_TIMESTAMP, FTN_CHAR, n_ftich, &o_ftich);
 .
 .
 .
\end{verbatim}

This example adds an operator to allow the assignment of a TIMESTAMP
field from a character string.  See {\tt dbaddtype} for the definition
of TIMESTAMP.
%\end{document}             % End of document.
