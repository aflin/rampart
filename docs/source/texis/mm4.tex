\chapter{Tailoring Metamorph's Linguistics}{\label{chp:mmling}}

\section{Editing Special Word Lists}

There are special word lists which are called upon at various points
in the Metamorph search routine and influence the way your search
requests are processed, as well as which hits are deemed valid and
presented for viewing.  These lists can be tailored to user
specification if desired.

While it is encouraged that you edit equivalences as often as you wish
to reflect what you have in mind to search for, it should be clearly
understood that editing equivalences that go with search words is a
very different activity from editing the special word lists which
govern Metamorph's approach to linguistically processing the English
language.

The default content of the special lists is the result of much
research on the part of the program developers, and reflects what has
been locked inside Metamorph for a long time for effective use in most
situations.  Still, these lists are open to editing to conform to
program purpose:  that just about everything that is related to how
Metamorph processes language is subject to modification.

The special word lists are what the program classifies as:  Noise
(which also includes Pronouns, Is-associations, and Question words),
Prefixes, Suffixes, and Equiv-Suffixes.

For example, here is the default Pronoun list:

%\begin{screen}
\begin{verbatim}
  anybody             ourselves         you
  anyone              she               your
  anything            somebody
  each                someone
  everyone            something
  everything          their
  he                  them
  her                 they
  him                 this
  his                 us
  i                   we
  me                  whatever
  mine                who
  my                  whoever
  myself              whom
  our                 whose
\end{verbatim}
%\end{screen}

The content of this list is stored in ASCII format in a Metamorph
profile, and can be edited there, if you have an application set up
which can read a special profile.  Otherwise trust the internal
defaults of the program, and override them with an inline APICP
directive if you need to change them, as shown under {\em Question
Parsing}.

\section{Question Parsing}

A Metamorph query will by default be parsed for noise before the main
words are sent out to the executed search.  The intent is that a user
can phrase his search request in a manner which reflects how he is
thinking about it, without requiring him to categorize his thoughts
into a syntactical procedure.

You do have the option to keep the noise words as significant words in
the query.  This choice can be invoked by setting the APICP flag
\verb`keepnoise` to \verb`on`.  If off, as is the Metamorph
default used by Texis, the question parser is invoked and allowed to
exercise discretion in determining what is most important about your
query.  This selection process determines what should be treated as a
search word, and therefore included in the set count and intersection
quantity assigned to each significant search item.

A need to edit these lists would be most likely to come up if one were
searching language from a different era or with a vastly different
language style; perhaps for example the Bible.  Where
``\verb`thee's`'' and ``\verb`thou's`'' replace ``\verb`you`'' and
``\verb`yours`'', certain things should be filtered out from the query
that might not be entered in the list.  These categories can be
defined as follows, where all of these categories are grouped for
APICP programming purposes under ``noise''.

\begin{description}
\item[Noise] Small, common, relational words which appear frequently
in a particular language and refine and fine tune specific
communications, but do not majorly affect the larger concepts under
discussion; e.g., about, in, on, whether.

\item[Pronouns] Words indicating person, number, and gender which are
used in place of other nouns; e.g., her, his, we, them, its.

\item[Is] Words indicating state of being or existence, or used as
auxiliary verbs as assistive in defining tense and use; e.g., is,
being, was.

\item[Questions] Words indicating a question is being asked and
requires an answer or response; e.g., what, where, when.
\end{description}

Syntax for setting a custom list inside a Vortex script is as
follows, where what you write in would override the default internal
noise list:

\begin{verbatim}
  <$noiselist = "am" "are" "was" "were" "thee" "thou" "has" "hast"
     "from" "hence">
  <apicp noise $noiselist>
\end{verbatim}

\section{Morpheme Processing}

Metamorph's whole search process is built around its ability to deal
with root morphemes in words, and in fact the program was originally
christened ``Metamorph'' based on that ability.  An inherent part of
the search process is to take search words, strip them of suffixes and
prefixes down to a recognizable morpheme, search for the morpheme,
find a possible match, then go through the process again with the
possible match to see if it is indeed what you were looking for.

The elements which affect morpheme processing are very subtle as they
affect all language and therefore all of any text you are searching.
Therefore, any changes entered into the defaults which dictate how or
whether morpheme processing is done, will affect broadly the nature of
your search results.

The options which affect morpheme processing are the following:

\begin{itemize}
\item  Content of the Equiv-Suffixes List;
\item  Content of the Prefix List;
\item  Content of the Suffix List;
\item  Prefix processing on/off toggle;
\item  Suffix processing on/off toggle;
\item  Morpheme rebuild on/off toggle;
\item  Minimum word length setting.
\end{itemize}

The content of the \verb`Equiv-Suffix` List determines which suffixes
will be stripped before doing lookup for a matching entry in the
Equivalence File.

The content of the \verb`Prefix` and \verb`Suffix` Lists determines
precisely what will be stripped from a search word to obtain the
morpheme which is passed to the Search Engine.  Specifically this
relates to all words in all equivalence sets.

To understand why the program is doing what it is doing you need to
know more than just what is contained in the lists; you must also
understand the sequence of the Morpheme Stripping process.  If you
understand these rules, you will have better judgment on how to edit
the prefix and suffix lists, what happens when you turn off prefix
and/or suffix processing altogether, what happens if you turn off the
morpheme rebuild step, and where the minimum word length setting comes
in.

\subsection{Morpheme Stripping Routine}

This routine is done by the Engine as a preliminary step before
actually executing any search, using the content of the Prefix and
Suffix Lists.  This routine is used to get words from the Equivalence
File, and only does the suffix stripping part, using the content of
the Equiv-Suffix List.

\begin{enumerate}

\item When it is time to execute a search, the suffix and prefix lists
are each sorted by descending size and ascending alphabetical order.
The reason for and importance of descending order is so that suffixes
and prefixes can be stripped largest to smallest.  There is no
particular reason for alphabetical order except to provide a
predictable ordering sequence.

\item Get the word to be checked from entered query; e.g.,
``\verb`antidisestablishmentarianism`''.

\item Check the word's length to see if it is greater than or equal to
the length set in minimum word length.  The default in Texis is 255.  A
setting of 5 normally produces the best results with the default suffix
list.  If so, carry
on.  If not, there is no need to morpheme strip the word; it would
just get searched for as is; e.g., ``\verb`red`''.

\item Check the word found against the list of suffixes to see if
there is a match, starting from largest suffix on the list.

\item If a match is found strip it from the word.  Note:  This is why
ordering by size is so important:  because you want to remove suffixes
(or prefixes) by the largest first, so as not to miss multiple
suffixes, where one suffix may be a subset of another.

\item Continue checking against the list for the next match.  Follow
steps 4-5 until no more matches found.  In the case of our example
above, based on the default suffix list, we would be left with the
following morpheme before prefix processing:
``\verb`antidisestablishmentarian`''.  Note the following things:

\begin{itemize}
\item The suffix ``\verb`ism`'' was on the list and was stripped.

\item Neither ``\verb`an`'', ``\verb`ian`'', nor ``\verb`arian`'' was
on the suffix list, so it was not stripped.

\item The suffix ``\verb`ment`'' is on the suffix list, but it was not
left at the end of the word at any point, and therefore was not
removed.

\item If ``\verb`arian`'' and ``\verb`ian`'' were both entered on the
suffix list, ``\verb`arian`'' would be removed first, so as not to
remove ``\verb`ian`'' and be left with ``\verb`ar`'' at the end of the
word which would not be strippable.
\end{itemize}

\item If suffix checking (only), remove any trailing vowels, or 1 of
any double trailing consonants.  This handles things like
``\verb`strive`'', which would be correctly stripped down to
``\verb`striv`'' so that it won't miss matches for
``\verb`striving`'', etc.  (trailing vowel).  And things like
``\verb`travelling`'' would be stripped to ``\verb`travell`''; you
have to strip the second `\verb`l`' so that you wouldn't miss the word
``\verb`travel`'' (trailing double consonant).  Note:  this is only
done for suffix checking, not prefix checking.

\item Now repeat Steps 4-6 for prefix stripping against the prefix
list.  In our example, ``\verb`antidisestablishmentarian`'' would get
stripped down to ``\verb`establishmentarian`''.  This is what you have
left and is what goes to the pattern matcher.

\item When something is found, the pattern matcher builds it back up
again to make sure it is truly a match to what you were looking for.
This prevents things like taking ``\verb`pressure`'' when you were
really looking for ``\verb`president`'', ``\verb`restive`'' when you
were really looking for ``\verb`restaurant`'', and other such
oddities.
\end{enumerate}

\subsection{Suffix List Editing Notes}

It should be noted that the Suffix List used for Equivalence Lookup is
different than the Suffix List used for matching words in the text.
One should make changes in the Suffix List with great care, as it is
so basic to what can be found in the text.

While caution should also be applied to editing the Equiv-Suffix List,
doing so has a different effect upon the search results.  The default
Equiv-Suffix List is quite short by comparison, containing only 3
suffixes.  This is because different forms of words tend to have
different sets of concepts, as contained in the Equivalence File.

If you feel the suffix list is not broad enough, you might wish to
carefully expand it.  An example of a valid suffix to add might be
``\verb`al`''.  Doing so would mean that typing in a query using the
word ``\verb`environmental`'' would pull up the equivalences listed
for ``\verb`environment`''.

If an exact entry for your word exists, only its equivalences will be
retrieved from the Equivalence File.  You want ``\verb`monumental`''
to match ``\verb`colossal`'', but you would probably be unhappy if it
matched ``\verb`tombstone`'' from the set associated with
``\verb`monument`''.  If no such exact entry existed, you would have
to be prepared for ``\verb`monumental`'' to dutifully retrieve the word
``\verb`obelisk`'' if it were linked to ``\verb`monument`''.

Keeping all these English nuances in mind, you do in fact have
complete control over exactly what you can make the program retrieve.

\subsection{Toggling Prefix/Suffix Processing On/Off}{\label{set:presuf}}

You can toggle prefix and suffix processing individually on and off.
If you don't like the way a particular word is being suffix or prefix
processed, the easiest thing to do is simply turn off suffix or prefix
processing temporarily and search for the word exactly as you want it.
This is done by setting the corresponding APICP flag for
\verb`suffixproc` or \verb`prefixproc`.

When one or both (suffix or prefix) morpheme processes are off, none
of that respective routine (or routines) will be done at all.  If both
are off, you get a search for exactly the word you were looking for
with no removal at all, and no check once found; in other words, the
word you enter is passed directly from the search line to the
\verb`PPM` (Parallel Pattern Matcher) search algorithm.

Note that prefix processing is not supported by the Metamorph index,
and thus enabling it will require a linear search (slower).  This is
why prefix processing is off by default in Vortex, \verb`tsql` and the
Search Appliance.

\subsection{Toggling Morpheme Rebuild Step On/Off}

You can also turn off the Morpheme Rebuild step if you want to, by
setting it with the corresponding APICP flag \verb`rebuild`.  The
Rebuild step, referenced in Step 9 above, is intended as a check to
rebuild morphemes back up into words once found, to verify hits.  If
not otherwise set, the default is ``\verb`on`''; i.e., that this step
is done.

It may happen however, that you either miss a hit you feel you should
have gotten, or are getting a hit that does not make sense to you.  If
so, you can try turning off the rebuild step, do your search, go into
context and see exactly what was retrieved (as it will be highlighted)
and why.  This will at least serve to explain to you exactly what is
going on in the search process.

Remember that the English language is based on more irregularities
than regularities.  Metamorph is designed to process the English
language as best as it can, in a regular fashion.  We leave it to
specific applications to adjust these language toggles to one's own
satisfaction, and expect that everything will not be set in exactly
the same way every time.

\subsection{Setting Minimum Word Length}{\label{set:minwrdlen}}

Minimum word length is the approximate number of significant
characters the program will deal with at a morpheme level.  You
increase it to obtain more exactness to the search pattern entered,
and decrease it for less exactness to that pattern.  The smaller the
minimum word length, the slower the search will be, although the
difference may be imperceptible.

Vortex scripts, \verb`tsql` and the Search Appliance use a default
minimum word length of 255, which essentially turns off morpheme
stripping and allows for exact searching in locating documents.  To
use morpheme processing on a content oriented, English-smart
application, set the APICP flag for \verb`minwordlen` to 5. From years
of experience we have established this as the best place to start, and
really do not advise changing this setting arbitrarily.

As applies to the Morpheme Stripping Routine, note the following:  in
general (about 90\% of the time) these rules are followed exactly, and
the word would never be stripped smaller than the set length.  But in
certain cases to take into account certain overlapping rules and/or
idiosyncrasies as it sees fit, it will sometimes strip down further
than minimum word length; but never more than 1 character.

\section{Warning to Linguistic Fiddlers}

The program defaults and guidelines as described herein pass on to the
user years of experience in what works best in terms of all these
linguistic elements to accomplish in the main generally satisfactory
search results.  If you are not getting good results for some reason,
or have allowed various situation specific choices at some points of
your program, restore the defaults before continuing. This can be done
by setting the APICP flag \verb`defaults`, which restores the
Vortex defaults to the next search.

\section{APICP Metamorph Control Parameters Summary}

You can modify these APICP settings by adjusting these flags to modify
the Metamorph control parameters below.  These are covered elsewhere
in relation to their use in Texis, Vortex, or the Metamorph API; see
the appropriate manual.  The settings are:

\begin{itemize}
  \item[suffixproc]: Whether to do suffix processing or not.
  \item[prefixproc]: Whether to do suffix processing or not.
  \item[rebuild]: Whether to do word rebuilding.
  \item[incsd]: Include start \verb`w/` delimiters in hits (always on for \verb`w/N`).
  \item[inced]: Include end \verb`w/` delimiters in hits (always on for \verb`w/N`).
  \item[withinproc]: Whether to respect the within (\verb`w/`) operator.
  \item[minwordlen]: Minimum word length for morpheme processing.
  \item[intersects]: Default number of intersections (if no \verb`@`).
  \item[see]: Whether to look up ``see also'' references.
  \item[keepeqvs]: Whether to keep equivalences for words/phrases.
  \item[keepnoise]: Whether to keep noise words in the query.
  \item[sdexp]: Default start delimiter REX expression.
  \item[edexp]: Default end delimiter REX expression.
  \item[eqprefix]: Main equivalence file name.
  \item[ueqprefix]: User equivalence file name.
  \item[suffix]: Suffix list for suffix processing during search.
  \item[suffixeq]: Suffix list for suffix processing during equivalence lookup.
  \item[prefix]: Prefix list for prefix processing during search.
  \item[noise]: Noise word list.
  \item[defaults]: Restore defaults for all APICP settings.
\end{itemize}

  Note that the default values for these (and other) settings may vary
between Vortex, the Search Appliance, \verb`tsql` and the Metamorph
API.  See the section on ``Differences Between Vortex, tsql and
Metamorph API'' in the Vortex manual for a list of differences.

  The easiest way to make use of these settings is in a Vortex script, e.g.:

\begin{verbatim}
<apicp suffixproc 1>
\end{verbatim}

In \verb`tsql`, they can be altered with a SQL \verb`set` statement:

\begin{verbatim}
tsql "set suffixproc=1; select ..."
\end{verbatim}

The various C APIs have their own calls, e.g. \verb`n_setsuffixproc()`.

Some of the settings can also be changed inline with the query, for
example: {\tt horse @minwordlen=3 cat}.  Such inline settings will
apply to all terms after the setting.

You can also create thesaurus entries that apply specific settings,
for example the thesaurus entry:

\begin{verbatim}
battery={@suffixproc=0,battery,batteries}
\end{verbatim}

could be used to give exact word forms for just a specific word in a
query.  I.e. only the words {\tt battery} and {\tt batteries} will
match for {\tt battery}: no suffix processing for other suffixes will
be done for that term, though such processing may be done for other
terms.  The {\tt \{} and {\tt \}} curly braces are required to enable
inline settings in an equiv list.  Also, the settings should be given
first in the list, so that they apply to all the words in the list.

To use inline settings in a query-specified equiv list, you will need
to escape the {\tt =} which is special, e.g.:

\begin{verbatim}
"cell phone" ({@suffixproc\=0,battery,batteries})
\end{verbatim}

The following settings may be specified inline:

\begin{itemize}
  \item[rebuild]
  \item[defsufrm]
  \item[defsuffrm]
  \item[suffixproc]
  \item[minwordlen]
  \item[prefixproc]
\end{itemize}
