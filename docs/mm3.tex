\chapter{Thesaurus Customization}

\section{Specializing Knowledge}

You could say that Metamorph's general knowledge approximates that of
the man on the street without the benefit of specialized knowledge.
This knowledge is stored in a large Thesaurus containing some 250,000
word associations supplied with the program.

It is not necessary for a query item to be stored in the Thesaurus for
Metamorph to search for it.  In fact, for many technical terms you
would want only forms of that exact word rather than any abstract
associations.

If you wanted someone to intelligently correlate information from a
medical, legal, or other specialized or technical data base, you would
expect him to have a working knowledge of the nomenclature used in
discussing those subjects.  Therefore, we have created a facility for
teaching Metamorph specialized vocabulary and domain specific
inference sets to draw upon for its concept set expansion.

You are customizing Metamorph for your own use by adding semantic
knowledge to the web of equivalences.  In this way, the program can be
made ``smarter'' and able to take on the ``viewpoints'' of the user
where required.  Metamorph becomes an increasingly powerful retrieval
tool through this process of the user teaching it new language and
associations.

By passing this ability onto the user, and allowing for thesaurus
customization to be an integral part of a search environment, the job
of evolving vocabulary becomes entirely the user's responsibility and
under his control.  We provide a basic foundation of over 250,000
associations which is adequate to process any query with a reasonable
amount of discretion and intelligence.  The job of refining that
vocabulary and keeping it up to date as language evolves is up to you.

\section{Equivalence File Explanation}

Metamorph draws upon a large interconnected web of root words and
their equivalence sets called the Equivalence File.  For every English
word entered as a search item, a lookup is done in this file, and
whatever list of associations exists therein is included with the
specified root word.

The entered keyword is called the ``root''.  The associations pulled
from the Equivalence File are called the equivalences, or ``equivs''.

Each list of words created from such Equivalence File lookup is passed
to PPM (the Parallel Pattern Matcher) for text searching.  Morpheme
processing is done on all equivalences of all root words.  Therefore,
all valid word forms of all words in a concept set are searched for.

There are over 250,000 of these word associations contained in the
Equivalence File, making a Metamorph search fairly intelligent without
the need for customization before trying it.

While the Equivalence File is similar to a thesaurus, it is more
accurately comprised of associations of different types, signalling
paths of equivalent weightings through a volume of vocabulary.

Where special nomenclature, slang, acronyms, or technical terminology
is in abundance, one can customize the Equivalence File by making a
User Equivalence File, through which tailored priorities can be
ordered and new entries added.

\section{Editing Concept Sets With Backref}

It is intended that permanent Equivalence editing be done with the
whole group's needs in mind, rather than to please just one individual
user.  Vocabulary additions, specialized nomenclature, slang and
acronyms that would be common to a particular group, subgroup, purpose
or project, would be entered and saved in a User Equivalence File, as
named with the corresponding APICP flag \verb`ueqprefix`, and pointed
to in the managing program; e.g., a Vortex script.

While it is possible to create several User Equivalence Files for
different groups with differing acronyms and vocabulary, this is not
always necessary.  More often than not there is only one User
Equivalence File per system, and multiples are created only to resolve
conflicts in terminology.

Running with See References on produces anywhere from 75 to 200
equivalences associated with a particular word; more than 256 words
are truncated.  This is usually much more than is prudent, so it is
intended that with See References on, you would edit such a list down
to correspond more closely to what you really had in mind for the
search question at hand.

To check existing equivalence set for a word, use the BACKREF program,
where the syntax is:

\begin{verbatim}
     backref -e input_file output_file
\end{verbatim}

where \verb`input_file` is your ASCII equiv source filename, and
\verb`output_file` is the backreferenced and indexed binary file which
the Metamorph engine will use.  The default User Equiv source file is
\verb`eqvsusr.lst` and its binary counterpart is \verb`eqvsusr`.

When you enter a query like ``\verb`power struggle`'' in a query input
box, where concept search has been enabled, your search already knows
57 equivalences for \verb`power`, and 23 equivalences for
\verb`struggle`.  Using the above command for BACKREF, you can see
what those sets are composed of, and you can edit them.  When asked to
enter a root term, enter \verb`power`, and the list below would be
revealed:

\begin{screen}
\begin{verbatim}
power;n  57 equivalences
  ability;n          intensity;n        primacy;n          might;u
  acquistion;n       jurisdiction;n     regency;n          reign;u
  ascendency;n       justice;n          restraint;n        rule;u
  authority;n        kingship;n         scepter;n          sovereignty;u
  carte blanche;n    leadership;n       skill;n            sway;u
  clutches;n         majesty;n          strength;n         electrify;v
  command;n          mastership;n       suction;n
  control;n          mastery;n          superiority;n
  domination;n       militarism;n       supremacy;n
  dominion;n         monarchy;n         vigor;n
  efficiency;n       nuclear fission;n  weight;n
  electricity;n      omnipotence;n      ability;u
  energy;n           persuasiveness;n   capability;u
  force;n            potency;n          control;u
  hegemony;n         predominance;n     energy;u
  imperialism;n      preponderance;n    faculty;u
  influence;n        pressure;n         function;u
\end{verbatim}
\end{screen}

\begin{screen}
\begin{verbatim}
struggle;n  23 equivalences
  battle;n       agonize;v
  combat;n       compete;v
  competition;n  contest;v
  conflict;n     fight;v
  effort;n       flounder;v
  exertion;n     strive;v
  experience;n
  fight;n
  scuffle;n
  strife;n
  attempt;u
  clash;u
  conflict;u
  endeavor;u
  fight;u
  flight;u
  oppose;u
\end{verbatim}
\end{screen}

The root entry appears at the top, with its equivalences (each with an
assigned class, or part of speech) listed underneath.  The `\verb`n`'
following (or preceding) ``\verb`power`'' stands for ``\verb`noun`'',
and is the class to which ``\verb`power`'' has been assigned.
(`\verb`v`' means ``\verb`verb`''; `\verb`u`' means
``\verb`unclassed`''.)

Once editing a root word's set of equivalences, you'll have these
choices, offered below:

\begin{tabbing}
\verb`xxxxxxxxxxxxxxxx`xx \= Saves all changes made to list to User Equivalence File. \kill
          \verb`Delete` \> Deletes equivalence (named by number).                   \\
             \verb`Add` \> Opens word entry line below list to add new equivalence. \\
             \verb`Zap` \> Deletes entire equivalence list.                         \\
    \verb`Change-class` \> Prompts on line below for new word class assignment.     \\
 \verb`By-class-delete` \> Deletes all words in the entered class.                  \\
    \verb`Save Changes` \> Saves all changes made to list to User Equiv File.      \\
    \verb`Undo Changes` \> Restores previous root word entry screen.                \\
       \verb`Redisplay` \> Refreshes the list with any changes made (or as it was).   \\
\end{tabbing}

If you choose ``\verb`Save Changes`'', any changes made to the
list will be saved to the named User Equivalence File when you
quit the program with \verb`qq`.

When a new word is added, you are prompted to enter its class.  When
the new word is added to the list, it will be sorted in alphabetically
at its appropriate place in the list, under the class to which it
belongs.  Existing thesaurus entries have been classed according to
the standard parts of speech as described below.  However, you may
create and assign any class you like.

In the example above under ``\verb`struggle`'', you might want to
delete-by-class all those entries listed as verbs.  Doing so would
eliminate with one keystroke all equivalences classed as `\verb`v`':
``\verb`agonize`'', ``\verb`compete`'', ``\verb`contest`'',
``\verb`fight`'', ``\verb`flounder`'', and ``\verb`strive`''.  The
classes in use are as follows:

\begin{tabbing}
Pxxx \= Pronoun         \kill
   P \> Pronoun         \\
   c \> conjunction     \\
   i \> interjection    \\
   m \> modifier        \\
   n \> noun            \\
   p \> preposition     \\
   u \> unclassed       \\
   v \> verb
\end{tabbing}

You can \verb`Undo Changes` anytime while editing, to escape from the
action you are in.  This restores the entry screen, which lets you
choose another root word to edit, add to, or delete.  When you are
finished editing a word, either \verb`Save` or \verb`Undo` Changes,
then \verb`qq` to Quit.  At that point the source file will be indexed
into its binary form usable by the search, if you have saved any
changes.

\section{Toggling Equiv Expansion On or Off}

The APICP flag \verb`keepeqvs` lets you set the default for concept
expansion.  If set on, unless otherwise marked, the search will use
any equivalences found in the Equivalence File associated with any
word entered in a query.  This global condition can be selectively
reversed by preceding a word in a query with a \verb`~` tilde.  If the
global setting is off, the \verb`~` will selectively enable
concept expansion (i.e., equivalence look-up) on that word; if
the global setting is on, the \verb`~` will turn it off for that word.
(Note that equivalences can also be explicitly specified in
parentheses; see p.~\pageref{`MetamorphParenSet'}.)

While use of the Equivalence File is an integral part of the
intelligence of a Metamorph search, in certain kinds of particularly
specialized or technical data such abstraction may not be desired.
Turning the \verb`keepeqvs` flag off prevents any automatic lookup in the
Equivalence File, changing the nature of the search so that the
emphasis is rather on intersections of valid word forms of specified
English words, mixed in with special expressions. For example:

\begin{verbatim}
     What RESEARCH has been done about HEALTH DRINKS?
\end{verbatim}

Morpheme processing will be retained on the important (non-noise)
words, but equivalences will not be included.  An example of a
sentence this question would retrieve, would be:

\begin{quote}
The company had been {\bf researching} ingredients which would taste
good in a {\bf drink} while still promoting good {\bf health}.
\end{quote}

Where the global setting is on and you wish to selectively restrict
Equivalence Lookup on some but not all words, you use tilde `\verb`~`'
in front of those words where no equivalences are desired.  This would
be the preferred method of search for some types of technical
material, such as medical case data.

Although at first glance it would seem that an effective Metamorph
search could not be done on technical data until much specialized
nomenclature was taught to the Equivalence File, this is not always
the case.  Often a technical term means only that, and the power is in
intersecting some valid English form of that word with some other
concept set.

An example of a very discrete query that requires no knowledge
engineering beyond what comes ``out of the box'' with Metamorph might
be as in this query (assuming \verb`keepeqvs` turned on):

\begin{verbatim}
     stomach ~cancer operation
\end{verbatim}

The tilde `\verb`~`' is used to restrict equivalence lookup on
``\verb`cancer`'' (toggle the \verb`keepeqvs` setting, to off), as
references to such things as ``\verb`illness`'' rather than
``\verb`cancer`'' would be too abstract.  However, what you do want is
concept expansion on the related words.  Therefore, such a search
would retrieve the sentence:

\begin{quote}
     Suffering severe pains in his {\bf abdomen}, it was first thought to be
     appendicitis; however this led to exploratory {\bf surgery} which revealed
     {\bf cancerous} tissue.
\end{quote}

In this example, ``\verb`cancerous`'' is a valid word form of
``\verb`cancer`'' included through the morpheme process;
``\verb`abdomen`'' was found because it was in the ``\verb`stomach`''
equivalence list; ``\verb`surgery`'' was found because it was in the
``\verb`operation`'' equivalence list.

Again, with the \verb`keepeqvs` flag set on, use of the tilde
(\verb`~`) reverses its meaning.  Therefore, the above example would
seek only valid word forms of the root words ``\verb`stomach`'' and
``\verb`operation`'', but the tilde preceding ``\verb`cancer`'' would
selectively enable Equivalence Lookup on that word alone.  So, with
\verb`keepeqvs` off, we might retrieve instead the following:

\begin{quote}
      In the midst of the {\bf operation} to remove her appendix, an abnormal
      {\bf growth} was found in the {\bf stomach} area.
\end{quote}

This last response is matched because ``\verb`operation`'' and
``\verb`stomach`'' are forms of those root words; ``\verb`growth`'' is
in the equivalence list for ``\verb`cancer`'' and was included in the
set of possibilities due to the tilde (\verb`~`) preceding it.

\section{Creating a User Equivalence File By Hand}

The User Equivalence File is read by Metamorph as an overlay to the
Main Equivalence File.  The search looks for matching root entries
first in the User Equivalence File, and then in the Main Equivalence
File.  The information found in both places is combined, following
certain rules.

When editing equivalences using the BACKREF program as shown, the
changes you make are written to the named User Equivalence File.  It
is also possible to hand edit a User Equivalence File if you
understand the syntax which is used when writing directly to it.

In order to precisely deal with issues such as precedence,
substitution, removal, assignment, back referencing, and see
references, a strict format must be adhered to.  Any erroneous
characters included in the User Equivalence File could be
misinterpreted, causing unseen difficulties.  Therefore, one must take
care to ensure the User Equivalence file is flat ASCII.

If you want to create a User Equivalence file independent of
the \verb`backref -e` feature, follow these steps:

\begin{enumerate}

\item Using an ASCII only editor, open a file called
``\verb`eqvsusr.lst`''.  If you already have a list of words and
equivalences you want to add in a flat ASCII file, you can edit the
entries into the prescribed format.  Otherwise, simply begin entering
root words with their equivalences as outlined in the following
sections.

\item After creating the above User Equivalence File, it must still be
indexed by the ``\verb`backref`'' program supplied with your Texis
package.  \verb`Backref` takes an ASCII filename as the first
argument, and creates a file of the name given in the second argument.
For example, use this command on your ASCII User Equivalence File:

\begin{verbatim}
     backref eqvsusr.lst eqvsusr
\end{verbatim}
   Where ``\verb`eqvsusr.lst`'' is the ASCII file containing your User Equivalence
   entries, and ``\verb`eqvsusr`'' is the file ready for use by the
   Metamorph search engine.

\item If you have not otherwise named a special User Equiv File in
some managing program such as a vortex script, the indexed User
Equivalence File must be called ``\verb`eqvsusr`'', and must be
located in the ``\verb`morph3`'' directory, in order for it to be used
by a Metamorph search.
\end{enumerate}

\section{User Equivalence File Format}

A User Equivalence File is an ASCII file created by the user, which
corresponds to information in the larger 2+ megabyte Main Equivalence
File which comes with the Texis package.

Each root word is listed as a separate entry, on its own line with a
list of known associations or equivalences (equivs).

The root words go down the lefthand side of the file, each one a new
entry; the equivalences go out left to right as separated by commas.
Word class (part of speech) and other optional weighting information
may be stored with each entry.

Here is an example of a User Equivalence File.  It contains no special
information beyond root words and their equivalences.  Its chief
purpose would be the addition of domain specific vocabulary.  Phrases
are acceptable as roots or as equivalences.

\begin{verbatim}
     chicken,bird,seed,egg,aviary,rooster
     seed,food,feed,sprout
     ranch,farm,pen,hen house,chicken house,pig sty
     Farmer's Almanac,research,weather forecast,book
     rose,flower,thorn,red flower
     water,moisture,dew,dewdrop,Atlantic Ocean
     bee pollen,mating,flower,pollination,Vitamin B
     grow,mature,blossom,ripen
\end{verbatim}

Root word entries should be kept to a reasonable line length; around
80 characters for standard screen display is prudent.  In no case
should a root word entry exceed 1K per line.  Where more equivalences
exist for a root word than can be fit onto one line, enter multiple
entries where the root word is repeated as the first item.  For
example:

\begin{verbatim}
     abort,cancel,cease,destroy,end,fail,kill
     abort,miscarry,nullify,terminate,zap
\end{verbatim}

It is important to remember that these are not just synonyms.  They
can be anything you wish to associate with a particular word:  i.e.,
identities, generalities, or specifics of the word entry, plus
associated phrases, acronyms, or spelling variations.  Even antonyms
could be listed if you wished, although that wouldn't generally be
advisable.

The word ``equivalence'' is used in a programming sense, to indicate
that each equivalence will be treated in weight exactly the same as
every other equivalence in that set grouping when a search is
executed.


\section{Back Referencing}

Both the Main Equivalence File and the User Equivalence File include a
provision for ``back referencing''.  That is, where a word is stored
as the equivalence to another root entry, there is an inherent
connection ``backwards'' to the root, when that equivalence is entered
as a keyword (root) on the Query Line.

For example, imagine the following Equivalence File entry for the root
acronym ``\verb`A&R`'':

\begin{verbatim}
     A&R,automation,robotics,automation and robotics
\end{verbatim}

Metamorph, and so therefore Texis, knows when you enter
``\verb`robotics`'' as a root word in a query, that it should back
search and associate it with ``\verb`A&R`''.  Therefore, when
``\verb`robotics`'' is used as a keyword, ``\verb`A&R`'' will be
automatically associated as one of its equivalences.

Back referencing means that the following association is implicitly
understood, and need not be separately entered:

\begin{verbatim}
     robotics,A&R
\end{verbatim}

Such automatic back searching capability exponentially increases the
density of association and connection within the Equivalence File and
User Equivalence File.


\section{See Referencing}

You can exponentially increase the denseness of connectivity in the
Equivalence File by invoking ``See References'', set with the APICP
flag \verb`see`.

The concept of a ``\verb`See`'' reference is the same as in a
dictionary.  When looking up the word ``\verb`cat`'', you'll get a
description and a few definitions for cat.  Then it may say at the end
``\verb`see also pet`''.  With See Referencing on, the equivalence
list associated with the word ``\verb`cat`'' is expanded to also
include all the equivalences associated with the word ``\verb`pet`''.

See Referencing greatly increases the general size of word sets in use
in any search, increasing the chance for abstraction of concept.  One
would not normally invoke See Referencing, and would do so only where
such abstraction was desired.

Not all equivalences have ``\verb`See also`'' notations; but with See
Referencing on, all equivalences associated with any ``\verb`See
also`'' root word will be included as part of the original.  With See
References off, only the root word and its equivalences will be
included in that word set, regardless of whether a see reference
exists or not.

See Referencing is restricted to only one level of reference, to
prevent inadvertent ``endless'' looping or overlap of concepts.  In
any case, a word set will be truncated at the point it approaches 256
equivalences.

See References are denoted with the at sign (\verb`@`).  Enter the
word preceded by `\verb`@`'.  For example:

\begin{verbatim}
     cowboy,horse,cows,@rancher
     rancher,plains,landowner
\end{verbatim}

In normal usage, the search item ``\verb`cowboy`'' expands to the set
``\verb`cowboy`'', ``\verb`horse`'', ``\verb`cows`'' and
``\verb`rancher`''.  With see referencing invoked, ``\verb`cowboy`''
expands to ``\verb`cowboy`'', ``\verb`horse`'', and ``\verb`cows`'' as
well as ``\verb`rancher`'', ``\verb`plains`'', and
``\verb`landowner`''.

The see reference `\verb`@`' marking in a User Equivalence file will
only connect entries which are in the User Equivalence file.  In the
above example, the entry for ``\verb`rancher`'' must exist in the User
Equivalence to be so linked.


\section{Word Classes and Parts of Speech}

Part of speech or other word class information can be stored with
equivalence entries.  Use the following abbreviations:

\begin{verbatim}
     P: Pronoun           n: noun
     c: conjunction       p: preposition
     i: interjection      u: unclassed
     m: modifier          v: verb
\end{verbatim}

A part of speech abbreviation follows a semicolon (\verb`;`), where
the part of speech designation applies to that word and to all
equivalences which follow it up to the next part of speech
abbreviation on the line.  For example:

\begin{verbatim}
     wish;n,pie in the sky,dream;v,yearn,long,pine
\end{verbatim}

The above is an entry for the root word ``\verb`wish`''.  The
equivalences ``\verb`pie in the sky`'', and ``\verb`dream`'' are
classed as nouns.  The equivalences ``\verb`yearn`'', ``\verb`long`'',
and ``\verb`pine`'' are classed as verbs.


\section{Rules of Precedence and Syntax}

A root word and its equivalences are separated by commas.  The comma
(\verb`,`) signifies addition of an item to a set.

Where an entry exists in a User Equivalence File and also in the Main
Equivalence File, equivalences found for all entries of that root word
are combined into one set.  Example:

\begin{verbatim}
     constellation,celestial heavens
\end{verbatim}

Phrases are acceptable as roots or as equivalences, and locate matches
as separated by a hyphen or any kind of white space, provided the
separation is only one character long.  Use spaces rather than hyphens
to enter normally hyphenated words.

When ``\verb`constellation`'' is entered as a search item on the Query
Line, ``\verb`celestial heavens`'' from the User Equivalence File will
be added to the existing set, making the complete concept set:

\begin{verbatim}
     constellation
     celestial heavens
     configuration of stars
     galaxy
     group of stars
     nebula
     star
     zodiac
\end{verbatim}

Equivs can be removed from a larger set by preceding them with a tilde
(\verb`~`) in the User Equivalence File.  For example:

\begin{verbatim}
     constellation~nebula~zodiac,big dipper
\end{verbatim}

This entry for constellation reads ``remove `\verb`nebula`', remove
`\verb`zodiac`', and add `\verb`big dipper`'''; making the complete
concept set:

\begin{verbatim}
     constellation
     big dipper
     configuration of stars
     galaxy
     group of stars
     star
\end{verbatim}

A whole equivalence set can be substituted for what is in the Main
Equivalence File with a User Equivalence File entry which uses the
equal sign (\verb`=`) preceding the favored list of equivalences.  For
example:

\begin{verbatim}
     constellation=constellation,galaxy,nebula,star
\end{verbatim}

This entry for constellation replaces any entries in the Main
Equivalence File, making the complete concept set:

\begin{verbatim}
     constellation
     galaxy
     nebula
     star
\end{verbatim}

Don't forget to include the root word following the equal sign
(\verb`=`), as the substitution is literal for the whole set, and the
root word must be repeated to be included.

To permanently swap one word for another, you could make one entry
only, having the effect of assignment.  For example:

\begin{verbatim}
     constellation=andromeda
\end{verbatim}

Subsequent searches for ``\verb`constellation`'' where concept
searching is invoked will swap ``\verb`constellation`'' for
``\verb`andromeda`''.

To permanently disable concept expansion for an item, use the equal
sign (\verb`=`) to replace a keyword with itself only.  For example:

\begin{verbatim}
     constellation=constellation
\end{verbatim}

Any equivalences from the Main Equivalence File would be ignored, as
the set is replaced by this entry.

The above rules for substitution apply where what immediately follows
the equal sign (\verb`=`) is alphanumeric.  In the special case where
the 1st character following the equal operator (\verb`=`) is not
alphanumeric, the entirety of what follows on the line is grabbed as a
unit, rather than as a list of equivalences.  Example:

\begin{verbatim}
     lots=#>100
\end{verbatim}

The root word ``\verb`lots`'' will be replaced on the Query Line by
the NPM expression which follows the equal sign ``\verb`#>100`'',
therefore finding numeric quantities greater than 100, rather than
finding English occurrences of the word ``\verb`lots`''.

All root and equivalence entries are case insensitive.  If you need
case sensitivity you must so specify with \verb`REX` syntax on the
Query Line.  \verb`REX`, \verb`NPM`, \verb`XPM`, and \verb`*`
(Wildcard) expressions cannot be entered as equivalences, as
equivalences are sent directly to \verb`PPM` which processes lists of
English words.

The only way an English word may be linked in this way to a special
expression is through the use of substitution.  In this case the
expression which follows an equal sign (\verb`=`) will be substituted
for the root word.  Example:

\begin{verbatim}
     bush=/\RBush
\end{verbatim}

The root word ``\verb`bush`'' will be replaced on the Query Line by
the \verb`REX` expression which follows the equal sign
``\verb`/\RBush`''; therefore finding only the proper name
``\verb`Bush`'', rather than the common noun ``\verb`bush`'' along
with any of its equivalences (``\verb`jungle`'', ``\verb`shrub`'',
``\verb`hedge`'') as listed in the Main Equivalence File.

\section{Specialized User Equivalence Files}

Where you are creating your own specialized lexicon of terms, such as
for medical, legal, technical, or acronym laden fields, you may be in
a position to obtain your own digitized compilation of such which
could be of some size.

Rather than starting from scratch, take the digitized file, flatten it
to ASCII, then follow the User Equivalence rules as outlined in the
previous sections to edit it into the correct format.

If you have questions on how you might speed up this activity, call
Thunderstone technical support for discussion of details.

