
\documentstyle[12pt,epiman,chgbar]{book}

\author{Thunderstone Software \\ Expansion Programs International, Inc.}

\begin{document}

\tableofcontents
\chapter{Introduction}

This document answers the most frequently asked  questions about
Texis. It is intended for use by salespersons, support personnel and
consultants.

\chapter{Why Is This Technology Important?}

\subsection{Why is Texis important?}

Customers are demanding fully integrated systems that manage information -
regardless of
data source.  Data could be conventional structured data,
text, graphics,
images, video or sound.

They want the ability to store, search, retrieve and view these data
objects together in a single, integrated application and deploy that
application over a variety of network topologies.

\subsection{What does a Text Retrieval RDBMS provide?}

An effective system provides the ability to store, manage
and subsequently retrieve textual information.  The text can consist of
complete documents, or be abstracts/bibliographies, email, notes,
keywords, names, titles etc.  Documents can of course be multi-media
made up of text, graphics, images, drawings, etc.

A key requirement is that text can be placed into the system in an
unstructured (free) format, i.e. the format of books, memos, etc.

Another key requirement is that the user must be able to search and
retrieve information based upon the content of documents (words), together
with (and not solely on) conventional structured fields, e.g. document
number date, author name.

\subsection{Why is Thunderstone Corporation interested in an RDBMS?}

The marketplace for internetworked information
management/retrieval systems is
rapidly expanding.  Texis represents a
crucial part of the
 information requirement by symmetrically merging
relational data and full text retrieval.

No other product can make this claim, and no integration of other products
can easily replicate Texis functionality.

\subsection{What is Texis' approach?}

Texis' approach is to provide text retrieval capability, within the
framework of a database management system.

By providing this integrated approach two key benefits are obtained by
users:

(i) Required applications, containing both documents and structured data,
can be generated effectively.

(ii) All applications can be created, maintained and run in a common
software environment.  This means that return on investment in software,
staff training and support can be maximized.

\subsection{What is the opportunity for Thunderstone?}

By taking this unique approach we realize the following benefits:

(i) It will expand the number of applications that customers can use
Thunderstone products for.  This will generate significant product revenue for
Texis, as well as leverage additional revenue from incremental
product sales, consultancy and support fees.

(ii)    The technology will be used in other Thunderstone products, e.g. CASE, Personnel,
4GL Tools, and Office Automation, adding unique facilities to these
important product areas.

(iii)   We will provide a unique product to Thunderstone's Original Equipment
Manufacturers (OEM's) and VAR's - making us more attractive to them.

(iv) Thunderstone's consultancy operation can include a True database in
all the applications we specify design and build for customers.

(v) It confirms Thunderstone as the company with the best product
portfolio for serving the ever growing Internet application market.

(vi) It differentiates us greatly from our competitors.

\chapter{Product Features}

\section{General}

\subsection{What is Texis?}

Texis is a highly portable information retrieval and management product that
fully integrates
documents into strategic business applications, and deploys these applications
over a variety of Intra/Inter-net topologies.

\subsection{What does it do?}

It enables both structured data and unstructured data to be stored and
retrieved efficiently, so that ALL relevant information can be retrieved
when required.

The product provides you with the ability to:

(i) Load documents or text fields into a TEXIS database, in several
different ways from several sources, e.g.  HTML, WP systems, OCR., existing
files, archive systems.  Or, alternatively leave your text within the
operating system files in its native format.

(ii) Index the text automatically, via several options, which enables the
text to be searched and retrieved efficiently and in a totally flexible
manner.

(iii) Generate queries that will retrieve the relevant textual and
structured information.

(iv) Allow the retrieved information to be displayed in a chosen sequence
onto a web browser, selected output device, or launch an application e.g.
image viewer.

(v) Maintain the information, automatically updating the text if desired.

(vi) Provides usage audit and access control mechanisms.

\subsection{What are the components of Texis?}

Texis consists of:
\begin{itemize}
 \item  a powerful scripting language for creating Intra/Inter-net applications.

 \item  utilities for maintenance of the text data dictionary

 \item  utilities for text loading/indexing

 \item  retrieval language (SQL with the addition of a sophisticated
\verb`LIKE` clauses)

 \item  various client interfaces for generation of user applications via
third party 5GL tools.

 \item  an ad hoc command language

 \item  set of demonstration samples

\end{itemize}

\subsection{What are the software dependencies?}

Unix, Windows NT/2000 for a server, and any web browser for a client.
Client APIs are supported on the same platforms as the server.

TCP/IP if operating in a client server environment.

\subsection{Are there any hardware dependencies?}

No.

Texis is portable to all hardware environments that
have the required software ported (see item above).

\subsection{What additional functionalility does Texis provide over Metamorph and 3DB?}

Texis provides a completely structured environment for management, control
and dissemination of data around the core functionality of Metamorph.

Texis provides the ability to create any application the user
desires instead of accepting a single retrieval methodology.

\subsection{What major performance improvements does Texis provide over Metamorph and 3DB?}

Texis contains many technology advances that provide improved performance:
\begin{itemize}
\item Variable key size Btree's for word index storage which greatly
reduce disk activity during searching and indexing.

\item Improved index compression techniques which result in faster
retrieval and lower overhead.

\item The zero-latency index queue which dramatically reduces the
time required to keep text indexes current.
\end{itemize}

\subsection{How have these terrific performance improvements been achieved?}

Texis = 17 Years of experience + 7 Years development time + A ton of
customer feedback.


\subsection{With all this new technology and functionality in how
compatible is it with Metamorph and 3DB?}

Texis is 100 percent compatible with the existing Metamorph Query
Language, API calls, and network interface.

Texis represents the next generation of the 3DB product, and is not binary
compatible with 3DB's databases.  However, we provide an an API layer over
TEXIS which will fully emulate 3DB's functionality.

\section{Text Storage}

\subsection{How Is text stored in an Texis database?}

Documents can be stored in a single field per record, or split into
several fields within a Texis table structure.  These fields might be for
example, author, title, abstract and the body of the text.  Also
additional fields can be designed containing traditional structured data,
for example numeric or date information.

\subsection{What size of text database can Texis handle?}

There is no limit to the size of a database (see also question 2.7).

\subsection{What size of individual document can Texis handle?}

Texis can handle documents of up to 2 Gigabytes each on all platforms,
and on operating systems that support it, it can handle 64-bit files.

\subsection{What is the index overhead}

There are two different indexing schemes available, one has roughly
10-15\% overhead, and the other has roughly 40\%.

\subsection{How fast can it create an index?}

Texis can index text data at around 2 Gigabytes / Hour on a 200MHz Pentium Pro.
Actual index times will vary with the number of documents and their average size.

\subsection{Can there be multiple text columns per text table?}

Yes

Unlimited columns may be present in any table, each one defined to
be of the available data types.

\subsection{Can the text be compressed to reduce storage requirements?}

Text may be stored compressed, but we don't recommend this practice.
The decompression can cause delays and increases server load
when a users want to view documents.

\subsection{Do all documents need to be stored in Texis tables?}

Documents, either formatted or not, do not have to be stored in Texis
tables.  Applications may be created where the text remains outside the
Texis database.  The Texis environment is used to provide the required
security and control of the text indexes and application.  This
functionality is provided through the INDIRECT data type - with pointers
in the field relating to where the documents are stored externally.  Texis
supports operating system files (which could have been created from CDROM
or optical disks) and word processor files.

There are no changes required to run a client application with external
documents.  Also full text retrieval functionality is available
to external text, e.g. indexing, searching, display, editing.

Retrieval times are faster however when full-text is actually stored in
a Texis table.

\subsection{Can we store additional information alongside the text?}
Yes

Structured data can be stored alongside the text in a Texis table in the
usual way.  It can either exist on the database already or be loaded as
part of the text load operation.

\subsection{Can we work with compound documents that contain text, data and
 images?}

Yes

Text and data of any type can be stored in the same Texis table.  Compound
documents created by say, MS Word or WordPerfect, that contain both text
and graphics
 can also be managed.  The lexical 'filter' will ignore graphic
information
and other non-text content as it indexes the files.

\subsection{How do we get text into the database?}

Texis provides several utilities for loading/indexing data.  Timport
(Texis import) program is the most common method. It also supports
ODBC, XBASE/dBase file import, and delimited text.

\subsection{Where can it originate from?}

Text can be generated from a range of sources - word processing files,
OCR, electronic mail, existing Texis tables, etc. and is loaded onto the
database in its original format.

\subsection{Are facilities for dumping and loading data supplied with the
product?}

Yes

Comprehensive utilities are supplied that enable a wide range of options
for handling text and structured data.

\section{Indexing}

\subsection{What indexing strategy does Texis provide?}

Texis has two main indexing strategies available.
It has a coarse-grained inverted index method, which provides
the benefits of both speed for many queries and index compactness.

The second strategy is a complete inversion, which provides improved
performance for many searches at the cost of a slightly larger index.

\subsection{What is a Text Index?}

A Text Index is a special indexing facility which enables efficient
retrieval of text information.
 Text indexes are not natively available in
any other RDBMS.

See index Schema diagram.

\subsection{Why is the Text Index important?}

Text retrieval is based on the identification of individual words or
phrases within natural language.  If no index was implemented, the whole
of the document file would have to be scanned, looking for the words and
phrases of the query.  This scan would have to be repeated for each user
search request and the search time would increase linearly with the text
database size and may become impracticable.

\subsection{What indexing utilities are supplied?}

To index a text field of any type you use a SQL {\tt CREATE INDEX} statement.

There is a background utility program that will monitor the tables within
a database to see if the indexes are in need of maintenance. This utility
operates on a per index basis using both size change and time as
controlling parameters.
(See CHKIND).

At no time is the user query results affected by the state of the index.
Texis will always provide the same results without regard to index status.

\subsection{Is there one index per application or table or database?}

Each text field may have its own index and index parameters.  These parameters
include:  index update time rate, index update size rate, and the lexical
'filter' that applies to that field.

Text indexes may also be created on Virtual Fields which are logically created
combinations of existing fields.

\subsection{Is the indexing separate from Texis and where is the resultant text index
held?}

Text indexing is performed by Texis and the text indexes themselves
are always stored within the Texis database (the same as any other field
type).

You may direct Texis to place index files on separate physical devices
if desired. This is usually done to increase performance.

\subsection{How are non-text objects indexed, e.g. graphical images?}

Through a combination of 'structured data' attributes, such as Creation
date, Customer name, possible reference number PLUS a textual description
of the object, or on text that has been generated alongside the object, as
say, part of an article containing both text and graphics (compound
documents).

\subsection{Where are the indexes for external documents?}

External documents are logically part of the text table, even though they
are physically outside.  Therefore they are indexed in the same way as all
other documents, with the index being held within the database for
security and integrity.

\subsection{Is there an overhead in space utilization?}

Overhead for a coarse-grained text index will range between 7\% and 25\% of
the size of the data to which it refers.  The average overhead tends to
be 15%.

The overhead for a fully inverted index will range between 20\% and 60\%,
depending on the nature of the text.
The average overhead is about 40% overhead.

\subsection{Is Texis case sensitive?}

Not by default, but users may perform case sensitive queries on any
text field if they wish.

\subsection{Can the wordlist be viewed?}

Yes, a utility for viewing the contents of a word index is provided.

\subsection{Can the wordlist be modified?}

No, it is an index and it would be inappropriate to perform modifications
directly to it.

The lexical 'filter' provides mechanisms for controlling the content of
the words within an index.

\subsection{Do you supply a STOP LIST and can it be modified?}

Yes

We supply a list of about 150 words with the software.  It is stored in a
master profile and can be modified.

\subsection{In what languages do you provide the stop list?}

The stop list provided with the software is in English.  Users
may redefine this list for other languages if desired.

\subsection{Do you control suffixes of words in the index?}

The index will contain the full form of every word.  Morpheme processing
(word form changing) is performed at query time to allow a user to find
and validate the differing versions of the same word.

\subsection{Does Texis support automatic reindexing?}

Yes.

Texis supports both timed and incremental reindexing. Details on
indexing may be found under the Text Maintenance section.

\section{Document Retrieval}

\subsection{Is the Retrieval Language based upon SQL?}

Yes

Texis is based on the industry standard SQL language.  To do searching on
text fields we have extended SQL by adding a new \verb`LIKE` clause.  This new
clause provides the full power of the Metamorph search within the
database environment.

\subsection{Is this as powerful as other text retrieval software?}

Even more so, because the full SQL language, together with the text
content searching extensions, can be used in the same query.

Texis doesn't just rely on one retrieval method. It currently supports
five different retrieval algorithms which may be used individually
or in combination.

\subsection{Are the extensions to SQL proprietary?}

Yes, at the moment, since there are no defined standards for such
extensions to SQL.

\subsection{How has 'extended SQL' been extended?}

The extended SQL has an added \verb`LIKE` clause that defines the text to be
searched for within the document set.  These search conditions have a rich
functionality that includes truncation, wild cards, proximity fuzzy
matching, saved query expressions and a wide range of concept search
facilities.  All of these can be used with standard Boolean logic to build
up comprehensive topics.

\subsection{How complicated is the extended SQL?}

About the same as SQL.  Simple queries could be written by end users, but
in general you would develop an application that shields end users from
the language.

\subsection{Can Natural Language queries be used as input?}

Natural language queries may be used inside a \verb`LIKE` clause to match against
the content of a field.  However, putting in a query of the form:  "Show
me all the consultants who know about Texis RDBMS and make less than
\$50,000." cannot be handled.  This would require extensions via
third-party products such as QandA or EasyTalk that generate SQL from
english.

\subsection{What are the options then for generating queries?}

There are several Options:
\begin{itemize}
 \item  Using Texis Webscript from a Web browser.

 \item  Use any Windows ODBC client package.  (like:  Powerbuilder,
         Microsoft Access, Visual Basic..)

 \item  Use a  C-based application and invoke supplied C function calls.

 \item  Query via the TSQL command line application.
\end{itemize}
\subsection{Can we search on 'noise' words?}

Yes

Phrase searching and proximity will take account of noise words, so you can
search on such phrases as 'United Sates of America' or 'state of the art'.

\subsection{Can we do Boolean searches?}

Yes, it can do full Boolean searches based on SQL.

\subsection{Can we search on numbers or dates?}

Yes

Any quantity, range, date, or pattern may be located within a text field
through the use of the features provided by the Metamorph search.  Numbers
and dates stored within fields may be located with traditional SQL.

\subsection{Can we use wildcard searches?}

Yes:
\begin{itemize}
 \item        word* Anything beginning with word (forward truncation)
 \item        w*d Anything ending in 'd' and beginning with 'w'
 \item        w*d*f Multiple wildcards.
 \item        Use the Reqular Expression matcher for more complex items
\end{itemize}
\subsection{Do you have a Thesaurus?}

Texis comes with a 250,000 word association thesaurus.  Add to that its
ability to derive word forms and its vocabulary is huge.  The provided
thesuarus contains only commonly used English.  Industry specific terms,
phrases, and acronyms may be added by the user.

\subsection{What relationships are allowed?}

Texis allows true thesaural and synonym relationships. The concepts that can
be used in a query to expand the search include:
\begin{itemize}
 \item        link to other thesauri

 \item        synonyms

 \item        related terms

 \item        saved macro expressions

 \item        rules of inclusion and exclusion
\end{itemize}
\section{The Thesuarus}

\subsection{Do the thesaural relationships conform to any standard?}

Yes

It includes a full implementation of the ISO 2788 standard for Thesaurus and
Synonym Relationships.

\subsection{Can we import/export conceptual definitions across databases?}

Yes

Forms-based interactive browsing and maintenance of the 'concepts'
database is provided, together with dump, load and remove facilities (via
utilities).

\subsection{Can an independently-supplied thesaurus handler be used in conjunction with
Texis?}

No

However if the thesaurus handler allows the unloading of its data, then
this data could be loaded into the Texis thesaurus management system.

        Note: There may need to be some data format conversion.

\section{Search and Retrieval}


\subsection{What is fuzzy matching?}

Fuzzy matching is the searching for words that are 'similar' to the search term. It is
used to compensate for errors in data entry and phonetics.

\subsection{How is fuzzy matching implemented?}

Texis uses a proprietary Thunderstone algorithm that generates a
similarity measure between any two words or patterns. This measure
is expresses as a percentage of "closeness" to the users input.

\subsection{Do we have to view all the documents retrieved by a query?}

No

The queries are ordinary select statements, so you can choose which
fields to see.  If one of these fields is a key to the table it can
later be used to retrieve the desired document.  This allows an
application to build a 'Hitlist' that allows you to decide which
documents you are interested in.

\subsection{Can we browse the database as well as the hitlist?}

Yes

At any time the application can switch from hitlist browse to database
browse mode, and visa versa.

\subsection{Is the hitlist available for other application processes?}

Yes.

The hitlist is simply the result set from the query.  The application
that executed the query can do whatever it wants with the information.

\subsection{Can Texis do relevance ranking of retrieved documents?}

Yes.  It is possible to have Texis return the result set ordered by
relevance, as well as produce a rank associated with the document.
Many controls are provided to tune the ranking to your needs.

\subsection{After a search can the resulting text be highlighted?}

Yes

Texis Webscript has all the functionality builtin to highlight text.

The data specifying the location and size of the match is
available to the application program via the APIs, to display the hit
highlighted.

\subsection{Having done one query, I wish to search on a more limited range.  Can
I do this without re-executing the entire query?}

Absolutely!

The results of a query can be saved into a temporary table, and then this
temporary table searched in a more refined manner.

\subsection{What is User Profiling and can Texis do it?}

User Profiling (sometimes called Automated Message Handling (AMHS) or
Selective Dissemination of Information (SDl) in Europe) is the ability to
store predefined queries and execute them automatically against new
information in the database.  Texis can store queries under users - so an
application could be developed to periodically run them against the
database, or the insertion of the data can fire a trigger to perform the
dissemination.  This works well with data in the form of 'real-time' data
feeds (see question 5.1).

\subsection{What is fuzzy matching?}

Fuzzy matching is the searching for words that are 'similar' to the search term. It is
used to compensate for errors in data entry and phonetics.

\subsection{How is fuzzy matching implemented?}

Texis uses a proprietary Thunderstone algorithm that generates a
similarity measure between any two words or patterns. This measure
is expresses as a percentage of "closeness" to the users input.


\section{Text Maintenance}

\subsection{Can we edit documents from our text retrieval application?}

Yes

Any operating system editor, word processor or desktop publishing system
may be used providing that:
\begin{itemize}
 \item  for non-ASCii documents, there is a conversion
filter

 \item  the text retrieval application has been set up to call the
appropriate application software
\end{itemize}
A chosen document is 'passed' to the editing software - which can be
called from the text application.  At the end of editing control is passed
back to the text application.

Note:  The Zero Latency feature will handle the indexing of the revised
text -  making it always current and available for searching.

\subsection{Is the text index automatically updated after editing?}

When control passes back to the application it will inform Texis that
the text has changed.  This will be immediately noted in the index, so
subsequent searches will work as expected.  A seperate maintenance
program monitors the index to keep it in an efficient form.

(See prior question)

\subsection{Can documents be printed as well as displayed?}

Yes

This is handled in a manner similar to editing.

\subsection{Is there an AUDIT trail facility?}

Yes

In that you can use the Texis DBMS facilities.

\subsection{Can documents be deleted from the Texis system?}

As documents are treated the same as any other record in Texis they can
be deleted with a SQL \verb`DELETE` command.

\subsection{Once physically deleted, are external documents accessible from the
operating system and/or can they be reinstated within Texis?}

The external documents are always accessible from the operating system -
only the Texis record and index information are removed.  They can be
reinstated by reloading this information into Texis and then indexing.

\subsection{When maintaining text does the user come from the application or the
word processor used to create the text?}

The user is taken seamlessly from the application to the word processor
used to create the text.  If this route is not taken then the data in the
WP file might be 'out-of-step' with the text index used by the
application!

\section{External Data Sources}

\subsection{Will it load formatted Word Processor (WP) files?}

Texis currently supports about 100 different file formats.
Documents with full WP format control characters can be loaded. Texis'
lexical filter will reccognize the language content, and discard the
control information.

\subsection{How does Texis handle formatted WP files?}

Documents with full WP format control characters are stored in an Texis
table in their original format.

\subsection{Can we use a WP directly on retrieved text?}

A WP package can be used directly on the text when a document has been
retrieved. The WP package would be launched from the application. Also a
retrieved document could be placed in a standard file and then accessed by any
appropriate WP system.

\subsection{Can we interface to third party publishing packages like Interleaf, Ventura and
PageMaker?}

Yes

In exactly the same way as word processor documents are handled.

\subsection{Does Texis work with mass storage devices such as CD-ROM?}

By definition, CDROM is read-only.  Thus the only use that can be made of
it is as a SOURCE of data for Texis.  However there may be situations
where data is distributed on CDROM, and we want to select certain data
from it (using the retrieval system which is usually supplied with the
disk).  This data could then be loaded into Texis in the normal way.

The possible uses of CDROM are:
\begin{itemize}
 \item  a complete Texis database could be built on CDROM.  As a CDROM
is typically slower than a hard disk, a corresponding reduction in
retrieval rate will be seen.

 \item  The files on a CDROM can be indexed by Texis as external
documents.  Again the slower transfer rate must be taken into account.
\end{itemize}

\subsection{Can Texis handle structured documents?}

Yes. By correctly defining the 'look' of the doument structure to Texis-import.

\subsection{Do we support HTML/XML/SGML documents?}

The Texis program Webinator provides full HTML 3.0 import support.
XML/SGML tag / structure information may be described for use by Texis-import
either with XQL or by regular expression.

\subsection{Can we use spreadsheet information?}

Yes, in the same way as word processing files.

\subsection{Can we use PDF, Postscript and EPS?}

Thunderstone supports Adobe Acrobat / PDF files directly on platforms
where the Adobe toolkit is available.

Postscript files may be imported by Texis. Certain postscript files may not be
legible to the text index program however; it depends on the origin
of the file.

\section{Application Development}

\subsection{How can customers generate their required solution?}

Thunderstone's Texis Webscript is the best and fastest way to create
Internet / Intranet Web aplications.

Thunderstone provides a complete API that can be used to build
applications using standard C.  These calls allow the easy embedding
of SQL into C programs.

Texis also supports ODBC, and provides an ODBC driver for Windows.
The same ODBC call set is also available to developers on all
platforms supported by Texis.

\subsection{Can customers integrate documents into existing Texis applications?}

Document storage and retrieval can be easily added to existing Web, ODBC
and C-based applications.  It involves basically creating new tables
and indexes.

\subsection{Is ODBC the only API supported?}

No.

As well as supporting ODBC on all platforms we also provide our own
proprietary 'C' APIs which are faster and simpler than ODBC for talking to
Texis.

\subsection{Is there support for Network Code Generator applications?}

An NCG interface is supplied for use by Thunderstone customers to
integrate into their other NCG applications.

\subsection{Does Texis use Thunderstone's Metamorph API?}

Yes

You can use any of the normal metamorph API functions within a Texis
application, although this is not generally required.  Texis does not
however use the 3DB API.  In fact it is possible to implement the 3DB API
using Texis.

\subsection{Can Texis messages in an application be customized?}
Yes

Because the generated messages are passed to a function in the application
which can analyze and modify them, and optionally display them.

\subsection{What areas are covered by the API?}

Both toolkits consist of over 50 user exits or function calls that can be
classified into the following areas:
\begin{itemize}
 \item  Query Manipulation - creation and modification of the
extended-SQL statement.

 \item  Query Execution - performs the search against the database and
optionally generates the result set.

 \item  Result Set Control - builds the resulting results and provides
functions to access the data with in the result set.

 \item  Text Display/Printing - ability to determine where in the result
set the hit occurred.

 \item  Text Manipulation - provides the required updating capability,
including the immediate updating of the text index in online mode.

 \item  System Control - initialization and termination of sessions,
saving of queries, and invoking of operating system commands.
\end{itemize}
\subsection{Is there a new API with this version?}

Yes, Texis has it's own APIs, which are either standards (ODBC), or
are modelled after the rest of the Thunderstone APIs.

\subsection{Is a command language supplied?}

Yes

The command language used is SQL, with some added functions for loading
text.  Almost all of the required interaction can be performed entirely
within SQL.

\section{Client/Server Issues}

\subsection{Can Texis work in client/server mode?}

Absolutely, this is the intented purpose of Texis!

Because an NCG interface and an ODBC interface are provided the Texis
application they will work seamlessly over a network.  Any programs
written to the server level API will not work over a network.

\subsection{How do we invoke Texis over a WAN?}

Any WAN issue is handled by NCG or ODBC rather than Texis.

\subsection{Does the document, if external, have to be in the domain of the
database?}

Yes, the documents must be accessible to the server, otherwise it is
impossible for Texis to read them for the purpose of indexing them.

\subsection{Does Texis have to be installed on both the client and the server?}

Not if its Web Based.

In client server mode the product is split into two distinct parts.
The server and the client
library.  The only part required to reside on the client side is the
client library, which only provides access to Texis.

\subsection{Where does the processing take place?}

Query formulation, and text editing will occur at the client.  All
indexing and query execution occur on the server.

\subsection{Where are the text table indexes held?}

Each table can have its own index and each index can reside on different
physical devices.

\section{Miscellaneous}

\subsection{Will Texis work in a bitmapped environment?}

Yes

Since applications can run under ODBC then all environments that ODBC
operates in are available to text retrieval applications.  Also any
environment capable of calling a standard library can access Texis, so
this opens up almost all environments.

\subsection{Are hypertext links supported in the product?}

Yes, there are a plethora of uses for hyperlinks within HTML environments
using Texis Webscript.

\subsection{What languages does this version work with?}

The text can be in any language.  The definition of a word is completely
flexible.  Currently the morpheme processing has been tuned to work best
with English like suffixes and prefixes.

\subsection{What levels of security exist for accessing data within an application?}

All normal Texis access and security levels can be used.  In addition a
user of a Texis application must be a registered user in the Text
Dictionary (as well as in the Texis Dictionary).  Registered users have
username/password combinations.  Also, individual document (row) security
level could be easily coded in a 5GL application.

\subsection{How does Texis handle images?}

Texis stores, manages, and delivers these objects.

The display of images would be via another application or a
third party image system like a web browser, Visual-Basic, Powerbuilder,
Diamondhead, or
Microsoft Access.

The index for image objects normally consists of precoded attributes,
stored in the database.  If some text can be associated with an image, via
data entry or the use of Optical Character Recognition (OCR) software,
then a text index for the images can be created.

\subsection{In what type of applications would you use Texis?}
\begin{itemize}

 \item        any kind of Web site application

 \item        Integrated data application market

 \item        Largescale Document Retrieval applications

 \item        'Enterprise and wide' data-sharing market

 \item        Personnel records; career development plans, organization skills register.

 \item        Legal records; client history, caselaw.

 \item        Project Management; activity tracking, contract management.

 \item        Market research; competitive information, market analysis, demographic
data.

 \item        Company Information services; products, pricing, services, sales channels,
customers, applications, policies.

 \item        Materials and Services catalogues.

 \item        Document tracking, drawings management.

 \item        Bibliographics and Abstracts.

 \item        Research databases (all research fields).

 \item        Regulatory information.

 \item        Help desks.
\end{itemize}

\subsection{Which are the key vertical markets where Texis can be used?}

These are substantially the same as our installed base:
\begin{itemize}
 \item        Internet, Intranet - Texis has global apllicability here.

 \item        Pharmaceuticals - largescale document retrieval - FDA drug submission.

 \item        Energy - safety information, procedures, marketing and product data, project
information.

 \item        Legal - Texis will supplant Thunderstone's  already significant share of this vertical market.

 \item        Utilities - need Texis for Regulatory control - especially Nuclear!

 \item        Government - Military, Intelligence, and Police systems.
\end{itemize}
\subsection{Is it targeted at any particular industries?}

No

Texis can be used to create software solutions in all industries.  Our
customers constantly come up with new and innovative applications.  ( In
fact, they are better at it than we are.  )

\subsection{How does the user benefit?}

The ability to handle text (especially with structured data) means that
new applications can now be undertaken.  The user has control and use of
an important information resource - the written word.  We already have
many customers who have gained significant competitive edge by harnessing
the combination of text and structured data.


\end{document}

%
%\subsection{In Version 1.1 the utilities did not have a good 'look and feel'. Has anything been
%done to improve them?}
%
%You bet!
%All maintenance and set-up utilities have been randwrillen using SQL*Forms 3.0
%and Menu 5.0. They now look and work great! No more having to remember
%those names ... STRRD(, SThTDU, SThDBA, STROIX, etc.
%
%
%
%
%
%
%
%
%
%
%
%
%\chapter{Sales/Marketing}
%
%\subsection{What is the current status of Texis?}
%
%As at November 1st 1992, Texis has gone production and is available on two
%platforms, Sun/Sparc and VAX/VMS. Also a production~uality porting kit is
%available for all other Thunderstone porting groups. The latest porting kit, is a dual
%interoperable release working with Texis Version 6.0 and Texis7.
%
%\subsection{What is the market for this product?}
%
%By integrating text retrieval functionality with DBMS and 4GL tools and by
%providing an 'open' product which can be combined with other technologies such
%as image or drawing systems - Orade has redefined the market. Now text retrieval
%becomes an integrated component of every information application, rather than a
%niche application for librarians. As such its market potential is huge and will
%rapidly grow into a billion dollar market.
%
%We can now compete in:
%
% \item        Integrated data application market
%
% \item        Largescale Document Retrieval applications
%
% \item        'Enterpri sandwide' data-sharing market
%
%\subsection{Which main application area does it cover?}
%
%sQL*TextRetrieval is suitable for a very wide range of applications. Some example
%application areas are listed below:
%
% \item        Personnel records; career development plans, organization skills register.
%
% \item        Legal records; client history, caselaw.
%
% \item        Project Management; activity tracking, contract management.
%
% \item        Market research; competitive information, market analysis, demographic
%data.
%
% \item        Company Information services; products, pricing, services, sales channels,
%customers, applications, policies.
%
% \item        Materials and Services catalogues.
%
% \item        Document tracking, drawings management.
%
% \item        Bibliographics and Abstracts.
%
% \item        Research databases (all research fields).
%
% \item        Regulatory information.
%
% \item        Help desks.
%
%\subsection{Is it targeted at any particular industries?}
%
%No
%
%Texis can be used to create software solutions in all industries.
%
%\subsection{Which are the key vertical markets where Texis can be used to develop
%Thunderstone accounts?}
%
%There are already five key areas emerging:
%
% \item        Pharmaceuticals - largandscale document retrieval - FDA drug submission.
%
% \item        Energy - safety information, procedures, marketing and product data, project
%information.
%
% \item        Legal - will help Thunderstone to gain a significant new share of this vertical market worldwide.
%
% \item        Utilities - need Texis for Regulatory control - especially Nuclear!
%
% \item        Government - 'Intelligence'/Police systerns.
%
%\subsection{What new words must I learn to talk about text systems?}
%
%Please see the glossary of terms at the end of this document.
%
%\subsection{What are the top selling points of Texis?}
%
%1.      Application Development Tool
%
%Texis is an application development tool that enhances the
%Texis application environment by adding full function text retrieval
%capabilities to the database. It combines the management of text and structured
%data in an integrated system, thereby allowing organizations to create a much
%richer set of applications.
%
%2.      A Flexible and Complete Approach
%
%Texis enhances an organization's development environment by
%allowing any Thunderstone application to be tailored to a company's needs.
%Centralized and distributed approaches can be used, augrnented by
%easy-to-use ad hoc query capability to retrieve valuable information. This
%offers a great deal of flexibility in how applications are created and who can
%access the information.
%
%3.      Full Integration
%
%Texis is fully integrated with the Thunderstone Database Management
%System. lt extends SQL, the industry-standard language for data handling. The
%result is a comrnon language for storage and retrieval of structured data and
%text, and coordination of multi-media both from within the database and on
%any other file system.
%
%4.      Powerful Retrieval Options
%
%Comprehensive retrieval capabilities are an integral part of the system. By
%extending the SQL language to include a simple CONTAINS clause, it enables
%the user to easily retrieve data and text in the same query. Full Boolean logic,
%complete word, phrase and fuzzy matching logic provide the options to get the
%data that is needed. This improves productivity by delivering accurate, timely
%and complete information.
%
%5.      Efficient Application Methods
%
%Applications can be created by using Thunderstone's 4th generation application tools
%like SQL*Forms or C Language programs. Both provide all benefits of rapid
%application development and the ability to change applications fast.
%
%
%6.      Common Software Environment
%
%All applications can be aeated, maintained and run in a common software
%environment, Texis. This equates to a high return on an organization's
%investment in software, staff training and support.
%
%7.      Flexible End User Access
%
%SQL~extRetrieval and other Orade software allow the application access and
%functionality to be tailored to the end user's needs. The end user interacts with
%all data in a common way, using interfaces with a common 'look and feel'.
%
%8.      Efficient Organization Cf Data
%
%A sophisticated array index is the key to flexible fast retrieval of information.
%Words and data can be organized and indexed based on the need within the
%application. This option increases the efficiency of a search because it will
%provide specific information requested by the user in the form that will be most
%useful.
%
%9.      Concept (Topic) Searching
%
%Texis provides users with the ability to specify relationships with
%the data to include synonyms, thesauri and alternative words. This allows them
%to customize the data based on their particular business language. By
%integrating concepts within the application, powerful options can be created
%that define subjects and topics. ALL relevant information can now be retrieved.
%
%10.     Single Vendor/Competitive Price Point
%
%Texis is very competitively priced at 30% of the RDBMS. By
%adding this valuandadded tool to the suite of development products, Thunderstone
%delivers an RDBMS, 4th generation tools and text retrieval software from one
%source.
%
%None of Thunderstone's major competitors can offer this combined functionality.
%
%\subsection{What types of applications are {\em not} suitable for Texis?}
%
%Texis is {\em not} aimed at the 'real-time' news feed market nor the CD-ROM
%mastering process.
%
%\subsection{Is this version going to be marketed broadly?}
%
%Yes, it will be marketed in most of the 92 countries Thunderstone sells and supports its
%products in.
%
%\subsection{Are there any easy ways to sell Texis?}
%
%Yes
%
%The easiest way to sell your first copies will be to your customer base, where
%existing Texis applications can be enhanced to include a textual component.
%Most Texis applications fall into this category though obvious end-users
%will be:
%
% \item        Information officers/lnformation Resources Departments.
%
% \item        Commerdal /Contracts Executives.
%
% \item        Project Managers.
%
% \item        Marketing Departments.
%
% \item        Service Departments (especially computer departments).
%
% \item        Human Resource / Personnel Executives.
%
%
%\subsection{What should I focus on when selllng at departmental level?}
%
%The typical sale at departmental level is to an MIS or project manager. The key
%selling point to these decision makers is integration:
%
% \item        integration management of data and documents
%
% \item        one universal' query language
%
% \item        integrated CASE and development tools
%
% \item        support for integrating applications such as word processors
%
%\subsection{What should I focus on when selling at executive level?}
%
%Texis can be sold effectively at executive level by focussing on the
%strategic uses of documents as part of enterpris~wide information systems and on
%the high costs of not having a document management strategy. Sales at this level
%should stress that a large percentage of important corporate information is locked
%up in paper-based or proprietary electronic filing systems, but that
%Texis can provide ways of unlocking that information and integrating
%it with all other corporate data.
%
%\subsection{How can it leverage other sales?}
%
%By selling Texis the customer can expand his applications - which in
%turn will lead to more copies of Texis and tools. Also with many sales it has
%been possible to sell consultancy to design and implement a complete solution.
%
%\subsection{What is a typical sales cycle?}
%
%Since initial sales would be to existing Texis users the cycle could be very short
%- a matter of weeks, provided the customer has identified an application.
%
%\subsection{How many customers do you have currently?}
%
%Over 300 customers bought Version 1.1, comprising 450 copies. Texis has just
%really started to be sold.
%
%\subsection{What kinds of customers have purchased the product so far?}
%
%Texis customers typically tend to be chemical companies, police and
%intelligence forces, legal firms and departments.
%
%A partial customer list and active references can be obtained by emailing
%AJEFFRIE.UK, or your local marketing contact.
%
%\subsection{What are customers doing with the product?}
%
%The most common use is to manage large collections of word processor documents.
%Other common categories are imaging and managing paper flow.
%
%\subsection{Who inside a company will use this product?}
%
%Developers who create document management, workflow and imaging
%applications. The users of the application may be anyone: secretaries, attorneys,
%managers, clerks, research scientists, etc.
%
%\subsection{Has anyone bought this who did not have an Texis database?}
%
%Yes. An increasing number of users select Thunderstone because of Texis and
%buy the database as a requirement.
%
%\subsection{How does the user benefit?}
%
%The ability to handle text (especially with structured data) means that new
%applications can now be undertaken. The user has control and use of an important
%information resource - the written word. We already have many customers who
%have gained significant competitive edge by harnessing the combination of text and
%structured data.
%
%\subsection{How is the product priced?}
%
%Each individual country is responsible for pridng. The recommended price i~ 3()~
%of Texis kernel or equivalent single-user price. Please check with your local
%marketing group.
%
%\subsection{Has the price been increased over Version 1.1?}
%
%No
%
%The price of Texis is the same. Customers who have a maintenance
%contract for Version 1.1 receive a free upgrade.
%
%\subsection{How will my customers be supported?}
%
%Your customers will be supported through the normal Thunderstone support channels.
%
%\subsection{How will my customers be trained?}
%
%Your customers will be trained through the normal Thunderstone training channels. A
%threeday customer training course is now available for your country to run.
%
%\subsection{How is it used in Consultancy?}
%
%Consultants can now define and build systems for customers that use document
%storage and retrieval as part of a total solution. Several customers have already
%used Thunderstone consultancy in this way It is a unique dimension that we can bring to a
%sale.
%
%\subsection{What user documentation is available for the product?}
%
%        Documentation   Part No.
%        Texis Developer's Guide     0363200792
%        Texis Administrator's Guide 036~200792
%        Texis Messages      03~200792
%        Texis Installation Guide    Msg PIm INFO
%
%
%\subsection{What Sales/Marketing collateral is currently available for the product?}
%
%The Sales/Marketing collateral currently available is:
%
%1.      Product Demos
%
%        Collateral      Part No.
%        Texis V2.O  50836-0792
%        Scripts for the Suite of Demonstration Applications
%        This document completely describes the demos
%        included on the distribution tape, along with
%        instructions for installing them.
%
%
%2.      Customer Collateral
%
%        Collateral      Part No.
%        Texis Data Sheet    51~)o2#)792
%        Texis V2.O Technical Overview       50674~792
%        Thunderstone at Work - Centocor       52793#?692
%        Research, legal, and document management applica-
%        tions at a leading biotechnology firm.
%        Thunderstone at Work - Baring Securities      506764)190
%        Market research applications in the financial indus-
%        tries.
%        Thunderstone at Work - VACT   526674)592
%        Legislation, case law, and regulation retrieval sys-
%        tern used by judges of an Australian government
%agency.
%Thunderstone at Work - Thunderstone Worldwide Support       52245-1091
%Help desk application implemented by the UK
%Worldwide Support Center. Contains useful mea-
%sures for quantifying the benefits of SQL*ThxtRetri~
%val.
%
%
%3.      Sales Aids for lnternal Thunderstone Use Only
%
%
%        Collateral      Part No.
%        Texis Questions and Answers 50108-1192
%        Texis V2    506924)792
%        Product Overview Presentation
%This comprises a set of 35mm slides plus hardcopy
%
%
%In addition to the items listed above, all of which are available from US shipping,
%your local marketing contact for Texis may have information relating
%to competitors, internal sales/technical presentations and customer lists and
%references.
%
%\subsection{What demonstrations are available?}
%
%A suite of demonstration applications are shipped with Texis Version
%2.0. They cover a wide range of Texis's features and include
%applications for manufacturing, pharmaceuticals and legal systems. A full
%demonstration script is available, Part No.508360792.
%
%For major customer presentations prototypes of the customer's application can be
%created in a mafter of days by pr~sales consultants.
%
%\subsection{Are there any help lines available?}
%
%INFOSTh - sales/marketing questions
%
%HELPSTh - technical questions
%
%QUICKINFO - direct online information for browsing
%
%\subsection{Has the company really made a commitment to the product?}
%
%Yes. Texis is a tier 1 Thunderstone product, internally developed to corporate
%standards.
%
%\subsection{Who develops SQL*TextRetrleval?}
%
%The product is developed by the Text Retrieval SBU. This group, part of Thunderstone's
%Product Division, is located in Chertsey, England.
%
%\subsection{Are there any SBU contacts?}
%
%ALAN JEFFIcIES (AJEFFRlE.UK) - Worldwide Marketing Manager (ChertsQv, UK)
%
%ROGER FORD (RAFORD.UK) - Development Support Manager (Chertsey, UK)
%
%\subsection{What is our position with customers who want to store and retrieve both images
%and textual documents?}
%
%More customers are using Imaging Technology to store large volumes of
%documents. The problem is that images cannot be searched for content; only by
%piHiefined attributes. This can be overcome by a combination of storing
%documents in textual format (perhaps an abstract) with the full document in image
%format. The text documents can then be used to search for the relevant documents
%and then Image software/hardware can be used to retrieve and display the
%required images. Several customers are using or building these ldnds of
%applications.
%
%\subsection{Does Thunderstone provide any of this Image technology?}
%
%Not Today.
%
%However, oCSC (USA) has a strong capability as system integrators to build
%applications for customers. Also, there are many good image systems from
%third-parties. Most either use Texis for their attribute indexes or can be
%easily linked via a pointer mechanism. Thunderstone has two OEM's - FILENET and
%Oki-Data - who are strong in the Image marketplace (for more information call
%the OEM group in World Headquarters).
%
%\subsection{Do you see Texis becoming part of the database?}
%
%Eventually, either as part of the database or more closely associated with the
%database.
%
%
%\chapter{The Competition}
%
%
%\subsection{What broad areas do the competition fall into?}
%
%There are three main areas of competition for Texis in the
%content-based retrieval (CBR) market:
%
% \item        relatively small companies with uWt~ate technology
%
% \item        systems and hardware vendors that use bought-in or licensed products
%
% \item        older vendors of host-based systems
%
%\subsection{Do these small companies pose a threat to us?}
%
%These companies tend to be very young or Start-up companies. On their own they
%are too small to pose a threat to Thunderstone. However, they may pose a threat in the
%future if VARs and hardware vendors partner with them for their technology,
%leaving us out of the picture as suppliers.
%
%\subsection{What's the position of the systems and hardware vendors?}
%
%The current position of the systems and hardware vendors is as follows;
%
% \item        DEC has a joint development effort with Verity to provide the retrieval
%capabilities Topic as a NAS service. It is also combining with the startup
%company, Information Access systems (lAS) to incorporate similar
%technology into Rdb.
%
% \item        Hewlett-Packard is coupled with reOuest from Harris and Paulson and Saros
%Mezzanine library services. However, their long term strategic direction is to
%develop retrieval capabilities based on technology licensed from Verity.
%
% \item        IBM has the old Stairs, which has been upgraded and renamed
%BookManager. Their current development may be dropped in favour of a
%purchased product.
%
% \item        NCR licenses technology from Fulcrum.
%
% \item        Sun Micro systems licenses technology from Fulcrum.
%
%It is believed that none of these companies is so totally 'locked-in' to other text
%retrieval vendors that we cannot replace them.
%
%\subsection{Are the older vendors a threat to us?}
%
%Just the opposite in fact! Although these vendors have a large customer base, their
%products are monolithic and closed in architecture terms. As these become obsolete,
%Thunderstone can make use of this opportunity to increase its market share.
%
%\subsection{Are there no players from the RDBMS marketplace?}
%
%Not yet!
%
% \item        Sybase - are able to query long text fields directly, but this does not appear to
%be any more than essentially a \verb`LIKE` function in the SQL statement. There are
%certainly no concept search management facilities.
%
% \item        Informix - have some BLOB support but it does not compare against
%Texis in its query fadlities.
%
% \item        Ingres - have announced they will support BLOBs and provide extensions to
%SQL for this purpose in the next major release.
%
% \item        Interbase - have good support for BLOBs, but poor query facilities.
%
%
%\subsection{Which companies are our main competitors?}
%
%At the moment the market is highly fragmented and is populated by numen~u~
%small vendors, none of whom possess a commanding market share. Only two of
%them, Verity and Fulcrum, have competitive products based on client/server
%technology The current market share leader, Information Dimensions Inc (IDI),
%dominates the mainframe sector, but up to now has not entered the LANbased
%marketplace.
%
%\subsection{What is really unique about Texis?}
%
%There are three main differentiators for Texis:
%
%1.      Fully integrated with Texis DBMS and tools.
%
%2.      Uses an extension of the industry standard SQL language.
%
%3.      Uses 4GL application development tools for text retrieval applications.
%
%\subsection{What are our strengths against these competitors?}
%
%Texis and Thunderstone are strongly positioned against the rival products.
%Thunderstone has the advantage of its relatively large size, existing RDBMS base and an
%integrated development environment.
%
%\subsection{Do you compete directly with Verity and IDI?}
%
%Yes, these two companies are among our strongest rivals.
%
%\subsection{How does Thunderstone's approach compare against Verity?}
%
%Verity is young aggressive company that has excellent PR as well as product. They
%license technology to other companies and sell high end tools rather than customer
%applications. They have strong distribution capabilities from their close relationship
%with DEC. Their weakness lies in viewing text retrieval as a departmental concern
%rather than executive level, unlike Thunderstone which can provide corporatewide
%solutions. Also they are too small to support large, global companies.
%
%\subsection{What is Verity's product like?}
%
%Their product, TOPIC, is a client/server system available on mid-range to desktop
%platforms. It has strong integration with word processing packages and is used
%extensively as a filtering agent for newswire and other on-line data sources.
%
%Queries are difficult to design effectively and build; its data access language is
%proprietary. However, Topic does support the ranking of query results by
%user-supplied weights. It also has gateways to Texis, Sybase, Ingres,
%INFORMIX, and Rdb provided by SQL Bridge though the RDBMS interface
%functionality is very weak.
%
%Compared to Texis it is very expensive.
%
%\subsection{How does Thunderstone's approach compare against IDI?}
%
%IDI is the acknowledged market leader. They dominate the mainframe arena but
%have not yet entered the LAN-based systems market. (Though with their
%acquisition of Zylab they may integrate the two product ranges to cover both
%markets.) They supply high quality, high end systems. Their weakness is like
%Verity's, in that they are too small to support large global companies.
%
%\subsection{What is IDI's product like?}
%
%Their product, BASISplus, uses a client/server architecture and has been ported to
%several platforms, some Unix, VMS, MVS and VM. It is based on what IDI calls a
%"relational" database (but actually a network database) which does not support
%SQL and is not easily integrated with RDBMS applications. However their product
%set is well integrated and has good support for word processors.
%
%Again compared to Texis, it is very expensive.
%
%\subsection{How does Thunderstone's approach compare to Fulcrum Technologies Inc.'s?}
%
%Fulcrum sells a text retrieval engine and an API. You can implement the
%applications any way you want. However, you have to code the entire application
%yourself in C, there are no development tools to help you. Also their product ha~ no
%integration with RDBMS.
%
%\subsection{Who will be the competition irt the future?}
%
%Other DBMS vendors will develop or acquire a 'document storage and retrieval'
%capability Also this technology will be increasingly part of fully functional
%electronic document management offerings, from such companies as Lotus, Xerox,
%Hewlett-Packard and Interleaf.
%
%\chapter{Standards, Thunderstone and Third Party Products}
%
%\subsection{Can it be used with external news services (e.g. INFO GLOBE, Dow Jones, etc.)?}
%We do not target Texis at this application area.
%
%\subsection{How many ISV's (Independent Software Vendors) and VAR's do you have lined
%up for this?}
%
%Several are already using early copies of Texis to
%incorporate text retrieval facilities into their applications. Many more are interested
%now that Texis has reached production.
%
%\subsection{Does the software support the following:}
%
%(i)     DCA (Document Content Architecture)
%
%(ii)    DDIF (Distributed Document interchange Format)
%(iii) SGML (Standard Generalized Markup Language)
%(iv) ODA (Office Document Architecture)
%
%(v)     CDA (Content Document Architecture)
%
%These are all ISO or industry standards for document markup and document
%structure definition. These are supported in as much as we provide an interface for
%customers to write their own filters for documents of those types.
%
%\subsection{How does it relate to Thunderstone BookViewer?}
%
%Essentially the two products are complementary.
%
%BookViewer is aimed at the display of multi-media documents. It can also navigate
%through documents using hypertext techniques. Also once within a large document
%you can search for words or phrases.
%
%However, it is not targeted to be used for searching and retrieving documents in
%large databases, nor can it search on combined text and structured fields. So the two
%products could be used together, where sQL*TextRetrieval could be used to find a
%required set of documents and Thunderstone BookViewer could be used to display, browse
%and navigate through the results set.
%
%\subsection{What other Thunderstone products does Texis work with?}
%Tools
%
%Because sQL*TextRetrieval is integrated with Texis, most products that
%work with Texis can be used with document applications. These include
%SQL*Net, SQL*ReportWriter, SQL*Forms 3.0 and PRO*C.
%
%CASE
%
%CASE*Generator Texis for SQL*Forms/SQL*Menu will generate
%applications with text retrieval functionality~ CASE*Dictionary Version 5.0 has
%some embedded Texis technology.
%
%Personnel
%
%Texis has been built as an option to Thunderstone Personnel.
%
%\subsection{Is the ANSI SQL committee looking at text retrieval extensions?}
%
%Yes.
%
%ANSI has now become very active through the SQL3 committee. Thunderstone is very
%active in providing its experiences and customer feedback concerning text
%extensions.
%
%\subsection{What is the stance on SFQL?}
%
%SFQL superficially appears ver,' similar to Texis's extended SQL, but
%SFQL has no connections with either SQL or RDBMS. Thunderstone's position is that
%customers are best served by a single data specification language that can handle all
%types of data, structured and uns~ructured, together in the same quefl'.
%
%\subsection{Can you use Thunderstone Card as a front end?}
%
%Not currently, but a development plan is in place to use it with Texis of
%Texis in the future.
%
%\subsection{Is Texis designed for use with Texis7?}
%
%It works with both Texis Version 6.0 and Texis7.
%
%\subsection{Are there any Office Automation products coming out in the near future that
%relate to this?}
%
%The first Office Automation product that will have Texis fully
%integrated into it, will be Document Manager.
%
%\end{document}
%
%List of Questions
%
%1.      Why Is This Technology Important?
%        1.1     Why is Text important to our customers?
%        1.2     What does a Text Retrieval System provide?      3
%        1.3     Why is Thunderstone Corporation interested in Text Retrieval? 3
%        1.4     What is Thunderstone's approach?      3
%        1.5     What is the opportunity for Thunderstone?     4
%
%
%2.      Product Features
%
%        General         5
%        2.1     What is sQL*TextRetrieval?      5
%        2.2     What does it do?        5
%        2.3     What are the components of Texis?   5
%        2.4     What are the software dependencies?     5
%        2.5     Are there any hardware dependencies?    5
%        2.6     What additional functionality does Texis provide over Version 1.1?        6
%        2.7     What major performance improvements does Texis offer over Version 1.1?    6
%        2.8     How have these terrific performance improvements been achieved? 6
%        2.9     With all this new technology and functionality in Texis, how compatible is it
%                with Version 1.1?       7
%
%        Text    Storage 7
%        2.10    How is text stored in an Texis database?       7
%        2.11    What size of text database can Texis handle?      7
%        2.12    What size of individual document can Texis handle?        7
%        2.13    Can there be multiple text columns per text table?      7
%        2.14    Can the text be compressed to reduce storage requirements?      7
%        2.15    Does the tokenized text sit inside or outside the database?     7
%        2.16    Do all documents need to be stored in Texis tables?    8
%        2.17    Can we store additional information alongside the text? 8
%        2.18    Can we work with compound documents that contain text, data and images? 8
%        2.19    How do we get text into the database?   8
%        2.20    Where can it originate from?    8
%        2.21    Are facilities for dumping and loading data supplied with the product?  8
%
%        indexing                8
%        2.22    What indexing strategy does Texis provide?  8
%        2.23    What is a Text index?   8
%        2.24    Why is the Text Index important?        9
%        2.25    What indexing utilities are supplied?   9
%        2.26    Is there one index per application or table or database?        9
%        2.27    Is the indexing separate from Texis and where is the resultant text index held?        9
%        2.28    How are non-text objects indexed, e.g. graphical images?        9
%        2.29    Where are the indexes for external documents?   9
%        2.30    Is there an overhead in space utilization?      9
%        2.31    Is Texis case sensitive?    9
%        2.32    Can the wordlist be viewed?     9
%        2.33    Can the wordlist be modified?   9
%        2.34    Do you supply a STOP LIST and can it be modified?       10
%        2.35    In what languages do you provide the STOP LIST? 10
%
%
%Coinpany Confidential
%2.36    Do you control plurals of words in the index7   10
%2.37    Does Texis support automatic randindexing7    10
%
%Document Retrieval              10
%2.38    Is the Retrieval Language based upon SQL7       10
%2.39    is this as powerful as the query capability in the database?    10
%2A0     Are the extensions to SQL proprietary?  10
%2.41    How has 'extended SQL' been extended7   10
%2A2     How complicated is the extended SQL?    10
%2A3     Can Natural Language queries be used as input7  10
%2.44    What are the options then for generating queries7       11
%2A5     Can we search on 'noise' words7 11
%2A7     Can we search on numbers or dates7      11
%2.48    Can we use wildcard searches7   11
%2.49    Do you have a Thesaurus7        11
%2.50    What relationships are allowed7 11
%2.51    Do the thesaural relationships conform to any standard7 12
%2.52    Can we import/export conceptual definitions across databases7   12
%2.53    Can an independently-supplied thesaurus handler be used in conjunction
%        with SQL~extRetrieval7  12
%2.54    What is fuzzy matching? 12
%2.55    How is fuzzy matching implemented7      12
%2.56    Do we have to view all the documents retrieved by a query7      12
%2.57    Can we browse the database as well as the hitlist7      12
%2.58    is the hitlist available for other application processes?       12
%2.59    Can Texis do relevance ranking of retrieved documents7      12
%2.60    After a search can the resulting text be highlighted7   12
%2.61    Havin done one query, I wish to search on a more limited range.
%        Can I 0 this without ie~executing the entire query7     13
%2.62    What is User Profiling and can Texis do it~ 13
%
%Text    Maintenance     13
%2.63    Can we edit documents from our text retrieval application?      13
%2.64    Is the text index automatically updated after editing7  13
%2.65    Can documents be printed as well as displayed?  13
%2.66    Is there an AUDIT trail facility7       13
%2.67    Can documents be deleted from the Texis system7     14
%2.68    Once physically deleted, are external documents accessible from the operating system
%        and/or can they be reinstated within Texis7 14
%2.69    When maintaining text does the user come from the application or the word processor
%        used to create the text7        14
%
%External Data Sources           14
%2.70    Will it load formatted Word Processor (WP) files7       14
%2.71    How does Texis handle formatted WP files7   14
%2.72    Can we use a WP directly on retrieved text7     14
%2.73    Can we interface to third party publishing packages like Interleaf, Ventura
%        and PageMaker7  14
%2.74    Does Texis have any 'format managers' included in the product7      15
%2.75    Does Texis work with mass storage devices such as CD-ROM7   15
%2.76    Can Texis handle structured documents7      15
%2.77    Do we support Standard Graphical Markup Language (SGML) documents7      15
%2.78    Do you have a bridge to move from one word processor to another7        15
%2.79    Hdw is word processor independence implemented? 15
%2.80    Can we use spreadsheet information7     15
%2.81    Can we use EPS (Encapsulated Post Script)7      15
%
%
%2  SOL~extRetrieva1
%        Application Development         16
%        2.82    How can customers generate their required solution?     16
%        2.83    Can customers integrate documents into existing Texis applications?    16
%        2.84    Is PRO*C the only precompiler supported?        16
%        2.85    Is there support for Application Foundation applications?       16
%        2.86    Does Texis use Thunderstone's Adaptive User Interface (Toolkit)?  16
%        2.87    Can Texis messages in a SQL*Forms application be customized?        16
%        2.88    What areas are covered by the API?      16
%        2.89    Is there a new API with this version?   17
%        2.90    Is a command language supplied? 17
%
%        Client/Server Issues            17
%        2.91    Can Texis work in client/server mode?       17
%        2.92    How do we invoke Texis over a WAN?  17
%        2.93    Does the document, if external, have to be in the domain of the database?       17
%        2.94    Does Texis have to be installed on both the client and the server?  17
%        2.95    Where does the processing take place?   17
%        2.96    Where are the text table indexes held?  17
%
%        Miscellaneous           18
%        2.97    Will Texis work in a bitmapped environment? 18
%        2.98    Are hypertext links supported in the product?   18
%        2.99    What languages does this version work with?     18
%        2.100 What levels of security exist for accessing data within an application?           18
%        2.101   In Version 1.1 the utilities did not have a good 'look and feel'.
%                Has anything been done to improve them? 18
%        2.102 How does Texis handle images?         18
%
%
%3. Sales/Marketing
%        3.1     What is the current status of sQL*TextRetrieval?        19
%        3.2     What is the market for this product?    19
%        3.3     Which main application area does it cover?      19
%        3.4     Is it targeted at any particular industries?    19
%        3.5     Which are the key vertical markets where Texis can be used to develop
%                Thunderstone accounts?        20
%        3.6     What new words must I learn to talk about text systems? 20
%        3.7     What are the top selling points of Texis?   20
%        3.8     What types of applications are {\em not} suitable for Texis?      21
%        3.9     Is this version going to be marketed broadly?   21
%        3.10    Are there any easy ways to sell sQL*TextRetrieval?      21
%        3.11    What should I focus on when selling at departmental level?      22
%        3.12    What should I focus on when selling at executive level7 22
%        3.13    How can it leverage other sales?        22
%        3.14    What is a typical sales cycle?  22
%        3.15    How many customers do you have currently?       22
%        3.16    What kinds of customers have purchased the product so far7      22
%        3.17    What are customers doing with the product?      22
%        3.18    Who inside a company will use this product?     22
%        3.19    Has anyone bought this who did not have an Texis database?     22
%        3.20    How does the user benefit?      22
%        3.21    How is the product priced7      23
%        3.22    Has the price been increased over Version 1.1~  23
%        3.23    How will my customers be supported7     23
%        3.24    How will my customers be trained7       23
%
%
%        Company Confidential    3
%        3.25    How is it used in Consultancy7  23
%        3.26    What user documentation is available for the product7   23
%        3.27    What Sales/Marketing collateral is currently available for the product7 23
%        3.28    What demonstrations are available7      24
%        3.29    Are there any help lines available7     24
%        3.30    Has the company really made a commitment to the product7        24
%        3.31    Who develops SQL~extRetrieval7  24
%        3.32    Are there any SBU contacts7     25
%        3.33    What is our osition with customers who want to store and retrieve both images
%                and textual ocuments7   25
%        3.34    Does Thunderstone provide any of this Image technology7       25
%        3.35    Do you see SQL~extRetrieval becoming part of the database7      25
%
%
%4.      The     Competition
%        4.1     What broad areas do the competition fall into7  26
%        4.2     Do these small companies pose a threat to us7   26
%        4.3     What's the position of the systems and hardware vendors7        26
%        4A      Are the older vendors a threat to us7   26
%        4.5     Are there no players from the RDBMS marketplace7        26
%        4.6     Which companies are our main competitors7       27
%        4.7     What is really unique about sQL*TextRetrieval7  27
%        4.8     What are our strengths against these competitors7       27
%        4.9     Do you compete directly with Verity and 1D17    27
%        4.10    How does Thunderstone's approach compare against Verity7      27
%        4.11    What is Verity's product like7  27
%        4.12    How does Thunderstone's approach compare against 1D17 27
%        4.13    What is IDI's product like7     27
%        4.14    How does Thunderstone's approach compare to Fulcrum Technologies Inc.'s7      28
%        4.15    Who will be the competition in the future7      28
%
%5.      Standards, Thunderstone and Thfrd Party Products
%        5.1     Can it be used with external news services (e.g. INFOGLOBE, Dow Jones, etc.)7    29
%        5.2     How many ISVs (Independent Software Vendors) and VARSs do you have lined
%                up for this7    29
%        5.3     Does the software support the following:        29
%        5A      How does it relate to Thunderstone BookViewer7        29
%        5.5     What other Thunderstone products does Texis work with7    29
%        5.6     Is the ANSI SQL committee looking at text retrieval extensions7 29
%        5.7     What is the stance on SFQL7     30
%        5.8     Can you use Thunderstone Card as a front end7 30
%        5.9     Is Texis designed for use with Texis77   30
%        5.10    Are there an Office Automation products coming out in the near future that
%                relate to this  30
%
%
%Glossary
%
%Automatic Document Abstracting  The use of      query is placed in the storage using the same
%computer software, lexical tools, and expert systems    algorithms used to position the documents. The
%to automatically create synopses or abstracts of        location of the query becomes a centre around which
%documents. see also Document Abstract.  a retrieval sphere is formed. Documents that are
%        within a spedfied distance to the query are retrieved
%Boolean Operators  Used in search statements to as relevant.
%combine search terms. Usual operators are {\tt AND}, {\tt OR}
%and {\tt NOT}.  Concept-based Retrieval A textual search based
%        on a concept rather than an exact word match. A
%Breaking on a Predetermined Character  Use of a concept automatically defines a list of search terms,
%character (or characters) to delimit text at word,      phrases, and rules. Accomplished at various levels of
%sentence, paragraph, page or document level. See        sophistication through various tools and
%also Lexical Rules.     methodologies (i.e. a synonym list, thesaurus, topic
%        definitions, concept-based clustering). Also known
%Broader Term  A term that represents a more     as concept searching.
%general definition of another term. A thesaurus
%operator that allows the tracking of broader terms.     Compound Document  A document that contains
%Using Broader Term and Narrower Term operators, a       information in several formats: text, graphic, and
%thesaurus can establish a definition hierarchy. See     image.
%also Narrower Term.
%        Cross-database Searching  Ability to search across
%Browse on Index  Ability to search the index of a       more than one database in a single query. See also
%text database for specific words or look through the    Concatenated.
%entire index.
%        Current Awareness  Ability of an application to
%Clustering  A method for logically storing      automatically tell individual users about new
%documents based on relationships between        information on the database that matches their
%documents. Documents that are most relevant to one      personal predesigned selection criteria. See also
%another are logically stored close together. See also   Selective Dissemination of Information.
%Concept-based Clustering.
%        Document  Any item, printed or otherwise, that is
%Compound Term  An indexing term that can be     amenable to cataloguing and indexing. This
%factorized morphologically into separate        definition refers not only to written and printed
%components, each of which could be expressed, or        material in paper or microfilm versions (e.g. books,
%reexpressed, as a noun that is capable of serving       diagrams, maps), but also to non-print media (e.g.
%independently as an indexing term. See also     machin~readable records, films, sound recordings),
%Multi-word Term.        and threedimensional objects or realizations
%        (models) used as spedmens.
%Concatenated  Joined together. Enables more than
%one text database to be searched for the same query,    Document Abstract  A synopsis of a document's
%with the results combined. See also Crossdatabase       content which highlights key concepts, significant
%Searching.      data, and areas of interest in the document.
%        Document abstracts are used as a form of document
%Concept  A unit of thought. The semantic content of     management and retrieval by providing researchers
%a concept can be reexpressed by a combination of        with condse insights into the full document library.
%other and different concepts, which may vary from       The abstracting process has traditionally been
%one language or culture to another.     performed manually. Recent advances in technology
%        have made the automation of the process possible.
%Concept-based Clustering A full text-retrieval  See also Automatic Document Abstracting.
%search methodology based on document clustering.
%Documents are logically stored in a virtual storage     Field-directed Searching  Ability to limit a search
%area (clustering) based on document content and the     to a specified field (or fields); for example, author,
%defined relationship of terms to subject matter. A      title, date. See also Limited Search by Field.
%
%
%        Company Confidential    ~
%Free Text Scanning A full text-retrieval search fixed-length identifier field values (i.e. author, title,
%methodology that 'reads' the entire text database for   date). Also known as Image Processing.
%each query submitted. Typically utilizes a Text Array
%Processor (TAP) to fadlitate the text scarming  Index or Indexes An alphabetical or systematic
%process.        listing of subjects which refers to the position of each
%        subject in a document or collection of documents.
%Free text Text presented to the application in an
%unstructured format.    Indexing Term The representation of a concept in
%        the form of either: a term derived from natural
%Full-text Retrieval (Fin) A software or hardware        language, preferably a noun or a noun phrase, or a
%process that provides the functionality to retrieve     classification symbol.
%textual documents based on the words, phrases, or
%concepts contained in the documents. Also known as      Indicative Text (ITh) Text that identifies and points
%Text Information Management Systems and Text    to documents held elsewhere; for example, library
%Retrieval Systems.      catalogues, bibliographies, abstracts.
%
%Full Text (FTX) Unabridged documents; that is, not      Internal Character Masking A character used to
%bibliographies or abstracts.    mean 'any one' or 'contiguous set of characters'
%        within a word or phrase. See also Wild Card.
%
%Go List List of all allowable terms in the index. See
%also Pass list  Inverted Index An index created from an inversion
%        of all the words contained in all of the documents in
%
%
%History Can be used in a thesaurus for  a text database.
%documenting meaning (or changes in meaning) of a
%term.   Keywords Additional terms that can be added to
%        document records and used in searching. Typically
%
%        these would be used for classification of documents
%Hitlist An automatically generated list showing the     or to restrict the search vocabulary.
%results of a search. It will display details of how
%many occurrences (hits) of the query have been  Lemmatization Reduction of all words with the
%found and in which documents.
%same root to a single form. See also Stemming.
%
%Homograph One of two or more words spelled      Lexical Rules The rules whereby a stream of text is
%alike but different in meaning, derivation or   broken down into tokens. See also Token.
%pronunciation; for example, the noun conduct and
%the verb conduct.       Limited Search by Field It is normal to be able to
%
%        search for the occurrence of a term or word in a
%Homonym A class of words with more than one     specific field such as the field for name or the field
%meaning to the same spelling. Homonyms subdivide        for title. See also Field-directed Searching.
%into homographs and polysemes.
%        Location List List of words pointing to where they
%Hypermedia An extension of the hypertext        occur in the original document. See also Pointer File.
%concept. Hypermedia supports the linking of
%non-text nodes as well (i.e. images, graphics, audio,   Morpheme A component of a compound word or
%and video).     term.
%
%Hypertext A text-retrieval search methodology   Multi-word Term Such as Hydrogen Chloride,
%based on assodative memory process. Hypertext   where the combination of two or more words gives
%organizes textual information into sections known as    rise to a new meaning, which should be entered in
%nodes or chunks. The nodes can be linked together       the index. See also Compound Term.
%via macros that are automatically built by the system,
%based on user input.    Na~ower Term A term that represents a more
%        focused definition of another term. A thesaurus
%Imaging A method of electronic document operator that allows the tracking of narrower terms.
%management in which documents are stored as     Using Broader Term and Narrower Term operators, a
%bit-mapped representations or copies of the     thesaurus can establish a definition hierarchy. See
%document. Imaging systems limit retrieval access to     also Broader Term.
%
%
%
%2  SQL~ThxtRetrieva1
%Node Label  A 'dummy' term not assigned to      Posting  Individual entry in the pointer file
%documents when indexing, but inserted into the  representing the occurrence of one entry, of a word in
%systematic section of some types of thesauri to the original text.
%indicate the logical basis on which a category has
%been divided; sometimes known as a 'facet       Precision The measure of a text-retrieval system's
%indicator'.     ability to deliver only the relevant documents to a
%        user query. Precision is calculated as the ratio of
%Non-preferred Term The synonym or       relevant documents retrieved to the total number of
%quasi-synonym of a preferred term. A non-preferred      documents retrieved.
%term is not assigned to documents, but is provided      Preferred Term A term used consistently when
%as an entry point in a thesaurus or alphabetical
%index, the user being directed by an instruction (e.g.  indexing to represent a given concept; sometimes
%USE or SEE) to the appropriate preferred term;  known as a 'descriptor'. See also Use Term.
%sometimes known as a non~escriptor'.    Proximity Searching  Searching for two words,
%
%        where the user specifies their relative location to each
%Normalization A process that reduces all query  other; for example, words must be adjacent or within
%
%terms, including irregular forms of the word and        a given number of words of each other.
%verb tenses, and document terms to their common
%root word.      Query Language  Defined syntax for the
%
%Numerics Numbers ~9, represented as text        specification of queries. Language employed by the
%strings, unless specified otherwise     user to specify search criteria.
%        Range Searching  Searching for terms between
%ODA/ODIF An architecture for describing specified values. Normally relates to numeric or
%compound documents in such a way that they are  structured fields; for example, dates, ages.
%readable by any system. ISO standard 8613.
%        Recall The measure of a text-retrieval system's
%Optical Character Recognition Often known as    ability to deliver all of the documents relevant to a
%OCR, software used in conjunction with a scanner        user query. Recall is calculated as the ratio of the
%which translates scanned textual images into    documents retrieved to the total number of relevant
%ASCIl/EBCDlC text files.        documents available.
%
%Pass List List of all allowable terms in the index.     Related Term  A term that provides greater insight
%See also Go List.       on another term, but is not synonymous or a
%        broader/narrower definition of that term. Related
%Pattern Recognition A full-text retrieval search        Terms can be tracked in a thesaurus using the
%methodology in which the patterns created by the        Related Term operator.
%groups of letters in the document text are stored and
%indexed as bit vectors. User~ueries are processed       Relevancy ranking  Displaying retrieved
%using the same pattern creation algorithms. The documents in descending order of the level to which
%retrieval process looks for similarities between the    they are pertinent to the query statement.
%patterns in the query and in the text.  Relation, Relationship Indication  Hierarchical
%
%Phonetic Searching A soundex facility.  relationship between terms in the thesaurus
%        structure. The relationship is expressed in: preferred
%
%        term; non-preferred terms; synonyms;
%Pointer File  List of words pointing to where they      quasi-synonyms; abbreviations; acronyms; top term;
%
%occur in the original document. See also Location       broader term; narrower term; related terms.
%List.
%        Scope Note  lnformation that provides insight into
%Polyseme  A word with multiplicity of meaning; for     a Use Term in a thesaurus. Scope notes can provide
%
%example, the noun 'beams' as 'radiation' or     information regarding term qualities such as the
%structures'.
%        availability of related terms or dated definitions of a
%
%        term. These insights are tracked in a thesaurus using
%Positional Inverted Index A full-text retrieval the Scope Note operator..
%search methodology in which an inverted index is
%created which tracks the exact location of words in     Selective Dissemination of Information (SDI)
%documents. See also Inverted Index.     Ability of an application to automatically tell
%
%
%        Coinpany Confidential   3
%individual users about new information in the   Top Term  The broadest dass to which a concept
%database that matches their personal, predesigned       belongs; sometimes used in the alphabetical section
%selection criteria. see Current Awareness.      of a thesaurus.
%
%Serial Text Searching  Processing text by examining     Topic Definition A facility that permits concept
%each character in turn (scanning).      searching by providing a structure that links words
%
%                and phrases that are related in a hierarchical manner.
%Soundex Allows searches to be carried out on a  Each node in the topic definition can be a term,
%sounds like' basis for phonetically similar words and           compound term, Boolean search expression, a
%phrases.                wildcard term, soundex search, a date range
%
%Stemming  Reduction of all words with the same  regarding the source, or the source itself. Topic terms
%        can also be topics within themselves, allowing for
%root to a single form. See also Lemmatizatio~   intricate and sophisticated networked topic
%
%
%        definitions.
%Stop List  A list of insignificant words that are not
%to be included in the index and selected by the user.
%Such words might include: and, to, but, the, as.        Truncation, forward and backward  This is a search
%facility enabling the user to search on a part of a
%
%Structured  Structured data is typically numerical      word; for example, the truncated form of planning
%graphical or tabular.   would be plan. A search using the truncated form
%would also reveal occurrences of plan, planner,
%
%Subject Any concept or combination of concepts  planned.
%representing a theme in a document.
%        Up-posting Facility  The ability to post a word
%Synonyms  A collection of words that represent the      under more than one existing term in a word list.
%same, or similat concepts.      These terms would be linked via a thesaurus.
%
%Tagging  An alternative to the automatic
%production of an index by the software. Significant     Use Term The main or preferred term in a synonym
%words are manually tagged during input to produce       list or thesaurus entry. In a thesaurus the Use For
%        operator is used to track these terms. Also known as
%a go list.      a Preferred Term or Lead Term. See also Preferred
%
%
%        Term.
%Term  Entry in thesaurus as either preferred term or
%non-preferred term.
%User Profile  A continuously running text-retrieval
%
%Term Weighting  Assigning a numerical value to a        query submitted in background. User profiles will
%term representing its importance to a database of       typically notify their respective owner of any new
%text, a defined topic, or a query. Term weights can be  submissions to the text database that satisfy the
%assigned by a user or automatically determined by a     query. See also Selective Dissemination of
%text-retrieval system.  Information and Current Awareness.
%
%Text File  File where original text is stored   Variable Length Field  Field containing documents
%(normally in original format).  that need not be a defined, fixed length.
%
%Thesaurus  The vocabulary of a controlled indexing     Wild Card  A character used to mean 'any one
%language, formally organized so that the
%relationships between concepts (e.g. 'broader' and      character' or 'a contiguous set of characters' within a
%narrower') are made explidt.    word or phrase. See also Internal Character
%        Masking.
%
%
%Token The lowest level single unit of text. Often a
%word, but it may, for example, be a hyphenated word     Wordlist  A list of all the useful words in the text
%or a date.      file. See also Inverted Index.
%
%
%
%
%
%
%
%
%
%4  sQL*TextRetrieval
