\chapter{REX On Its Own}

\section{REX, The Regular Expression Pattern Matcher}

{\bf REX} is a very powerful pattern matcher which can be called
internally within the Metamorph Query Language.  It can also be used
as an external stand-alone tool.  Any pattern or expression which is
outlined herein can be specified from within a Metamorph query by
preceding it with a forward slash (\verb`/`).  Similarly, any {\bf
REX} pattern can be entered as Start or End Delimiter expression, or
used otherwise in a Vortex script which calls for \verb`rex`, there
requiring no forward slash (\verb`/`).

{\bf REX} (for {\bf R}egular {\bf EX}pression) allows you to search
for any fixed or variable length regular expression, and executes more
efficiently types of patterns you might normally try to look for with
the Unix \verb`Grep` family.  {\bf REX} puts all these facilities into
one program, along with a Search \& Replace capability, easier to
learn syntax, faster execution, ability to set delimiters within which
you want to search (i.e., it can search across lines), and goes beyond
what is possible with other search tools.

It may be somewhat new to those who haven't previously used such
tools, but you'll find that if you follow REX's syntax very literally
and practice searching for simple, then more complex expressions, that
it becomes quite understandable and easy to use.  One thing to keep in
mind is that REX looks both forwards and backwards, something quite
different from other types of tools.  Therefore, when you construct a
REX expression, make sure it makes sense from a global view of the
file; that is, whether you'd be looking forwards, or looking
backwards.

A REX pattern can be constructed from a series of {\bf F-REX'S}
(pronounced ``f-rex''), or Fixed length Regular EXpressions, where
repetition operators after each subexpression are used to state how
many of each you are looking for.  Unless otherwise delineated, REX
assumes you are looking for one occurrence of each subexpression
stated within the expression, as specified within quotation marks.

To begin learning to use REX, first try some easy F-REX (fixed length)
patterns.  For example, on the query line, type in:

\begin{verbatim}
     /life
\end{verbatim}

This pattern ``\verb`life`'' is the same as the pattern
``\verb`life=`'', and means that you are asking the pattern matcher
REX to look for one occurrence of the fixed length expression
``\verb`life`''.

The equal sign `\verb`=`' is used to designate one occurrence of a
fixed length pattern, and is assumed if not otherwise stated.

The plus sign `\verb`+`' is used to designate one or more occurrences
of a fixed length pattern.  If you were to search for ``\verb`life+`''
REX would look for one or more occurrences of the word
``\verb`life`'':  e.g., it would locate the pattern ``\verb`life`''
and it would also locate the pattern ``\verb`lifelife`''.

The asterisk `\verb`*`' is used to designate zero or more occurrences
of a fixed length pattern.  If you were to try to look for
``\verb`life*`'', REX would be directed to look for zero or more
occurrences of the word \verb`life` (rather than 1 or more
occurrences); therefore, while it could locate ``\verb`life`'' or
``\verb`lifelife`'', it would also have to look for 0 occurrences of
``\verb`life`'' which could be every pattern in the file.  This would
be an impossible and unsatisfactory search, and so is not a legal
pattern to look for.  The rule is that a `\verb`*`' search must be
rooted to something else requiring one or more occurrences.

If you root ``\verb`life*`'' to a fixed length pattern which must find
at least one occurrence of something, the pattern becomes legal.
Therefore, you could precede ``\verb`life*`'' with ``\verb`love=`'',
making the pattern ``\verb`love=life*`''.  Now it is rooted to
something which definitely can be found in the file; e.g., one
occurrence of the word ``\verb`love`'', followed by 0 or more
occurrences of the word ``\verb`life`''.  Such an expression
``\verb`love=life*`'' would locate ``\verb`love`'',
``\verb`lovelife`'', and ``\verb`lovelifelife`''.

If there is more than one subexpression within a REX pattern, any
designated repetition operator will attach itself to the longest
preceding fixed length pattern.  Therefore a pattern preceding a plus
sign `\verb`+`', even if it is made of more than one subexpression,
will be treated as one or more occurrences of the whole preceding
pattern.  Use the equal sign `\verb`=`', if necessary, to separate
these subexpressions and prevent an incorrect interpretation.

For example:  if you say ``\verb`lovelife*`'' (rather than
``\verb`love=life*`'') the `\verb`*`' operator will attach itself to
the whole preceding expression, ``\verb`lovelife`'', and will
therefore be translated to mean 0 or more occurrences of the entire
pattern ``\verb`lovelife`'', making it an illegal expression.  On the
other hand, in the expression ``\verb`love=life*`'', REX will
correctly look for 1 occurrence of ``\verb`love`'', followed by 0 or
more occurrences of ``\verb`life`''.

Use the ``\verb`-x`'' option in \verb`REX` (from the Windows or Unix
command line) to get feedback on how REX translates the syntax you
have entered to what it is really looking for.  In this way you can
debug your use of syntax and learn to use REX to its maximum power.
Using our same example, if you enter on the Windows or Unix command line:

\begin{verbatim}
     rex -x ``love=life*''
\end{verbatim}

you will get the following output back to show you how REX will
interpret your search request:

\begin{verbatim}
     1 occurrence(s) of : [Ll][Oo][Vv][Ee]
     followed by from 0 to 32000 occurrences of : [Ll][Ii][Ff][Ee]
\end{verbatim}

The brackets above, as:  ``\verb`[Ll]`'', mean either of the
characters shown inside the brackets, in that position; i.e., a
capital or small `\verb`l`' in the 1st character position, a capital
or small `\verb`o`' in the 2nd character position, and so on.

You can find REX syntax for all matters discussed above and delineated
below, by typing in ``\verb`REX`'' followed by a carriage return on
the Windows or Unix command line.

\section{REX Program Syntax and Options}

REX locates and prints lines containing occurrences of regular
expressions.  Beginning and end of line are the default beginning and
ending delimiters.  If starting (\verb`-s`) and ending (\verb`-e`)
delimiters are specified REX will print from the beginning delimiting
expression to the ending delimiting expression.  If so specified
(\verb`-p`) REX will print the expression only.  If files are not
specified, standard input is used.  If files are specified, the
filename will be printed before the line containing the expression,
unless the ``\verb`-n`'' option (don't print the filename) is used.

\begin{verbatim}
SYNTAX:  rex [options] expression [files]

 Where:  options are preceded by a hyphen (-)
         expressions may be contained in quotes (" ")
         filename(s) comes last and can contain wildcards

OPTIONS:
   -c       Don't print control characters, replace with space.
   -C       Count the number of times the expression occurs.
   -l       List file names (only) that contain the expression.
   -E"EX"   Specify and print the ending delimiting expression.
   -e"EX"   Specify the ending delimiting expression.
   -S"EX"   Specify and print the starting delimiting expression.
   -s"EX"   Specify the starting delimiting expression.
   -p       Begin printing at the start of the expression.
   -P       Stop printing at the end of the expression.
   -r"STR"  Replace the expression with "STR" to standard output.
   -R"STR"  Replace the expression with "STR" to original file.
   -t"Fn"   Use "Fn" as the temporary file. (default: "rextmp")
   -f"Fn"   Read the expression(s) from the file "Fn".
   -n       Do not print the file name.
   -O       Generate "FNAME@OFFSET,LEN" entries for mm3 subfile list.
   -x       Translate the expression into pseudo-code (debug).
   -v       Print lines (or delimiters) not containing the expression.
\end{verbatim}

\begin{itemize}
\item Each option must be placed individually on the command line.

\item ``\verb`EX`'' is a REX expression.

\item ``\verb`Fn`'' is file name.

\item ``\verb`STR`'' is a replacement string.

\item ``\verb`N`'' is a drive name  e.g.: \verb`A,B,C` \ldots
\end{itemize}


\section{REX Expression Syntax}

\begin{itemize}
\item  Expressions are composed of characters and operators.  Operators
       are characters with special meaning to REX. The following
       characters have special meaning: ``\verb`\=+*?{},[]^$.-!`'' and must
       be escaped with a `\verb`\`' if they are meant to be taken literally.
       The string ``\verb`>>`'' is also special and if it is to be matched,
       it should be written ``\verb`\>>`''.  Not all of these characters are
       special all the time; if an entire string is to be escaped so it
       will be interpreted literally, only the characters ``\verb`\=?+*{[^$.!>`''
       need be escaped.

\item  A `\verb`\`' followed by an `\verb`R`' or an `\verb`I`' mean to begin
       respecting or ignoring alphabetic case distinction. (Ignoring case is
       the default.) These switches {\em do not} apply inside range brackets.

\item  A `\verb`\`' followed by an `\verb`L`' indicates that the characters
       following are to be taken literally up to the next `\verb`\L`'. The
       purpose of this operation is to remove the special meanings from
       characters.

\item  A subexpression following `\verb`\F`' (followed by) or `\verb`\P`'
       (preceded by) can be used to root the rest of an expression to which
       it is tied.  It means to look for the rest of the expression ``as long as followed
       by \ldots'' or ``as long as preceded by \ldots'' the subexpression
       following the \verb`\F` or \verb`\P`, but the designated subexpression will be
       considered excluded from the located expression itself.

\item  A `\verb`\`' followed by one of the following `\verb`C`' language character
       classes matches that character class: \verb`alpha`, \verb`upper`, \verb`lower`,
       \verb`digit`, \verb`xdigit`, \verb`alnum`, \verb`space`, \verb`punct`,
       \verb`print`, \verb`graph`, \verb`cntrl`, \verb`ascii`.

\item  A `\verb`\`' followed by one of the following special characters
       will assume the following meaning: \verb`n`=newline, \verb`t`=tab,
       \verb`v`=vertical tab, \verb`b`=backspace, \verb`r`=carriage return,
       \verb`f`=form feed, \verb`0`=the null character.

\item  A `\verb`\`' followed by \verb`Xn` or \verb`Xnn` where \verb`n` is a
       hexadecimal digit will match that character.

\item  A `\verb`\`' followed by any single character (not one of the
       above) matches that character.  Escaping a character that is
       not a special escape is not recommended, as the expression
       could change meaning if the character becomes an escape in a
       future release.

\item  The character `\verb`^`' placed anywhere in an expression (except after a
       `\verb`[`') matches the beginning of a line. (same as: \verb`\x0A` in Unix or
       \verb`\x0D\x0A` in Windows)

\item  The character `\verb`$`' placed anywhere in an expression
       matches the end of a line. (\verb`\x0A` in Unix, \verb`\x0D\x0A` in Windows)

\item  The character `\verb`.`' matches any character.

\item  A single character not having special meaning matches that character.

\item  A string enclosed in brackets \verb`[]` is a set, and matches
       any single character from the string.
       Ranges of ASCII character codes may be
       abbreviated as in \verb`[a-z]` or \verb`[0-9]`.  A `\verb`^`'
       occurring as the first character of the set will invert the
       meaning of the set.  A literal `\verb`-`' must be preceded by a
       `\verb`\`'.  The case of alphabetic characters is always respected
       within brackets.

       A double-dash (``\verb`--`'') may be used inside a bracketed set
       to subtract characters from the set; e.g. ``\verb`[\alpha--x]`''
       for all alphabetic characters except ``\verb`x`''.  The
       left-hand side of a set subtraction must be a range, character
       class, or another set subtraction.  The right-hand side of a set
       subtraction must be a range, character class, or a single
       character.  Set subtraction groups left-to-right.  The range
       operator ``\verb`-`'' has precedence over set subtraction.
       Set subtraction was added in version 6.

\item   The `\verb`>>`' operator in the first position of a fixed expression
        will force REX to use that expression as the ``root'' expression
        off which the other fixed expressions are matched. This operator
        overrides one of the optimizers in REX. This operator can
        be quite handy if you are trying to match an expression
        with a `\verb`!`' operator or if you are matching an item that
        is surrounded by other items. For example: ``\verb`x+>>y+z+`''
        would force REX to find the ``\verb`y's`'' first then go
        backwards and forwards for the leading ``\verb`x's`'' and trailing
        ``\verb`z's`''.

\item  The `\verb`!`' character in the first position of an expression means
       that it is {\em not} to match the following fixed expression.
       For example: ``\verb`start=!finish+`'' would match the word ``\verb`start`''
       and anything past it up to (but not including the word ``\verb`finish`''.
       Usually operations involving the ``\verb`!`'' operator involve knowing
       what direction the pattern is being matched in. In these cases
       the `\verb`>>`' operator comes in handy. If the `\verb`>>`' operator is
       used, it comes before the `\verb`!`'. For example:
       ``\verb`>>start=!finish+finish`'' would match anything that began
       with ``\verb`start`'' and ended with ``\verb`finish`''.  {\em The ``\verb`!`''
       operator cannot be used by itself} in an expression, or as the root
       expression in a compound expression.  NOTE:  This ``\verb`!`'' operator
       ``nots'' the whole expression rather than its sequence of characters,
       as in earlier versions of REX.

       Note that ``\verb`!`'' expressions match a character at a time,
       so their repetition operators count characters, not
       expression-lengths as with normal expressions.
       E.g. ``\verb`!finish{2,4}`'' matches 2 to 4 characters, whereas
       ``\verb`finish{2,4}`'' matches 2 to 4 times the length of
       ``\verb`finish`''.

\end{itemize}

\section{REX Repetition Operators}

\begin{itemize}
\item  A regular expression may be followed by a repetition operator in
       order to indicate the number of times it may be repeated.

     NOTE: Under Windows the operation ``\verb`{X,Y}`'' has the syntax ``\verb`{X-Y}`''
       because Windows will not accept the comma on a command line.  Also, \verb`N`
       occurrences of an expression implies infinite repetitions but in
       this program \verb`N` represents the quantity \verb`32768` which should
       be a more than adequate substitute in real world text.

\item  An expression followed by the operator ``\verb`{X,Y}`'' indicates that
       from \verb`X` to \verb`Y` occurrences of the expression are to be
       located.  This notation may take on several forms: ``\verb`{X}`''
       means \verb`X` occurrences of the expression, ``\verb`{X,}`'' means
       from \verb`X` to \verb`N` occurrences of the expression, and
       ``\verb`{,Y}`'' means from \verb`0` (no occurrences) to \verb`Y`
       occurrences of the expression.

\item  The `\verb`?`' operator is a synonym for the operation ``\verb`{0,1}`''.
       Read as: ``Zero or one occurrence.''

\item  The `\verb`*`' operator is a synonym for the operation ``\verb`{0,}`''.
       Read as: ``Zero or more occurrences.''

\item  The `\verb`+`' operator is a synonym for the operation ``\verb`{1,}`''.
       Read as: ``One or more occurrences.''

\item  The `\verb`=`' operator is a synonym for the operation ``\verb`{1}`''.
       Read as: ``One occurrence.''
\end{itemize}

%#ifdef EPI_ENABLE_RE2
\section{RE2 Syntax}

In Texis version 7.06 and later, on some platforms the search
expression may be given in RE2 syntax instead of REX.  RE2 is a
Perl-compatible regular expression library whose syntax may be more
familiar to Unix users than Texis' REX syntax.  An RE2 expression in
REX is indicated by prefixing the expression with ``{\tt \char92<re2\char92>}''.
E.g. ``{\tt \char92<re2\char92>\char92w+}'' would search for one or more word
characters, as ``{\tt \char92w}'' means word character in RE2, but not REX.

REX syntax can also be indicated in an expression by prefixing it with
``{\tt \char92<rex\char92>}''.  Since the default syntax is already REX, this
flag is not normally needed; it is primarily useful in circumstances
where the syntax has already been changed to RE2, but outside of the
expression -- e.g. a Vortex \verb`<rex>` statement with the option
\verb`syntax=re2`.

Note that RE2 syntax is not supported on all {\tt texis -platform}
platforms; where it is unsupported, attempting to invoke an RE2
expression will result in the error message
``{\tt REX: RE2 not supported on this platform}''.  (Windows,
most Linux 2.6 versions except {\tt i686-unknown-linux2.6.17-64-32}
are supported.)

%#endif % EPI_ENABLE_RE2

\section{REX Replacement Strings}

NOTE:  Any expression may be searched for in conjunction with a replace
operation but the replacement string will always be of fixed size.

An expression will be defined one way when given to REX to search for;
it will be defined another way when used with the `\verb`-r`' or `\verb`-R`'
\verb`Replace` options as the replacement string.  The characters
``\verb`?#{}+`'' have entirely different meanings when used as part of the
syntax of a replacement string, than when used as part of the syntax of
expressions or repetition operators.

Replacement syntax is different from expression syntax, and is limited to the
following delineated rules.

\begin{itemize}

\item The characters ``\verb`?#{}+\`'' are special.  To use them
      literally, precede them with the escapement character
      ``\verb`\`'.

\item  Replacement strings may just be a literal string or they may
       include the ``ditto'' character ``\verb`?`''.  The ditto character
       will copy the character in the position specified in the
       replace string from the same position in the located expression.

\item  A decimal digit placed within curly braces (e.g. \verb`{5}`) will place
       that character of the located expression to the output.

\item A ``\verb`\`'' followed by a decimal number will place that
      subexpression to the output. Subexpressions are numbered starting at
      1.

%#ifdef EPI_ENABLE_RE2
\item The sequence ``\verb`\&`'' will place the entire expression match
      (sans \verb`\P` and \verb`\F` portions) to the output.  This escape
      was added in Texis version 7.06.
%#endif % EPI_ENABLE_RE2

\item  A plus character ``\verb`+`'' will place an incrementing decimal
       number to the output. One purpose of this operator is to number lines.

\item  A ``\verb`#`'' in the replace string will cause the character in that
       position to be printed in hexadecimal form.

\item  Any character in the replace string may be represented by the
       hexadecimal value of that character using the following syntax:
       \verb`\Xdd` where ``\verb`dd`'' is the hexadecimal value.
\end{itemize}


\section{REX Caveats and Commentary}

{\bf REX} is a highly optimized pattern recognition tool that has been
modeled after the UNIX family of tools:  \verb`GREP`, \verb`EGREP`,
\verb`FGREP`, and \verb`LEX`.  Wherever possible REX's syntax
has been held consistent with these tools, but there are several major
departures that may bite those who are used to using the \verb`GREP`
family.

REX uses a combination of techniques that allow it to surpass the
speed of anything similar to it by a very wide margin.

The technique that provides the largest advantage is called ``state
anticipation or state skipping'' which works as follows \ldots

If we were looking for the pattern:
\begin{verbatim}
                       ABCDE
\end{verbatim}
in the text:
\begin{verbatim}
                       AAAAABCDEAAAAAAA
\end{verbatim}
a normal pattern matcher would do the following:
\begin{verbatim}
                       ABCDE
                        ABCDE
                         ABCDE
                          ABCDE
                           ABCDE
                       AAAAABCDEAAAAAAA
\end{verbatim}

The state anticipation scheme would do the following:

\begin{verbatim}
                       ABCDE
                           ABCDE
                       AAAAABCDEAAAAAAA
\end{verbatim}

The normal algorithm moves one character at a time through the text,
comparing the leading character of the pattern to the current text
character of text, and if they match, it compares the leading pattern
character +1 to the current text character +1, and so on \ldots

The state anticipation pattern matcher is aware of the length of the
pattern to be matched, and compares the last character of the pattern
to the corresponding text character.  If the two are not equal, it
moves over by an amount that would allow it to match the next
potential hit.

If one were to count the number of comparison cycles for each pattern
matching scheme using the example above, the normal pattern matcher
would have to perform 13 compare operations before locating the first
occurrence; versus 6 compare operations for the state anticipation
pattern matcher.

One concept to grasp here is:  ``{\em The longer the pattern to be
found, the faster the state anticipation pattern matcher will be;}''
while a normal pattern matcher will slow down as the pattern gets
longer.

Herein lies the first major syntax departure:  REX always applies
repetition operators to the longest preceding expression.  It does
this so that it can maximize the benefits of using the state skipping
pattern matcher.

If you were to give \verb`GREP` the expression:  \verb`ab*de+`

It would interpret it as:  an `\verb`a`' then 0 or more `\verb`b's`'
then a `\verb`d`' then 1 or more `\verb`e's`'.

\verb`REX` will interpret this as:  0 or more occurrences of
`\verb`ab`' followed by 1 or more occurrences of `\verb`de`'.

The second technique that provides REX with a speed advantage is
ability to locate patterns both forwards and backwards
indiscriminately.

Given the expression:  ``\verb`abc*def`'', the pattern matcher is
looking for ``\verb`Zero` to \verb`N` occurrences of `\verb`abc`'
followed by a `\verb`def`'''.

The following text examples would be matched by this expression:

\begin{verbatim}
     abcabcabcabcdef
     def
     abcdef
\end{verbatim}

But consider these patterns if they were embedded within a body of text:

\begin{verbatim}
     My country 'tis of abcabcabcabcdef sweet land of def, abcdef.
\end{verbatim}

A normal pattern matching scheme would begin looking for
`\verb`abc*`'.  Since `\verb`abc*`' is matched by every position
within the text, the normal pattern matcher would plod along checking
for `\verb`abc*`' and then whether it's there or not it would try to
match `\verb`def`'.  REX examines the expression in search of the the
most efficient fixed length subpattern and uses it as the root of
search rather than the first subexpression.  So, in the example above,
REX would not begin searching for `\verb`abc*`' until it has located a
`\verb`def`'.

There are many other techniques used in REX to improve the rate at
which it searches for patterns, but these should have no effect on the
way in which you specify an expression.

The three rules that will cause the most problems to experienced
\verb`GREP` users are:

\begin{enumerate}

\item Repetition operators are always applied to the longest
expression.

\item There must be at least one subexpression that has one or more
repetitions.

\item No matched subexpression will be located as part of another.

\end{enumerate}

\begin{description}
\item[Rule 1 Example]:
\verb`abc=def*`  Means one `\verb`abc`' followed by 0 or more `\verb`def's`' .

\item[Rule 2 Example]:
\verb`abc*def*`  CANNOT be located because it matches every position within the text.

\item[Rule 3 Example]:
\verb`a+ab`  Is idiosyncratic because `\verb`a+`' is a subpart of `\verb`ab`'.
\end{description}

\section{Other REX Notes}

\begin{itemize}

\item When using REX external to Metamorph in Windows or Unix, it can
occasionally occur that a very complicated expression requires more
memory than is available.  In the event of an error message indicating
not enough memory, you can try the same expression using
``\verb`REXL`'', another version of \verb`REX` compiled for large
model.  As ``\verb`REXL`'' is slightly slower than ``\verb`REX`'' it
would not be recommended for use in most circumstances.

\item The beginning of line (\verb`^`) and end of line (\verb`$`)
notation expressions for Windows are both identified as a 2 character
notation; i.e., REX under Windows matches \verb`\X0d\X0a` (carriage
return \verb`[cr]`, line feed \verb`[lf]`) as beginning and end of
line, rather than \verb`\X0a` as beginning, and \verb`\X0d` as end.

\end{itemize}
