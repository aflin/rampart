% This comment needed for Makefile .tex.l2h rule.
\documentstyle[12pt,epiman,chgbar]{book}

% Note: `Texis FAQ' is edited to `Texis Frequently Asked Questions'
% in the online docs' <h1> header (but not title), to match other
% FAQs on www.  See Makefile:
\title{Texis FAQ}
\author{Thunderstone Software}

\begin{document}

\maketitle

\tableofcontents

% ****************************************************************************
\chapter{General}

% ============================================================================
\section{How is Texis different from other search engines? }
Texis is the only search engine with the structure of a SQL
relational database (rdbms). SQL as used here means Structured Query
Language -- not Microsoft's product named with that term!  SQL is an
industry standard defined by the American National Standards Institute
(ANSI), and its counterpart, the International Organization for
Standardization (ISO). All major database vendors use SQL as their
query language.

SQL provides many advantages for addressing complicated search
requirements. It also provides you with the confidence of a reliable,
well-defined path for implementing unanticipated new search
functionality in the future.  SQL is a rich, mature, open standard used
by hundreds of thousands of database application developers around the
world.

All other search engines provide a much narrower range of capabilities
based on proprietary interfaces. No other search engine provides the
versatility of using SQL as its application development model.

% ============================================================================
\section{How is Texis different from other relational databases? }
Texis is the only relational database that can store and search text
documents of unlimited size within standard database tables. All other
solutions that purport to accomplish this employ, either explicitly or
``under the covers,'' a loosely coupled external text index, and store
documents in a binary large object (blob) field. That approach causes
major bottlenecks.

% ============================================================================
\section{What's so hard about integrating text-search with an RDMBS? }

Text-searching and relational database management are radically
different paradigms for organizing and retrieving information. They
were developed over decades as completely separate technologies and do
not marry easily. Thunderstone has devoted more than 10 years to
solving this problem; it is our ``core competency.'' Thunderstone is
the only software vendor to have undertaken and solved this challenge
head-on.

% ============================================================================
\section{Is Texis a content management system? }

Many of our customers use Texis for content management. Examples
include legal document archiving and editorial publishing
systems. However, Texis ``out of the box'' is not a finished content
management application. You must customize it for that purpose -- see
below regarding application development.  One of the benefits of using
Texis for content management is that you can make it do what you want.
If off-the-shelf content management solutions don't quite do
everything you need, you'll end up bringing in consultants
(programmers) to customize it anyway. Sometimes it is preferable to
start with a more generic platform, and design exactly what you need
from the start.

% ============================================================================
\section{Is Texis an e-commerce system? }

Many e-commerce web sites use Texis. Some are powered entirely by
Texis technology. Others use Texis as a search engine, together with
commerce tools of other vendors. Texis is an ideal application
development platform for various e-commerce applications. Thunderstone
also provides some generic commerce application scripts that may be
customized with the tools discussed below.

% ============================================================================
\section{Is Texis a portal system? }

The word ``portal'' has come to mean a uniform web ``front-end'' to
disparate ``back-end'' computer systems.  Texis includes many
portal-building features.  The most important relate to web fetching,
i.e., retrieving content from other web applications in real-time.  A
typical use of this ability is to create a ``federated'' search
application that integrates results from disparate sources.  Texis
actually is a whole suite of tools that together enable a developer to
create, deploy, and maintain web-based applications. The tool suite
encompasses: a scripting environment; data importing utilities, a web
crawler; a database; a search engine; and even a web server.

% ============================================================================
\section{Is Texis a knowledge management (KM) system? }

Knowledge Management is a broad term for sharing of knowledge within
an organization systematically, instead of informally as is the human
tradition.  Tools for searching archives are fundamental to KM.  In
many cases, an effective internal search engine will give you the
greatest KM benefit for the least investment.  Texis is designed to
meet the most sophisticated internal search needs.  The idea of KM may
encompass a variety of other applications such as cataloging of
individuals' expertise.  Texis provides tools for creating such
applications, but they are not ``out of the box'' features.

% ============================================================================
\section{Database X has a full-text-search feature -- How is that different from what Texis does? }

Database X in fact has tied in a separate text-indexing module by
means of a ``foreign-key join,'' an extremely inefficient
technique. It is suitable only for small databases and light user
loads.  And text searches will be slow!

% ============================================================================
\section{Search engine Y can index a relational database -- How is that different from what Texis does? }

Search engine Y ignores SQL logic and treats database records as
documents. So you lose ability to do sorting on other columns,
real-time updates, and many other capabilities that are inherent in a
relational database.

% ============================================================================
\section{Search engine Z can communicate in SQL -- How is that different from what Texis does? }

Search engine Z uses only a subset of the SQL language, and does not
perform some of the basic functions of a database.  For example, it
probably doesn't support JOIN operations, the DISTINCT argument, the
{\tt GROUP BY} command, the {\tt HAVING} clause.  These are all
generic features of the SQL language.  And when records are updated
(changed), it must rebuild the search index in a time-consuming batch
operation.  In between index rebuilds -- which may occur only once a
day -- searches match the old stale data, not the new data.  A true
relational database (of which Texis is one) never allows the indexes
to get ``out of sync'' with the data.

% ============================================================================
\section{What is the difference between Texis and Thunderstone's other products Texis Web Script, Vortex, Webinator, and the Search Appliance? }

Texis is the core technology, encompassing the database and search
engine capabilities.

Texis Web Script also is known as Vortex.  This is an application
development (scripting) tool set.  It is bundled with Texis.  Texis
Web Script is a superset of HTML, with extensions for passing calls to
the Texis database.  Returned data is dynamically marked up, normally
with HTML but optionally with JavaScript or any other markup language.
Texis applications may alternately be created with a variety of other
technologies.

Webinator is an application of Texis. The Webinator crawler (dowalk)
extracts the content of web pages and stores each as a record in a
Texis database.  The database thus becomes a search engine for the
spidered pages.  It can be queried through normal Texis SQL {\tt
  SELECT}... {\tt LIKE} statements, but a generic web interface also
is provided.  Webinator is given out as an example Texis application,
but under conditions defined by its acceptable-use policy (see license
addendum), it may be used for indexing a web site(s) without
purchasing a full Texis license.

The Search Appliance also is based on Texis.  It encompasses the
features of Webinator, with some additional capabilities such as the
ability to crawl file systems and respect document permission
settings.  The Appliance is created to be a ``turn-key'' solution, so
that the customer does not need to install or configure software or
operating system features.  The Search Appliance is designed to be
administered by a business user, as opposed to a technical user.


% ============================================================================
\section{What is the ``target market'' for Texis? }

Target applications include online publishing, interactive catalogs,
classified advertising, digital asset management, intelligence, and of
course, web searching.  What these all have in common is that they
require both structured and unstructured types of searching.  For
example, product catalogs typically contain unstructured text (name,
description, etc.), as well as structured content (size, price,
inventory number, etc.).  Users may wish to search by description; or
navigate by price range; or both in combination.  Texis is the premier
solution for providing text searching tightly integrated with
traditional structured database querying.

% ****************************************************************************
\chapter{Texis as a Database }

% ============================================================================
\section{Which database (RDBMS) does Texis use? }

Texis does not ``use'' another database; it is a complete database
itself. However, it can be used as a search engine for content
residing in any other database.

% ============================================================================
\section{Is the Texis database proprietary? }

Texis is non-proprietary in the sense that is integrates the ANSI SQL
relational database standard. Texis is proprietary in the same sense
that Oracle is a proprietary database.  But see below about the {\tt
  LIKE} clause syntax.

% ============================================================================
\section{Does Texis have stored procedures? } 

Yes. Procedures are stored in server-side compiled scripts.

% ============================================================================
\section{Does Texis do joins? }

Yes. The technique of joining two or more tables together in a
relationship is what makes a database ``relational.'' As with any
database, in designing for scalability, care must be taken to avoid
excessive reliance on joins, which tend to be resource-intensive.

% ============================================================================
\section{How are search-engine type queries expressed in SQL? }

In Texis, full-text search terms go inside the {\tt LIKE} clause.  An
expanded syntax, including the familiar + and - operators, is
supported.  This expanded syntax by necessity is implemented as an
extension of SQL simply because traditional search-engine type queries
are not defined in standard SQL.

% ============================================================================
\section{What is a Metamorph index? }

This is Thunderstone's name for an inverted (search-engine) index on a
column of unstructured text, as distinguished from a b-tree
(sorted-order) index normally used on numeric or string database
fields.  This following is somewhat technical and is intended for
experienced database programmers.

A Metamorph index is set up and used in a manner analogous to any
other database index:

\begin{verbatim}
create metamorph index descriptionindex on products(description);
\end{verbatim}

As an illustration of the power this provides, consider a typical
Texis query of this model:

\begin{verbatim}
SELECT id, name FROM products WHERE description LIKE 'big fancy gizmo'
   ORDER BY price;
\end{verbatim}

The Texis database optimizer uses the metamorph index on the
description field, along with the B-tree index on the price field, to
quickly and efficiently resolve this query.  (From a more technical
point of view, you'd probably create a single compound index combining
the characteristics of those two indexes, for even better
performance.)

Any other database, to accomplish something similar, would receive no
help from the database optimizer.  The text index is a black box to
it.  It can only hand off the gizmo query to a separate text index;
then create a temporary table containing the text-search results; then
do a join between that and the table containing the price information.

Texis is the only relational database that can resolve a query of this
type without a join.  This makes Texis many times more efficient.

% ============================================================================
\section{Are documents stored within Texis, or as separate files? }

Either! It depends on the circumstances. Web-searching is a typical
example of indexing external documents: the Texis crawler extracts
information about the pages and builds an index (database) based on
that; search results consist of links to those pages. On the other
hand, in an auction application, the original information typically
exists entirely within the database: users input their listings
directly into the database; and search results consist of links to
records within the database.

% ============================================================================
\section{Do I need a Texis DBA (database administrator)? }

Probably not. Administering a Texis database is simpler than
administering other databases such as Oracle. Texis runs as an
application on top of the operating system and does not usurp
operating system functions. So backing up a Texis database, for
example, can be accomplished simply by copying a directory (but see
below regarding redundancy). Texis does have various administrative
aspects and configuration options, but overseeing those usually is
handled by the application developer(s).

% ============================================================================
\section{Can Texis handle BLOBs (binary large objects)? }

Yes. Texis has a blob-type field useful for storing graphics or other
binary data.  But note that in Texis, textual content of any size
usually is put in a variable-size varchar field.  This provides
superior text-indexing and searching functionality compared to storing
text into blobs.  But if you have binary content, Texis can manage the
storage of files much more efficiently than an OS file system!  That
is because Texis keeps track of each record's location on disk, and
can fetch it with a single disk seek-and-read operation; whereas
operating systems are un-indexed, so that fetching files typically
takes four or more seek-and-reads to search through the directory
structure.

% ============================================================================
\section{Does Texis do data mining? }
Yes, definitely.  The idea of data mining carries the connotation of a large archive of data collected from other business processes.  The archive is used for analysis, typically in a search for trends or relationships not obvious in the normal course of business.  Texis is an excellent platform for data mining and may be superior to traditional databases for this purpose, depending on the kind of data and analyses needed.  Texis is most versatile for querying data that combines structured and unstructured elements.  Many business databases containing unstructured text fields, such as customer correspondence or support logs.  Insights may be found in patterns either in the structured or unstructured data, or in a combination.  Texis is the market's leading tool for such research.

% ****************************************************************************
\chapter{Texis Search Technology }

% ============================================================================
\section{Does Texis handle natural language queries? }

Yes. Users may enter any natural language question. By default,
matching records or documents are presented in relevance rank
order. There are many settings for ``tuning'' the rankings.

% ============================================================================
\section{Can Texis index PDFs, word-processing documents, or other formats? }

Yes. Texis can index most common document formats.  It will also
extract and text in binary files, such as a photo containing a caption
in ASCII.

% ============================================================================
\section{Does Texis highlight ``hits'' (word matches) in the results? }

Yes. Texis can even pass the appropriate information to Adobe Acrobat
to perform the highlighting within a PDF document.  Hit highlighting
is completely customizable with CSS classes and stylesheets.

% ============================================================================
\section{Does Texis create a summary of the each result item? }

Yes, we call this an abstract.  Normally it is centered around the
most representative ``hit'' words in the results, and the size of the
abstract may be specified.

% ============================================================================
\section{Does Texis handle phrases? Wildcards? }

Yes, both. A typical search form will consider text within quote marks
as a phrase, and the asterisk character as a wildcard. If desired,
Texis will accept wildcards within or at the beginning of a word, as
well as at the end.  These features are under the control of the
application developer, who may turn them on or off, or change their
behavior in various ways.

% ============================================================================
\section{Does Texis support Boolean logic? }

Yes. Full Boolean logic is standard within the SQL language. Texis
also understands the + and - operators popularized by web search
engines. And Texis understands set logic, which can be used to express
a command of the style ``Find records containing $n$ or more words of
my query.'' Absent explicit operators, the default logic is specified
by the application developer.

% ============================================================================
\section{Does Texis have fuzzy logic? }

Yes. The Texis facility for accomplishing this is called approximate
pattern matching. This generates a similarity measure between any two
words or patterns, expressed as a percentage of closeness. The user or
application developer may control the degree of closeness. This
capability most commonly is desired to accommodate spelling mistakes
in either the queries or the data. It can be useful in searching
scanned documents, which tend to have errors resulting from the
imperfect OCR process.  Developers should use this feature with
caution, however. Fuzzy logic, by its nature, brings back some records
unrelated to the either the user's query words or the intended
meaning.  This tends to confuse and annoy users not expecting this
style of response.

% ============================================================================
\section{Can Texis index documents stored on multiple servers? }

Yes, elementary! Texis may create a searchable index of documents
anywhere on a network or on the Internet.

% ============================================================================
\section{Can Texis sort results by date (or by price, or rating, or whatever)? }

Yes. Texis's sorting power is one of its most popular features. You
may sort the results of a text search by any field in your data. For
example, if your database contains an {\tt author} field, you can sort
search results by author.  This works efficiently even on large result
sets, by taking advantage of the powerful sorting capability inherent
within relational database technology. Texis can quickly sort tens of
thousands of hits or more. Other search engines either bog down
sorting more than a few hundred items, or else their sorting
capabilities are much more limited. For example, one major search
engine cannot perform relevance-ranking together with sorting; another
can sort by date only, not by other fields.

% ============================================================================
\section{Can Texis organize results by category? }

Yes. This is another powerful feature benefiting from relational
technology. You're not limited to presenting results in one long
list. For example, if your data has a {\tt state} field, you might
want to present the results grouped by state. Categories also may be a
hierarchical structure, Yahoo-style.

% ============================================================================
\section{Can Texis find related results (``More like this'')? }

Yes, that is a standard feature. Texis can take any document or text
selection and turn it into a search for similar records. This is
sometimes called ``query by example.''

% ============================================================================
\section{Can Texis search document ``zones'' separately? }

Elementary! What some people call zones, are in database lingo,
fields. With Texis you may query any field separately or in
combination with other fields. And queries are not limited to text! If
one field (zone) contains a postal code, for example, you could query
that with a numeric range such as 90011 through 97000.

% ============================================================================
\section{What is the Texis relevance ranking algorithm? Is it tunable? }

Texis contains a sophisticated ranking system that may be tuned in
various ways. Factors it uses include: closeness of query words to the
beginning of a document; order of occurrence of the query words; and
proximity (closeness) of query words to each other within a
record. These factors may be weighted to change the ranking
behavior. As an example of how that might be useful: newspaper
articles tend to have the most important material close to the
beginning, so in a newspaper search application, you might give that
factor more weight.

% ****************************************************************************
\chapter{Texis for E-commerce }

% ============================================================================
\section{Does Thunderstone provide a web storefront solution (or auction or classified advertising solution)? }

Yes. Generic applications are available from Thunderstone for all of
these functions. However, they are not intended as turn-key
solutions. Sites of this kind typically each have a unique structure
based on the type of products they sell, which would be reflected in a
customized database schema and/or some additional application
scripting.

% ============================================================================
\section{Does Thunderstone offer a web-site personalization solution? }

Yes, but note that ``personalization'' is used to describe many kinds
of functionality. At the simplest level, you may store user
preferences, so that each user receives only desired information when
visiting the site. To accomplish this, Texis can recognize users by
login or cookie, and it provides a mechanism for carrying a user's
identity from page to page in an encrypted format, so that preferences
may be taken into account on every page visited. Personalization
sometimes refers to a more elaborate process of capturing user actions
(such as purchases or even individual clicks) on an ongoing basis, and
using that information to influence what is displayed to the user
subsequently. Such logic could be implemented in Texis but is not
provided as a finished solution.

% ============================================================================
\section{Can Texis search an existing product catalog? }

Yes. Whether your catalog exists in another database, or in some other
structure, it can easily be imported into a Texis search engine.

% ****************************************************************************
\chapter{Texis for Web Searching}

% ============================================================================
\section{Can Texis provide searching of web sites related to one specific industry? }

Maybe. If you have a list of the relevant sites, it's easy. But we
don't ourselves maintain lists of what are the appropriate sites to
spider for any particular topic.

% ============================================================================
\section{Can Texis search the web in addition to my local content? }

Yes. There are a variety of ways to accomplish that. If you have a
specific list of sites on the web you want to search, it's easy. If
not, the usual approach is to use a free partner site such as
www.master.com for a broad-based web search.

An alternative is to implement a ``meta-search'' against other search
engines.  Texis Web Script includes tools for setting that up.  Note,
however, that other search engines may or may not allow
meta-searching.

If you seek to spider the entire web and build a proprietary
web-search engine, Texis also is an excellent platform for that;
however, such an undertaking takes considerable resources for
bandwidth, machinery, administration, etc., and is normally not
practical as a ``sideline'' or by a thinly funded start-up business.

% ============================================================================
\section{Can Texis power a bid-for-keyword search service? }

Yes, Texis is an excellent platform for such a function.  Ideally,
this requires a sophisticated trade-off between the relevance ranking
and the advertiser's bid for each possible result item.  Texis has a
facility for tuning and accomplishing such a calculation efficiently.

% ============================================================================
\section{Can Texis power an Open Directory Project (dmoz.org) search service? }

Yes, Thunderstone provides a generic Open Directory solution that may
be customized in many ways.

% ============================================================================
\section{Can Texis extract product prices (or addresses or salaries or whatever) from web pages? }

Maybe.  It depends on whether that information is readily identifiable
on each page, either by means of a tag or a predictable structure.  If
so, then a Texis Web Script may easily accomplish the extraction and
save the information as a separate field in the database.

% ============================================================================
\section{Can Texis do federated searching? }

Yes.  A federated search refers to a query that is submitted to two or
more search engines or collections, then the results are displayed
together or combined.  This may also be called a meta-search.  Texis
Web Script has a comprehensive set of tools for setting up federated
searches.

% ============================================================================
\section{Can Texis crawl news publisher sites and provide a news search engine? }

Yes, Thunderstone provides a generic news crawling solution that may
be customized in many ways.

% ****************************************************************************
\chapter{Texis in Other Applications}

% ============================================================================
\section{Can Texis search a newswire feed? }

Yes.  Texis is an excellent platform for news searching.  A Texis
database can index a news feed in real-time, meaning there is no lag
between the time a story arrives and when it is searchable.  Texis
also provides an efficient mechanism for setting up stored queries for
automatic news filtering.

% ============================================================================
\section{Can Texis search e-mail or discussion groups? }

Yes.  Texis is an excellent platform for this.  Utilities for
importing email as well as usenet (NNTP) data are available from
Thunderstone.  In the USA, this type of application may be especially
important for compliance with laws mandating that significant business
data be saved and kept available for auditing. Indexing email archives
can also be an important management role.  Texis goes far beyond most
email archiving solutions that only allow one to search by header such
as the \verb`From`, \verb`To` and \verb`Date` fields.  Texis will
index the full text of both messages and attachments.  It will allow
sophisticated queries such as a phrase in attachment, between two
dates, with results grouped by author.

% ============================================================================
\section{Can Texis search graphics or other binary content? }

Yes, assuming those files have some descriptive text (metadata) such
as photo captions or song titles.  Texis does not know how to look at
a photo of a giraffe and recognize it as an animal, unless it is
labeled as a giraffe!

% ============================================================================
\section{Can Texis search scanned documents? }

Yes, if they have been ``OCR'ed'' (converted to text by optical
character recognition).  Texis does not perform the OCR function.
That must be accomplished with document conversion software from
another vendor.

% ============================================================================
\section{Can Texis classify documents into categories? }

Yes. In the simplest case, this depends on how well you can define the
categories. If you can specify a query or set of words describing each
category, it is quite straightforward.  Assigning a category does not
need to be a yes-or-no operation; multiple categories may be assigned
with a strength (relevance) rating.  In many cases that we see, the
source or author of a document, or other metadata, plays a significant
part in the rules determining category assignments.

In other situations, the rules for assigning categories are not so
clear cut.  Categories may have been assigned in the past by humans
who simply ``know one when they see it.''  For these needs, the Texis
Categorizer is available as an add-on module.  The Categorizer
``learns'' to reproduce these decisions from past the category
assignments.  It assigns the most likely categories, each with a
relevance score. New categories may be created by administrators on an
ad-hoc basis.

% ****************************************************************************
\chapter{Software Compatibility Issues}

% ============================================================================
\section{What other software is required to make use of Texis on the web? }

None other that the operating system and web server! Texis provides a
suite of tools that together enable a designer to create, deploy, and
maintain many web-based applications.

% ============================================================================
\section{Which web browsers and web servers is Texis compatible with? }

All browsers and most web servers work fine with Texis.

% ============================================================================
\section{Can Texis search an Oracle database (or SQL Server or Sybase, etc.)? }

Yes! Many of our customers use Texis side-by-side with another
database. Using two databases in this manner is straightforward
because both obey the SQL standard and thus they may easily exchange
data -- in real time if necessary.

% ============================================================================
\section{Does Texis use XML? }

It can, but it is not required. XML was developed principally as a
data interchange mechanism, and it is useful for acquiring data from,
or providing it to, an affiliate site. On output, Texis can apply XML
markup dynamically even if the data is stored without XML tags.  On
input, XML data typically would be parsed into fields; but XML mark-up
can be preserved in the database if necessary.  Texis Web Script
contains facilities for full manipulation of XML data.  Webinator (a
product built on top of Texis) has a SOAP and XML API for searching.

% ============================================================================
\section{Does Texis use metadata? }

Yes. Metadata is information that describes a document, such as
author, source, or subject categories. Texis stores any such data
routinely, and can enable searches on this data separately or together
with the body text. Texis also make use of metadata for sorting or
grouping text search results.

% ============================================================================
\section{What platforms does Texis run on? }

Texis runs on Linux and Windows Server, and other major OSes depending
on demand.  For an up-to-date list of supported platforms, see the
% Note: `Download Webinator' is made into a link to the download page:
Download Webinator page on our web site, or contact a Thunderstone
sales representative.

% ============================================================================
\section{What hardware resources does Texis need? }

Resource requirements depend on the database structure, record count,
record sizes, query complexity, and of course any other functionality
handled at the application (script) layer. As a benchmark, a Texis
database containing one million records of typical web page content
can serve typical web-search queries at a sustained rate of at least
10 per second on a single-CPU Unix server with 1GB RAM.  The biggest
single variable affecting performance is usually RAM, which will be
used for index caching.  Disk space needs are approximately the ASCII
text size of the content plus 20 percent for basic full-text
indexes. The space needed for indexes could be up to several times
more for applications requiring additional or more detailed indexes.

% ============================================================================
\section{Does Texis need a dedicated server? }

Not necessarily. Texis can share a server with other software such as
a web server or another database. It just depends on the resources
available in comparison to the combined workload of everything running
on the machine.

% ============================================================================
\section{Can I use Texis if our site is hosted by a third-party service? }

Yes, if you have a dedicated machine at that hosting service. If your
site is run on a shared machine, hosting services usually frown on
your installing third-party software.

% ****************************************************************************
\chapter{Application Development Issues}

% ============================================================================
\section{How do I interface to Texis? Is there an API? }

There are a variety of ways to connect an application program to
Texis, but the most common is by HTTP.  This has the advantage of
being simple and high-performance.

A feature-rich, C-callable API is available for special situations,
but for most web applications, the HTTP interface is easier to use and
almost as fast.

% ============================================================================
\section{Does Texis have an ODBC interface? }

Yes. ODBC (Open Database Connectivity) is a protocol developed by
Microsoft as a generic interface to any relational database.
Unfortunately, the specification has considerable ``overhead'' that
tends to make it slow, and it may not be suitable for high transaction
rates. HTTP is much faster for those situations.  Texis also provides
a DBI-DBD interface (Perl module), but performance cautions similar to
those of ODBC apply.

% ============================================================================
\section{What is the typical implementation effort or time to make use of Texis? }

A simple search of web pages, with customized user interface, might
take a couple of days.  A search of a typical product catalog would
typically take a week of scripting effort to get to prototype stage,
and another week for refinements. More elaborate applications usually
take no more than a month.

% ============================================================================
\section{What experience or skills are needed to set up a Texis search engine? }

There are two relevant skills.  One is familiarity with the SQL
language for database application development.  The other is either
simple programming experience, especially using loops; or html
scripting, which may be with a wide range of tools.

% ============================================================================
\section{Can I customize the Texis results presentation (look-and-feel)? }

Yes, completely!  Texis may be used as a ``back-end'' technology, and
imposes no requirements as to user interface.  Texis Web Script, a
tool for creating web applications, is ``neutral'' with regard to what
HTML mark-up (or other user interface technology such as JavaScript)
is used for the user presentation.  Webinator supports complete
customization of results output via administrator-created XSL
stylesheets.

% ============================================================================
\section{Does Texis have a graphical user interface (GUI)? }

Yes, Webinator and the Search Appliance (products built on top of
Texis) have web-based GUIs.  Texis itself does not; the set of
application development tools provided with Texis use the SQL and HTML
programming languages.  Texis scripts are created as text files
containing special HTML tags for defining database interaction.  The
HTML look-and-feel may be created with any HTML authoring tool, then
combined with the tags defining the database calls.

% ============================================================================
\section{Can I interface Texis to a server-side Java (or Perl or VB or ASP etc.) application? }

Yes.  Any web application program can communicate with Texis via HTTP.
Use of Texis Web Script for creating the user interface is optional.

% ============================================================================
\section{Can Texis index a combination of web pages and database content? }

Yes.  Search results may be presented together or separately, as
desired.

% ============================================================================
\section{Can Texis restrict groups of people from seeing certain docs?}

Yes.  This capability is very flexible.  Entitlements may be set
document-by-document, or by groups of documents. Unauthorized users
will not see restricted documents in search results and will not be
able to retrieve them any other way.  Texis recognizes users by login
or cookie, and it provides a mechanism for carrying a user's identity
from page to page in an encrypted URL, so that entitlements may be
taken into account on every page visited.

% ============================================================================
\section{Does Thunderstone provide training in the use of Texis? }

Yes, although that may not be necessary.  Tutorials and tech support
often provide sufficient background to become proficient.  Training
classes, if desired, usually are organized on a custom basis and
conducted at the customer site.

% ============================================================================
\section{Does Thunderstone install and configure Texis? }

Ordinarily, the customer can accomplish this with the possible help of
Thunderstone's technical support group.  However, consulting services
are available for special situations where needed.

% ****************************************************************************
\chapter{Performance Issues}

% ============================================================================
\section{How well does Texis scale up? What are the benchmarks? }

Texis is by far the highest-performance product in the marketplace
providing full-text search within a relational database framework.  It
powers some of the largest search sites on the internet.  Texis
provided the search engine at eBay from their earliest days, and
scaled up to serve more than 40 million searches a day. Databases of
tens of millions of records are not considered large.

% ============================================================================
\section{How many documents or records can Texis search? }

There is no inherent limit.  Texis is routinely used on the most
heavily trafficked web sites for searching databases of tens of
millions of large records.  It has been used with hundreds of millions
of records with no significant complications.

% ============================================================================
\section{How quickly are Texis text indexes updated? }

Instantly!  Texis performs standard database record locking,
unlocking, and management of contention.  It keeps the data consistent
and available for all users while records are being inserted, updated,
or deleted.  No other search engine performs these database-type
functions.

% ============================================================================
\section{Does Texis do incremental indexing? }

Yes.  Items added to the database are searchable instantly.  Texis
takes care of all index updating in background.

% ============================================================================
\section{Isn't CGI scripting slow? How can Texis be fast and use CGI? }

One of the methods of searching Texis data is through a CGI script.
CGI inherently is a very simple, very efficient mechanism.  However,
it has become associated with Perl and other interpreted scripts that
are relatively slow due to the interpreter overhead used at every
invocation.  Texis scripts are compiled and very fast.

% ============================================================================
\section{Is Texis fault-tolerant? }

Yes. At the user's option, Texis can run in a fault-tolerant mode.
Whenever changes are being made to the database, backup data enabling
recovery are preserved in case of power failure or similar disruption.

% ============================================================================
\section{Can Texis be used in a distributed/clustered/redundant architecture? }

Yes, most of our larger customers have implemented some sort of
redundancy strategy.  The designs tend to vary based on many
particulars including the size of the application, user load,
frequency of update, etc.

% ****************************************************************************
\chapter{Linguistic Issues}

% ============================================================================
\section{Can Texis search data in languages other than English? Does it handle the ``accented'' characters of Spanish or French etc.? }

Yes, Texis is used in many languages.  It probably will work
automatically for most European languages.  Support for UTF-8
(Unicode) is standard.  For other encodings, some configuration
settings may be necessary.  Accent characters and any other
non-English characters will be preserved in the data and become fully
searchable, if desired.  There are settings to control how searches
respond to case, diacritical marks (accents etc.), ligatures,
character width etc.  For example, searches may be configured to
ignore accents, so that users unable to enter accented characters may
still find accented-character data, and vice-versa.

% ============================================================================
\section{Can Texis index multi-byte languages (Chinese, Japanese, etc.)? }

Yes, Texis has been used in these languages.  A simple configuration
setting tells Texis to index multi-byte patterns (if other than
UTF-8).  Our customers report satisfactory results with this approach.
However, there are some options to improve the accuracy.  For example,
a specific character in Chinese may sometimes be a word on its own,
and other times part of a different word.  Chinese readers discern the
difference from the context, but there is no indication in the text as
to which it is.  If you need to index one of these languages, please
contact us to discuss these and related issues.

% ============================================================================
\section{Does Texis do ``stemming''? How about in other languages? }

Stemming refers to a process of stripping a word down to its root by
removing suffixes or prefixes (such as the ``s'' on the end of English
plurals), and then searching for valid variations of the root (known
as morphemes).  Texis provides very sophisticated morpheme processing,
with default rules that apply to English.  Various aspects of morpheme
processing may be turned on or off, and the rules customized.  A set
of morpheme processing rules may be specified for any language.
However, we are not linguists and have not defined the stripping and
rebuilding rules for most other languages.  A user organization
typically will wish to customize these rules not only for your
language, but for a particular type of data or search style.

% ============================================================================
\section{Does Texis have a thesaurus capability? }

Yes, a very extensive one.  The thesaurus may be customized for any
special subject.

% ============================================================================
\section{Does Texis search noise words (such as ``the'' or ``is'' etc.) ? }

Yes or no, as desired.  The default behavior is to remove noise words
from queries.  However, the noise-word list can be customized or
turned off, allowing users to search for ``the'' or any word, if
needed.

% ****************************************************************************
\chapter{Other Technical Issues}

% ============================================================================
\section{Can Texis search according to geographical locations, such as zip code? }

Yes.  Texis is unique in its ability to store text records containing
geographical locations, and efficiently perform a text search
restricted to some distance from a particular point (``swimming pool
repair within 10 miles of Columbus, Ohio'').  This is accomplished by
converting the locations into by longitude and latitude.  More details
are available from Thunderstone.

% ============================================================================
\section{Can Texis index dynamic content such as JSP, ColdFusion, or PHP pages? }

Yes.  Dynamic content usually implies that data is stored in a
database.  Texis can easily make any database content searchable, and
can provide the search engine behind a Java application.

% ============================================================================
\section{Does Texis do agent searching? }

Yes.  This usually refers to a query submitted by the user and stored
on the server; a process notifies the user whenever new data is found.
We call this stored query a ``profile.''  Texis contains a very
versatile profile processing capability.  It can handle a high volume
of profiles, matching them in real-time against either incoming data
(such as news); or against the results of a web crawler process; or
against any other data source.

% ============================================================================
\section{Can Texis do a sub-search i.e., only within previous search results? }

Yes!

% ****************************************************************************
\chapter{Business Issues}

% ============================================================================
\section{How much does Texis cost? }

Entry-level Texis prices are in the range of US\$8K to \$14K
one-time. Prices are scaled according to two variables: (a) maximum
records per table, and (b) maximum transactions per day (usually
corresponding to Texis results page views per day).  Leasing also is
available.  Please contact us for more details.

% ============================================================================
\section{How can I get a trial or evaluation copy of Texis? }

We provide a free downloadable example application built on Texis.
That is Webinator.  It includes the source code of the user
application layer so that you may examine or modify the SQL
\verb`SELECT` statements and other functionality.

% ============================================================================
\section{Is there a developer version of Texis? }

Every Texis license comes as a bundle with a complete set of developer
tools.  An entry-level Texis license may be considered the developer
version.

% ============================================================================
\section{Is there a single-user version of Texis? }

Although Texis will run on a PC or workstation, we do not offer
single-user pricing.  Texis is fundamentally a server product designed
for a network.  Texis licensees are free to install the software on
multiple machines at no additional cost, so it is common for
application developers to install copies on their desktop machines.

% ============================================================================
\section{Can I get a site license for Texis? }

All Texis licenses are site licenses in the sense that the software
may be installed on an unlimited number of machines at the customer's
site. See above about the price structure.  (Other Thunderstone
products may be licensed differently.)

% ============================================================================
\section{Does an application service provider (ASP) need a separate license for each web site? }

No.  Applications hosted at one site are bundled under one master
license.

% ============================================================================
\section{Is there special pricing for educational or nonprofit institutions? }

We offer Webinator in part to meet the needs of this market. Webinator
provides an actual Texis database with a subset of Texis functions,
and the entry-level version is free!  Many universities use Webinator
for site-search functions.

% ============================================================================
\section{Must Texis customers display a Thunderstone emblem on their sites? }

No.  Texis customers may totally customize the look-and-feel of their
applications, and a Thunderstone acknowledgement in the HTML content
is not required.  However, Webinator users must display a Thunderstone
copyright and logo.

% ============================================================================
\section{Does Thunderstone have a VAR (value-added-reseller) program? }

Yes, we encourage VAR relationships.  The key aspect is in the ``value
added.''  We expect our VARs to add value by developing applications
on top of Texis for third-party clients.  Please contact us for more
information on becoming a VAR.

%We do not have resellers in the sense of a simple distributor or retailer.

% ============================================================================
\section{Does Thunderstone have a European representative?}

Yes. See the reseller information page at: {\tt
  http://www.thunderstone.com/texis/site/pages/Resellers.html}
Thunderstone also sells and supports Texis directly worldwide from our
USA location.  Please contact us regarding your requirements.

% ============================================================================
\section{Will Thunderstone develop an application or solution to my specifications? }

Maybe.  We do have a consulting arm that undertakes such efforts.  The
requirements need to be well developed before we can bid on a project.
A good way to define the requirements is to mock up the entire
application in static HTML pages.  We are not graphic designers and do
not create the look-and-feel. But we can incorporate a design scheme
defined by others.

% ============================================================================
\section{Can Thunderstone recommend any third-party Texis consultants? }

Maybe.  There actually is some misunderstanding about what skills
Texis consulting entails.  The relevant experience is mostly related
to standard rdbms application development and web scripting.  Very
little of it is Texis-specific.  Thus, the best consultant may be a
local database application developer who can sit down with you to
understand your project.  Ideally, a consultant will have implemented
a web front-end to a large SQL database, such as one involving a
million-row table.  With that in mind, however, please let us know
your needs.  We might be able to refer you to a consultant or VAR for
a custom solution to a specific problem.

% ============================================================================
\section{Is Thunderstone stock publicly traded? What are Thunderstone's annual revenues? }

Thunderstone is privately held and does not disclose internal
financial information. But we can tell you that we are a 30-year-old
company that is conservatively managed and has never required venture
capital. We do not subscribe to the philosophy of trying to build
market share by spending millions of dollars more on promotion than
comes in as revenue. We have grown based on re-invested profits, and
we stay away from ``bet-your-company'' strategies!

% ============================================================================
\section{What fallback do I have if Thunderstone ever goes out of business or discontinues supporting Texis?}

Texis is a mature and stable product that has been on the market for
17 years. At most customer sites, Texis runs for many months without
restarting. However, we recognize that many customers are wary of
becoming overly reliant on any one software vendor. The ultimate
fallback, should a customer ever need to discontinue using Texis, lies
precisely in Thunderstone's adherence to standards. Texis data may
easily be transferred into any competitive SQL database. Likewise,
Texis scripts consist essentially of standard HTML with encapsulated
SQL calls that could, with some effort but not an unreasonable amount,
be converted to a competitive scripting environment. Of course, other
databases at this time do not come close to Texis's performance in
text-intensive applications. However, if that situation changes in the
future, the migration path to an alternate solution would be
straightforward.

% ============================================================================
\section{Does Thunderstone provide a hosted solution?}

Yes, Thunderstone Data Services provides hosting services to Texis
customers.  Please contact us for further details.

\end{document}


