\subsection{Internet/IP address functions}

The following functions manipulate IP network and/or host addresses;
most take \verb`inet` style argument(s).
This is an IPv4
address string, optionally followed by a netmask.

For IPv4, the format is dotted-decimal, i.e.
$N$[{\tt .}$N$[{\tt .}$[N${\tt .}$N$]]] where $N$ is a decimal, octal
or hexadecimal integer from 0 to 255.  If $x < 4$ values of $N$ are
given, the last $N$ is taken as the last $5-x$ bytes instead of 1
byte, with missing bytes padded to the right.  E.g. {\tt 192.258} is
valid and equivalent to {\tt 192.1.2.0}: the last $N$ is 2 bytes in
size, and covers 5 - 2 = 3 needed bytes, including 1 zero pad to the
right.  Conversely, {\tt 192.168.4.1027} is not valid: the last $N$
is too large.

An IPv4 address may optionally be followed by a netmask, either of
the form {\tt /}$B$ or {\tt :}$IPv4$, where $B$ is a decimal, octal or
hexadecimal netmask integer from 0 to 32, and $IPv4$ is a
dotted-decimal IPv4 address of the same format described above.  If an
{\tt :}$IPv4$ netmask is given, only the largest contiguous set of
most-significant 1 bits are used (because netmasks are contiguous).
If no netmask is given, it will be calculated from standard IPv4 class
A/B/C/D/E rules, but will be large enough to include all given bytes
of the IP.  E.g. {\tt 1.2.3.4} is Class A which has a netmask of 8,
but the netmask will be extended to 32 to include all 4 given bytes.


In version 7.07.1554395000 20190404 and later, error messages are
reported.

  The \verb`inet` functions were added in version 5.01.1113268256 of
Apr 11 2005 and include the following.  See also the Vortex
\verb`<urlutil>` equivalents:

\begin{itemize}
  \item \verb`inetabbrev(inet)` \\

    Returns a possibly shorter-than-canonical representation of
    \verb`$inet`, where trailing zero byte(s) of an IPv4 address may
    be omitted.  All bytes of the network, and leading non-zero bytes
    of the host, will be included.  E.g.  {\tt <urlutil inetabbrev
      "192.100.0.0/24">} returns {\tt 192.100.0/24}.  The {\tt /}$B$
    netmask is included, except if (in version 7.07.1554840000
    20190409 and later) the network is host-only (i.e. netmask is the
    full size of the IP address).  Empty string is returned on error.

  \item \verb`inetcanon(inet)` \\

    Returns canonical representation of \verb`$inet`.  For IPv4, this
    is dotted-decimal with all 4 bytes.
    The {\tt /}$B$ netmask is included, except if (in version
    7.07.1554840000 20190409 and later) the network is host-only
    (i.e. netmask is the full size of the IP address).  Empty string
    is returned on error.

  \item \verb`inetnetwork(inet)` \\
    Returns string IP address with the network bits of \verb`inet`,
    and the host bits set to 0.  Empty string is returned on error.

  \item \verb`inethost(inet)` \\
    Returns string IP address with the host bits of \verb`inet`,
    and the network bits set to 0.  Empty string is returned on error.

  \item \verb`inetbroadcast(inet)` \\
    Returns string IP broadcast address for \verb`inet`, i.e. with
    the network bits, and host bits set to 1.  Empty string is
    returned on error.

  \item \verb`inetnetmask(inet)` \\
    Returns string IP netmask for \verb`inet`, i.e. with the
    network bits set to 1, and host bits set to 0.  Empty string is
    returned on error.

  \item \verb`inetnetmasklen(inet)` \\
    Returns integer netmask length of \verb`inet`.  -1 is returned
    on error.
% $
  \item \verb`inetcontains(inetA, inetB)` \\
    Returns 1 if \verb`inetA` contains \verb`inetB`, i.e. every
    address in \verb`inetB` occurs within the \verb`inetA` network.
    0 is returned if not, or -1 on error.

  \item \verb`inetclass(inet)` \\
    Returns class of \verb`inet`, e.g. {\tt A}, {\tt B}, {\tt C},
    {\tt D}, {\tt E} or {\tt classless} if a different netmask is
    used (or the address is IPv6).  Empty string is returned on error.

  \item \verb`inet2int(inet)` \\

    Returns integer representation of IP network/host bits of
    \verb`$inet` (i.e. without netmask); useful for compact storage of
    address as integer(s) instead of string.
    Returns -1 is returned on error (note that -1 may also be
    returned for an all-ones IP address, e.g. {\tt 255.255.255.255}).

  \item \verb`int2inet(i)` \\
    Returns \verb`inet` string for
    1- or 4-value {\tt varint} \verb`$i`
    taken as an IP address.  Since no netmask can be stored in the
    integer form of an IP address, the returned IP string will not
    have a netmask.  Empty string is returned on error.

\end{itemize}

% - - - - - - - - - - - - - - - - - - - - - - - - - - - - - - - - - - - - - -
\subsubsection{urlcanonicalize}

Canonicalize a URL.  Usage:
\begin{verbatim}
   urlcanonicalize(url[, flags])
\end{verbatim}

Returns a copy of \verb`url`, canonicalized according to
case-insensitive comma-separated \verb`flags`, which are zero or more of:

\begin{itemize}

  \item \verb`lowerProtocol` \\

    Lower-cases the protocol.

  \item \verb`lowerHost` \\

    Lower-cases the hostname.

  \item \verb`removeTrailingDot` \\

    Removes trailing dot(s) in hostname.

  \item \verb`reverseHost` \\

    Reverse the host/domains in the hostname.  E.g.
    {\tt http://host.example.com/} becomes
    {\tt http://com.example.host/}.  This can be used to put the
    most-significant part of the hostname leftmost.

  \item \verb`removeStandardPort` \\

    Remove the port number if it is the standard port for the protocol.

  \item \verb`decodeSafeBytes` \\

    URL-decode safe bytes, where semantics are unlikely to change.
    E.g. ``\verb`%41`'' becomes ``\verb`A`'', but ``\verb`%2F`''
    remains encoded, because it would decode to ``\verb`/`''.

  \item \verb`upperEncoded` \\

    Upper-case the hex characters of encoded bytes.

  \item \verb`lowerPath` \\

    Lower-case the (non-encoded) characters in the path.  May be used
    for URLs known to point to case-insensitive filesystems,
    e.g. Windows.

  \item \verb`addTrailingSlash` \\

    Adds a trailing slash to the path, if no path is present.

\end{itemize}

Default flags are all but \verb`reverseHost`, \verb`lowerPath`.  A
flag may be prefixed with the operator \verb`+` to append the flag to
existing flags; \verb`-` to remove the flag from existing flags; or
\verb`=` (default) to clear existing flags first and then set the
flag.  Operators remain in effect for subsequent flags until the next
operator (if any) is used.  Function added in Texis version 7.05.

% ----------------------------------------------------------------------------
