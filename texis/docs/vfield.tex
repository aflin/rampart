\section{Virtual Fields}

To improve the capabilities of Texis, especially with regard to Metamorph
searching multiple fields we implemented the concept of virtual fields.
This allows you to treat the concatentation of any number of text fields
as a single field.  As a single field you can create an index on the fields,
search the fields, and perform any other operation allowable on a field.
Concatenation is represented by the \verb`\` operator.  For example:

\begin{verbatim}
	SELECT TITLE
	FROM   PAPERS
	WHERE  ABSTRACT\BODY LIKE 'ink coloration';
\end{verbatim}

would display the title of all papers whose abstract or body matched the
query \verb`ink coloration`.  By itself this is helpful, but the real
change is that you could create an index on this virtual field as follows:

\begin{verbatim}
	CREATE METAMORPH INDEX IXMMABSBOD ON PAPERS(ABSTRACT\BODY);
\end{verbatim}

which could greatly improve the performance of this query.  You can create
any type of index on a virtual field, although it is important to remember
that for non Metamorph indices the sum of the fields should not exceed
2048 bytes.  If your keys are text fields this method allows you to create
a unique index across several fields.
