\subsection{Bit manipulation functions}

  These functions are used to manipulate integers as bit fields.  This
can be useful for efficient set operations (e.g. set membership,
intersection, etc.).  For example, categories could be mapped to
sequential bit numbers, and a row's category membership stored
compactly as bits of an \verb`int` or \verb`varint`, instead of using a
string list.  Category membership can then be quickly determined with
\verb`bitand` on the integer.

  In the following functions, bit field arguments \verb`a` and
\verb`b` are \verb`int` or \verb`varint` (32 bits per integer, all platforms).
Argument \verb`n` is any integer type.  Bits are numbered starting
with 0 as the least-significant bit of the first integer.  31 is the
most-significant bit of the first integer, 32 is the least-significant
bit of the second integer (if a multi-value \verb`varint`), etc.
These functions were added in version 5.01.1099455599 of Nov 2 2004.

\begin{itemize}
  \item \verb`bitand(a, b)` \\
    Returns the bit-wise AND of \verb`a` and \verb`b`.  If one
    argument is shorter than the other, it will be expanded with
    0-value integers.

  \item \verb`bitor(a, b)` \\
    Returns the bit-wise OR of \verb`a` and \verb`b`.  If one argument
    is shorter than the other, it will be expanded with 0-value integers.

  \item \verb`bitxor(a, b)` \\
    Returns the bit-wise XOR (exclusive OR) of \verb`a` and \verb`b`.
    If one argument is shorter than the other, it will be expanded with
    0-value integers.

  \item \verb`bitnot(a)` \\
    Returns the bit-wise NOT of \verb`a`.

  \item \verb`bitsize(a)` \\
    Returns the total number of bits in \verb`a`, i.e. the highest
    bit number plus 1.

  \item \verb`bitcount(a)` \\
    Returns the number of bits in \verb`a` that are set to 1.

  \item \verb`bitmin(a)` \\
    Returns the lowest bit number in \verb`a` that is set to 1.
    If none are set to 1, returns -1.

  \item \verb`bitmax(a)` \\
    Returns the highest bit number in \verb`a` that is set to 1.
    If none are set to 1, returns -1.

  \item \verb`bitlist(a)` \\
    Returns the list of bit numbers of \verb`a`, in ascending order,
    that are set to 1, as a \verb`varint`.  Returns a single -1 if
    no bits are set to 1.

  \item \verb`bitshiftleft(a, n)` \\
    Returns \verb`a` shifted \verb`n` bits to the left, with 0s padded
    for bits on the right.  If \verb`n` is negative, shifts right instead.

  \item \verb`bitshiftright(a, n)` \\
    Returns \verb`a` shifted \verb`n` bits to the right, with 0s padded
    for bits on the left (i.e. an unsigned shift).  If \verb`n` is
    negative, shifts left instead.

  \item \verb`bitrotateleft(a, n)` \\
    Returns \verb`a` rotated \verb`n` bits to the left, with left
    (most-significant) bits wrapping around to the right.  If \verb`n`
    is negative, rotates right instead.

  \item \verb`bitrotateright(a, n)` \\
    Returns \verb`a` rotated \verb`n` bits to the right, with right
    (least-significant) bits wrapping around to the left.  If \verb`n`
    is negative, rotates left instead.

  \item \verb`bitset(a, n)` \\
    Returns \verb`a` with bit number \verb`n` set to 1.  \verb`a` will
    be padded with 0-value integers if needed to reach \verb`n` (e.g.
    \verb`bitset(5, 40)` will return a \verb`varint(2)`).

  \item \verb`bitclear(a, n)` \\
    Returns \verb`a` with bit number \verb`n` set to 0.  \verb`a` will
    be padded with 0-value integers if needed to reach \verb`n` (e.g.
    \verb`bitclear(5, 40)` will return a \verb`varint(2)`).

  \item \verb`bitisset(a, n)` \\
    Returns 1 if bit number \verb`n` is set to 1 in \verb`a`, 0 if not.
\end{itemize}

% ----------------------------------------------------------------------------
