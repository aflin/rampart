% >>>>> DO NOT EDIT: this file generated from function.src <<<<<
% -*- mode: LaTeX -*-
% $Id$
%
%  Use:        For:               PDF render:   Online render:
%  ----        ----               ----------    --------------
%  \verb`...`  Keywords           Fixed-font    <code> fixed-font red-on-pink
%  {\tt ... }  User input         Fixed-font    <tt> fixed-font
%  {\bf ...}   Settings/sections  Bold          <b> bold

\section{Server functions}
The Texis server has a number of functions built into it which can operate
on fields.  This can occur anywhere an expression can occur in a SQL
statement.  It is possible that the server at your site has been extended
with additional functions.  Each of the arguments can be either a single
field name, or another expression.

% ----------------------------------------------------------------------------
\subsection{File functions}

% - - - - - - - - - - - - - - - - - - - - - - - - - - - - - - - - - - - - - -
\subsubsection{fromfile, fromfiletext}

The \verb`fromfile` and \verb`fromfiletext` functions read a file.
The syntax is
\begin{verbatim}
   fromfile(filename[, offset[, length]])
   fromfiletext(filename[, offset[, length]])
\end{verbatim}

These functions take one required, and two optional arguments.  The
first argument is the filename.  The second argument is an offset into
the file, and the third argument is the length of data to read.  If the
second argument is omitted then the file will be read from the
beginning.  If the third argument is omitted then the file will be read
to the end.  The result is the contents of the file.  This can be used
to load data into a table.  For example if you have an indirect field
and you wish to see the contents of the file you can issue SQL similar
to the following.

The difference between the two functions is the type of data that is
returned.  \verb`fromfile` will return varbyte data, and
\verb`fromfiletext` will return varchar data.  If you are using the
functions to insert data into a field you should make sure that you
use the appropriate function for the type of field you are inserting
into.

\begin{verbatim}
     SELECT  FILENAME, fromfiletext(FILENAME)
     FROM    DOCUMENTS
     WHERE   DOCID = 'JT09113' ;
\end{verbatim}

The results are:

\begin{screen}
\begin{verbatim}
  FILENAME            fromfiletext(FILENAME)
  /docs/JT09113.txt   This is the text contained in the document
  that has an id of JT09113.
\end{verbatim}
\end{screen}

% - - - - - - - - - - - - - - - - - - - - - - - - - - - - - - - - - - - - - -
\subsubsection{totext}

Converts data or file to text.  The syntax is
\begin{verbatim}
   totext(filename[, args])
   totext(data[, args])
\end{verbatim}

This function will convert the contents of a file, if the argument given is
an indirect, or else the result of the expression, and convert it to text.
It does this by calling the program \verb'anytotx', which must be in the
path.  The \verb`anytotx` program (obtained from Thunderstone) will handle
\verb`PDF` as well as many other file formats.

As of version 2.06.935767000 the \verb`totext` command will take an optional
second argument which contains arguments to the \verb`anytotx` program.
See the documentation for \verb`anytotx` for details on its arguments.

\begin{verbatim}
     SELECT  FILENAME, totext(FILENAME)
     FROM    DOCUMENTS
     WHERE   DOCID = 'JT09113' ;
\end{verbatim}

The results are:

\begin{screen}
\begin{verbatim}
  FILENAME            totext(FILENAME)
  /docs/JT09113.pdf   This is the text contained in the document
  that has an id of JT09113.
\end{verbatim}
\end{screen}

% - - - - - - - - - - - - - - - - - - - - - - - - - - - - - - - - - - - - - -
\subsubsection{toind}

Create a Texis managed indirect file.  The syntax is
\begin{verbatim}
   toind(data)
\end{verbatim}

This function takes the argument, stores it into a file, and returns the
filename as an \verb`indirect` type.  This is most often used in combination
with \verb`fromfile` to create a Texis managed file.  For example:

\begin{verbatim}
     INSERT  INTO DOCUMENTS
     VALUES('JT09114', toind(fromfile('srcfile')))
\end{verbatim}

The database will now contain a pointer to a copy of \verb|srcfile|, which
will remain searchable even if the original is changed or removed.  An
important point to note is that any changes to \verb|srcfile| will not be
reflected in the database, unless the table row's \verb`indirect` column
is modified (even to the save value, this just tells Texis to re-index it).

% - - - - - - - - - - - - - - - - - - - - - - - - - - - - - - - - - - - - - -
\subsubsection{canonpath}

  Returns canonical version of a file path, i.e. fully-qualified and
without symbolic links:

\begin{verbatim}
  canonpath(path[, flags])
\end{verbatim}

The optional \verb`flags` is a set of bit flags: bit 0 set if error
messages should be issued, bit 1 set if the return value should be
empty instead of \verb`path` on error.  Added in version
5.01.1139446515 20060208.

% - - - - - - - - - - - - - - - - - - - - - - - - - - - - - - - - - - - - - -
\subsubsection{pathcmp}

  File path comparison function; like C function \verb`strcmp()` but
for paths:

\begin{verbatim}
  pathcmp(pathA, pathB)
\end{verbatim}

Returns an integer indicating the sort order of \verb`pathA` relative
to \verb`pathB`: 0 if \verb`pathA` is the same as \verb`pathB`, less
than 0 if \verb`pathA` is less than \verb`pathB`, greater than 0 if
\verb`pathA` is greater than \verb`pathB`.  Paths are compared
case-insensitively if and only if the OS is case-insensitive for
paths, and OS-specific alternate directory separators are considered
the same (e.g. ``\verb`\`'' and ``\verb`/`'' in Windows).  Multiple
consecutive directory separators are considered the same as one.  A
trailing directory separator (if not also a leading separator) is
ignored.  Directory separators sort lexically before any other
character.

  Note that the paths are only compared lexically: no attempt is made
to resolve symbolic links, ``{\tt ..}'' path components, etc.  Note
also that no inference should be made about the magnitude of negative
or positive return values: greater magnitude does not necessarily
indicate greater lexical ``separation'', nor should it be assumed that
comparing the same two paths will always yield the same-magnitude
value in future versions.  Only the sign of the return value is
significant.  Added in version 5.01.1139446515 20060208.

% - - - - - - - - - - - - - - - - - - - - - - - - - - - - - - - - - - - - - -
\subsubsection{basename}

  Returns the base filename of a given file path.

\begin{verbatim}
  basename(path)
\end{verbatim}

The basename is the contents of \verb`path` after the last path separator.
No filesystem checks are performed, as this is a text/parsing function;
thus ``\verb`.`'' and ``\verb`..`'' are not significant.
Added in version 7.00.1352510000 20121109.

% - - - - - - - - - - - - - - - - - - - - - - - - - - - - - - - - - - - - - -
\subsubsection{dirname}

  Returns the directory part of a given file path.

\begin{verbatim}
  dirname(path)
\end{verbatim}

The directory is the contents of \verb`path` before the last path separator
(unless it is significant -- e.g. for the root directory -- in which case
it is retained).  Added in version 7.00.1352510000 20121109.
No filesystem checks are performed, as this is a text/parsing function;
thus ``\verb`.`'' and ``\verb`..`'' are not significant.

% - - - - - - - - - - - - - - - - - - - - - - - - - - - - - - - - - - - - - -
\subsubsection{fileext}

  Returns the file extension of a given file path.

\begin{verbatim}
  fileext(path)
\end{verbatim}

  The file extension starts with and includes a dot.  The file extension
is only considered present in the basename of the path, i.e. after the
last path separator.  Added in version 7.00.1352510000 20121109.

% - - - - - - - - - - - - - - - - - - - - - - - - - - - - - - - - - - - - - -
\subsubsection{joinpath}

Joins one or more file/directory path arguments into a merged path,
inserting/removing a path separator between arguments as needed.
Takes one to 5 path component arguments.  E.g.:

\begin{verbatim}
  joinpath('one', 'two/', '/three/four', 'five')
\end{verbatim}

yields

\begin{verbatim}
  one/two/three/four/five
\end{verbatim}

Added in version 7.00.1352770000 20121112.  Redundant path separators
internal to an argument are not removed, nor are ``{\tt .}'' and ``{\tt
  ..}'' path components removed.  Prior to version 7.07.1550082000
20190213 redundant path separators between arguments were not
removed.

% - - - - - - - - - - - - - - - - - - - - - - - - - - - - - - - - - - - - - -
\subsubsection{joinpathabsolute}

Like \verb`joinpath`, except that a second or later argument that is
an absolute path will overwrite the previously-merged path.  E.g.:

\begin{verbatim}
  joinpathabsolute('one', 'two', '/three/four', 'five')
\end{verbatim}

yields

\begin{verbatim}
  /three/four/five
\end{verbatim}

Under Windows, partially absolute path arguments -- e.g. ``{\tt
  /dir}'' or ``{\tt C:dir}'' where the drive or dir is still relative
-- are considered absolute for the sake of overwriting the merge.

Added in version 7.00.1352770000 20121112.  Redundant path separators
internal to an argument are not removed, nor are ``{\tt .}'' and
``{\tt ..}'' path components removed.  Prior to version
7.07.1550082000 20190213 partially absolute arguments were not
considered absolute.

% ----------------------------------------------------------------------------

\subsection{String Functions}

% - - - - - - - - - - - - - - - - - - - - - - - - - - - - - - - - - - - - - -
\subsubsection{abstract}

Generate an abstract of a given portion of text.  The syntax is
\begin{verbatim}
   abstract(text[, maxsize[, style[, query]]])
\end{verbatim}

  The abstract will be less than \verb`maxsize` characters long, and
will attempt to end at a word boundary.  If \verb`maxsize` is not
specified (or is less than or equal to 0) then a default size of 230
characters is used.

  The \verb`style` argument is a string or integer, and allows a
choice between several different ways of creating the abstract.
Note that some of these styles require the \verb`query` argument as
well, which is a Metamorph query to look for:

\begin{itemize}
  \item \verb`dumb` (0) \\
    Start the abstract at the top of the document.

  \item \verb`smart` (1) \\
    This style will look for the first meaningful chunk of text,
    skipping over any headers at the top of the text.  This is the
    default if neither \verb`style` nor \verb`query` is given.

  \item \verb`querysingle` (2) \\
    Center the abstract contiguously on the best occurence of
    \verb`query` in the document.

  \item \verb`querymultiple` (3) \\
    Like \verb`querysingle`, but also break up the abstract into
    multiple sections (separated with ``\verb`...`'') if needed to
    help ensure all terms are visible.  Also take care with URLs to
    try to show the start and end.

  \item \verb`querybest` \\
    An alias for the best available query-based style; currently the
    same as \verb`querymultiple`.  Using \verb`querybest` in a script
    ensures that if improved styles become available in future
    releases, the script will automatically ``upgrade'' to the best
    style.
\end{itemize}

  If no \verb`query` is given for the \verb`query`$...$ modes, they
fall back to \verb`dumb` mode.  If a \verb`query` is given with a {\em
non-}\verb`query`$...$ mode (\verb`dumb`/\verb`smart`), the mode is
promoted to \verb`querybest`.  The current locale and index
expressions also have an effect on the abstract in the
\verb`query`$...$ modes, so that it more closely reflects an
index-obtained hit.

\begin{verbatim}
     SELECT     abstract(STORY, 0, 1, 'power struggle')
     FROM       ARTICLES
     WHERE      ARTID = 'JT09115' ;
\end{verbatim}

% - - - - - - - - - - - - - - - - - - - - - - - - - - - - - - - - - - - - - -
\subsubsection{text2mm}

Generate \verb`LIKEP` query.  The syntax is
\begin{verbatim}
   text2mm(text[, maxwords])
\end{verbatim}

This function will take a text expression, and produce a list of words
that can be given to \verb`LIKER` or \verb`LIKEP` to find similar
documents.  \verb`text2mm` takes an optional second argument which
specifies how many words should be returned.  If this is not specified
then 10 words are returned.  Most commonly \verb`text2mm` will be given the
name of a field.  If it is an \verb`indirect` field you will need to call
\verb|fromfile| as shown below:

\begin{verbatim}
     SELECT     text2mm(fromfile(FILENAME))
     FROM       DOCUMENTS
     WHERE      DOCID = 'JT09115' ;
\end{verbatim}

You may also call it as \verb`texttomm()` instead of \verb`text2mm()` .

% - - - - - - - - - - - - - - - - - - - - - - - - - - - - - - - - - - - - - -
\subsubsection{keywords}

Generate list of keywords.  The syntax is
\begin{verbatim}
   keywords(text[, maxwords])
\end{verbatim}

{\tt keywords} is similar to {\tt text2mm} but produces a list of
phrases, with a linefeed separating them.  The difference between
{\tt text2mm} and {\tt keywords} is that {\tt keywords} will maintain
the phrases.
{\tt keywords} also takes an optional second
argument which indicates how many words or phrases should be returned.

% - - - - - - - - - - - - - - - - - - - - - - - - - - - - - - - - - - - - - -
\subsubsection{length}

Returns the length in characters of a \verb`char` or \verb`varchar`
expression, or number of strings/items in other types.  The syntax is
\begin{verbatim}
  length(value[, mode])
\end{verbatim}

For example:

\begin{verbatim}
     SELECT  NAME, length(NAME)
     FROM    SYSTABLES
\end{verbatim}

The results are:

\begin{screen}
\begin{verbatim}
  NAME                length(NAME)
 SYSTABLES               9
 SYSCOLUMNS             10
 SYSINDEX                8
 SYSUSERS                8
 SYSPERMS                8
 SYSTRIG                 7
 SYSMETAINDEX           12
\end{verbatim}
\end{screen}

  The optional \verb`mode` argument is a
\verb`stringcomparemode`-style compare mode to use; see the Vortex manual
on {\tt <apicp stringcomparemode>} for details on syntax and the
default.  If \verb`mode` is not given, the current {\tt apicp
stringcomparemode} is used.  Currently the only pertinent \verb`mode`
flag is ``{\tt iso-8859-1}'', which determines whether to interpret
\verb`value` as ISO-8859-1 or UTF-8.  This can alter how many characters long
the string appears to be, as UTF-8 characters are variable-byte-sized,
whereas ISO-8859-1 characters are always mono-byte.  The \verb`mode`
argument was added in version 6.

  In version 5.01.1226622000 20081113 and later, if given a \verb`strlst`
type \verb`value`, \verb`length()` returns the number of string values
in the list.  For other types, it returns the number of values, e.g.
for \verb`varint` it returns the number of integer values.

% - - - - - - - - - - - - - - - - - - - - - - - - - - - - - - - - - - - - - -
\subsubsection{lower}

Returns the text expression with all letters in lower-case. The syntax is
\begin{verbatim}
  lower(text[, mode])
\end{verbatim}

For example:

\begin{verbatim}
     SELECT  NAME, lower(NAME)
     FROM    SYSTABLES
\end{verbatim}

The results are:

\begin{screen}
\begin{verbatim}
  NAME                lower(NAME)
 SYSTABLES            systables
 SYSCOLUMNS           syscolumns
 SYSINDEX             sysindex
 SYSUSERS             sysusers
 SYSPERMS             sysperms
 SYSTRIG              systrig
 SYSMETAINDEX         sysmetaindex
\end{verbatim}
\end{screen}

Added in version 2.6.932060000.

  The optional \verb`mode` argument is a string-folding mode in the
same format as {\tt <apicp stringcomparemode>}; see the Vortex manual
for details on the syntax and default.  If \verb`mode` is unspecified,
the current {\tt apicp stringcomparemode} setting -- with ``{\tt +lowercase}''
aded -- is used.  The \verb`mode` argument was added in version 6.

% - - - - - - - - - - - - - - - - - - - - - - - - - - - - - - - - - - - - - -
\subsubsection{upper}

Returns the text expression with all letters in upper-case. The sytax is
\begin{verbatim}
  upper(text[, mode])
\end{verbatim}

For example:

\begin{verbatim}
     SELECT  NAME, upper(NAME)
     FROM    SYSTABLES
\end{verbatim}

The results are:

\begin{screen}
\begin{verbatim}
  NAME                upper(NAME)
 SYSTABLES            SYSTABLES
 SYSCOLUMNS           SYSCOLUMNS
 SYSINDEX             SYSINDEX
 SYSUSERS             SYSUSERS
 SYSPERMS             SYSPERMS
 SYSTRIG              SYSTRIG
 SYSMETAINDEX         SYSMETAINDEX
\end{verbatim}
\end{screen}

Added in version 2.6.932060000.

  The optional \verb`mode` argument is a string-folding mode in the
same format as {\tt <apicp stringcomparemode>}; see the Vortex manual
for details on the syntax and default.  If \verb`mode` is unspecified,
the current {\tt apicp stringcomparemode} setting -- with ``{\tt
+uppercase}'' added -- is used.  The \verb`mode` argument was added in
version 6.

% - - - - - - - - - - - - - - - - - - - - - - - - - - - - - - - - - - - - - -
\subsubsection{initcap}

Capitalizes text.  The syntax is
\begin{verbatim}
  initcap(text[, mode])
\end{verbatim}

Returns the text expression with the first letter of each word in
title case (i.e. upper case), and all other letters in lower-case.
For example:

\begin{verbatim}
     SELECT  NAME, initcap(NAME)
     FROM    SYSTABLES
\end{verbatim}

The results are:

\begin{screen}
\begin{verbatim}
  NAME                initcap(NAME)
 SYSTABLES            Systables
 SYSCOLUMNS           Syscolumns
 SYSINDEX             Sysindex
 SYSUSERS             Sysusers
 SYSPERMS             Sysperms
 SYSTRIG              Systrig
 SYSMETAINDEX         Sysmetaindex
\end{verbatim}
\end{screen}

Added in version 2.6.932060000.

  The optional \verb`mode` argument is a string-folding mode in the
same format as {\tt <apicp stringcomparemode>}; see the Vortex manual
for details on the syntax and default.  If \verb`mode` is unspecified,
the current {\tt apicp stringcomparemode} setting -- with ``{\tt +titlecase}''
added -- is used.  The \verb`mode` argument was added in version 6.

% - - - - - - - - - - - - - - - - - - - - - - - - - - - - - - - - - - - - - -
\subsubsection{sandr}

Search and replace text.
\begin{verbatim}
   sandr(search, replace, text)
\end{verbatim}

Returns the text expression with the search REX expression replaced
with the replace expression.  See the REX documentation and the
Vortex sandr function documentation for complete syntax of the search
and replace expressions.

\begin{verbatim}
     SELECT  NAME, sandr('>>=SYS=', 'SYSTEM TABLE ', NAME) DESC
     FROM    SYSTABLES
\end{verbatim}

The results are:

\begin{screen}
\begin{verbatim}
  NAME                DESC
 SYSTABLES            SYSTEM TABLE TABLES
 SYSCOLUMNS           SYSTEM TABLE COLUMNS
 SYSINDEX             SYSTEM TABLE INDEX
 SYSUSERS             SYSTEM TABLE USERS
 SYSPERMS             SYSTEM TABLE PERMS
 SYSTRIG              SYSTEM TABLE TRIG
 SYSMETAINDEX         SYSTEM TABLE METAINDEX
\end{verbatim}
\end{screen}

Added in version 3.0

% - - - - - - - - - - - - - - - - - - - - - - - - - - - - - - - - - - - - - -
\subsubsection{separator}

Returns the separator character from its \verb`strlst` argument,
as a \verb`varchar` string:

\begin{verbatim}
   separator(strlstValue)
\end{verbatim}

  This can be used in situations where the \verb`strlstValue` argument
may have a nul character as the separator, in which case simply
converting \verb`strlstValue` to \verb`varchar` and looking at the
last character would be incorrect.  Added in version 5.01.1226030000
20081106.

% - - - - - - - - - - - - - - - - - - - - - - - - - - - - - - - - - - - - - -
\subsubsection{stringcompare}

  Compares its string (\verb`varchar`) arguments \verb`a` and
\verb`b`, returning -1 if \verb`a` is less than \verb`b`, 0 if they
are equal, or 1 if \verb`a` is greater than \verb`b`:

\begin{verbatim}
  stringcompare(a, b[, mode])
\end{verbatim}

  The strings are compared using the optional \verb`mode` argument,
which is a string-folding mode in the same format as
{\tt <apicp stringcomparemode>}; see the Vortex manual for details on
the syntax and default.  If \verb`mode` is unspecified, the current
{\tt apicp stringcomparemode} setting is used.  Function added
in version 6.00.1304108000 20110429.

% - - - - - - - - - - - - - - - - - - - - - - - - - - - - - - - - - - - - - -
\subsubsection{stringformat}

  Returns its arguments formatted into a string (\verb`varchar`), like
the equivalent Vortex function \verb`<strfmt>` (based on the C
function \verb`sprintf()`):

\begin{verbatim}
  stringformat(format[, arg[, arg[, arg[, arg]]]])
\end{verbatim}

  The \verb`format` argument is a \verb`varchar` string that describes
how to print the following argument(s), if any.  See the Vortex manual
for \verb`<strfmt>` for details.  Added in version 6.00.1300386000
20110317.

% ----------------------------------------------------------------------------

\subsection{Math functions}

The following basic math functions are available in Texis:
\verb`acos`, \verb`asin`, \verb`atan`, \verb`atan2`, \verb`ceil`, \verb`cos`,
\verb`cosh`, \verb`exp`, \verb`fabs`, \verb`floor`, \verb`fmod`,
\verb`log`, \verb`log10`, \verb`pow`, \verb`sin`, \verb`sinh`, \verb`sqrt`,
\verb`tan`, \verb`tanh`.

All of the above functions call the ANSI C math library function
of the same name, and return a result of type \verb`double`.
\verb`pow`, \verb`atan2` and \verb`fmod` take two double arguments,
the remainder take one double argument.  Added in version:
2.6.931790000

In addition, the following math-related functions are available:

\begin{itemize}
  \item \verb`isNaN(x)` \\
    Returns 1 if \verb`x` is a float or double NaN (Not a Number) value,
    0 if not.  This function should be used to test for NaN, rather than
    using the equality operator (e.g. \verb`x = 'NaN'`), because the
    IEEE standard defines \verb`NaN == NaN` to be false, not true as
    might be expected.  Added in version 5.01.1193955406 20071101.
\end{itemize}

% ----------------------------------------------------------------------------

\subsection{Date functions}

The following date functions are available in Texis:
\verb`dayname`, \verb`month`, \verb`monthname`, \verb`dayofmonth`,
\verb`dayofweek`, \verb`dayofyear`, \verb`quarter`, \verb`week`,
\verb`year`, \verb`hour`, \verb`minute`, \verb`second`.

All the functions take a date as an argument.  \verb`dayname` and \verb`monthname` will return a string with the full day or month name based on the current
locale, and the others return a number.

The \verb`dayofweek` function returns 1 for Sunday.  The quarter is
based on months, so April 1st is the first day of quarter 2.  Week 1
begins with the first Sunday of the year.

Added in version: 3.0.948300000

The following functions were added in version 3.01.990400000:
\verb`monthseq`, \verb`weekseq` and \verb`dayseq` which will return the
number of months, weeks and days since an arbitrary past date.  These
can be used when comparing dates to see how many months, weeks or days
separate them.

% ----------------------------------------------------------------------------

\subsection{Bit manipulation functions}

  These functions are used to manipulate integers as bit fields.  This
can be useful for efficient set operations (e.g. set membership,
intersection, etc.).  For example, categories could be mapped to
sequential bit numbers, and a row's category membership stored
compactly as bits of an \verb`int` or \verb`varint`, instead of using a
string list.  Category membership can then be quickly determined with
\verb`bitand` on the integer.

  In the following functions, bit field arguments \verb`a` and
\verb`b` are \verb`int` or \verb`varint` (32 bits per integer, all platforms).
Argument \verb`n` is any integer type.  Bits are numbered starting
with 0 as the least-significant bit of the first integer.  31 is the
most-significant bit of the first integer, 32 is the least-significant
bit of the second integer (if a multi-value \verb`varint`), etc.
These functions were added in version 5.01.1099455599 of Nov 2 2004.

\begin{itemize}
  \item \verb`bitand(a, b)` \\
    Returns the bit-wise AND of \verb`a` and \verb`b`.  If one
    argument is shorter than the other, it will be expanded with
    0-value integers.

  \item \verb`bitor(a, b)` \\
    Returns the bit-wise OR of \verb`a` and \verb`b`.  If one argument
    is shorter than the other, it will be expanded with 0-value integers.

  \item \verb`bitxor(a, b)` \\
    Returns the bit-wise XOR (exclusive OR) of \verb`a` and \verb`b`.
    If one argument is shorter than the other, it will be expanded with
    0-value integers.

  \item \verb`bitnot(a)` \\
    Returns the bit-wise NOT of \verb`a`.

  \item \verb`bitsize(a)` \\
    Returns the total number of bits in \verb`a`, i.e. the highest
    bit number plus 1.

  \item \verb`bitcount(a)` \\
    Returns the number of bits in \verb`a` that are set to 1.

  \item \verb`bitmin(a)` \\
    Returns the lowest bit number in \verb`a` that is set to 1.
    If none are set to 1, returns -1.

  \item \verb`bitmax(a)` \\
    Returns the highest bit number in \verb`a` that is set to 1.
    If none are set to 1, returns -1.

  \item \verb`bitlist(a)` \\
    Returns the list of bit numbers of \verb`a`, in ascending order,
    that are set to 1, as a \verb`varint`.  Returns a single -1 if
    no bits are set to 1.

  \item \verb`bitshiftleft(a, n)` \\
    Returns \verb`a` shifted \verb`n` bits to the left, with 0s padded
    for bits on the right.  If \verb`n` is negative, shifts right instead.

  \item \verb`bitshiftright(a, n)` \\
    Returns \verb`a` shifted \verb`n` bits to the right, with 0s padded
    for bits on the left (i.e. an unsigned shift).  If \verb`n` is
    negative, shifts left instead.

  \item \verb`bitrotateleft(a, n)` \\
    Returns \verb`a` rotated \verb`n` bits to the left, with left
    (most-significant) bits wrapping around to the right.  If \verb`n`
    is negative, rotates right instead.

  \item \verb`bitrotateright(a, n)` \\
    Returns \verb`a` rotated \verb`n` bits to the right, with right
    (least-significant) bits wrapping around to the left.  If \verb`n`
    is negative, rotates left instead.

  \item \verb`bitset(a, n)` \\
    Returns \verb`a` with bit number \verb`n` set to 1.  \verb`a` will
    be padded with 0-value integers if needed to reach \verb`n` (e.g.
    \verb`bitset(5, 40)` will return a \verb`varint(2)`).

  \item \verb`bitclear(a, n)` \\
    Returns \verb`a` with bit number \verb`n` set to 0.  \verb`a` will
    be padded with 0-value integers if needed to reach \verb`n` (e.g.
    \verb`bitclear(5, 40)` will return a \verb`varint(2)`).

  \item \verb`bitisset(a, n)` \\
    Returns 1 if bit number \verb`n` is set to 1 in \verb`a`, 0 if not.
\end{itemize}

% ----------------------------------------------------------------------------

\subsection{Internet/IP address functions}

The following functions manipulate IP network and/or host addresses;
most take \verb`inet` style argument(s).
This is an IPv4
address string, optionally followed by a netmask.

For IPv4, the format is dotted-decimal, i.e.
$N$[{\tt .}$N$[{\tt .}$[N${\tt .}$N$]]] where $N$ is a decimal, octal
or hexadecimal integer from 0 to 255.  If $x < 4$ values of $N$ are
given, the last $N$ is taken as the last $5-x$ bytes instead of 1
byte, with missing bytes padded to the right.  E.g. {\tt 192.258} is
valid and equivalent to {\tt 192.1.2.0}: the last $N$ is 2 bytes in
size, and covers 5 - 2 = 3 needed bytes, including 1 zero pad to the
right.  Conversely, {\tt 192.168.4.1027} is not valid: the last $N$
is too large.

An IPv4 address may optionally be followed by a netmask, either of
the form {\tt /}$B$ or {\tt :}$IPv4$, where $B$ is a decimal, octal or
hexadecimal netmask integer from 0 to 32, and $IPv4$ is a
dotted-decimal IPv4 address of the same format described above.  If an
{\tt :}$IPv4$ netmask is given, only the largest contiguous set of
most-significant 1 bits are used (because netmasks are contiguous).
If no netmask is given, it will be calculated from standard IPv4 class
A/B/C/D/E rules, but will be large enough to include all given bytes
of the IP.  E.g. {\tt 1.2.3.4} is Class A which has a netmask of 8,
but the netmask will be extended to 32 to include all 4 given bytes.


In version 7.07.1554395000 20190404 and later, error messages are
reported.

  The \verb`inet` functions were added in version 5.01.1113268256 of
Apr 11 2005 and include the following.  See also the Vortex
\verb`<urlutil>` equivalents:

\begin{itemize}
  \item \verb`inetabbrev(inet)` \\

    Returns a possibly shorter-than-canonical representation of
    \verb`$inet`, where trailing zero byte(s) of an IPv4 address may
    be omitted.  All bytes of the network, and leading non-zero bytes
    of the host, will be included.  E.g.  {\tt <urlutil inetabbrev
      "192.100.0.0/24">} returns {\tt 192.100.0/24}.  The {\tt /}$B$
    netmask is included, except if (in version 7.07.1554840000
    20190409 and later) the network is host-only (i.e. netmask is the
    full size of the IP address).  Empty string is returned on error.

  \item \verb`inetcanon(inet)` \\

    Returns canonical representation of \verb`$inet`.  For IPv4, this
    is dotted-decimal with all 4 bytes.
    The {\tt /}$B$ netmask is included, except if (in version
    7.07.1554840000 20190409 and later) the network is host-only
    (i.e. netmask is the full size of the IP address).  Empty string
    is returned on error.

  \item \verb`inetnetwork(inet)` \\
    Returns string IP address with the network bits of \verb`inet`,
    and the host bits set to 0.  Empty string is returned on error.

  \item \verb`inethost(inet)` \\
    Returns string IP address with the host bits of \verb`inet`,
    and the network bits set to 0.  Empty string is returned on error.

  \item \verb`inetbroadcast(inet)` \\
    Returns string IP broadcast address for \verb`inet`, i.e. with
    the network bits, and host bits set to 1.  Empty string is
    returned on error.

  \item \verb`inetnetmask(inet)` \\
    Returns string IP netmask for \verb`inet`, i.e. with the
    network bits set to 1, and host bits set to 0.  Empty string is
    returned on error.

  \item \verb`inetnetmasklen(inet)` \\
    Returns integer netmask length of \verb`inet`.  -1 is returned
    on error.
% $
  \item \verb`inetcontains(inetA, inetB)` \\
    Returns 1 if \verb`inetA` contains \verb`inetB`, i.e. every
    address in \verb`inetB` occurs within the \verb`inetA` network.
    0 is returned if not, or -1 on error.

  \item \verb`inetclass(inet)` \\
    Returns class of \verb`inet`, e.g. {\tt A}, {\tt B}, {\tt C},
    {\tt D}, {\tt E} or {\tt classless} if a different netmask is
    used (or the address is IPv6).  Empty string is returned on error.

  \item \verb`inet2int(inet)` \\

    Returns integer representation of IP network/host bits of
    \verb`$inet` (i.e. without netmask); useful for compact storage of
    address as integer(s) instead of string.
    Returns -1 is returned on error (note that -1 may also be
    returned for an all-ones IP address, e.g. {\tt 255.255.255.255}).

  \item \verb`int2inet(i)` \\
    Returns \verb`inet` string for
    1- or 4-value {\tt varint} \verb`$i`
    taken as an IP address.  Since no netmask can be stored in the
    integer form of an IP address, the returned IP string will not
    have a netmask.  Empty string is returned on error.

\end{itemize}

% - - - - - - - - - - - - - - - - - - - - - - - - - - - - - - - - - - - - - -
\subsubsection{urlcanonicalize}

Canonicalize a URL.  Usage:
\begin{verbatim}
   urlcanonicalize(url[, flags])
\end{verbatim}

Returns a copy of \verb`url`, canonicalized according to
case-insensitive comma-separated \verb`flags`, which are zero or more of:

\begin{itemize}

  \item \verb`lowerProtocol` \\

    Lower-cases the protocol.

  \item \verb`lowerHost` \\

    Lower-cases the hostname.

  \item \verb`removeTrailingDot` \\

    Removes trailing dot(s) in hostname.

  \item \verb`reverseHost` \\

    Reverse the host/domains in the hostname.  E.g.
    {\tt http://host.example.com/} becomes
    {\tt http://com.example.host/}.  This can be used to put the
    most-significant part of the hostname leftmost.

  \item \verb`removeStandardPort` \\

    Remove the port number if it is the standard port for the protocol.

  \item \verb`decodeSafeBytes` \\

    URL-decode safe bytes, where semantics are unlikely to change.
    E.g. ``\verb`%41`'' becomes ``\verb`A`'', but ``\verb`%2F`''
    remains encoded, because it would decode to ``\verb`/`''.

  \item \verb`upperEncoded` \\

    Upper-case the hex characters of encoded bytes.

  \item \verb`lowerPath` \\

    Lower-case the (non-encoded) characters in the path.  May be used
    for URLs known to point to case-insensitive filesystems,
    e.g. Windows.

  \item \verb`addTrailingSlash` \\

    Adds a trailing slash to the path, if no path is present.

\end{itemize}

Default flags are all but \verb`reverseHost`, \verb`lowerPath`.  A
flag may be prefixed with the operator \verb`+` to append the flag to
existing flags; \verb`-` to remove the flag from existing flags; or
\verb`=` (default) to clear existing flags first and then set the
flag.  Operators remain in effect for subsequent flags until the next
operator (if any) is used.  Function added in Texis version 7.05.

% ----------------------------------------------------------------------------

\subsection{Geographical coordinate functions}

The geographical coordinate functions allow for efficient processing
of latitude / longitude operations.  They allow for the conversion of
a latitude/longitude pair into a single ``geocode'', which is a single
\verb`long` value that contains both values.  This can be used to
easily compare it to other geocodes (for distance calculations) or for
finding other geocodes that are within a certain distance.

% - - - - - - - - - - - - - - - - - - - - - - - - - - - - - - - - - - - - - -
\subsubsection{azimuth2compass}

\begin{verbatim}
  azimuth2compass(double azimuth [, int resolution [, int verbosity]])
\end{verbatim}

The \verb`azimuth2compass` function converts a numerical azimuth value
(degrees of rotation from 0 degrees north) and converts it into a
compass heading, such as \verb`N` or \verb`Southeast`.  The exact text
returned is controlled by two optional parameters, \verb`resolution`
and \verb`verbosity`.

\verb`Resolution` determines how fine-grained the values returned
are.  There are 4 possible values:

\begin{itemize}
\item \verb`1` - Only the four cardinal directions are used (N, E, S,
W)
\item \verb`2` \em{(default)} - Inter-cardinal directions (N, NE, E,
etc.)
\item \verb`3` - In-between inter-cardinal directions (N, NNE, NE, ENE,
E, etc.)
\item \verb`4` - ``by'' values (N, NbE, NNE, NEbN, NE, NEbE, ENE, EbN,
E, etc.)
\end{itemize}

\verb`Verbosity` affects how verbose the resulting text is.  There are
two possible values:

\begin{itemize}
\item \verb`1` \em{(default)} - Use initials for direction values (N,
NbE, NNE, etc.)
\item \verb`2` - Use full text for direction values (North, North by
east, North-northeast, etc.)
\end{itemize}

For an azimuth value of \verb`105`, here are some example results of
\verb`azimuth2compass`:

\begin{verbatim}
azimuth2compass(105): E
azimuth2compass(105, 3): ESE
azimuth2compass(105, 4): EbS
azimuth2compass(105, 1, 2): East
azimuth2compass(105, 3, 2): East-southeast
azimuth2compass(105, 4, 2): East by south
\end{verbatim}

% - - - - - - - - - - - - - - - - - - - - - - - - - - - - - - - - - - - - - -
\subsubsection{azimuthgeocode}

\begin{verbatim}
  azimuthgeocode(geocode1, geocode2 [, method])
\end{verbatim}

The \verb`azimuthgeocode` function calculates the directional heading
going from one geocode to another.  It returns a number between 0-360
where 0 is north, 90 east, etc., up to 360 being north again.

The third, optional \verb`method` parameter can be used to specify
which mathematical method is used to calculate the direction.  There
are two possible values:

\begin{itemize}
\item \verb`greatcircle` {\em(default)} - The ``Great Circle'' method
is a highly accurate tool for calculating distances and directions on
a sphere.  It is used by default.

\item \verb`pythagorean` - Calculations based on the pythagorean
method can also be used.  They're faster, but less accurate as the
core formulas don't take the curvature of the earth into
consideration.  Some internal adjustments are made, but the values are
less accurate than the \verb`greatcircle` method, especially over long
distances and with paths that approach the poles.
\end{itemize}

\EXAMPLE 

For examples of using \verb`azimuthgeocode`, see the \verb`geocode`
script in the \verb`texis/samples` directory.

% - - - - - - - - - - - - - - - - - - - - - - - - - - - - - - - - - - - - - -
\subsubsection{azimuthlatlon}

\begin{verbatim}
  azimuthlatlon(lat1, lon1, lat2, lon2, [, method])
\end{verbatim}

The \verb`azimuthlatlon` function calculates the directional heading
going from one latitude-longitude point to another.  It operates
identically to \verb`azimuthgeocode`, except azimuthlatlon takes its
parameters in a pair of latitude-longitude points instead of geocode
values.

The third, optional \verb`method` parameter can be used to specify
which mathematical method is used to calculate the direction.  There
are two possible values:

\begin{itemize}
\item \verb`greatcircle` {\em(default)} - The ``Great Circle'' method
is a highly accurate tool for calculating distances and directions on
a sphere.  It is used by default.

\item \verb`pythagorean` - Calculations based on the pythagorean
method can also be used.  They're faster, but less accurate as the
core formulas don't take the curvature of the earth into
consideration.  Some internal adjustments are made, but the values are
less accurate than the \verb`greatcircle` method, especially over long
distances and with paths that approach the poles.

\end{itemize}

% - - - - - - - - - - - - - - - - - - - - - - - - - - - - - - - - - - - - - -
\subsubsection{dms2dec, dec2dms}

\begin{verbatim}
  dms2dec(dms)
  dec2dms(dec)
\end{verbatim}

The \verb`dms2dec` and \verb`dec2dms` functions are for changing back
and forth between the deprecated Texis/Vortex ``degrees minutes
seconds'' (DMS) format (west-positive) and ``decimal degree'' format
for latitude and longitude coordinates.  All SQL geographical functions
expect decimal degree parameters (the Vortex \verb`<code2geo>` and
\verb`<geo2code>` Vortex functions expect Texis/Vortex DMS).

Texis/Vortex DMS values are of the format $DDDMMSS$.  For example,
35\degree 15' would be represented as 351500.

In decimal degrees, a degree is a whole digit, and minutes \& seconds
are represented as fractions of a degree.  Therefore, 35\degree 15'
would be 35.25 in decimal degrees.

Note that the Texis/Vortex DMS format has {\em west}-positive
longitudes (unlike ISO 6709 DMS format), and decimal
degrees have {\em east}-positive longitudes.  It is up to the caller
to flip the sign of longitudes where needed.

% - - - - - - - - - - - - - - - - - - - - - - - - - - - - - - - - - - - - - -
\subsubsection{distgeocode}

\begin{verbatim}
  distgeocode(geocode1, geocode2 [, method] )
\end{verbatim}

The \verb`distgeocode` function calculates the distance, in miles,
between two given geocodes.  It uses the ``Great Circle'' method for
calculation by default, which is very accurate.  A faster, but less
accurate, calculation can be done with the Pythagorean theorem.  It is
not designed for distances on a sphere, however, and becomes somewhat  
inaccurate at larger distances and on paths that approach the poles.
To use the Pythagorean theorem, pass a third string parameter,
``\verb`pythagorean`'', to force that method.  ``\verb`greatcircle`''
can also be specified as a method. 

For example:
\begin{itemize}
\item New York (JFK) to Cleveland (CLE), the Pythagorean method is off by 
.8 miles (.1\%)
\item New York (JFK) to Los Angeles (LAX), the Pythagorean method is off by 
22.2 miles (.8\%)
\item New York (JFK) to South Africa (PLZ), the Pythagorean method is off by
430 miles (5.2\%)
\end{itemize}

\EXAMPLE 

For examples of using \verb`distgeocode`, see the \verb`geocode`
script in the \verb`texis/samples` directory.

\SEE

\verb`distlatlon`

For a very fast method that leverages geocodes for selecting cities
within a certain radius, see the \verb`<code2geo>` and \verb`<geo2code>`
functions in the Vortex manual.

% - - - - - - - - - - - - - - - - - - - - - - - - - - - - - - - - - - - - - -
\subsubsection{distlatlon}

\begin{verbatim}
  distlatlon(lat1, lon1, lat2, lon2 [, method] )
\end{verbatim}

The \verb`distlatlon` function calculates the distance, in miles,
between two points, represented in latitude/longitude pairs in decimal
degree format.

Like \verb`distgeocode`, it uses the ``Great Circle'' method by
default, but can be overridden to use the faster, less accurate
Pythagorean method if ``\verb`pythagorean`'' is passed as the optional
\verb`method` parameter.

For example:
\begin{itemize}
\item New York (JFK) to Cleveland (CLE), the Pythagorean method is off by 
.8 miles (.1\%)
\item New York (JFK) to Los Angeles (LAX), the Pythagorean method is off by 
22.2 miles (.8\%)
\item New York (JFK) to South Africa (PLZ), the Pythagorean method is off by
430 miles (5.2\%)
\end{itemize}

\SEE

\verb`distgeocode`

% - - - - - - - - - - - - - - - - - - - - - - - - - - - - - - - - - - - - - -

\subsubsection{latlon2geocode, latlon2geocodearea}

\begin{verbatim}
  latlon2geocode(lat[, lon])
  latlon2geocodearea(lat[, lon], radius)
\end{verbatim}

The \verb`latlon2geocode` function encodes a given latitude/longitude
coordinate into one \verb`long` return value.  This encoded value -- a
``geocode'' value -- can be indexed and used with a special variant of
Texis' \verb`BETWEEN` operator for bounded-area searches of a
geographical region.

The \verb`latlon2geocodearea` function generates a bounding area
centered on the coordinate.  It encodes a given latitude/longitude
coordinate into a {\em two-} value \verb`varlong`.  The returned
geocode value pair represents the southwest and northeast corners of a
square box centered on the latitude/longitude coordinate, with sides
of length two times \verb`radius` (in decimal degrees).  This bounding
area can be used with the Texis \verb`BETWEEN` operator for fast
geographical searches.  \verb`latlon2geocodearea` was added in
version 6.00.1299627000 20110308; it replaces the deprecated Vortex
\verb`<geo2code>` function.

The \verb`lat` and \verb`lon` parameters are \verb`double`s in the
decimal degrees format.  (To pass $DDDMMSS$ ``degrees minutes
seconds'' (DMS) format values, convert them first with \verb`dms2dec`
or \verb`parselatitude()`/\verb`parselongitude()`.).  Negative numbers
represent south latitudes and west longitudes, i.e. these functions
are east-positive, and decimal format (unlike Vortex \verb`<geo2code>`
which is west-positive, and DMS-format).

Valid values for latitude are -90 to 90 inclusive.  Valid values for
longitude are -360 to 360 inclusive.  A longitude value less than -180
will have 360 added to it, and a longitude value greater than 180 will
have 360 subtracted from it.  This allows longitude values to continue
to increase or decrease when crossing the International Dateline, and
thus avoid a non-linear ``step function''.  Passing invalid \verb`lat`
or \verb`lon` values to \verb`latlon2geocode` will return -1.  These
changes were added in version 5.01.1193956000 20071101.

In version 5.01.1194667000 20071109 and later, the \verb`lon`
parameter is optional: both latitude and longitude (in that order) may
be given in a single space- or comma-separated text (\verb`varchar`)
value for \verb`lat`.  Also, a \verb`N`/\verb`S` suffix (for latitude)
or \verb`E`/\verb`W` suffix (for longitude) may be given; \verb`S` or
\verb`W` will negate the value.

  In version 6.00.1300154000 20110314 and later, the latitude and/or
longitude may have just about any of the formats supported by
\verb`parselatitude()`/\verb`parselongitude()`
(p.~\pageref{parselatitudeSqlFunc}), provided they are disambiguated
(e.g. separate parameters; or if one parameter, separated by a comma
and/or fully specified with degrees/minutes/seconds).

\EXAMPLE
\begin{samepage}
% NOTE: this is tested in Vortex test454:
\begin{verbatim}
  -- Populate a table with latitude/longitude information:
  create table geotest(city varchar(64), lat double, lon double,
                       geocode long);
  insert into geotest values('Cleveland, OH, USA', 41.4,  -81.5,  -1);
  insert into geotest values('Seattle, WA, USA',   47.6, -122.3,  -1);
  insert into geotest values('Dayton, OH, USA',    39.75, -84.19, -1);
  insert into geotest values('Columbus, OH, USA',  39.96, -83.0,  -1);
  -- Prepare for geographic searches:
  update geotest set geocode = latlon2geocode(lat, lon);
  create index xgeotest_geocode on geotest(geocode);
  -- Search for cities within a 3-degree-radius "circle" (box)
  -- of Cleveland, nearest first:
  select city, lat, lon, distlatlon(41.4, -81.5, lat, lon) MilesAway
  from geotest
  where geocode between (select latlon2geocodearea(41.4, -81.5, 3.0))
  order by 4 asc;
\end{verbatim}
\end{samepage}

For more examples of using \verb`latlon2geocode`, see the
\verb`geocode` script in the \verb`texis/samples` directory.

\CAVEATS

The geocode values returned by \verb`latlon2geocode` and
\verb`latlon2geocodearea` are platform-dependent in format and
accuracy, and should not be copied across platforms. On platforms with
32-bit \verb`long`s a geocode value is accurate to about 32 seconds
(around half a mile, depending on latitude).  -1 is returned for
invalid input values (in version 5.01.1193955804 20071101 and later).

NOTES

The geocodes produced by these functions are compatible with the codes
used by the deprecated Vortex functions \verb`<code2geo>` and
\verb`<geo2code>`.  However, the \verb`<code2geo>` and
\verb`<geo2code>` functions take Texis/Vortex DMS format ($DDDMMSS$
``degrees minutes seconds'', as described in the \verb`dec2dms` and
\verb`dms2dec` SQL functions).

\SEE

\verb`geocode2lat`, \verb`geocode2lon`

% - - - - - - - - - - - - - - - - - - - - - - - - - - - - - - - - - - - - - -
\subsubsection{geocode2lat, geocode2lon}

\begin{verbatim}
  geocode2lat(geocode)
  geocode2lon(geocode)
\end{verbatim}

The \verb`geocode2lat` and \verb`geocode2lon` functions decode a
geocode into a latitude or longitude coordinate, respectively.  The
returned coordinate is in the decimal degrees format.  In
version 5.01.1193955804 20071101 and later, an invalid geocode value
(e.g. -1) will return NaN (Not a Number).

If you want $DDDMMSS$ ``degrees minutes seconds'' (DMS) format, you
can use \verb`dec2dms` to convert it.

\EXAMPLE

\begin{verbatim}
  select city, geocode2lat(geocode), geocode2lon(geocode) from geotest;
\end{verbatim}

\CAVEATS

As with \verb`latlon2geocode`, the \verb`geocode` value is platform-dependent
in accuracy and format, so it should not be copied across platforms,
and the returned coordinates from \verb`geocode2lat` and
\verb`geocode2lon` may differ up to about half a minute from the
original coordinates (due to the finite resolution of a \verb`long`).
In version 5.01.1193955804 20071101 and later, an invalid geocode value
(e.g. -1) will return \verb`NaN` (Not a Number).

\SEE

\verb`latlon2geocode`

% - - - - - - - - - - - - - - - - - - - - - - - - - - - - - - - - - - - - - -
\subsubsection{parselatitude, parselongitude}
\label{parselatitudeSqlFunc}

\begin{verbatim}
  parselatitude(latitudeText)
  parselongitude(longitudeText)
\end{verbatim}

The \verb`parselatitude` and \verb`parselongitude` functions parse a
text (\verb`varchar`) latitude or longitude coordinate, respectively,
and return its value in decimal degrees as a \verb`double`.  The
coordinate should be in one of the following forms (optional parts in
square brackets):

[$H$] $nnn$ [$U$] [\verb`:`] [$H$] [$nnn$ [$U$] [\verb`:`] [$nnn$ [$U$]]] [$H$] \\
$DDMM$[$.MMM$...] \\
$DDMMSS$[$.SSS$...]

where the terms are:

\begin{itemize}
  \item $nnn$ \\

    A number (integer or decimal) with optional plus/minus sign.  Only
    the first number may be negative, in which case it is a south
    latitude or west longitude.  Note that this is true even for
    $DDDMMSS$ (DMS) longitudes -- i.e. the ISO 6709 east-positive
    standard is followed, not the deprecated Texis/Vortex
    west-positive standard.

  \item $U$ \\
    A unit (case-insensitive):
    \begin{itemize}
      \item \verb`d`
      \item \verb`deg`
      \item \verb`deg.`
      \item \verb`degrees`
      \item \verb`'` (single quote) for minutes
      \item \verb`m`
      \item \verb`min`
      \item \verb`min.`
      \item \verb`minutes`
      \item \verb`"` (double quote) for seconds
      \item \verb`s` (iff \verb`d`/\verb`m` also used for degrees/minutes)
      \item \verb`sec`
      \item \verb`sec.`
      \item \verb`seconds`
      \item Unicode degree-sign (U+00B0), in ISO-8559-1 or UTF-8
    \end{itemize}
    If no unit is given, the first number is assumed to be degrees,
    the second minutes, the third seconds.  Note that ``\verb`s`'' may
    only be used for seconds if ``\verb`d`'' and/or ``\verb`m`'' was
    also used for an earlier degrees/minutes value; this is to help
    disambiguate ``seconds'' vs. ``southern hemisphere''.

  \item $H$ \\
    A hemisphere (case-insensitive):
    \begin{itemize}
      \item \verb`N`
      \item \verb`north`
      \item \verb`S`
      \item \verb`south`
      \item \verb`E`
      \item \verb`east`
      \item \verb`W`
      \item \verb`west`
    \end{itemize}
    A longitude hemisphere may not be given for a latitude, and
    vice-versa.

  \item $DD$ \\

    A two- or three-digit degree value, with optional sign.  Note that
    longitudes are east-positive ala ISO 6709, not west-positive like
    the deprecated Texis standard.

  \item $MM$ \\

    A two-digit minutes value, with leading zero if needed to make two digits.

  \item $.MMM$... \\

    A zero or more digit fractional minute value.

  \item $SS$ \\

    A two-digit seconds value, with leading zero if needed to make two digits.

  \item $.SSS$... \\

    A zero or more digit fractional seconds value.

\end{itemize}

  Whitespace is generally not required between terms in the first
format.  A hemisphere token may only occur once.
Degrees/minutes/seconds numbers need not be in that order, if units
are given after each number.  If a 5-integer-digit $DDDMM$[$.MMM$...]
format is given and the degree value is out of range (e.g. more than
90 degrees latitude), it is interpreted as a $DMMSS$[$.SSS$...] value
instead.  To force $DDDMMSS$[$.SSS$...] for small numbers, pad with
leading zeros to 6 or 7 digits.

\EXAMPLE

\begin{verbatim}
insert into geotest(lat, lon)
  values(parselatitude('54d 40m 10"'),
         parselongitude('W90 10.2'));
\end{verbatim}

\CAVEATS

An invalid or unparseable latitude or longitude value will return
\verb`NaN` (Not a Number).  Extra unparsed/unparsable text may be
allowed (and ignored) after the coordinate in most instances.
Out-of-range values (e.g. latitudes greater than 90 degrees) are
accepted; it is up to the caller to bounds-check the result.  The
\verb`parselatitude` and \verb`parselongitude` SQL functions were
added in version 6.00.1300132000 20110314.

% ----------------------------------------------------------------------------

\subsection{JSON functions}

The JSON functions allow for the manipulation of \verb`varchar` fields
and literals as JSON objects.

\subsubsection{JSON Path Syntax}
The JSON Path syntax is standard Javascript object access, using \verb`$` to
represent the entire document.  If the document is an object the path must
start \verb`$.`, and if an array \verb`$[`.

\subsubsection{JSON Field Syntax}
\label{jsoncomputedfield}
In addition to using the JSON functions it is possible to access elements
in a \verb`varchar` field that holds JSON as if it was a field itself.
This allows for creation of indexes, searching and sorting efficiently.
Arrays can also be fetched as \verb`strlst` to make use of those features,
e.g. \verb`SELECT Json.$.name FROM tablename WHERE 'SQL' IN Json.$.skills[*];`

% - - - - - - - - - - - - - - - - - - - - - - - - - - - - - - - - - - - - - -
\subsubsection{isjson}

\begin{verbatim}
  isjson(JsonDocument)
\end{verbatim}

The \verb`isjson` function returns 1 if the document is valid JSON,
0 otherwise.

\begin{verbatim}
isjson('{ "type" : 1 }'): 1
isjson('{}'): 1
isjson('json this is not'): 0
\end{verbatim}

% - - - - - - - - - - - - - - - - - - - - - - - - - - - - - - - - - - - - - -
\subsubsection{json\_format}

\begin{verbatim}
  json_format(JsonDocument, FormatOptions)
\end{verbatim}

The \verb`json_format` formats the \verb`JsonDocument` according to
\verb`FormatOptions`.  Multiple options can be provided either space
or comma separated.

Valid \verb`FormatOptions` are:
\begin{itemize}
\item COMPACT - remove all unnecessary whitespace
\item INDENT(N) - print the JSON with each object or array member on a new line,
indented by N spaces to show structure
\item SORT-KEYS - sort the keys in the object.  By default the order is preserved
\item EMBED - omit the enclosing \verb`{}` or \verb`[]` is using the snippet in another object
\item ENSURE\_ASCII - encode all Unicode characters outside the ASCII range
\item ENCODE\_ANY - if not a valid JSON document then encode into a JSON literal, e.g. to encode a string.
\item ESCAPE\_SLASH - escape forward slash \verb`/` as \verb`\/`
\end{itemize}

% - - - - - - - - - - - - - - - - - - - - - - - - - - - - - - - - - - - - - -
\subsubsection{json\_type}

\begin{verbatim}
  json_type(JsonDocument)
\end{verbatim}

The \verb`json_type` function returns the type of the JSON object or element.
Valid responses are:
\begin{itemize}
\item OBJECT
\item ARRAY
\item STRING
\item INTEGER
\item DOUBLE
\item NULL
\item BOOLEAN
\end{itemize}

Assuming a field \verb`Json` containing:
{
  "items" : [
    {
      "Num" : 1,
      "Text" : "The Name",
      "First" : true
    },
    {
      "Num" : 2.0,
      "Text" : "The second one",
      "First" : false
    }
    ,
    null
  ]
}
\begin{verbatim}
json_type(Json): OBJECT
json_type(Json.$.items[0]): OBJECT
json_type(Json.$.items): ARRAY
json_type(Json.$.items[0].Num): INTEGER
json_type(Json.$.items[1].Num): DOUBLE
json_type(Json.$.items[0].Text): STRING
json_type(Json.$.items[0].First): BOOLEAN
json_type(Json.$.items[2]): NULL
\end{verbatim}

% - - - - - - - - - - - - - - - - - - - - - - - - - - - - - - - - - - - - - -
\subsubsection{json\_value}

\begin{verbatim}
  json_value(JsonDocument, Path)
\end{verbatim}

The \verb`json_value` extracts the value identified by \verb`Path` from
\verb`JsonDocument`.  \verb`Path` is a varchar in the JSON Path Syntax.
This will return a scalar value.  If \verb`Path` refers to an array,
object, or invalid path no value is returned.

Assuming the same Json field from the previous examples:
\begin{verbatim}
json_value(Json, '$'):
json_value(Json, '$.items[0]'):
json_value(Json, '$.items'):
json_value(Json, '$.items[0].Num'): 1
json_value(Json, '$.items[1].Num'): 2.0
json_value(Json, '$.items[0].Text'): The Name
json_value(Json, '$.items[0].First'): true
json_value(Json, '$.items[2]'):
\end{verbatim}


% - - - - - - - - - - - - - - - - - - - - - - - - - - - - - - - - - - - - - -
\subsubsection{json\_query}

\begin{verbatim}
  json_query(JsonDocument, Path)
\end{verbatim}

The \verb`json_query` extracts the object or array identified by \verb`Path`
from \verb`JsonDocument`.  \verb`Path` is a varchar in the JSON Path Syntax.
This will return either an object or an array value.  If \verb`Path` refers
to a scalar no value is returned.

Assuming the same Json field from the previous examples:

\verb`json_query(Json, '$')`\\
\verb`---------------------`\\
\verb`{"items":[{"Num":1,"Text":"The Name","First":true},`\split
\verb`{"Num":2.0,"Text":"The second one","First":false},null]}`

\verb`json_query(Json, '$.items[0]')`\\
\verb`------------------------------`\\
\verb`{"Num":1,"Text":"The Name","First":true}`

\verb`json_query(Json, '$.items')`\\
\verb`---------------------------`\\
\verb`[{"Num":1,"Text":"The Name","First":true},`\split
\verb`{"Num":2.0,"Text":"The second one","First":false},null]`

The following will return an empty string as they refer to scalars
or non-existent keys.
\begin{verbatim}
json_query(Json, '$.items[0].Num')
json_query(Json, '$.items[1].Num')
json_query(Json, '$.items[0].Text')
json_query(Json, '$.items[0].First')
json_query(Json, '$.items[2]')
\end{verbatim}


% - - - - - - - - - - - - - - - - - - - - - - - - - - - - - - - - - - - - - -
\subsubsection{json\_modify}

\begin{verbatim}
  json_modify(JsonDocument, Path, NewValue)
\end{verbatim}

The \verb`json_modify` function returns a modified version of JsonDocument
with the key at Path replaced by NewValue.

If \verb`Path` starts with {\tt append\textvisiblespace } then the NewValue is appended to the
array referenced by {\tt Path}.  It is an error it {\tt Path} refers to anything
other than an array.

\begin{verbatim}
json_modify('{}', '$.foo', 'Some "quote"')
------------------------------------------
{"foo":"Some \"quote\""}

json_modify('{ "foo" : { "bar": [40, 42] } }', 'append $.foo.bar', 99)
----------------------------------------------------------------------
{"foo":{"bar":[40,42,99]}}

json_modify('{ "foo" : { "bar": [40, 42] } }', '$.foo.bar', 99)
---------------------------------------------------------------
{"foo":{"bar":99}}
\end{verbatim}
% - - - - - - - - - - - - - - - - - - - - - - - - - - - - - - - - - - - - - -
\subsubsection{json\_merge\_patch}

\begin{verbatim}
  json_merge_patch(JsonDocument, Patch)
\end{verbatim}

The \verb`json_merge_patch` function provides a way to patch a target JSON
document with another JSON document.  The patch function conforms to
(href=https://tools.ietf.org/html/rfc7386) RFC 7386

Keys in \verb`JsonDocument` are replaced if found in \verb`Patch`.  If the
value in \verb`Patch` is \verb`null` then the key will be removed in the
target document.

\begin{verbatim}
json_merge_patch('{"a":"b"}',          '{"a":"c"}'
--------------------------------------------------
{"a":"c"}

json_merge_patch('{"a": [{"b":"c"}]}', '{"a": [1]}'
---------------------------------------------------
{"a":[1]}

json_merge_patch('[1,2]',              '{"a":"b", "c":null}'
------------------------------------------------------------
{"a":"b"}

% - - - - - - - - - - - - - - - - - - - - - - - - - - - - - - - - - - - - - -
\subsubsection{json\_merge\_preserve}

\begin{verbatim}
  json_merge_preserve(JsonDocument, Patch)
\end{verbatim}

The \verb`json_merge_preserve` function provides a way to patch a target JSON
document with another JSON document while preserving the content that exists
in the target document.

Keys in \verb`JsonDocument` are merged if found in \verb`Patch`.  If the same
key exists in both the target and patch file the result will be an array with
the values from both target and patch.

If the
value in \verb`Patch` is \verb`null` then the key will be removed in the
target document.

\begin{verbatim}
json_merge_preserve('{"a":"b"}',          '{"a":"c"}'
-----------------------------------------------------
{"a":["b","c"]}

json_merge_preserve('{"a": [{"b":"c"}]}', '{"a": [1]}'
------------------------------------------------------
{"a":[{"b":"c"},1]}

json_merge_preserve('[1,2]',              '{"a":"b", "c":null}'
---------------------------------------------------------------
[1,2,{"a":"b","c":null}]

\end{verbatim}


% - - - - - - - - - - - - - - - - - - - - - - - - - - - - - - - - - - - - - -
% ----------------------------------------------------------------------------

\subsection{Other Functions}

% - - - - - - - - - - - - - - - - - - - - - - - - - - - - - - - - - - - - - -
\subsubsection{exec}

Execute an external command.  The syntax is
\begin{verbatim}
   exec(commandline[, INPUT[, INPUT[, INPUT[, INPUT]]]]);
\end{verbatim}

Allows execution of an external command.  The first argument is the
command to execute.  Any subsequent arguments are written to the standard
input of the process.  The standard output of the command is read as the
return from the function.

This function allows unlimited extensibility of Texis, although if a
particular function is being used often then it should be linked into the
Texis server to avoid the overhead of invoking another process.

For example this could be used to OCR text.  If you have a program which
will take an image filename on the command line, and return the text on
standard out you could issue SQL as follows:

\begin{verbatim}
     UPDATE     DOCUMENTS
     SET        TEXT = exec('ocr '+IMGFILE)
     WHERE      TEXT = '';
\end{verbatim}

Another example would be if you wanted to print envelopes from names and
addresses in a table you might use the following SQL:

\begin{verbatim}
     SELECT	exec('envelope ', FIRST_NAME+' '+LAST_NAME+'
     ', STREET + '
     ', CITY + ', ' + STATE + ' ' + ZIP)
     FROM ADDRESSES;
\end{verbatim}

Notice in this example the addition of spaces and line-breaks between the
fields.  Texis does not add any delimiters between fields or arguments
itself.

% - - - - - - - - - - - - - - - - - - - - - - - - - - - - - - - - - - - - - -
\subsubsection{mminfo}

This function lets you obtain Metamorph info.  You have the choice of
either just getting the portions of the document which were the hits, or
you can also get messages which describe each hit and subhits.

The SQL to use is as follows:

\begin{verbatim}
    SELECT mminfo(query,data[,nhits,[0,msgs]]) from TABLE
           [where CONDITION];
\end{verbatim}

\begin{description}
\item[query]

Query should be a string containing a metamorph query.

\item[data]

The text to search. May be literal data or a field from the table.

\item[nhits]

The maximum number of hits to return.  If it is 0, which
is the default, you will get all the hits.

\item[msgs]

An integer; controls what information is returned. A bit-wise OR of
any combination of the following values:
\begin{itemize}
\item 1 to get matches and offset/length information
\item 2 to suppress text from \verb`data` which matches; printed by default
\item 4 to get a count of hits (up to \verb`nhits`)
\item 8 to get the hit count in a numeric parseable format
\item 16 to get the offset/length in the original query of each search set
\end{itemize}

Set offset/length information (value 16) is of the form:
\begin{verbatim}
Set N offset/len in query: setoff setlen
\end{verbatim}
Where \verb`N` is the set number (starting with 1), \verb`setoff` is
the byte offset from the start of the query where set \verb`N` is,
and \verb`setlen` is the length of the set.
This information is available in version 5.01.1220640000 20080905
and later.

Hit offset/length information is of the form:
\begin{verbatim}
300 <Data from Texis> offset length suboff sublen [suboff sublen]..
301 End of Metamorph hit
\end{verbatim}
Where:
\begin{itemize}
\item offset is the offset within the data of the overall hit context
      (sentence, paragraph, etc...)
\item length is the length of the overall hit context
\item suboff is the offset within the hit of a matching term
\item sublen is the length of the matching term
\item suboff and sublen will be repeated for as many terms as are
      required to satisfy the query.
\end{itemize}

\end{description}


\begin{verbatim}
Example:
   select mminfo('power struggle @0 w/.',Body,0,0,1) inf from html
          where Title\Meta\Body like 'power struggle';
Would give something of the form:

300 <Data from Texis> 62 5 0 5
power
301 End of Metamorph hit
300 <Data from Texis> 2042 5 0 5
power
301 End of Metamorph hit
300 <Data from Texis> 2331 5 0 5
POWER
301 End of Metamorph hit
300 <Data from Texis> 2892 8 0 8
STRUGGLE
301 End of Metamorph hit
\end{verbatim}

% - - - - - - - - - - - - - - - - - - - - - - - - - - - - - - - - - - - - - -
\subsubsection{convert}
\label{convertSqlFunction}

The convert function allows you to change the type of an expression.
The syntax is
\begin{verbatim}
   CONVERT(expression, 'type-name'[, 'mode'])
\end{verbatim}
The type name should in general be in lower case.

This
can be useful in a number of situations.  Some cases where you might want
to use convert are
\begin{itemize}
\item  The display format for a different format is more useful.  For example
you might want to convert a field of type COUNTER to a DATE field, so you
can see when the record was inserted, for example:

\begin{verbatim}
    SELECT convert(id, 'date')
    FROM   LOG;
\end{verbatim}

\begin{smscreen}
\begin{verbatim}
    CONVERT(id, 'date')
    1995-01-27 22:43:48
\end{verbatim}
\end{smscreen}
\item  If you have an application which is expecting data in a particular
type you can use convert to make sure you will receive the correct type.
\end{itemize}


Caveat: Note that in Texis version 7 and later, \verb`convert()`ing
data from/to \verb`varbyte`/\verb`varchar` no longer converts the data
to/from hexadecimal by default (as was done in earlier versions) in
programs other than \verb`tsql`; it is now preserved as-is (though
truncated at nul for \verb`varchar`).  See the \verb`bintohex()` and
\verb`hextobin()` functions (p.~\pageref{bintohexSqlFunction}) for
hexadecimal conversion, and the \verb`hexifybytes` SQL property
(p.~\pageref{hexifybytesProperty}) for controlling automatic hex
conversion.

Also in Texis version 7 and later, an optional third argument may be
given to \verb`convert()`, which is a \verb`varchartostrlstsep` mode
value (p.~\pageref{`varchartostrlstsep'}).  This third argument may
only be supplied when converting to type \verb`strlst` or
\verb`varstrlst`.  It allows the separator character or mode to be
conveniently specified locally to the conversion, instead of having to
alter the global \verb`varchartostrlstsep` mode.


% - - - - - - - - - - - - - - - - - - - - - - - - - - - - - - - - - - - - - -
\subsubsection{seq}

Returns a sequence number.  The number can be initialized to any value,
and the increment can be defined for each call.  The syntax is:

\begin{verbatim}
	seq(increment [, init])
\end{verbatim}

If {\tt init} is given then the sequence number is initialized to that value,
which will be the value returned.  It is then incremented by {\tt increment}.
If {\tt init} is not specified then the current value will be retained.  The
initial value will be zero if {\tt init} has not been specified.

Examples of typical use:

\begin{verbatim}
     SELECT  NAME, seq(1)
     FROM    SYSTABLES
\end{verbatim}

The results are:

\begin{screen}
\begin{verbatim}
  NAME                seq(1)
 SYSTABLES               0
 SYSCOLUMNS              1
 SYSINDEX                2
 SYSUSERS                3
 SYSPERMS                4
 SYSTRIG                 5
 SYSMETAINDEX            6
\end{verbatim}
\end{screen}

\begin{verbatim}
     SELECT  seq(0, 100)
     FROM    SYSDUMMY;

     SELECT  NAME, seq(1)
     FROM    SYSTABLES
\end{verbatim}

The results are:

\begin{screen}
\begin{verbatim}
  seq(0, 100)
     100

  NAME                seq(1)
 SYSTABLES             100
 SYSCOLUMNS            101
 SYSINDEX              102
 SYSUSERS              103
 SYSPERMS              104
 SYSTRIG               105
 SYSMETAINDEX          106
\end{verbatim}
\end{screen}

% - - - - - - - - - - - - - - - - - - - - - - - - - - - - - - - - - - - - - -
\subsubsection{random}

Returns a random {\tt int}.  The syntax is:

\begin{verbatim}
	random(max [, seed])
\end{verbatim}

If {\tt seed} is given then the random number generator is seeded to
that value.  The random number generator will only be seeded once in
each session, and will be randomly seeded on the first call if no seed
is supplied.  The {\tt seed} parameter is ignored in the second and
later calls to {\tt random} in a process.

The returned number is always non-negative, and never larger than the
limit of the C lib's random number generator (typically either 32767
or 2147483647).  If {\tt max} is non-zero, then the returned number
will also be less than {\tt max}.

This function is typically used to either generate a random number for
later use, or to generate a random ordering of result records by adding
{\tt random} to the {\tt ORDER BY} clause.

Examples of typical use:

\begin{verbatim}
     SELECT  NAME, random(100)
     FROM    SYSTABLES
\end{verbatim}

The results might be:

\begin{screen}
\begin{verbatim}
  NAME                random(100)
 SYSTABLES               90
 SYSCOLUMNS              16
 SYSINDEX                94
 SYSUSERS                96
 SYSPERMS                 1
 SYSTRIG                 84
 SYSMETAINDEX            96
\end{verbatim}
\end{screen}

\begin{verbatim}
     SELECT  ENAME
     FROM    EMPLOYEE
     ORDER BY random(0);
\end{verbatim}

The results would be a list of employees in a random order.

% - - - - - - - - - - - - - - - - - - - - - - - - - - - - - - - - - - - - - -
\subsubsection{bintohex}
\label{bintohexSqlFunction}

  Converts a binary (\verb`varbyte`) value into a hexadecimal string.

\begin{verbatim}
    bintohex(varbyteData[, 'stream|pretty'])
\end{verbatim}

  A string (\verb`varchar`) hexadecimal representation of the
\verb`varbyteData` parameter is returned.  This can be useful to
visually examine binary data that may contain non-printable or nul
bytes.  The optional second argument is a comma-separated string of
any of the following flags:

\begin{itemize}
  \item \verb`stream`: Use the default output mode: a continuous
    stream of hexadecimal bytes, i.e. the same format that
    \verb`convert(varbyteData, 'varchar')` would have returned in
    Texis version 6 and earlier.

  \item \verb`pretty`: Return a ``pretty'' version of the data: print
    16 byte per line, space-separate the hexadecimal bytes, and print
    an ASCII dump on the right side.
\end{itemize}

The \verb`bintohex()` function was added in Texis version 7.  Caveat:
Note that in version 7 and later, \verb`convert()`ing data from/to
\verb`varbyte`/\verb`varchar` no longer converts the data to/from
hexadecimal by default (as was done in earlier versions) in programs
other than \verb`tsql`; it is now preserved as-is (though truncated at
nul for \verb`varchar`).  See the \verb`hexifybytes` SQL property
(p.~\pageref{hexifybytesProperty}) to change this.

% - - - - - - - - - - - - - - - - - - - - - - - - - - - - - - - - - - - - - -
\subsubsection{hextobin}

  Converts a hexadecimal stream to its binary representation.

\begin{verbatim}
    hextobin(hexString[, 'stream|pretty'])
\end{verbatim}

  The hexadecimal \verb`varchar` string \verb`hexString` is converted
to its binary representation, and the \verb`varbyte` result returned.
The optional second argument is a comma-separated string of any of the
following flags:

\begin{itemize}
  \item \verb`stream`: Only accept the \verb`stream` format of
    \verb`bintohex()`, i.e. a stream of hexadecimal bytes, the same
    format that \verb`convert(varbyteData, 'varchar')` would have
    returned in Texis version 6 and earlier.  Whitespace is
    acceptable, but only between (not within) hexadecimal bytes.
    Case-insensitive.  Non-conforming data will result in an error
    message and the function failing.

  \item \verb`pretty`: Accept either \verb`stream` or \verb`pretty`
    format data; if the latter, only the hexadecimal bytes are parsed
    (e.g. ASCII column is ignored).  Parsing is more liberal, but
    may be confused if the data deviates significantly from either
    format.
\end{itemize}

The \verb`hextobin()` function was added in Texis version 7.  Caveat:
Note that in version 7 and later, \verb`convert()`ing data from/to
\verb`varbyte`/\verb`varchar` no longer converts the data to/from
hexadecimal by default (as was done in earlier versions) in programs
other than \verb`tsql`; it is now preserved as-is (though truncated at
nul for \verb`varchar`).  See the \verb`hexifybytes` SQL property
(p.~\pageref{hexifybytesProperty}) to change this.


% - - - - - - - - - - - - - - - - - - - - - - - - - - - - - - - - - - - - - -
\subsubsection{identifylanguage}

  Tries to identify the predominant language of a given string.  By
  returning a probability in addition to the identified language, this
  function can also serve as a test of whether the given string is
  really natural-language text, or perhaps binary/encoded data
  instead.  Syntax:

\begin{verbatim}
    identifylanguage(text[, language[, samplesize]])
\end{verbatim}

  The return value is a two-element \verb`strlst`: a probability and a
  language code.  The probability is a value from {\tt 0.000} to
  {\tt 1.000} that the {\tt text} argument is composed in the
  language named by the returned language code.  The language code is
  a two-letter ISO-639-1 code.

  If an ISO-639-1 code is given for the optional {\tt language}
  argument, the probability for that particular language is returned,
  instead of for the highest-probability language of the
  known/built-in languages (currently {\tt de}, {\tt es}, {\tt fr},
  {\tt ja}, {\tt pl}, {\tt tr}, {\tt da}, {\tt en}, {\tt eu},
  {\tt it}, {\tt ko}, {\tt ru}).

  The optional third argument {\tt samplesize} is the initial integer
  size in bytes of the {\tt text} to sample when determining language;
  it defaults to 16384.  The {\tt samplesize} parameter was added in
  version 7.01.1382113000 20131018.

  Note that since a \verb`strlst` value is returned, the probability
  is returned as a \verb`strlst` element, not a \verb`double` value,
  and thus should be cast to \verb`double` during comparisons.  In
  Vortex with \verb`arrayconvert` on (the default), the return value
  will be automatically split into a two-element Vortex \verb`varchar`
  array.

  The \verb`identifylanguage()` function is experimental, and its
  behavior, syntax, name and/or existence are subject to change
  without notice.  Added in version 7.01.1381362000 20131009.

% - - - - - - - - - - - - - - - - - - - - - - - - - - - - - - - - - - - - - -
\subsubsection{lookup}
\label{lookup_SqlFunction}

By combining the \verb`lookup()` function with a \verb`GROUP BY`, a
column may be grouped into bins or ranges -- e.g. for price-range
grouping -- instead of distinct individual values.  Syntax:

\begin{verbatim}
    lookup(keys, ranges[, names])
\end{verbatim}

The {\tt keys} argument is one (or more, e.g. \verb`strlst`) values to
look up; each is searched for in the {\tt ranges} argument, which is
one (or more, e.g. \verb`strlst`) ranges.  All range(s) that the given
key(s) match will be returned.  If the {\tt names} argument is given,
the corresponding {\tt names} value(s) are returned instead; this
allows ranges to be renamed into human-readable values.  If {\tt
  names} is given, the number of its values must equal the number of
ranges.

Each range is a pair of values (lower and upper bounds) separated by
``{\tt ..}'' (two periods).  The range is optionally surrounded by
square (bound included) or curly (bound excluded) brackets.  E.g.:

\begin{verbatim}
[10..20}
\end{verbatim}

denotes the range 10 to 20: including 10 (``{\tt [}'') but not
  including (``{\tt \}}'') 20.  Both an upper and lower bracket must
  be given if either is present (though they need not be the same
  type).  The default if no brackets are given is to include the lower
  bound but exclude the upper bound; this makes consecutive ranges
  non-overlapping, if they have the same upper and lower bound and no
  brackets (e.g. ``0..10,10..20'').  Either bound may be omitted, in
  which case that bound is unlimited.  Each range's lower bound must
  not be greater than its upper bound, nor equal if either bound is
  exclusive.

If a {\tt ranges} value is not {\tt varchar}/{\tt char}, or does not
contain ``{\tt ..}'', its entire value is taken as a single inclusive
lower bound, and the exclusive upper bound will be the next {\tt
  ranges} value's lower bound (or unlimited if no next value).
E.g. the \verb`varint` lower-bound list:

\begin{verbatim}
0,10,20,30
\end{verbatim}

is equivalent to the \verb`strlst` range list:

\begin{verbatim}
[0..10},[10..20},[20..30},[30..]
\end{verbatim}

By using the \verb`lookup()` function in a \verb`GROUP BY`, a column may
be grouped into ranges.  For example, given a table {\tt Products}
with the following SKUs and \verb`float` prices:

\begin{verbatim}
    SKU    Price
    ------------
    1234   12.95
    1235    5.99
    1236   69.88
    1237   39.99
    1238   29.99
    1239   25.00
    1240   50.00
    1241   -2.00
    1242  499.95
    1243   19.95
    1244    9.99
    1245  125.00
\end{verbatim}

they may be grouped into price ranges (with most-products first) with
this SQL:

\begin{samepage}
{\small
\begin{verbatim}
SELECT   lookup(Price, convert('0..25,25..50,50..,', 'strlst', 'lastchar'),
     convert('Under $25,$25-49.99,$50 and up,', 'strlst', 'lastchar'))
       PriceRange, count(SKU) NumberOfProducts
FROM Products
GROUP BY lookup(Price, convert('0..25,25..50,50..,', 'strlst', 'lastchar'),
     convert('Under $25,$25-49.99,$50 and up,', 'strlst', 'lastchar'))
ORDER BY 2 DESC;
\end{verbatim}
}
\end{samepage}

or this Vortex:

\begin{samepage}
\begin{verbatim}
<$binValues =   "0..25"      "25..50"     "50..">
<$binDisplays = "Under $$25" "$$25-49.99" "$$50 and up">
<sql row "select lookup(Price, $binValues, $binDisplays) PriceRange,
              count(SKU) NumberOfProducts
          from Products
          group by lookup(Price, $binValues, $binDisplays)
          order by 2 desc">
  <fmt "%10s: %d\n" $PriceRange $NumberOfProducts>
</sql>
\end{verbatim}
\end{samepage}

which would give these results:

\begin{verbatim}
  PriceRange NumberOfProducts
------------+------------+
Under $25,              4
$50 and up,             4
$25-49.99,              3
                        1
\end{verbatim}

The trailing commas in {\tt PriceRange} values are due to them being
\verb`strlst` values, for possible multiple ranges.  Note the empty
{\tt PriceRange} for the fourth row: the -2 {\tt Price} matched
no ranges, and hence an empty {\tt PriceRange} was returned for
it.

\CAVEATS

The \verb`lookup()` function as described above was added in Texis
version 7.06.1528745000 20180611.

A different version of the \verb`lookup()` function was first added in
version 7.01.1386980000 20131213: it only took the second range syntax
variant (single lower bound); range values had to be in ascending
order (by {\tt keys} type); only the first matching range was
returned; and if a key did not match any range the first range was
returned.

\SEE

\verb`lookupCanonicalizeRanges`, \verb`lookupParseRange`

% - - - - - - - - - - - - - - - - - - - - - - - - - - - - - - - - - - - - - -
\subsubsection{lookupCanonicalizeRanges}
\label{lookupCanonicalizeRanges_SqlFunction}

The \verb`lookupCanonicalizeRanges()` function returns the canonical
version(s) of its {\tt ranges} argument, which is zero or more ranges
of the syntaxes acceptable to \verb`lookup()`
(p.~\pageref{lookup_SqlFunction}):

\begin{verbatim}
    lookupCanonicalizeRanges(ranges, keyType)
\end{verbatim}

The canonical version always includes both a lower and upper
inclusive/exclusive bracket/brace, both lower and upper bounds (unless
unlimited), the ``{\tt ..}'' range operator, and is independent of
other ranges that may be in the sequence.

The {\tt keyType} parameter is a \verb`varchar` string denoting the
SQL type of the key field that would be looked up in the given
range(s).  This ensures that comparisons are done correctly.  E.g. for
a \verb`strlst` range list of ``{\tt 0,500,1000}'', {\tt keyType}
should be ``{\tt integer}'', so that ``{\tt 500}'' is not compared
alphabetically with ``{\tt 1000}'' and considered invalid (greater than).

This function can be used to verify the syntax of a range, or to
transform it into a standard form for \verb`lookupParseRange()`
(p.~\pageref{lookupParseRange_SqlFunction}).

\CAVEATS

For an implicit-upper-bound range, the upper bound is determined by
the {\em next} range's lower bound.  Thus the full list of ranges (if
multiple) should be given to \verb`lookupCanonicalizeRanges()` -- even
if only one range needs to be canonicalized -- so that each range gets
its proper bounds.

The \verb`lookupCanonicalizeRanges()` function was added in version
7.06.1528837000 20180612.  The {\tt keyType} parameter was added
in version 7.06.1535500000 20180828.

\SEE

\verb`lookup`, \verb`lookupParseRange`

% - - - - - - - - - - - - - - - - - - - - - - - - - - - - - - - - - - - - - -
\subsubsection{lookupParseRange}
\label{lookupParseRange_SqlFunction}

The \verb`lookupParseRange()` function parses a single
\verb`lookup()`-style range into its constituent parts, returning them
as strings in one \verb`strlst` value.  This can be used by Vortex
scripts to edit a range.  Syntax:

\begin{verbatim}
    lookupParseRange(range, parts)
\end{verbatim}

The {\tt parts} argument is zero or more of the following part tokens
as strings:

\begin{itemize}
  \item \verb`lowerInclusivity`: Returns the inclusive/exclusive operator
    for the lower bound, e.g. ``{\tt \{}'' or ``{\tt [}''
  \item \verb`lowerBound`: Returns the lower bound
  \item \verb`rangeOperator`: Returns the range operator, e.g. ``{\tt ..}''
  \item \verb`upperBound`: Returns the upper bound
  \item \verb`upperInclusivity`: Returns the inclusive/exclusive operator
    for the upper bound, e.g. ``{\tt \}}'' or ``{\tt ]}''
\end{itemize}

If a requested part is not present, an empty string is returned for
that part.  The concatenation of the above listed parts, in the above
order, should equal the given range.  Non-string range arguments are
not supported.

The \verb`lookupParseRange()` function was added in
version 7.06.1528837000 20180612.

\EXAMPLE

\begin{verbatim}
    lookupParseRange('10..20', 'lowerInclusivity')
\end{verbatim}

would return a single empty-string \verb`strlst`, as there is no
lower-bound inclusive/exclusive operator in the range ``{\tt 10..20}''.

\begin{verbatim}
    lookupParseRange('10..20', 'lowerBound')
\end{verbatim}

would return a \verb`strlst` with the single value ``{\tt 10}''.

\CAVEATS

For an implicit-upper-bound range, the upper bound is determined by
the {\em next} range's lower bound.  Since \verb`lookupParseRange()`
only takes one range, passing such a range to it may result in an
incorrect (unlimited) upper bound.  Thus the full list of ranges (if
multiple) should always be given to \verb`lookupCanonicalizeRanges()`
first, and only then the desired canonicalized range passed to
\verb`lookupParseRange()`.

\SEE

\verb`lookup`, \verb`lookupCanonicalizeRanges`

% - - - - - - - - - - - - - - - - - - - - - - - - - - - - - - - - - - - - - -
\subsubsection{hasFeature}

Returns 1 if given feature is supported, 0 if not (or unknown).
The syntax is:

\begin{verbatim}
	hasFeature(feature)
\end{verbatim}

where {\tt feature} is one of the following \verb`varchar` tokens:

\begin{itemize}
  \item \verb`RE2`  For RE2 regular expression support in REX
\end{itemize}

This function is typically used in Vortex scripts to test if a feature
is supported with the current version of Texis, and if not, to work
around that fact if possible.  For example:

\begin{verbatim}
     <if hasFeature( "RE2" ) = 1>
       ... proceed with RE2 expressions ...
     <else>
       ... use REX instead ...
     </if>
\end{verbatim}

Note that in a Vortex script that does not support {\tt hasFeature()}
itself, such an \verb`<if>` statement will still compile and run,
but will be false (with an error message).

Added in version 7.06.1481662000 20161213.  Some feature tokens were
added in later versions.

% - - - - - - - - - - - - - - - - - - - - - - - - - - - - - - - - - - - - - -
\subsubsection{ifNull}

Substitute another value for NULL values.  Syntax:

\begin{verbatim}
   ifNull(testVal, replaceVal)
\end{verbatim}

  If \verb`testVal` is a SQL NULL value, then \verb`replaceVal` (cast
to the type of \verb`testVal`) is returned; otherwise \verb`testVal`
is returned.  This function can be used to ensure that NULL value(s)
in a column are replaced with a non-NULL value, if a non-NULL value
is required:

\begin{verbatim}
    SELECT ifNull(myColumn, 'Unknown') FROM myTable;
\end{verbatim}

Added in version 7.02.1405382000 20140714.  Note that SQL NULL is not
yet fully supported in Texis (including in tables).  See also
\verb`isNull`.

% - - - - - - - - - - - - - - - - - - - - - - - - - - - - - - - - - - - - - -
\subsubsection{isNull}

Tests a value, and returns a \verb`long` value of 1 if NULL, 0 if not.
Syntax:

\begin{verbatim}
   isNull(testVal)
\end{verbatim}

\begin{verbatim}
    SELECT isNull(myColumn) FROM myTable;
\end{verbatim}

Added in version 7.02.1405382000 20140714.  Note that SQL NULL is
not yet fully supported in Texis (including in tables).  Also note
that Texis \verb`isNull` behavior differs from some other SQL
implementations; see also \verb`ifNull`.

% - - - - - - - - - - - - - - - - - - - - - - - - - - - - - - - - - - - - - -
\subsubsection{xmlTreeQuickXPath}

Extracts information from an XML document.

\begin{verbatim}
	xmlTreeQuickXPath(string xmlRaw, string xpathQuery
        [, string[] xmlns)
\end{verbatim}

Parameters:
\begin{itemize}
\item \verb`xmlRaw` - the plain text of the xml document you want to
  extract information from
\item \verb`xpathQuery` - the XPath expression that identifies the
  nodes you want to extract the data from
\item \verb`xmlns` {\em(optional)} - an array of \verb`prefix=URI`
  namespaces to use in the XPath query
\end{itemize}

Returns:
\begin{itemize}
\item String values of the node from the XML document \verb`xmlRaw`
  that match \verb`xpathQuery`
\end{itemize}

\verb`xmlTreeQuickXPath` allows you to easily extract information from
an XML document in a one-shot function.  It is intended to be used in
SQL statements to extract specific information from a field that
contains XML data.

It is essentially a one statement version of the following:
\begin{verbatim}
    <$doc = (xmlTreeNewDocFromString($xmlRaw))>
    <$xpath = (xmlTreeNewXPath($doc))>
    <$nodes = (xmlTreeXPathExecute($xpathQuery))>
    <loop $nodes>
        <$ret = (xmlTreeGetAllContent($nodes))>
        <$content = $content $ret>
    </loop>
\end{verbatim}

\EXAMPLE
if the \verb`xmlData` field of a table has content like this:
\begin{verbatim}
<extraInfo>
    <price>8.99</price>
    <author>John Doe</author>
    <isbn>978-0-06-051280-4</isbn>
</extraInfo>
\end{verbatim}

Then the following SQL statement will match that row:
\begin{verbatim}
SELECT * from myTable where xmlTreeQuickXPath(data,
'/extraInfo/author') = 'John Doe'
\end{verbatim}
