\subsection{JSON functions}

The JSON functions allow for the manipulation of \verb`varchar` fields
and literals as JSON objects.

\subsubsection{JSON Path Syntax}
The JSON Path syntax is standard Javascript object access, using \verb`$` to
represent the entire document.  If the document is an object the path must
start \verb`$.`, and if an array \verb`$[`.

\subsubsection{JSON Field Syntax}
\label{jsoncomputedfield}
In addition to using the JSON functions it is possible to access elements
in a \verb`varchar` field that holds JSON as if it was a field itself.
This allows for creation of indexes, searching and sorting efficiently.
Arrays can also be fetched as \verb`strlst` to make use of those features,
e.g. \verb`SELECT Json.$.name FROM tablename WHERE 'SQL' IN Json.$.skills[*];`

% - - - - - - - - - - - - - - - - - - - - - - - - - - - - - - - - - - - - - -
\subsubsection{isjson}

\begin{verbatim}
  isjson(JsonDocument)
\end{verbatim}

The \verb`isjson` function returns 1 if the document is valid JSON,
0 otherwise.

\begin{verbatim}
isjson('{ "type" : 1 }'): 1
isjson('{}'): 1
isjson('json this is not'): 0
\end{verbatim}

% - - - - - - - - - - - - - - - - - - - - - - - - - - - - - - - - - - - - - -
\subsubsection{json\_format}

\begin{verbatim}
  json_format(JsonDocument, FormatOptions)
\end{verbatim}

The \verb`json_format` formats the \verb`JsonDocument` according to
\verb`FormatOptions`.  Multiple options can be provided either space
or comma separated.

Valid \verb`FormatOptions` are:
\begin{itemize}
\item COMPACT - remove all unnecessary whitespace
\item INDENT(N) - print the JSON with each object or array member on a new line,
indented by N spaces to show structure
\item SORT-KEYS - sort the keys in the object.  By default the order is preserved
\item EMBED - omit the enclosing \verb`{}` or \verb`[]` is using the snippet in another object
\item ENSURE\_ASCII - encode all Unicode characters outside the ASCII range
\item ENCODE\_ANY - if not a valid JSON document then encode into a JSON literal, e.g. to encode a string.
\item ESCAPE\_SLASH - escape forward slash \verb`/` as \verb`\/`
\end{itemize}

% - - - - - - - - - - - - - - - - - - - - - - - - - - - - - - - - - - - - - -
\subsubsection{json\_type}

\begin{verbatim}
  json_type(JsonDocument)
\end{verbatim}

The \verb`json_type` function returns the type of the JSON object or element.
Valid responses are:
\begin{itemize}
\item OBJECT
\item ARRAY
\item STRING
\item INTEGER
\item DOUBLE
\item NULL
\item BOOLEAN
\end{itemize}

Assuming a field \verb`Json` containing:
{
  "items" : [
    {
      "Num" : 1,
      "Text" : "The Name",
      "First" : true
    },
    {
      "Num" : 2.0,
      "Text" : "The second one",
      "First" : false
    }
    ,
    null
  ]
}
\begin{verbatim}
json_type(Json): OBJECT
json_type(Json.$.items[0]): OBJECT
json_type(Json.$.items): ARRAY
json_type(Json.$.items[0].Num): INTEGER
json_type(Json.$.items[1].Num): DOUBLE
json_type(Json.$.items[0].Text): STRING
json_type(Json.$.items[0].First): BOOLEAN
json_type(Json.$.items[2]): NULL
\end{verbatim}

% - - - - - - - - - - - - - - - - - - - - - - - - - - - - - - - - - - - - - -
\subsubsection{json\_value}

\begin{verbatim}
  json_value(JsonDocument, Path)
\end{verbatim}

The \verb`json_value` extracts the value identified by \verb`Path` from
\verb`JsonDocument`.  \verb`Path` is a varchar in the JSON Path Syntax.
This will return a scalar value.  If \verb`Path` refers to an array,
object, or invalid path no value is returned.

Assuming the same Json field from the previous examples:
\begin{verbatim}
json_value(Json, '$'):
json_value(Json, '$.items[0]'):
json_value(Json, '$.items'):
json_value(Json, '$.items[0].Num'): 1
json_value(Json, '$.items[1].Num'): 2.0
json_value(Json, '$.items[0].Text'): The Name
json_value(Json, '$.items[0].First'): true
json_value(Json, '$.items[2]'):
\end{verbatim}


% - - - - - - - - - - - - - - - - - - - - - - - - - - - - - - - - - - - - - -
\subsubsection{json\_query}

\begin{verbatim}
  json_query(JsonDocument, Path)
\end{verbatim}

The \verb`json_query` extracts the object or array identified by \verb`Path`
from \verb`JsonDocument`.  \verb`Path` is a varchar in the JSON Path Syntax.
This will return either an object or an array value.  If \verb`Path` refers
to a scalar no value is returned.

Assuming the same Json field from the previous examples:

\verb`json_query(Json, '$')`\\
\verb`---------------------`\\
\verb`{"items":[{"Num":1,"Text":"The Name","First":true},`\split
\verb`{"Num":2.0,"Text":"The second one","First":false},null]}`

\verb`json_query(Json, '$.items[0]')`\\
\verb`------------------------------`\\
\verb`{"Num":1,"Text":"The Name","First":true}`

\verb`json_query(Json, '$.items')`\\
\verb`---------------------------`\\
\verb`[{"Num":1,"Text":"The Name","First":true},`\split
\verb`{"Num":2.0,"Text":"The second one","First":false},null]`

The following will return an empty string as they refer to scalars
or non-existent keys.
\begin{verbatim}
json_query(Json, '$.items[0].Num')
json_query(Json, '$.items[1].Num')
json_query(Json, '$.items[0].Text')
json_query(Json, '$.items[0].First')
json_query(Json, '$.items[2]')
\end{verbatim}


% - - - - - - - - - - - - - - - - - - - - - - - - - - - - - - - - - - - - - -
\subsubsection{json\_modify}

\begin{verbatim}
  json_modify(JsonDocument, Path, NewValue)
\end{verbatim}

The \verb`json_modify` function returns a modified version of JsonDocument
with the key at Path replaced by NewValue.

If \verb`Path` starts with {\tt append\textvisiblespace } then the NewValue is appended to the
array referenced by {\tt Path}.  It is an error it {\tt Path} refers to anything
other than an array.

\begin{verbatim}
json_modify('{}', '$.foo', 'Some "quote"')
------------------------------------------
{"foo":"Some \"quote\""}

json_modify('{ "foo" : { "bar": [40, 42] } }', 'append $.foo.bar', 99)
----------------------------------------------------------------------
{"foo":{"bar":[40,42,99]}}

json_modify('{ "foo" : { "bar": [40, 42] } }', '$.foo.bar', 99)
---------------------------------------------------------------
{"foo":{"bar":99}}
\end{verbatim}
% - - - - - - - - - - - - - - - - - - - - - - - - - - - - - - - - - - - - - -
\subsubsection{json\_merge\_patch}

\begin{verbatim}
  json_merge_patch(JsonDocument, Patch)
\end{verbatim}

The \verb`json_merge_patch` function provides a way to patch a target JSON
document with another JSON document.  The patch function conforms to
(href=https://tools.ietf.org/html/rfc7386) RFC 7386

Keys in \verb`JsonDocument` are replaced if found in \verb`Patch`.  If the
value in \verb`Patch` is \verb`null` then the key will be removed in the
target document.

\begin{verbatim}
json_merge_patch('{"a":"b"}',          '{"a":"c"}'
--------------------------------------------------
{"a":"c"}

json_merge_patch('{"a": [{"b":"c"}]}', '{"a": [1]}'
---------------------------------------------------
{"a":[1]}

json_merge_patch('[1,2]',              '{"a":"b", "c":null}'
------------------------------------------------------------
{"a":"b"}

% - - - - - - - - - - - - - - - - - - - - - - - - - - - - - - - - - - - - - -
\subsubsection{json\_merge\_preserve}

\begin{verbatim}
  json_merge_preserve(JsonDocument, Patch)
\end{verbatim}

The \verb`json_merge_preserve` function provides a way to patch a target JSON
document with another JSON document while preserving the content that exists
in the target document.

Keys in \verb`JsonDocument` are merged if found in \verb`Patch`.  If the same
key exists in both the target and patch file the result will be an array with
the values from both target and patch.

If the
value in \verb`Patch` is \verb`null` then the key will be removed in the
target document.

\begin{verbatim}
json_merge_preserve('{"a":"b"}',          '{"a":"c"}'
-----------------------------------------------------
{"a":["b","c"]}

json_merge_preserve('{"a": [{"b":"c"}]}', '{"a": [1]}'
------------------------------------------------------
{"a":[{"b":"c"},1]}

json_merge_preserve('[1,2]',              '{"a":"b", "c":null}'
---------------------------------------------------------------
[1,2,{"a":"b","c":null}]

\end{verbatim}


% - - - - - - - - - - - - - - - - - - - - - - - - - - - - - - - - - - - - - -
% ----------------------------------------------------------------------------
