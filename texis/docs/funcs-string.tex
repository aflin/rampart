\subsection{String Functions}

% - - - - - - - - - - - - - - - - - - - - - - - - - - - - - - - - - - - - - -
\subsubsection{abstract}

Generate an abstract of a given portion of text.  The syntax is
\begin{verbatim}
   abstract(text[, maxsize[, style[, query]]])
\end{verbatim}

  The abstract will be less than \verb`maxsize` characters long, and
will attempt to end at a word boundary.  If \verb`maxsize` is not
specified (or is less than or equal to 0) then a default size of 230
characters is used.

  The \verb`style` argument is a string or integer, and allows a
choice between several different ways of creating the abstract.
Note that some of these styles require the \verb`query` argument as
well, which is a Metamorph query to look for:

\begin{itemize}
  \item \verb`dumb` (0) \\
    Start the abstract at the top of the document.

  \item \verb`smart` (1) \\
    This style will look for the first meaningful chunk of text,
    skipping over any headers at the top of the text.  This is the
    default if neither \verb`style` nor \verb`query` is given.

  \item \verb`querysingle` (2) \\
    Center the abstract contiguously on the best occurence of
    \verb`query` in the document.

  \item \verb`querymultiple` (3) \\
    Like \verb`querysingle`, but also break up the abstract into
    multiple sections (separated with ``\verb`...`'') if needed to
    help ensure all terms are visible.  Also take care with URLs to
    try to show the start and end.

  \item \verb`querybest` \\
    An alias for the best available query-based style; currently the
    same as \verb`querymultiple`.  Using \verb`querybest` in a script
    ensures that if improved styles become available in future
    releases, the script will automatically ``upgrade'' to the best
    style.
\end{itemize}

  If no \verb`query` is given for the \verb`query`$...$ modes, they
fall back to \verb`dumb` mode.  If a \verb`query` is given with a {\em
non-}\verb`query`$...$ mode (\verb`dumb`/\verb`smart`), the mode is
promoted to \verb`querybest`.  The current locale and index
expressions also have an effect on the abstract in the
\verb`query`$...$ modes, so that it more closely reflects an
index-obtained hit.

\begin{verbatim}
     SELECT     abstract(STORY, 0, 1, 'power struggle')
     FROM       ARTICLES
     WHERE      ARTID = 'JT09115' ;
\end{verbatim}

% - - - - - - - - - - - - - - - - - - - - - - - - - - - - - - - - - - - - - -
\subsubsection{text2mm}

Generate \verb`LIKEP` query.  The syntax is
\begin{verbatim}
   text2mm(text[, maxwords])
\end{verbatim}

This function will take a text expression, and produce a list of words
that can be given to \verb`LIKER` or \verb`LIKEP` to find similar
documents.  \verb`text2mm` takes an optional second argument which
specifies how many words should be returned.  If this is not specified
then 10 words are returned.  Most commonly \verb`text2mm` will be given the
name of a field.  If it is an \verb`indirect` field you will need to call
\verb|fromfile| as shown below:

\begin{verbatim}
     SELECT     text2mm(fromfile(FILENAME))
     FROM       DOCUMENTS
     WHERE      DOCID = 'JT09115' ;
\end{verbatim}

You may also call it as \verb`texttomm()` instead of \verb`text2mm()` .

% - - - - - - - - - - - - - - - - - - - - - - - - - - - - - - - - - - - - - -
\subsubsection{keywords}

Generate list of keywords.  The syntax is
\begin{verbatim}
   keywords(text[, maxwords])
\end{verbatim}

{\tt keywords} is similar to {\tt text2mm} but produces a list of
phrases, with a linefeed separating them.  The difference between
{\tt text2mm} and {\tt keywords} is that {\tt keywords} will maintain
the phrases.
{\tt keywords} also takes an optional second
argument which indicates how many words or phrases should be returned.

% - - - - - - - - - - - - - - - - - - - - - - - - - - - - - - - - - - - - - -
\subsubsection{length}

Returns the length in characters of a \verb`char` or \verb`varchar`
expression, or number of strings/items in other types.  The syntax is
\begin{verbatim}
  length(value[, mode])
\end{verbatim}

For example:

\begin{verbatim}
     SELECT  NAME, length(NAME)
     FROM    SYSTABLES
\end{verbatim}

The results are:

\begin{screen}
\begin{verbatim}
  NAME                length(NAME)
 SYSTABLES               9
 SYSCOLUMNS             10
 SYSINDEX                8
 SYSUSERS                8
 SYSPERMS                8
 SYSTRIG                 7
 SYSMETAINDEX           12
\end{verbatim}
\end{screen}

  The optional \verb`mode` argument is a
\verb`stringcomparemode`-style compare mode to use; see the Vortex manual
on {\tt <apicp stringcomparemode>} for details on syntax and the
default.  If \verb`mode` is not given, the current {\tt apicp
stringcomparemode} is used.  Currently the only pertinent \verb`mode`
flag is ``{\tt iso-8859-1}'', which determines whether to interpret
\verb`value` as ISO-8859-1 or UTF-8.  This can alter how many characters long
the string appears to be, as UTF-8 characters are variable-byte-sized,
whereas ISO-8859-1 characters are always mono-byte.  The \verb`mode`
argument was added in version 6.

  In version 5.01.1226622000 20081113 and later, if given a \verb`strlst`
type \verb`value`, \verb`length()` returns the number of string values
in the list.  For other types, it returns the number of values, e.g.
for \verb`varint` it returns the number of integer values.

% - - - - - - - - - - - - - - - - - - - - - - - - - - - - - - - - - - - - - -
\subsubsection{lower}

Returns the text expression with all letters in lower-case. The syntax is
\begin{verbatim}
  lower(text[, mode])
\end{verbatim}

For example:

\begin{verbatim}
     SELECT  NAME, lower(NAME)
     FROM    SYSTABLES
\end{verbatim}

The results are:

\begin{screen}
\begin{verbatim}
  NAME                lower(NAME)
 SYSTABLES            systables
 SYSCOLUMNS           syscolumns
 SYSINDEX             sysindex
 SYSUSERS             sysusers
 SYSPERMS             sysperms
 SYSTRIG              systrig
 SYSMETAINDEX         sysmetaindex
\end{verbatim}
\end{screen}

Added in version 2.6.932060000.

  The optional \verb`mode` argument is a string-folding mode in the
same format as {\tt <apicp stringcomparemode>}; see the Vortex manual
for details on the syntax and default.  If \verb`mode` is unspecified,
the current {\tt apicp stringcomparemode} setting -- with ``{\tt +lowercase}''
aded -- is used.  The \verb`mode` argument was added in version 6.

% - - - - - - - - - - - - - - - - - - - - - - - - - - - - - - - - - - - - - -
\subsubsection{upper}

Returns the text expression with all letters in upper-case. The sytax is
\begin{verbatim}
  upper(text[, mode])
\end{verbatim}

For example:

\begin{verbatim}
     SELECT  NAME, upper(NAME)
     FROM    SYSTABLES
\end{verbatim}

The results are:

\begin{screen}
\begin{verbatim}
  NAME                upper(NAME)
 SYSTABLES            SYSTABLES
 SYSCOLUMNS           SYSCOLUMNS
 SYSINDEX             SYSINDEX
 SYSUSERS             SYSUSERS
 SYSPERMS             SYSPERMS
 SYSTRIG              SYSTRIG
 SYSMETAINDEX         SYSMETAINDEX
\end{verbatim}
\end{screen}

Added in version 2.6.932060000.

  The optional \verb`mode` argument is a string-folding mode in the
same format as {\tt <apicp stringcomparemode>}; see the Vortex manual
for details on the syntax and default.  If \verb`mode` is unspecified,
the current {\tt apicp stringcomparemode} setting -- with ``{\tt
+uppercase}'' added -- is used.  The \verb`mode` argument was added in
version 6.

% - - - - - - - - - - - - - - - - - - - - - - - - - - - - - - - - - - - - - -
\subsubsection{initcap}

Capitalizes text.  The syntax is
\begin{verbatim}
  initcap(text[, mode])
\end{verbatim}

Returns the text expression with the first letter of each word in
title case (i.e. upper case), and all other letters in lower-case.
For example:

\begin{verbatim}
     SELECT  NAME, initcap(NAME)
     FROM    SYSTABLES
\end{verbatim}

The results are:

\begin{screen}
\begin{verbatim}
  NAME                initcap(NAME)
 SYSTABLES            Systables
 SYSCOLUMNS           Syscolumns
 SYSINDEX             Sysindex
 SYSUSERS             Sysusers
 SYSPERMS             Sysperms
 SYSTRIG              Systrig
 SYSMETAINDEX         Sysmetaindex
\end{verbatim}
\end{screen}

Added in version 2.6.932060000.

  The optional \verb`mode` argument is a string-folding mode in the
same format as {\tt <apicp stringcomparemode>}; see the Vortex manual
for details on the syntax and default.  If \verb`mode` is unspecified,
the current {\tt apicp stringcomparemode} setting -- with ``{\tt +titlecase}''
added -- is used.  The \verb`mode` argument was added in version 6.

% - - - - - - - - - - - - - - - - - - - - - - - - - - - - - - - - - - - - - -
\subsubsection{sandr}

Search and replace text.
\begin{verbatim}
   sandr(search, replace, text)
\end{verbatim}

Returns the text expression with the search REX expression replaced
with the replace expression.  See the REX documentation and the
Vortex sandr function documentation for complete syntax of the search
and replace expressions.

\begin{verbatim}
     SELECT  NAME, sandr('>>=SYS=', 'SYSTEM TABLE ', NAME) DESC
     FROM    SYSTABLES
\end{verbatim}

The results are:

\begin{screen}
\begin{verbatim}
  NAME                DESC
 SYSTABLES            SYSTEM TABLE TABLES
 SYSCOLUMNS           SYSTEM TABLE COLUMNS
 SYSINDEX             SYSTEM TABLE INDEX
 SYSUSERS             SYSTEM TABLE USERS
 SYSPERMS             SYSTEM TABLE PERMS
 SYSTRIG              SYSTEM TABLE TRIG
 SYSMETAINDEX         SYSTEM TABLE METAINDEX
\end{verbatim}
\end{screen}

Added in version 3.0

% - - - - - - - - - - - - - - - - - - - - - - - - - - - - - - - - - - - - - -
\subsubsection{separator}

Returns the separator character from its \verb`strlst` argument,
as a \verb`varchar` string:

\begin{verbatim}
   separator(strlstValue)
\end{verbatim}

  This can be used in situations where the \verb`strlstValue` argument
may have a nul character as the separator, in which case simply
converting \verb`strlstValue` to \verb`varchar` and looking at the
last character would be incorrect.  Added in version 5.01.1226030000
20081106.

% - - - - - - - - - - - - - - - - - - - - - - - - - - - - - - - - - - - - - -
\subsubsection{stringcompare}

  Compares its string (\verb`varchar`) arguments \verb`a` and
\verb`b`, returning -1 if \verb`a` is less than \verb`b`, 0 if they
are equal, or 1 if \verb`a` is greater than \verb`b`:

\begin{verbatim}
  stringcompare(a, b[, mode])
\end{verbatim}

  The strings are compared using the optional \verb`mode` argument,
which is a string-folding mode in the same format as
{\tt <apicp stringcomparemode>}; see the Vortex manual for details on
the syntax and default.  If \verb`mode` is unspecified, the current
{\tt apicp stringcomparemode} setting is used.  Function added
in version 6.00.1304108000 20110429.

% - - - - - - - - - - - - - - - - - - - - - - - - - - - - - - - - - - - - - -
\subsubsection{stringformat}

  Returns its arguments formatted into a string (\verb`varchar`), like
the equivalent Vortex function \verb`<strfmt>` (based on the C
function \verb`sprintf()`):

\begin{verbatim}
  stringformat(format[, arg[, arg[, arg[, arg]]]])
\end{verbatim}

  The \verb`format` argument is a \verb`varchar` string that describes
how to print the following argument(s), if any.  See the Vortex manual
for \verb`<strfmt>` for details.  Added in version 6.00.1300386000
20110317.

% ----------------------------------------------------------------------------
