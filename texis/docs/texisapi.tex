\documentstyle[epiman,12pt]{book}
\begin{document}           % End of preamble and beginning of text.

\title{Draft Network Texis API Documentation}
\tableofcontents
%\PART{The Texis API}
\chapter{Overview}
\NAME{Texis Client Functions}
\SYNOPSIS
{\bf Server Connection Management }
\begin{verbatim}
SERVER *openserver(char *hostname,char *port)
SERVER *closeserver(SERVER *serverhandle)
int     serveruser(char *username);
int     servergroup(char *groupname);
int     serverpass(char *password);
\end{verbatim}

{\bf Hit registration }
\begin{verbatim}
int n_regtexiscb(SERVER *se,void *usr,func cb);
    func cb(void *usr,TEXIS *tx,FLDLST *fl);
\end{verbatim}

{\bf Hit information }
\begin{verbatim}
SRCHLST *n_getsrchlst(SERVER *se,TEXIS *tx);
SRCHLST *n_freesrchlst(SERVER *se,SRCHLST *sl);
SRCHI   *n_srchinfo(SERVER *se,SRCH *sr,int i);
SRCHI   *n_freesrchi(SERVER *se,SRCHI *si);
XPMI    *n_xpminfo(SERVER *se,SRCH *sr,int i);
XPMI    *n_freexpmi(SERVER *se,XPMI *xi);
\end{verbatim}

{\bf Object manipulation }
\begin{verbatim}
BLOB *newblob(char *database,char *table,char *buf,int n);
BLOB *putblob(BLOB *bl,char *buf,int n);
BLOB *closeblob(BLOB *bl);
int   getblob(BLOB *bl,char **buf,int *n);
char *newindirect(char *database,char *table,char *filename);
\end{verbatim}

{\bf Texis control parameters }
\begin{verbatim}
int n_setdatabase(SERVER *se,char *database);
int n_getdatabase(SERVER *se,char *database);
\end{verbatim}

{\bf SQL interface }
\begin{verbatim}
int n_texis(SERVER *se,char *queryformat,...);
\end{verbatim}

\DESCRIPTION
The Texis API provides the client program with an easy to use interface to
ant Texis server without regard to machine type.  This is the preferred
interface and completely unrelated to the ODBC interface which is
documented elswhere.

The programmer is {\em strongly} encouraged to play with the example
progams provided with the package before attempting their own
implementation.

\begin{description}
\item[Server Connection Management]

    These functions allow you to connect and disconnect from a server.
    The return value from \verb`openserver()` is a \verb`SERVER *`, and
    will be used as the first parameter in all subsequent calls to
    the Server.

\item[Hit registration]

    These functions tell the server which client function to ``{\em call back}''
    when it has located information pertinent to the query.

\item[Hit information]

    These functions allow you to obtain detailed information about
    any Metamorph queries that may have been used in the query.

\item[Object manipulation]

    These functions allow you to manipulate indirect and blob fields
    within a Texis database.

\item[Texis control parameters]

    This set of functions is for getting and changing the various
    operational parameters that define how a remote Texis performs.

\item[SQL interface]

    This group of operations is for passing a user's query onto the
    server for processing and subsequently obtaining the results.

\end{description}

\chapter{Network API common functions}

\NAME{openserver() , closeserver() , serveruser() , servergroup() ,
serverpass() - Connect and disconnect from service}

\SYNOPSIS
\begin{verbatim}
#include "tstone.h"
SERVER *openserver(char *hostname,char *port);
SERVER *closeserver(SERVER *serverhandle);
int     serveruser(char *username);
int     servergroup(char *groupname);
int     serverpass(char *password);

\end{verbatim}

\DESCRIPTION

These rather generic sounding functions are for establishing or
disconnecting a communication channel to a Thunderstone Server.  The
presumption is made that the host and port you open is the correct one for
the type of calls you will be making.

In general, the \verb`openserver()` call returns a handle to the requested
service at the specified \verb`hostname` and \verb`port`.  The returned
handle is of the type \verb`SERVER *` and will have the value
\verb`SERVERPN`\footnote{\verb`SERVERPN` is a synonym for \verb`(SERVER*)NULL`} if there's a problem.

The \verb`closeserver()` call shuts the open communication channel with
a server and frees the allocated memory on both the client and the
server associated with the connection. \verb`closeserver()` will
return the value \verb`SERVERPN`.

The \verb`hostname` is a string that is the given internet name
or number of the machine on which the reqested service is running.
For example: \verb`thunder.thunderstone.com` or \verb`198.49.220.81`.
The port number is also a string, and is the port number assigned to
the service when it is brought up initially on the server machine. The
port number may also be assigned a name of a service that is enumerated
in the file \verb`/etc/services`.

Either \verb`hostname` or \verb`port` or both may be the empty string
(\verb`""`) indicating that the compiled in default is to be used.  The
default for \verb`hostname` is \verb`"localhost"` indicating the same
machine that the client application is running on.  The default
\verb`port` is \verb`"10000"` for Metamorph, \verb`"10001"` for 3DB
and \verb`"10002"` for Texis.

The \verb`serveruser()`, \verb`servergroup()`, and \verb`serverpass()`
calls set the user name, group name, and password, respectively, that
\verb`openserver()` will use when logging into the server.  These
functions will return zero on error.  Non-zero is returned on success.
If \verb`servergroup()` is not used, the user will be logged into their
default group as defined by the server.

The default user name and password are both the empty string (\verb`""`).

\EXAMPLE
\begin{verbatim}
 SERVER *se;                                  /* network server pointer */
 SRCH   *sr;                                          /* search pointer */

                                               /* connect to the server */
 if(serveruser("me")         &&
    servergroup("somegroup") &&
    serverpass("mypassword") &&
    (se=openserver("thunder.thunderstone.com","666"))!=SERVERPN)
    {
     n_reghitcb(se,(void *)NULL,mycallback);      /* setup hit callback */
     if((sr=n_setqry(se,"power struggle"))!=SRCHPN)  /* setup the query */
         {
          if(!n_search(se,sr))                         /* find all hits */
              puts("n_search() failed");
          n_closesrch(se,sr);                      /* cleanup the query */
         }
     closeserver(se);                         /* disconnect from server */
    }

\end{verbatim}

\CAVEATS
Make sure you are talking to the right port!

\SEE
\verb`/etc/services`, \verb`services(4)`

\chapter{Texis specific functions}

\NAME{n\_regtexiscb() - Register hit callback function}

\SYNOPSIS
\begin{verbatim}
#include "tstone.h"
FLDLST
{
 int    n;              /* number of items in each of following lists */
 int   *type;                             /* types of field (FTN_xxx) */
 void **data;                                   /* data array pointer */
 int   *ndata;                    /* number of elements in data array */
 char **name;                                        /* name of field */
};
int n_regtexiscb(SERVER *se,void *usr,func cb(void *usr,TEXIS *tx,FLDLST *fl));
\end{verbatim}

\DESCRIPTION
This function assigns the callback routine that the server is to call each
time it locates an item that matches the user's query.  The client program
can pass along a pointer to a ``user'' object to the register function
that will be in turn passed to the callback routine on each invocation.

The callback routine also gets a \verb`TEXIS` handle that can be used to
get further information about the hit (see \verb`n_getsrchlst()`) and a
\verb`FLDLST` handle that contains the \verb`select`ed fields.

A \verb`FLDLST` is a structure containing parallel arrays of information
about the selected fields and a count of those fields. \verb`FLDLST` members:

\verb`int n` --- The number of fields in the following lists.

\verb`int *type` --- An array of types of the fields.  Each element will
be one of the \verb`FTN_xxxx` macros.

\verb`void **data` --- An array of data pointers.  Each element will
point to the data for the field. The data for a field is an
array of \verb`type`s.

\verb`int *ndata` ---  An array of data counts. Each element says how
many elements are in the \verb`data` array for this field.

\verb`char **name` --- An array of strings containing the names of the
fields.

Possible types in \verb`FLDLST->type` array:

\verb`FTN_BYTE` --- An 8 bit \verb`byte`.

\verb`FTN_CHAR` --- A \verb`char`.

\verb`FTN_DOUBLE` --- A \verb`double`.

\verb`FTN_DWORD` --- A 32 bit \verb`dword`.

\verb`FTN_FLOAT` --- A \verb`float`.

\verb`FTN_INT` --- An \verb`int`.

\verb`FTN_INTEGER` --- An \verb`int`.

\verb`FTN_LONG` --- A \verb`long`.

\verb`FTN_INT64` --- An \verb`int64`.

\verb`FTN_UINT64` --- A \verb`uint64`.

\verb`FTN_SHORT` --- A \verb`short`.

\verb`FTN_SMALLINT` --- A \verb`short`.

\verb`FTN_WORD` --- A 16 bit \verb`word`.

\verb`FTN_INDIRECT` --- A \verb`char` string URL indicating the file that
the data for this field is stored in.

\verb`FTN_DATE` --- A \verb`time_t` in the same form as that from the
\verb`time()` system call.

The \verb`type` may also be \verb`|`ed with \verb`FTN_NotNullableFlag` (formerly \verb`DDNULLBIT`) and/or
\verb`DDVARBIT`. If \verb`FTN_NotNullableFlag` is set, it indicates that the field is not
allowed to be NULL. \verb`DDVARBIT` indicated that the field is
variable length instead of fixed length. When handling hit results
these bits can generally be ignored.

\verb`n_regtexiscb()` will return true if the registration was successful,
and will return false (0) on error.

\EXAMPLE
\begin{verbatim}
#include "tstone.h"

#define USERDATA my_data_structure
USERDATA
{
 FILE   *outfile;  /* where I will log the hits */
 long   hitcount;  /* just for fun */
};

void
dispnames(ud,fl)
USERDATA *ud;
FLDLST *fl;
{
 int i;

 for(i=0;i<fl->n;i++)                                    /* every field */
    printf("%s ",fl->name[i]);
 putchar('\n');                                      /* end header line */
}

void
dispfields(ud,fl)
USERDATA *ud;
FLDLST *fl;
{
 int i, j;

 for(i=0;i<fl->n;i++)
    {
     int   type=fl->type[i];                       /* the type of field */
     void *p   =fl->data[i];                     /* pointer to the data */
     int   n   =fl->ndata;                         /* how many elements */
     if(i>0) putchar('\t');      /* tab between fields, but not leading */
     switch(type&~(DDNULLBIT|DDVARBIT))/* ignore NULL and VARIABLE bits */
        {
                            /* loop through each element of field via j */
                                    /* cast and print according to type */
         case FTN_BYTE    : for(j=0;j<n;j++)
                             printf("0x%x ",((byte *)p)[j]);
                            break;
         case FTN_INDIRECT: /*nobreak; just print the filename as a string*/
         case FTN_CHAR    : for(j=0;j<n && ((char *)p)[j]!='\0';j++)
                             printf("%c"  ,((char *)p)[j]);
                            break;
         case FTN_DOUBLE  : for(j=0;j<n;j++)
                             printf("%lf ",((double *)p)[j]);
                            break;
         case FTN_DWORD   : for(j=0;j<n;j++)
                             printf("%lu ",(unsigned long)((dword *)p)[j]);
                            break;
         case FTN_FLOAT   : for(j=0;j<n;j++)
                             printf("%f " ,((float *)p)[j]);
                            break;
         case FTN_INT     : for(j=0;j<n;j++)
                             printf("%d " ,((int *)p)[j]);
                            break;
         case FTN_INTEGER : for(j=0;j<n;j++)
                             printf("%d " ,((int *)p)[j]);
                            break;
         case FTN_LONG    : for(j=0;j<n;j++)
                             printf("%ld ",((long *)p)[j]);
                            break;
         case FTN_SHORT   : for(j=0;j<n;j++)
                             printf("%d " ,((short *)p)[j]);
                            break;
         case FTN_SMALLINT: for(j=0;j<n;j++)
                             printf("%d " ,((short *)p)[j]);
                            break;
         case FTN_WORD    : for(j=0;j<n;j++)
                             printf("%u " ,(unsigned int)((word *)p)[j]);
                            break;
             /* assume exactly one element on FTN_DATE for this example */
         case FTN_DATE    : printf("%.25s",ctime(p));   /* human format */
                            break;
         default          : printf("unknowntype(%d)",type);
        }
    }
}

int
hit_handler(usr,tx,fl)
void *usr;  /* my USERDATA POINTER */
TEXIS *tx;  /* texis api handle */
FLDLST *fl; /* The field list data structure */
{
 USERDATA *ud=(USERDATA *)usr;      /* cast the void into the real type */

 ++ud->hitcount;                          /* add one to the hit counter */

 if(ud->hitcount==1)                            /* before the first hit */
    dispnames(ud,fl);                 /* display all of the field names */
 dispfields(ud,fl);                        /* display all of the fields */

 if(ud->hitcount>=100) /* tell the server to stop if I've seen too many */
    return(0);
 return(1);              /* tell the server to keep giving me more hits */
}

int
main()
{
 SERVER  *se;
 USERDATA mydata;
 ...
 mydata.outfile=(FILE *)NULL;
 mydata.hitcount=0;
 n_regtexiscb(se,mydata,hit_handler);
 ...
}

\end{verbatim}

\SEE
The example program \verb`netex3.c`.

\NAME{n\_getsrchlst(), n\_freesrchlst(), n\_srchinfo(), n\_freesrchinfo()
- Hit information}

\SYNOPSIS
\begin{verbatim}
#include "tstone.h"
SRCHLST
{
 int n;                                  /* number of elements in lst */
 SRCH lst[];                            /* list of Metamorph searches */
};
SRCHLST *n_getsrchlst(SERVER *se,TEXIS *tx);
SRCHLST *n_freesrchlst(SERVER *se,SRCHLST *sl);

SRCHI
{
 char *what;                                 /* what was searched for */
 char *where;                                       /* what was found */
 int   len;                                 /* length of where buffer */
};
SRCHI   *n_srchinfo(SERVER *se,SRCH *sr,int i);
SRCHI   *n_freesrchi(SERVER *se,SRCHI *si);
\end{verbatim}

\DESCRIPTION

These functions may be used within the hit callback function to obtain
detailed information about any Metamorph queries that may have been used
in the query.  \verb`n_getsrchlst()` takes the TEXIS handle passed to
the hit callback function and returns a list of handles to all Metamorph
searches associated with the query.  These handles may then be used in
calls to \verb`n_srchinfo()`.  \verb`SRCHLSTPN` is returned on error.
\verb`SRCHLST` members:

\verb`int n` --- The number of searches contained in \verb`lst`.

\verb`SRCH lst[]` --- The array of searches.

The \verb`SRCHLST` returned by \verb`n_getsrchlst()` should be freed by
calling \verb`n_freesrchlst()` when it is no longer needed.

\verb`n_srchinfo()` takes a search handle and the index of the sub-hit
to return information about.  It returns a \verb`SRCHI` pointer on
success or SRCHIPN on error or if the index is out of range.  The index
may be controlled by a loop to get information about all parts of the
hit.

Index values and what they return:
\begin{tabbing}
xxxxxx \= xxxxxxxxxxxxxxxxxxxxxxxxxxxxxx \= xxxxxxxxxxxxxxxxxxxxx \kill
Index  \> \verb`SRCHI->what` points to   \> \verb`SRCHI->where` contains \\
0      \> The original query             \> The whole hit         \\
1      \> A regular expression           \> The start delimiter   \\
2      \> A regular expression           \> The end delimiter     \\
3-n    \> The ``set'' being searched for \> The match             \\
       \> as listed below                \>                       \\
\end{tabbing}

\begin{tabbing}
xxxxxxxxx \= xxxxxxxxxxxxxxxxxxxxxxxxxxxxxxxxxx \kill
Set type  \> \verb`SRCHI->what` points to       \\
REX       \> A regular expression               \\
NPM       \> The npm query expression           \\
PPM       \> The root word of the list of words \\
XPM       \> The ``approximate'' string         \\
\end{tabbing}

Each \verb`SRCHI` returned by \verb`n_srchinfo()` should be freed by
calling \verb`n_freesrchi()` when it is no longer needed.

\SEE
\verb`n_xpminfo()` and its example.

\NAME{n\_xpminfo(), n\_freexpmi() - Hit information}

\SYNOPSIS
\begin{verbatim}
#include "tstone.h"
XPMI *n_xpminfo(SERVER *se,SRCH *sr,int index);
XPMI *n_freexpmi(SERVER *se,XPMI *xi);

\end{verbatim}

\DESCRIPTION

These functions may be used within the hit callback function to obtain
detailed information about any search terms that may have used the
approximate pattern matcher (XPM).  \verb`n_xpminfo()` is called with
the index of the desired XPM.

It returns a structure containing everything about that XPM.  It returns
XPMIPN\footnote{\verb`XPMIPN` is a synonym for \verb`(XPMI*)NULL`} if
index is out of bounds.

To get all XPM subhits put \verb`n_xpminfo()` in a loop with index
starting at zero and incrementing until \verb`XPMIPN` is returned.

Each valid structure returned by \verb`n_xpminfo()` should be freed by
calling \verb`n_freexpmi()` when it is no longer needed.

The XPMI structure contains the following members:
\begin{verbatim}
XPMI                                                      /* XPM Info */
{
 word  thresh;               /* threshold above which a hit can occur */
 word  maxthresh;                            /* exact match threshold */
 word  thishit;                               /* this hit's threshold */
 word  maxhit;                        /* max threshold located so far */
 char *maxstr;                      /* string of highest value so far */
 char *srchs;                              /* string being search for */
};

\end{verbatim}

\CAVEATS

Don't expect \verb`XPMI.thresh` to be the percentage entered in the
query passed to \verb`n_setqry()`.  It is an absolute number calculated
from that percentage and the search string.

\EXAMPLE
\begin{verbatim}

int
hit_handler(usr,tx,fl)
void *usr;  /* my USERDATA POINTER */
TEXIS *tx;  /* texis api handle */
FLDLST *fl; /* The field list data structure */
{
 ...
 MYAPP   *ts=(MYAPP *)usr;
 SERVER  *se=ts->se;
 SRCHLST *sl;
 SRCHI   *si;
 XPMI    *xi;
 int      i, j, k;

 ...
 sl=n_getsrchlst(se,tx);          /* get list of Metamorphs for query */
 if(sl!=SRCHLSTPN)
    {
     for(i=0;i<sl->n;i++)                       /* for each Metamorph */
         {
          SRCH *sr= &sl->lst[i];                             /* alias */
                                            /* loop thru all sub-hits */
              /* the zero index for n_srchinfo is the whole hit       */
              /* the one  index for n_srchinfo is the start delimiter */
              /* the two  index for n_srchinfo is the end delimiter   */
              /* the remaining indicies are the subhits               */
          for(j=0;(si=n_srchinfo(se,sr,j))!=SRCHIPN;j++)
              {
               char *p, *e;                /* scratch buffer pointers */
               switch(j)
               {
                case 0 :printf(" HIT    (%s):%d:",si->what,si->len); break;
                case 1 :printf(" S-DELIM(%s):%d:",si->what,si->len); break;
                case 2 :printf(" E-DELIM(%s):%d:",si->what,si->len); break;
                default:printf(" SUB-HIT(%s):%d:",si->what,si->len); break;
               }
               for(p=si->where,e=p+si->len;p<e;p++)
                   if(*p<32 && *p!='\n' && *p!='\t') printf("\\x%02x",*p);
                   else                              putchar(*p);
               printf("\n");
               n_freesrchi(se,si);              /* free any mem in si */
              }
          for(k=0;(xi=n_xpminfo(se,sr,k))!=XPMIPN;k++)/* loop thru all XPMs */
             {
              printf("XPM: \"%s\": thresh %u, maxthresh %u, thishit %u\n",
                     xi->srchs,xi->thresh,xi->maxthresh,xi->thishit);
              printf("   : maxhit %u, maxstr \"%s\"\n",xi->maxhit,xi->maxstr);
              n_freexpmi(se,xi);                    /* free mem in xi */
             }
         }
     n_freesrchlst(se,sl);
    }
 ...

 return(1);              /* tell the server to keep giving me more hits */
}

\end{verbatim}

\SEE
The example program \verb`netex1.c`, \verb`n_reghitcb()`,
\verb`n_getsrchlst()`, \verb`n_srchinfo()`.

\NAME{newblob(), putblob(), closeblob(), newindirect() - Object manipulation}

\SYNOPSIS
\begin{verbatim}
BLOB *newblob(char *database,char *table,char *buf,int n);
BLOB *putblob(BLOB *bl,char *buf,int n);
BLOB *closeblob(BLOB *bl);
int   getblob(BLOB *bl,char **buf,int *n);
char *url=newindirect(char *database,char *table,char *fn);
\end{verbatim}

\DESCRIPTION
These functions allow you to manipulate indirect and blob fields within
a Texis database.

\verb`newblob()` will create a new blob associated with the specified
database and table, and copy the buffer passed into the blob.  The
\verb`n` parameter says how many bytes from buf to copy.  It returns the
handle of the created blob.

\verb`putblob()` will put the contents of the buffer into the blob.  The
\verb`n` parameter says how many bytes from buf to copy.  It returns the
blob where the data was actually put.

{\bf Warning:} The returned \verb`BLOB` may be different than that
passed to \verb`putblob()`.  If so, the passed \verb`BLOB` will then be
invalid.

\verb`closeblob()` cleans up the memory associated with the blob handle.
The blob itself is left intact.  It returns \verb`BLOBPN`.

\verb`getblob()` retrieves data from a blob.  It will read in the data
from the blob, and return an allocated buffer containing the data, and
the number of bytes actually read in.  \verb`*buf` will be set to the
buffer pointer.  \verb`*n` will be set to the number of bytes in
\verb`*buf`.  It returns non-0 on success, 0 on failure.

\verb`newindirect()` generates a url that can be used to store data on
the server.  If \verb`fn` is \verb`NULL` or \verb`""` it will create a
url which can be used to store data in that is owned by Texis.  If
\verb`fn` is not \verb`NULL` and not \verb`""` and is not a url already
then it will be made into a url owned by you.  If url is a full path, it
will be respected.  Otherwise the standard path of indirect files for
the table will be prepended.

\verb`newindirect()` returns a url that can be stored into.  The url
that is returned is an allocated string that {\bf MUST} be freed by the
caller.

The blob handles and urls returned by these functions may then be used
as the field contents of blob and indirect fields respectively.

\NAME{n\_setXXX(), n\_getXXX() - Texis control parameters}

\SYNOPSIS
\begin{verbatim}
int   n_setdefaults(SERVER *se)
int   n_setdatabase(SERVER *se,str dbname)
str   n_getdatabase(SERVER *se)
\end{verbatim}

\DESCRIPTION
This collection of functions provide the needed control over how a {\bf
Texis} server will behave.  They are to be used prior to a call to
\verb`n_texis()`.  All of the functions have a common first argument
which is the omnipresent \verb`SERVER *`.  If a \verb`set` function
returns an \verb`int`, the value 0 means failure and \verb`not` 0 means
the operation was successful.  Those functions that have a \verb`void`
return value return nothing.  If a \verb`get` function returns a pointer
type, the value \verb`(type *)NULL` indicates a problem getting memory.
Otherwise the pointer should be freed when no longer needed.

\begin{description}
\item[void  n\_setdefaults(SERVER *se)]

     resets all server parameters to their initial state.

\item[int   n\_setdatabase(SERVER *se,str dbname)]

     sets \verb`dbname` as the name of the {\bf Texis} database that is
     to be queried against.

\item[str   n\_getdatabase(SERVER *se)]

     gets the name of the {\bf Texis} database that is
     to be queried against.
\end{description}

\NAME{n\_texis() - SQL interface}

\SYNOPSIS
\begin{verbatim}
int n_texis(SERVER *se,char *queryformat,...);
\end{verbatim}

\DESCRIPTION
This function comprises the real work that is to be performed by the
network Texis server.  To initiate the actual search the program makes a
call to the \verb`n_texis()` function.  The server will begin to call
the client's callback routine that was set in the \verb`n_regtexiscb()`
call.  The \verb`n_texis()` function will return 0 on error or true if
all goes well.  {\em NOTE:  It is not considered an error for there to
be zero hits located by a search.  A client's callback routine will
never be invoked in this instance.  }

The \verb`queryformat` argument is a \verb`printf()` style format string
that will be filled in by any subsequent arguments and then executed.

\EXAMPLE
\begin{verbatim}
#include "tstone.h"
main(argc,argv)
int argc;
cahr **argv;
{
 SERVER *se;
 char buf[80];
 USERDATA mydata;

 ...
 n_regtexiscb(se,mydata,hit_handler);           /* setup hit callback */
 n_setdatabase(se,argv[1]);                 /* set database to search */
 while(gets(buf)!=(char *)NULL)                   /* crude user input */
    if(!n_texis(se,"%s;",buf))       /* add required ';' for the user */
         puts("ERROR in n_texis");
 ...
}
\end{verbatim}
\end{document}             % End of document.
